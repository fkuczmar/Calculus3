\documentclass{ximera}
\title{Motion CW1}

\newcommand{\pskip}{\vskip 0.1 in}

\begin{document}
\begin{abstract}
Introduction to Motion.
\end{abstract}
\maketitle



\begin{question} \label{QodfdfEVCe}

The figure below shows the path of a beetle moving in the plane, with tracks at equally-spaced time intervals. The beetle's velocity vector (black) is shown at one particular position. 
\begin{enumerate}

\item Drag the tips of the two orange vectors to approximate the beetle's velocity at the two other positions. Explain your reasoning.

\item Activate the \emph{Solution} folder in Line 2 to see how you did.

\item Play the slider $u$ in Line 4 to start the motion.  Play particular attention to how the length of the velocity vector changes and make sure you understand these changes.

\begin{onlineOnly}
    \begin{center}
\desmos{onxs5dhqsm}{900}{600}
\end{center}
\end{onlineOnly}

\href{https://www.desmos.com/calculator/onxs5dhqsm}{Intro to Velocity}

\end{enumerate}
\end{question}


\begin{question} \label{QPdof3eDS}
The vector
\[
   {\bf p} = \langle 3, -2, 8\rangle \text{ meters}
\]
gives the position of a fly relative to the origin at some time.

The vector
\[
  {\bf v} = \langle -2, 1, 5 \rangle \text{ m/s}
\]
gives the velocity of the fly at that time.

\begin{enumerate}
\item Approximate the fly's position relative to the origin one-tenth of a second later.

\item Approximate the fly's position relative to the point $(10,4,-10)$ one-tenth of a second later.
\end{enumerate}
\end{question}



\begin{question} \label{Qpdfoer}
The function
\[
       {\bf p}_1(t) = \langle t-5, 0.5t^2, 1-t^2  \rangle \, , \, 0\leq t \leq 2 
\]
expresses the position (in feet) of a mosquito relative to the origin in terms of the number of seconds past noon.

At time $t=2$ seconds past noon the mosquito starts to fly with a constant velocity and does so for the next three seconds.

Find a function 
\[
 {\bf p}_2(t) \, , \, 2\leq t \leq 5 ,
\]
that expresses the position of the mosquito (in feet) relative to the origin in terms of the number of seconds past noon.
\end{question}




\begin{question} \label{QLDfer333}

The function 
\[
  {\bf p}(t) = \langle  t+2,t^{2}-t-3,2t \rangle \, , \, -3\leq t \leq 3,
\]
expresses the position (in meters) of a mosquito relative to the origin in terms of the number of seconds past noon.


\begin{onlineOnly}
    \begin{center}
\desmosThreeD{rwetbzlawt}{900}{600}
\end{center}
\end{onlineOnly}

\href{https://www.desmos.com/3d/rwetbzlawt}{Fly Without Plane}

\begin{enumerate}
\item Parameterize the tangent line to the mosquito's path at the point on the path with position ${\bf p}(w)$ relative to the origin. Enter your parameterization in Line 5 of the above worksheet. Unfortunately,  you'll need to use $t$ as the parameter. Keep in mind that this is not the same as the parameter $t$ in the motion parameterized above.

\item Drag the slider $w$ (another name for $t$) in Line 1 to approximate the time(s), if any, when the mosquito is flying directly toward the point $Q(5, -1, 6)$. Then ignore this approximation and use calculus to compute the exact time(s).

\item Drag the slider $w$ (another name for $t$) in Line 1 to approximate the time(s), if any, when the mosquito is flying directly away from the point $R(1,-5,-3)$. Then ignore this approximation and use calculus to compute the exact time(s).

\item Find all times when the mosquito is flying at a speed of $3$ m/sec.

\end{enumerate}

\end{question}



\begin{question} \label{QPfererer}

The function 
\[
  {\bf p}(t) = \Bigl \langle  \frac{t}{3}+2,\, \frac{t^2}{9}-\frac{t}{3}-3,\, \frac{2}{3}t \Bigr \rangle \, , \, -9\leq t \leq 9,
\]
expresses the position (in meters) of a mosquito relative to the origin in terms of the number of seconds past noon.

 
\begin{onlineOnly}
    \begin{center}
\desmosThreeD{gfr9ybm7mm}{900}{600}
\end{center}
\end{onlineOnly}

\href{https://www.desmos.com/3d/gfr9ybm7mm}{Fly and Plane}

\begin{enumerate}
\item Drag the slider $w$ (another name for $t$) in Line 1 above to approximate the time(s) when the mosquito is flying parallel to the plane
\[
   x + 3y + 2z =  5 .
\]
Then use algebra to find the exact time(s).

\item Drag the slider $w$ to approximate when the mosquito is flying directly away from the plane. Then use algebra to find the exact time(s).

\item Drag the slider to determine if the mosquito is flying toward or away from the plane at time $t=-3$ seconds past noon.

\item Use vectors to determine the rate at which the distance between the mosquito and the plane is changing at time $t=-3$ seconds past noon. Start by resolving the velocity vector into components parallel and normal to the plane.

\item Use calculus to determine the rate at which the distance between the mosquito and the plane is changing at time $t=-3$ seconds past noon. Start by finding a function that expresses the distance between the mosquito and the plane in terms of the number of seconds past noon.

\end{enumerate}




\end{question}

\end{document}

\documentclass{ximera}
\title{Statuary Hall and the Gradient Vector}

\newcommand{\pskip}{\vskip 0.1 in}

\begin{document}
\begin{abstract}
Statuary Hall and John Quincy Adams
\end{abstract}
\maketitle


\begin{question}  \label{Q656g5y4546}

\begin{center}
\youtube{watch?v=FX6rUU_74kk}
\end{center}

\href{https://www.youtube.com/watch?v=FX6rUU_74kk}{Statuary Hall}


Up until the middle of the 19th centuary, Statuary Hall in the U.S. Capitol was the meeting place of the House of Representatives. The room is in the shape of an ellipse and John Quincy Adams had his desk at one of the foci. Legend has it that he was able to eavesdrop on the whispered conversations of his political opponents when they were standing at the other focus.

The aim of this problem is to explain this and prove the \emph{reflective property of ellipses.}

\pskip

\emph{A light wave (or a sound wave) emitted from one focus of an ellipse passes through the other focus after reflecting off the ellipse.}

\pskip

We need to know that an ellipse is the set of points, the sum of whose distances from two fixed points (the foci) is constant. 

\begin{onlineOnly}
    \begin{center}
\desmosThreeD{3inyk1dxv4}{900}{600}
\end{center}
\end{onlineOnly}

Access Desmos interactives through the online version of this text at
 
\href{https://www.desmos.com/3d/3inyk1dxv4}{163: Statuary Hall 2}.

\begin{enumerate}
\item Let $\rho_0 = f_0(x,y)$ be the function that gives the distance (say in meters) between the origin and the point with coordinates $(x,y)$ (also in meters).

\begin{enumerate}
\item Without doing any computation, describe the gradient vector $\nabla f(x,y)$. In which direction does it point? What is its magnitude?

\item See if you are correct by computing the gradient.
\end{enumerate}

\item Repeat the previous question for the function $\rho_1 = f_1(x,y)$ that gives the distance (say in meters) between the point $(x_1, y_1)$ and the point $(x,y)$ (coordinates also measured in meters).

\item Find an equation of the ellipse with foci $(x_0,y_0)$ and $(x_1, y_1)$ that passes through the point $Q(x_2,y_2)$.

\item Use the results of the previous parts to describe a vector normal to the ellipse at the point $P(x,y)$.

\item Use the result of the previous question to prove the reflective property of ellipses.


\end{enumerate}
\end{question}


\begin{question}  \label{Q5445rggfbhyhyrdt}
Here's another way to prove the reflective property of ellipses..

The idea is that for a point moving around an ellipse, the rate of change (with respect to time) of its distances to the two foci sum to zero (why?).

\begin{onlineOnly}
    \begin{center}
\desmosThreeD{c9iyyxoqre}{900}{600}
\end{center}
\end{onlineOnly}

Access Desmos interactives through the online version of this text at
 
\href{https://www.desmos.com/3d/c9iyyxoqre}{163: Statuary Hall 3}.


To start the proof, forget about the ellipse for the moment and consider a general motion ${\bf p}(t)$, that gives the position (in meters) of a point $P$ relative to the origin in terms of time (measured in seconds). We wish to find an expression for the rate of change (with respect to time) in the distance between the point and the origin.

\begin{enumerate}
\item Work with vectors only (and not their components), to first find an expression for the distance from the origin to $P$.

\item Still working with vectors and not their components, differentiate your expression with respect to $t$.

\item Use the geometry of vectors to interpret the derivative geometrically.

\item Then prove the reflective property of ellipses. 
\end{enumerate}
\end{question}


\end{document}
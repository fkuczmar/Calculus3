\documentclass{ximera}
\title{Sailing in Circles}


\newcommand{\pskip}{\vskip 0.1 in}

\begin{document}
\begin{abstract}
An introduction to relative motion and frames of reference.
\end{abstract}
\maketitle


\section{Relative Position and Relative Velocity}

\begin{question} \label{QPFef3rfmMFz}

Let $P$, $Q$, and $R$ be distinct points not on the same line.

\begin{enumerate}
\item What vector gives the position of $P$ relative to $Q$?

\item What vector gives the position of $R$ relative to $Q$?

\item What vector gives the position of $P$ relative to $R$?

\item Draw a picture showing the three points and the three vectors.

\item Express the third vector above in terms of the first two.

\end{enumerate}
\end{question}


\begin{question}  \label{QLFr3dfmds}

Given the following:

\begin{enumerate}
\item The vector ${\bf w}$ gives the velocity of the air relative to the water.

\item The vector ${\bf v}$ gives the velocity of a boat relative to the water.

\end{enumerate} 

Express the velocity ${\bf u}$ of the air relative to the boat in terms of the vectors ${\bf w}$ and ${\bf v}$. Draw a picture to help with your explanation.
\end{question}


\section{Sailing in Circles}

Given the following:

\begin{enumerate}
\item Relative the water, the air moves due east at a speed of $w$ miles/hour. %Assuming the water is at rest relative to the land

\item A boat moves counterclockwise around a circle of radius $r$ miles at a constant speed of $v$ miles/hour.

\item In the $xy$ rectangular coordinate system below, the $x$-axis points due east, the $y$-axis due north, and the boat circles about the origin.

\item The $x^\prime y^\prime$ rectangular coordinate system is at rest relative to the boat and the positive $y^\prime$-axis points in the direction of motion.

\end{enumerate}

\begin{onlineOnly}
    \begin{center}
\desmos{y2o9zqibsv}{900}{600}
\end{center}
\end{onlineOnly}





\end{document}
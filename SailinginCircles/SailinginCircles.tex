\documentclass{ximera}
\title{Sailing in Circles}


\newcommand{\pskip}{\vskip 0.1 in}

\begin{document}
\begin{abstract}
An introduction to relative motion and frames of reference.
\end{abstract}
\maketitle


\section{Relative Position and Relative Velocity}

\begin{question} \label{QPFef3rfmMFz}

Let $P$, $Q$, and $R$ be distinct points not on the same line.

\begin{enumerate}
\item What vector gives the position of $P$ relative to $Q$?

\item What vector gives the position of $R$ relative to $Q$?

\item What vector gives the position of $P$ relative to $R$?

\item Draw a picture showing the three points and the three vectors.

\item Express the third vector above in terms of the first two.

\end{enumerate}
\end{question}


\begin{question}  \label{QLFr3dfmds}

Given the following:

\begin{enumerate}
\item The vector ${\bf w}$ gives the velocity of the air relative to the water.

\item The vector ${\bf v}$ gives the velocity of a boat relative to the water.

\end{enumerate} 

Express the velocity ${\bf u}$ of the air relative to the boat in terms of the vectors ${\bf w}$ and ${\bf v}$. Draw a picture to help with your explanation.
\end{question}


\section{Sailing in Circles}

\begin{question} \label{QLFever3r}

A boat moves counterclockwise around a circle at a constant speed of $v$ miles/hour relative the water.

The wind blows due east at a constant speed of $w$ miles/hr. We'll suppose the water is motionless relative to the land so that the velocity of the air is due east with a speed of $w$ miles/hour relative to the water. 

This problem is mainly about the velocity of the air relative to the boat. A flag on the boat would point in the direction of this velocity and a passenger on the boat would feel the wind blowing at a speed equal to the magnitude of this vector. But how the flag (and hence how the velocity vector) appears to change direction depends on the reference frame of the observer. A picknicker on land would see something dramatically different than a passenger on the boat.

To make this precise, we will plot the three velocity vectors (of the air relative to the water, of the boat relative to the water, and of the air relative to the boat) in two different coordinate systems. One system, the $xy$-coordinate system, is attached to the land and records the perspective of the picnicker. We'll suppose the positive $x$-axis points due east and the positive $y$-axis due north.

The other coordinate system, what we'll call the $x^\prime y^\prime$-system, is attached to the boat. It records the perspective of the passenger. The positive $y^\prime$-axis points forward in the direction of motion, the positive $x^\prime$-axis to the right of a forward-facing passenger.

\begin{onlineOnly}
    \begin{center}
\desmos{nhu1fnwbo0}{900}{600}
\end{center}
\end{onlineOnly}

\href{https://www.desmos.com/calculator/nhu1fnwbo0}{163: Sailing in Circles 4}



\begin{enumerate}
    \item The first step is to complete the picture above.
    \begin{enumerate}
    \item Express the vector ${\bf u}$ (the velocity of the air relative to the boat) in terms of the vectors ${\bf v}$ (the velocity of the boat relative to the water) and ${\bf w}$ (the velocity of the air relative to the water).

    \item Draw the vector ${\bf u}$ in a copy of the worksheet below. Then activate the folder \emph{Velocity of air relative to boat 1} in Line 13 to see how you did.

    \item Then turn off the folder in Line 13 and activate the folder \emph{Velocity of air relative to boat 2} in Line 14 to see the same vector ${\bf u}$ with its tail pinned at the boat $B$.
    \end{enumerate}

\item The next step is to compare how the vector ${\bf u}$ changes direction and magnitude in the reference frames of the passenger and picnicker.

\begin{enumerate}
    \item Play or drag the slider $U$ in Line 2. Then describe or sketch how the vector ${\bf u}$ changes direction from the perspective of the picnicker as the boat moves once around its circular path. Then do the same from the perspective of the passenger.

    \item Activate the folders \emph{Vectors in $xy$-frame} and \emph{Vectors in $x^\prime y^\prime$- frame} on Lines 21 and 32 to see how you did. Is there anything here that surprised you?

    \item Now change either the wind speed or the boat speed by dragging points $R$ or $S$ near the bottom of the worksheet. Summarize your observations about how these speeds affect the motions of the vector ${\bf u}$ in the two reference frames.
    
\end{enumerate}

\item Now we'll compute the components of the three vectors in the $xy$-coordinate system.

\begin{enumerate}
\item Find the components of the vector ${\bf w}$ in this frame.

\item Express the components of the vector ${\bf v}$ in terms of the marked angle $\phi$, $0\leq \phi \leq 2\pi$, measured counterclockwise from ${\bf w}$ to ${\bf v}$.

\item Use parts (a) and (b) to express the components of the vector ${\bf u}$ in terms of $\phi$.

\end{enumerate}
    
\item Now express the components of these same three vectors in terms of the same angle $\theta$ in the $x^\prime y^\prime$-coordinate system.

\item Find an expression for the speed of the wind relative to the boat in each coordinate system. Compare the two. Do this twice - once using the components of these vectors, and again using triangle trigonometry.

\item For what wind/boat speeds does a flag on the boat reverse its sense of rotation for the passenger? What fraction of the time does the flag rotate counterclockwise from the perspective of the passenger?

\item For what wind/boat speeds does a flag on the boat reverse its sense of rotation for the picnicker? What fraction of the time does the flag rotate counterclockwise from the perspective of the picnicker?

    
\end{enumerate}



\end{question}





\end{document}
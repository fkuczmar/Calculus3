\documentclass{ximera}
\title{Motion With a Constant Velocity}

\newcommand{\pskip}{\vskip 0.1 in}

\begin{document}
\begin{abstract}
Parameterizing motion with a constant velocity.
\end{abstract}
\maketitle


\section{A Traveling Mosquito}

\begin{question} \label{QLfder35r}
The continuous function 
\[
    {\bf v}(t) \, , \, 2 \leq t \leq 10 ,
\]
expresses the velocity (in meters/sec) of a mosquito moving in space in terms of the number of seconds past noon.

The mosquito passes the point $(3,2,5)$ meters at time $t=7$ seconds.

\begin{onlineOnly}
    \begin{center}
\desmosThreeD{idobesvy1e}{900}{600}
\end{center}
\end{onlineOnly}

\href{https://www.desmos.com/3d/idobesvy1e}{163: Mosquito Constant Velocity}

\begin{enumerate}
\item Find an expression for the function 
\[
   {\bf p}(t) \, , \, 2\leq t \leq 10,
\]
that gives the position (in meters) of the mosquito relative to the origin at time $t$ seconds past noon.

\item Suppose the mosquito moves with a constant velocity with a speed of $5$ m/sec and that it passes the point $(7,10,13)$ some time after time $t=7$ seconds past noon. 

\begin{enumerate}

\item Use the result of part (a) to find an expression for the function ${\bf p}(t)$ giving the mosquito's position relative to the origin (in meters) at time $t$ seconds past noon.

\item Enter your expression fro the position function in Line 2 of the above worksheet. 

\item Play the slider $t_1$ in Line 4 to check your work. Explain the meanings of the three vectors.
\begin{freeResponse}
\end{freeResponse}

\item When does the mosquito pass through the $xz$-plane? Use the worksheet above as a check.

\end{enumerate}
\end{enumerate}

\end{question}




\end{document}
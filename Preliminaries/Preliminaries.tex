\documentclass{ximera}
\title{Introduction}

\newcommand{\pskip}{\vskip 0.1 in}

\begin{document}
\begin{abstract}
A preview of some applications.
\end{abstract}
\maketitle


\section*{Planetary Motion}
Kepler discovered that the planets orbit the sun in elliptical paths with the sun at a focus of the ellipse. Later Newton used calculus to prove this to be true. 

The animation below shows a point $P$ orbiting the origin, its position (in meters) at time $t$ (measured in seconds) is given by the coordinate functions
\[
       (x,y) = (a \cos (\omega t), a \sin(\omega t)) \, , \, t \geq 0.
\]

\begin{exploration}
\begin{onlineOnly}
    \begin{center}
\desmos{5aq0sta5aa}{800}{600}  
\end{center}
\end{onlineOnly}

\href{https://www.desmos.com/calculator/5aq0sta5aa}{163: Central Force}
\end{exploration}


While the path is elliptical, the motion is not planetary-like because the ``sun'' is at the path's center and not at a focus. That is,  $P$'s acceleration points toward the origin at all times and not toward a focus of the ellipse. Equivalently, the force acting on $P$ points toward the center of the ellipse and not toward a focus as it would for a planet. %Or in terms of forces, the gravitational force of the sun acting on a planet points directly toward the sun and therefore toward 

Nevertheless, the force acting on $P$ is \emph{central} as it always points toward a fixed point $S$. And for any such force, the segment $\overline{SP}$ from the central point (the origin in our animation) to the moving point sweeps out area at a constant rate. And we don't need to rely on Newton to show this. You already know enough from your first two calculus classes. Here's the idea.


\begin{exploration}
Shown below is the differential triangle swept by $\overline{SP}$ as $\overline{SP}$ turns through the differential angle $d\theta = \angle PSP^\prime$.
\begin{onlineOnly}
    \begin{center}
\desmos{vtvt8znhxu}{800}{600}  
\end{center}
\end{onlineOnly}

\href{https://www.desmos.com/calculator/vtvt8znhxu}{163: Central Force 2}
\end{exploration}


\begin{question}  \label{Qdf4t5tht5544}
(a) (Integral Calculus) We'll start by letting $\theta$ be the radian measure of $\angle ASP$ and $r$ the length of $\overline{SP}$. Then as segment $\overline{SP}$ turns through the differential angle $d\theta$, it sweeps out
a differential triangle with height $r$, base ($\overline{PP^\prime}$ in the figure above) with differential length
\[
             db = \answer{r d\theta},
\]
and differential area
\[
    dA = \answer{\frac{1}{2}r^2 d\theta} .
\]

(b) Next divide both sides of the last equation by the differential time $dt$ seconds it takes $P$ to turn through the differential angle $d\theta$. This tells us that $\overline{SP}$ sweeps out area at the rate
\[
  \frac{dA}{dt} = \answer{\frac{1}{2}r^2 \frac{d\theta}{dt}} .
\]

\begin{freeResponse}
(a) What are the units of $dA/dt$ and $d\theta/dt$ above? 

(b) What is the meaning of $d\theta/dt$?
\end{freeResponse}

(c) Now just one more step. Remember, our goal is to show that $\overline{SP}$ sweeps out area at a constant rate. That is, we need to show that $dA/dt$ is constant. For this, we need an expression for the derivative $d\theta/dt$. I'll leave this to you. For a hint, click the \emph{Hint} tab at the start of this question.

\begin{hint}
Start by expressing $\theta$ in terms of $t$. You may assume $0 \leq \theta < \pi/2$. 
\[
    \theta = \answer{\arctan\left( \frac{b\sin (\omega t)}{a \cos (\omega t)}\right)} .
\]
\end{hint}

\pskip

This proof was not easy and we'll learn a better way in this class, one that's easier and works in greater generality.

\begin{freeResponse}   \label{Q:L98d7g33}
Here's a related question. How would you cut an elliptical pizza into eight equal slices (having the same area)?
\end{freeResponse}


\end{question}

\section*{The Doppler Effect}

\begin{onlineOnly}
    \begin{center}
\desmos{p2ismetboq}{800}{600}  
\end{center}
\end{onlineOnly}

\href{https://www.desmos.com/calculator/p2ismetboq}{Doppler Stationary Observer 1}



\begin{onlineOnly}
    \begin{center}
\desmos{alw5l83a0m}{800}{600}  
\end{center}
\end{onlineOnly}

\href{https://www.desmos.com/calculator/alw5l83a0m}{Doppler Effect}


alw5l83a0m


\section*{Small Changes}
Imagine flying a kite and letting out a little bit of string and the kite getting blown a small distance by the wind. How can we relate these two small changes, in the length of string and the little distance? We'll look at this general question later in the course, but you already know how to tackle a few special cases.


\begin{question}  \label{Q:09df0g4444e}
The general problem is three-dimensional, but we'll restrict ourselves to two dimensions by supposing the kite stays at a constant height, say $h$ meters, and moves along a straight path that lies directly above us. We'll let $s$ be the length of the string (in meters) and $x\geq 0$ the horizontal displacement of the kite (also in meters) measured from the point $B$ directly overhead at height $h$. Also, we'll let $\theta$ be the radian measure of the angle the string makes with the horizontal.

The problem now is to approximate $\Delta s$ (the change in the string length) in terms of the angle $\theta$ and $\Delta x$ (the small horizontal displacement of the kite). You might recognize this kind of approximation problem from calculus and I'll let you try this yourself. But before starting, you should ask yourself a simple question. 

\begin{freeResponse}
Which would you expect to be greater, $|\Delta s|$ or $|\Delta x|$? Explain.
\end{freeResponse}

Be sure to solve the problem in general, not using any numbers from the demonstration below. But you should use the demonstration to check your final result.

Click the Hint tab if you need help getting started.

\begin{hint}
Use the Pythagorean theorem to relate $x$, $h$, and $s$. 
\end{hint} 

%\begin{exploration}
Move the slider $u$ a little and record the small changes $\Delta x$ and $\Delta s$. Use this information to check your result. 
\begin{onlineOnly}
    \begin{center}
\desmos{7yad8xxxt8}{800}{600}  
\end{center}
\end{onlineOnly}

\href{https://www.desmos.com/calculator/7yad8xxxt8}{163: Kite}
%\end{exploration}

As a check, the final result is that
\[
    \Delta s \sim \answer{\cos\theta} \Delta x .
\]

\end{question}

\section*{Reflections}


\section*{Pendulums}

\begin{question}  \label{QDfdfgrenhht44}
The demonstration below shows the motion of a simple pendulum revolving about its center.

%\begin{exploration}
Stop the motion by pausing the Slider $v$. 
\begin{onlineOnly}
    \begin{center}
\desmos{nsfqtjr8fd}{800}{600}  
\end{center}
\end{onlineOnly}

\href{https://www.desmos.com/calculator/nsfqtjr8fd}{163: Revolving Pendulum}
%\end{exploration}

To describe a pendulum's motion, we could express the angular displacement from the low point $E$ as a function $\phi = f(t)$ of time. While this is not possible, at least explicitly, we can write the inverse function as
\[
     t = f^{-1}(\phi) = \int_0^{\phi/2} \frac{R\, du}{v_0\sqrt{1-k^2\sin^2u}} ,
\]
where $R$ is the pendulum's radius (in meters) and $v_0$ is the speed of the pendulum (in m/sec) as it passes $E$. The constant $k$ determines the shape of the motion and is equal to
\[
        k = \frac{2\sqrt{Rg}}{v_0} ,
\]
where $g$ is the magnitude of the gravitational acceleration (in $m/sec^2$).

Our goal here is to study the motion of a revolving pendulum with radius $R=7$ and launch speed $v_0=7$ in a uniform graviational field where $g=3$. The question is to compare the speeds of the pendulum at points $E$, $F$, and $G$ above by computing the double ratio
\[
  v_E : v_F : v_G
\]
of the respective speeds.

\begin{freeResponse}
(a) Use the animation to approximate this double ratio. Explain your reasoning.
\end{freeResponse}


(b) Use the expression for the function $f^{-1}$ above to compute the exact double ratio. You need not know anything more about simple pendulums to do this. All the necessary information is encoded in the expression for $f^{-1}$. 

The ratio is
\[
      v_E : v_F : v_G = \answer{7}: \answer{5}:\answer{1} .
\]



\begin{hint}
Start by finding an expression for the derivative $dt/d\phi$.
\end{hint}

\end{question}


\section*{Conservation of Energy}


\section*{Bicycle Paths}




\section*{Review Questions}

\begin{question}  \label{Qdfdsftt5466544}
(a) Give your understanding of what the Fundamental Theorem of Calculus says.

(b) Explain your understanding of \emph{why} the Fundamental Theorem is true. This question is not asking for any kind of proof, but for an inuitive explanation.
\end{question}


\begin{question}  \label{Q45rfdg55}
The continuous function
\[
  r = g(t) \, ,  \, 12 \leq t \leq 40,
\]
expresses the net rate (measured in gallons/minute) at which water flows into a tank in terms of the number of minutes past noon. The rate is positive when water flows into the tank and negative when water flows out. The tank holds $53$ gallons of water at 12:15pm.

Find an expression for the function
\[
    V = f(t)\, ,  \, 12 \leq t \leq 40,
\]
that expresses the volume of water in the tank (measured in gallons) in terms of the number of minutes past noon. Explain your reasoning.

\end{question}

\begin{question}  \label{Q5434g54t5t}
Kite?

\end{question}


\end{document}
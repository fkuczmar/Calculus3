\documentclass{ximera}
\title{Circlular Motion}

\newcommand{\pskip}{\vskip 0.1 in}

\begin{document}
\begin{abstract}
Circular motion, uniform and not.
\end{abstract}
\maketitle

\section{Uniform Circular Motion}
\begin{question}  \label{Q34trgr5t5t344}
Uniform circular motion means motion around a circle at a constant speed.

We'll take our circle to be centered at the origin and have radius $r$. We'll suppose our motion has position 
\[
     {\bf p}(0) = \langle r, 0 \rangle \, \text{meters}
\]
at time $t=0$ seconds and that it rotates counterclockwise about the origin at the constant rate of $\omega$ rad/sec.

\begin{onlineOnly}
    \begin{center}
\geogebra{uu2jshcp}{900}{600}
\end{center}
\end{onlineOnly}

\href{https://www.geogebra.org/classic/uu2jshcp}{Uniform Circlular Motion}

\begin{enumerate}
\item Find an expression for the function ${\bf p}(t)$ that gives the positon of the motion relative to the origin (in meters) in terms of the number of seconds since time $t=0$.

\item Find an expression for the velocity ${\bf v}(t)$ of the motion. What are its units? Check that your expression has these units.

\item Find an expression for the acceleration ${\bf a}(t)$ of the motion. What are its units? Check that your expression has these units.

\item In which direction does the velocity vector point? The accelaration vector?

\item Find an expression for the speed of the motion. Check that your expression has the correct units.

\item Find an expression for the magnitude of the acceleration. Check that your expression has the correct units.

\item How would you change one of the parameters $a$ and $\omega$ to keep the path and sense of rotation unchanged, and at the same time double the velocity at each point of the path?

\item Predict how the change you made in the previous question would affect the acceleration vector at each point of the path. Then use the animation above to check if you were correct. Can you explain this change intuitively, without referencing the expression for the acceleration?

\item Express the rotation rate in terms of the magnitude $a = | {\bf a}(t)|$ of the acceleration and the radius $r$.

\end{enumerate}
\end{question}

\section{General Circular Motion}
A motion around a circle need not be at a constant speed. 

We'll assume as before that the path is a circle of radius $r$ centered at the origin. But to allow for a general motion, we'll suppose the angle $\theta$ from the positive $x$-axis to the position vector ${\bf p}(t)$ is a twice differentiable function
\[
   \theta = f(t)
\]
of time. As always, the angle $\theta$ is measured counterclockwise. 

\begin{enumerate}
\item Find an expression for the function ${\bf p}(t)$ that gives the positon of the motion relative to the origin (in meters) in terms time (measured in seconds).

\item Find an expression for the velocity ${\bf v}(t)$ of the motion. What are its units? Check that your expression has these units.

\item Find an expression for the acceleration ${\bf a}(t)$ of the motion. What are its units? Check that your expression has these units.

\item Resolve the acceleration vector into two components, one tangent to the path, the other normal to the path. Write the acceleration vector as the sum of these components.

\item Express the scalar projection of the acceleration vector in the direction of the velocity vector in terms of $r$, ${\bf v}(t)$ and $d\theta/dt$. For this you can assume that $d\theta / dt \geq 0$. Interpret the meaning of this scalar projection.

\item Express the scalar projection of the acceleration vector in the direction of the inward pointing unit normal ${\bf n} = \langle -\cos\theta, -\sin\theta \rangle$ in terms of $r$, ${\bf v}(t)$ and $d\theta/dt$. Interpret the meaning of this scalar projection.

\item A point mass moving counterclockwise around an origin-centered circle of radius $5$ meters passes the point $(-3,4)$ meters with acceleration $\langle 2, -1\rangle \, m/s^2$.  

\begin{enumerate}
\item Is the mass speeding up or slowing down at this instant? At what rate?

\item At what rate is the mass turning about the origin at this instant?

\item What is the speed of the mass at this instant?
\end{enumerate}

\end{enumerate}


\begin{question}  \label{Q6756456443gg}
A disco ball in a circular dance hall of radius $r$ meters rotates at a constant rate of $\omega$ rad/sec. The ball casts a small spot of light on the wall (point $P$ in the animation below) that moves about the room. We suppose the ball (point $A$ below) is $a$ meters from the room's center.  We wish to parameterize the motion of $P$ and determine where on the path the motion is speeding up at the fastest rate.

\begin{onlineOnly}
    \begin{center}
\geogebra{m38mqcpa}{900}{600}
\end{center}
\end{onlineOnly}

\href{https://www.geogebra.org/classic/m38mqcpa}{163: Disco Dancing 2}

\begin{enumerate}
\item Begin by looking at the animiation above. Along what part of the path is the motion speeding up? Slowing down?

\item Use your observations to sketch a few acceleration vectors along the path. Think about both their directions and magnitudes. It should help to watch the hodograph as well.

\item Where on the path does it look like the motion is speeding up at the fastest rate? Slowing down at the fastest rate?

\begin{onlineOnly}
    \begin{center}
\geogebra{vtjgmmjt5c}{900}{600}
\end{center}
\end{onlineOnly}

\href{https://www.desmos.com/calculator/vtjgmmjt5c}{163: Disco Dancing 2}


\end{enumerate}



\end{question}




\end{document}
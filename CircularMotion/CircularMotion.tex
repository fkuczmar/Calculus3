\documentclass{ximera}
\title{Circlular Motion}

\newcommand{\pskip}{\vskip 0.1 in}

\begin{document}
\begin{abstract}
Circular motion, uniform and not.
\end{abstract}
\maketitle

\section{Uniform Circular Motion}
\begin{question}  \label{Q34trgr5t5t344}
Uniform circular motion means motion around a circle at a constant speed.

We'll take our circle to be centered at the origin and have radius $r$. We'll suppose our motion has position 
\[
     {\bf p}(0) = \langle r, 0 \rangle \, \text{meters}
\]
at time $t=0$ seconds and that it rotates about the origin at the constant rate of $\omega$ rad/sec. We'll allow negative values for $\omega$, in which case we'll suppose the sense of rotation is clockwise instead of counterclockwise (when $\omega > 0$). 

\begin{onlineOnly}
    \begin{center}
\geogebra{uu2jshcp}{900}{600}
\end{center}
\end{onlineOnly}

\href{https://www.geogebra.org/classic/uu2jshcp}{Uniform Circlular Motion}

\begin{enumerate}
\item Find an expression for the function ${\bf p}(t)$ that gives the positon of the motion relative to the origin (in meters) in terms of the number of seconds since time $t=0$.

\item Find an expression for the velocity ${\bf v}(t)$ of the motion. What are its units? Check that your expression has these units.

\item Find an expression for the acceleration ${\bf a}(t)$ of the motion. What are its units? Check that your expression has these units.

\item In which direction does the velocity vector point? The accelaration vector?

\item Find an expression for the speed of the motion. Check that your expression has the correct units.

\item Find an expression for the magnitude of the acceleration. Check that your expression has the correct units.

\item How would you change one of the parameters $r$ and $\omega$ to keep the path and sense of rotation unchanged but double the velocity at each point of the path?

\item Predict how the change you made in the previous question would affect the acceleration vector at each point of the path. Then use the animation above to check if you were correct. Can you explain this change intuitively, without referencing the expression for the acceleration vector?

\item Express the magntitude of the rotation rate in terms of the magnitude $a = | {\bf a}(t)|$ of the acceleration and the radius $r$.

\end{enumerate}
\end{question}

\section{General Circular Motion}
A motion around a circle need not be at a constant speed. 

\begin{question} \label{Q324234fgfg}
We'll assume as before that the path is a circle of radius $r$ centered at the origin. But to allow for a general motion, we'll suppose the angle $\theta$ from the positive $x$-axis to the position vector ${\bf p}(t)$ is a twice differentiable function
\[
   \theta = f(t)
\]
of time. As always, the angle $\theta$ is measured counterclockwise. 

\begin{enumerate}
\item Find an expression for the function ${\bf p}(t)$ that gives the positon of the motion relative to the origin (in meters) in terms time (measured in seconds). Click the arrow at the lower right for the answer.

\begin{expandable}
\[
   {\bf p} = r \langle \cos \theta , \sin\theta \rangle = r \langle \cos (f(t)) , \sin (f(t)) \rangle .
\]
\end{expandable}

\item Find an expression for the velocity ${\bf v}(t)$ of the motion. What are its units? Check that your expression has these units.
\begin{expandable}
%\begin{align*}
\[
   {\bf v} = \frac{d{\bf p}}{dt}  = \left( r \frac{d\theta}{dt} \right) \langle -\sin \theta , \cos\theta \rangle  \\
               %&= \left( r \frac{d\theta}{dt} \right) \langle -\sin \theta , \cos\theta \rangle                                 
\]
%\end{align*}
The units of the velocity $d{\bf p}/dt$ are meters/sec. To check this, note that the radius $r$ is measured in meters and the signed rotation rate (counterclockwise if postive, clockwise if negative) of the motion about the origin $d\theta/dt$ is measured in rad/sec.
\end{expandable}

\item Find an expression for the speed of the motion.

\begin{expandable}
The speed is
\begin{align*}
  | {\bf v} |  & = \Big|  \left( r \frac{d\theta}{dt} \right) \langle -\sin \theta , \cos\theta \rangle \Big|   \\
                & =  \Big| \left( r \frac{d\theta}{dt} \right) \Big| |  \langle -\sin \theta , \cos\theta \rangle  | \\
                &= r \Big|\frac{d\theta}{dt} \Big| .
\end{align*}
The speed is equal to the product of the radius (in meters) and the absolute (ie. unsigned) rotation rate (in rad/sec).
\end{expandable}

\item Find an expression for the acceleration ${\bf a}(t)$ of the motion. What are its units? Check that your expression has these units.

\begin{expandable}
Use the product rule to differentiate the velocity function in part (b) (because $r (d\theta/dt)$ and $\langle -\sin\theta, \cos\theta\rangle$ are both functions of $t$). This gives
\[
  {\bf a} =  \left( r \frac{d^2\theta}{dt^2} \right) \langle -\sin \theta , \cos\theta \rangle - r \left( \frac{d\theta}{dt} \right)^2 \langle \cos\theta , \sin\theta \rangle .
\]
\end{expandable}


\item Resolve the acceleration vector into two components, one tangent to the path, the other normal to the path. Write the acceleration vector as the sum of these vectors.

\begin{expandable}
The expression in part (d) resolves the acceleration vector in exactly this way. The first term is parallel (or anti-parallel if $d^2\theta/dt^2<0$) to the velocity vector and hence parallel to the path, the second term points in the direction opposite the position vector and is thefore normal to the path.
\end{expandable}

\item Express the scalar projection of the acceleration vector in the direction of the velocity vector in terms of $r$, ${\bf v}(t)$ and $d^2\theta/dt^2$. For this you can assume that $d\theta / dt>0$. Interpret the meaning of this scalar projection.

\begin{expandable}
We first use the expression for the velocity vector from part (b) to find the unit vector parallel to the velocity. This unit vector is
\[
   \frac{{\bf v}}{|{\bf v|}} = \frac{\frac{d\theta}{dt}}{|\frac{d\theta}{dt}|} \langle -\sin\theta, \cos\theta \rangle.
\]

Next we use the expression for the acceleration vector from part (d) and the unit vector above to compute the scalar projection of the acceleration vector in the direction of the velocity vector as
\[
    {\bf a}\cdot \frac{{\bf v}}{|{\bf v|}} = r \left( \frac{d^2\theta}{dt^2} \right)\left(\frac{\frac{d\theta}{dt}}{|\frac{d\theta}{dt}|}\right).
\]
So if $d\theta/dt >0$, the scalar projection is
\[
   {\bf a}\cdot \frac{{\bf v}}{|{\bf v|}} =  r \left( \frac{d^2\theta}{dt^2} \right)
\]
And if $d\theta/dt <0$, 
\[
   {\bf a}\cdot \frac{{\bf v}}{|{\bf v|}} =  - r \left( \frac{d^2\theta}{dt^2} \right) .
\]

To interpret the meaning of these scalar projections, first suppose $d\theta/dt >0$. Then we know the speed is
\[
   v =   r \Big| \frac{d\theta}{dt} \Big|    = r \left( \frac{d\theta}{dt} \right) .
\] 
So
\[
  \frac{dv}{dt} = r \left( \frac{d^2\theta}{dt^2} \right) =  {\bf a}\cdot \frac{{\bf v}}{|{\bf v|}} .
\]
This tells us that the scalar projection of the acceleration vector onto the velocity vector gives the rate at which the speed is changing with respect to time. The units of this rate of change are (meters/sec)/sec.

If on the other hand, $d\theta/dt < 0$, then (I'll leave this for you to work out) the result is still the same.

\emph{To summarize, the key points are as follows.}

\begin{enumerate}

\item For any circular motion, the scalar projection of the acceleration vector (measured in $m/s^2$) onto the velocity vector (measured in $m/s$) gives the rate at which the speed is changing with respect to time. The units of this rate of change are $m/sec^2$.

\item Given the angular position function $\theta = f(t)$ of the motion, we can compute this rate of change of speed with respect to time as either
\[
  \frac{dv}{dt} = r \left( \frac{d^2\theta}{dt^2} \right)  \text{ if } d\theta/dt > 0
\]
or
\[
   \frac{dv}{dt} = -r \left( \frac{d^2\theta}{dt^2} \right)  \text{ if } d\theta/dt < 0 .
\] 

\end{enumerate}

\end{expandable}

\item Express the scalar projection of the acceleration vector in the direction of the inward pointing unit normal ${\bf n} = \langle -\cos\theta, -\sin\theta \rangle$ in terms of $r$, ${\bf v}(t)$ and $d\theta/dt$. Interpret the meaning of this scalar projection.

\begin{expandable}
The scalar projection is
\[
   {\bf a}\cdot \frac{{\bf n}}{|{\bf n|}} =  r \left( \frac{d\theta}{dt} \right)^2 .
\]
The units are $m/s^2$. This scalar component of the acceleration vector in the direction of the inward-pointing normal is the product of the radius (in meters) and the square of the rotation rate (in $\text{rad}/\text{sec}^2$). More on this later in the course if time permits.
\end{expandable}

\item A point mass moving counterclockwise around an origin-centered circle of radius $5$ meters passes the point $(-3,4)$ meters with acceleration $\langle 2, -1\rangle \, m/s^2$.  

\begin{enumerate}
\item Is the mass speeding up or slowing down at this instant? At what rate?

\item At what rate is the mass turning about the origin at this instant?

\item What is the speed of the mass at this instant?

\begin{freeResponse}
\end{freeResponse}
\end{enumerate}

\end{enumerate}

\end{question}


\begin{question} \label{Qdsf3r34r5rt}
The animation below models the oscillation of a simple pendulum with an angular amplitude $a=0.2$ radians. This means the angular displacement $\theta  = \angle OCP$ oscillates between $-0.2$ and $0.2$ radians. The angle is measured counterclockwise from $\overrightarrow{OC}$ and is positive when $P$ is in the fourth quadrant, negative when $P$ is in the third quadrant.

The function 
\[
      \theta = f(t) = 0.2 \cos (\pi t) \, , \, t\geq 0, 
\]
expresses the angular displacement in terms of the number of seconds since the pendulum was released. 

The pendulum has length $L=1$ meter and period
\[
  T  = 2\pi \sqrt{L/g} \sim 2 \text{ sec}
\]
in a uniform graviational field with acceleration of magnitude g=9.8 $m/s^2$.

Because we are not measuring the angle $\theta$ from the positive $x$-axis, we need to adjust our function ${\bf p}(t)$ that gives the position of the pendulum $P$. Here we have
\begin{align*}
   {\bf p}(t) &= \langle \sin \theta, -\cos\theta   \rangle  \\
                  &= \langle \sin (0.2 \cos (\pi t)), - \cos (0.2 \cos (\pi t)) \rangle  , t \geq 0.
\end{align*}

But don't let this bother you. You'll need to modify our earlier results and adjust the components of the velocity and acceleration vectors accordingly, but the coordinate-free results (about the rate of change of speed with respect to time) still apply exactly as before.


\begin{onlineOnly}
    \begin{center}
\geogebra{gkbsdkrd}{900}{600}
\end{center}
\end{onlineOnly}

\href{https://www.geogebra.org/classic/gkbsdkrd}{163: Simple Pendulum 1}

\begin{enumerate}
\item The first step is to watch the animation above and then draw the acceleration vectors (by hand) at various points along the path. Don't do any computations for this. Just use the animation. Then click the Acceleration box at the upper left to see how you did. Then do the following.
\begin{enumerate}
\item Draw your picture of what you thought the acceleration vectors would look like.

\item Describe how the acceleration vector actually varies along the pendulum's path. Include a picture.

\item Is there anything about the acceleration vector that surprised you? Explain.

\item What did you learn from this animation?

\item What questions do you have?
\end{enumerate}

\begin{freeResponse}
\end{freeResponse}

\item Use the position function above to find expressions for the velocity and acceleration vectors.

\item Find the velocity and acceleration vectors the first time $\theta = 0$ radians. Is the pendulum speeding up or slowing down at this time? At what rate?

\item  Find the velocity and acceleration vectors the second time $\theta = 0$ radians. Is the pendulum speeding up or slowing down at this time? At what rate?

\item Is the pendulum speeding up or slowing down the first time $\theta = 0.1$ radian? At what rate? 

\item Is the pendulum speeding up or slowing down the first time $\theta = -0.1$ radian? At what rate?

\item Is the pendulum speeding up or slowing down the first time $\theta = -0.2$ radians? At what rate? 

\item What can you say about the maximum rate at which the speed of the pendulum increases and the times when this occurs?

\item Sketch a graph of the speed function and a graph of its derivative. 

\end{enumerate}


\end{question}

\begin{question}  \label{Q32frr3r4r4}
This question is also about circular motions.
\begin{enumerate}

\item Let $\phi$ be the angle between the acceleration and velocity vectors, both assumed to be non-zero. Show that $0\leq \phi \leq \pi/2$ radians.

\begin{hint}
See the work from part (g) of the previous question.
\end{hint}

\item The figure below shows possible acceleration and velocity vectors (color-coded) at different points on a circular path. Which configurations are actually possible? Explain your reasoning. \emph{Note: A missing vector indicates the vector vanishes (ie. is equal to the zero vector}.

\begin{freeResponse}
\end{freeResponse}

\begin{onlineOnly}
    \begin{center}
\geogebra{yzcwddhb}{900}{600}
\end{center}
\end{onlineOnly}

\href{https://www.geogebra.org/classic/yzcwddhb}{163: Circular Motion Conceptual Question}

\end{enumerate}

\end{question}


\begin{question}  \label{Q6756456443gg}
A disco ball in a circular dance hall of radius $r$ meters rotates at a constant rate of $\omega$ rad/sec. The ball casts a small spot of light on the wall (point $P$ in the animation below) that moves about the room. We suppose the ball (point $A$ below) is $a$ meters from the room's center.  We wish to parameterize the motion of $P$ and determine where on the path the motion is speeding up at the fastest rate.

\begin{onlineOnly}
    \begin{center}
\geogebra{m38mqcpa}{900}{600}
\end{center}
\end{onlineOnly}

\href{https://www.geogebra.org/classic/m38mqcpa}{163: Disco Dancing 2}

\begin{enumerate}
\item Begin by looking at the animation above. Along what part of the path is the motion speeding up? Slowing down?

\item Use your observations to sketch a few acceleration vectors along the path. Think about both their directions and magnitudes. It should help to watch the hodograph as well.

\item Where on the path does it look like the motion is speeding up at the fastest rate? Slowing down at the fastest rate?

\begin{onlineOnly}
    \begin{center}
\desmos{vtjgmmjt5c}{900}{600}
\end{center}
\end{onlineOnly}

\href{https://www.desmos.com/calculator/vtjgmmjt5c}{163: Disco Dancing 2}

\item Show that with $\phi = \omega t$, the speed of the motion is
\[
     v = f(\phi) = r \omega \left( 1- \frac{a\cos\phi}{\sqrt{r^2-a^2\sin^2\phi}}   \right) .
\]

\item Show that
\[
      \frac{dv}{d \phi} = \frac{ r\omega(r^2-a^2)\sin\phi}{r^2- a^2\sin^2\phi} .
\]

\begin{hint}
Activate the folder on Line 31 in the desmos demonstration above. 
\begin{enumerate}
\item Use the law of sines in $\Delta OAP$ to express $\alpha = \angle OPA$ in terms of $a$ and $\angle PAB = \pi - \angle PAB =  \pi -\phi$.
\[
   \alpha  = \arcsin \left( \answer{\frac{a\sin \phi}{r}} \right)
\]

\item Then express $\theta = \angle POB$ in terms of $\phi$ and $\alpha$. 
\[
  \theta = \answer{\phi - \alpha} .
\]

\item Finally, note that the speed of $P$ is
\[
   v = \frac{d}{dt} (r\theta) = r\omega \frac{d\theta}{d\phi}
\]

\end{enumerate}
\end{hint}

\item What values of $\phi$ maximize/minimize the derivative above? No need for calculus here.

\end{enumerate}



\end{question}




\end{document}
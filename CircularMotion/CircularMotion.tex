\documentclass{ximera}
\title{Circlular Motion}

\newcommand{\pskip}{\vskip 0.1 in}

\begin{document}
\begin{abstract}
Circular motion, uniform and not.
\end{abstract}
\maketitle

\section{Uniform Circular Motion}
Uniform circular motion means motion around a circle at a constant speed.

We'll take our circle to be centered at the origin and have radius $a$. We'll suppose our motion starts at position 
\[
     {\bf p}(0) = \langle a, 0 \rangle \, \text{meters}
\]
at time $t=0$ seconds and that it rotates counterclockwise about the origin at the constant rate of $\omega$ rad/sec.

\begin{onlineOnly}
    \begin{center}
\geogebra{uu2jshcp}{900}{600}
\end{center}
\end{onlineOnly}

\href{https://www.geogebra.org/classic/uu2jshcp}{Uniform Circlular Motion}

\begin{enumerate}
\item Find an expression for the position function ${\bf p}(t)$ that gives the positon of the motion relative to the origin (in meters) in terms of the number of seconds since time $t=0$.

\item Find an expression for the velocity ${\bf v}(t)$ of the motion. What are its units? Check that your expression has these units.

\item Find an expression for the acceleration ${\bf a}(t)$ of the motion. What are its units? Check that your expression has these units.

\item In which direction does the velocity vector point? The accelaration vector?

\item Find an expression for the speed of the motion. Check that your expression has the correct units.

\item Find an expression for the magnitude of the accerlation. Check that your expression has the correct units.

\item How would you change the parameters $a$ and $\omega$ to keep the path and sense of rotation unchanged and double the velocity at each point of the path?

\item Predict how the change you made in the previous question would affect the acceleration vector at each point of the path. Then use the animation above to check if you were correct. Can you explain this change intuitively, without referencing the expression for the acceleration?






\end{enumerate}

\end{document}
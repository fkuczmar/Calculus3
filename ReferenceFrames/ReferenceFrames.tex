\documentclass{ximera}
\title{Frames of Reference}

\newcommand{\pskip}{\vskip 0.1 in}

\begin{document}
\begin{abstract}
Viewing the same motion from different reference frames.
\end{abstract}
\maketitle


\section{Projectile Motion}
\begin{question}  \label{Qdsf435r45te}

\begin{onlineOnly}
    \begin{center}
\desmosThreeD{sqvm9cvfyd}{900}{600}
\end{center}
\end{onlineOnly}

\href{https://www.desmos.com/3d/sqvm9cvfyd}{163: Rotating Frame}


\end{question}


\section{Einstein's Trains}
\begin{question}  \label{Qsdfds44445te}
We use SI units as usual.

A train moves along the line $y=h$ with constant acceleration ${\bf a}$ and position function
\[
 {\bf p}_t = {\bf p}(0) + {\bf v}(0)t + \frac{1}{2} {\bf a} \, , \, t\geq 0 .
\]
At time $t=0$ you throw a rock from the train with initial velocity
\[
         {\bf w} = \langle w_0, 0 \rangle \, m/s
\]
\emph{relative to the train.}

Our aim is to describe the rock's motion first relative to the ground and then relative to the train.

\begin{onlineOnly}
    \begin{center}
\desmos{uzywalzuny}{900}{600}
\end{center}
\end{onlineOnly}

\href{https://www.desmos.com/calculator/uzywalzuny}{163: Train Problem}


\end{question}

\section{Adam's Apple}

\begin{onlineOnly}
    \begin{center}
\desmos{xvfucsguj9}{900}{600}
\end{center}
\end{onlineOnly}

\href{https://www.desmos.com/calculator/xvfucsguj9}{163: Adams Apple}



\section{Retrograde Motion of Mars}

\begin{center}
\youtube{https://www.youtube.com/watch?v=1nVSzzYCAYk}
\end{center}

\href{https://www.youtube.com/watch?v=1nVSzzYCAYk}{Retrograde Motion of Mars}


\end{document}
\documentclass{ximera}
\title{Errors in Approximations}

\newcommand{\pskip}{\vskip 0.1 in}

\begin{document}
\begin{abstract}
Bounding the error in approximations.
\end{abstract}
\maketitle


\begin{question} \label{PDLf3909bv}

This question is about the series
\[
    \sum_{k=1}^\infty \frac{1}{1+k^2} .
\]

\begin{enumerate}
\item Determine if the series converges or diverges by comparing it with the appropriate improper integral. Explain your reasoning fully in complete sentences. Draw a picture to help with your explanation.

\item You approximate the series with the sum of the first 10 terms. Evaluate an improper integral to find an upper bound for the error in your approximation. Draw a picture to help with your explanation.

\item At least how many terms of the series would you need to sum to approximate the series with an error of at most $10^{-8}$?

\end{enumerate}
\end{question}


\begin{question} \label{QLkdf3rr33}
This question is about the MacLaurin series
\[
     \sum_{k=0}^\infty \frac{x^k}{k!} 
\]
for $e^x$.

\begin{enumerate}
\item Use the ratio test to determine the radius of convergence.

\item Use summation notation to write the sum of the first 10 terms of the series that approximates $e$.

\item Use a geometric series to find an upper bound for the error in part (b).

\item Use a geometric series to find an upper bound for the error in approximating $e$ with the first $n$ terms of the appropriate series.

\item Use the result of part (d) and technology to approximate the minimum number of terms needed to approximate $e$ with an error of at most $10^{-12}$.
\end{enumerate}

\begin{onlineOnly}
    \begin{center}
\desmos{vygj2rwyhm}{900}{600}
\end{center}
\end{onlineOnly}

\href{https://www.desmos.com/calculator/vygj2rwyhm}{163: Approximating e}
\end{question}

\begin{question} \label{QOfe48xqzq}
This question  is about the MacLaurin series
\[
     x - \frac{x^2}{2} + \frac{x^3}{3} - \frac{x^4}{4} + \ldots
\]
for $\ln(1+x)$.

\begin{enumerate}
\item Write the above series in summation notation.

\item Use the ratio test to determine the open interval of convergence.

\item Check the endpoints to determine the interval of convergence.

\item Find an upper bound for the error in approximating $\ln 2$ with the first 10 terms of the appropriate series.

\item Approximate the minumum number of terms needed to approximate $\ln 2$ with an error of at most $10^{-12}$.

\item Find an upper bound for the error in approximating $\ln (1/2)$ with the first 10 terms of the appropriate series.

\item Use a geometric series to find an upper bound for the error in approximating $\ln(1/2)$ with the first $n$ terms of the appropriate series.

\item Use the result of part (g) and technology to approximate the minimum number of terms needed to approximate $\ln(1/2)$ with an error of at most $10^{-12}$.



\end{enumerate}

\begin{onlineOnly}
    \begin{center}
\desmos{50jirk9jgg}{900}{600}
\end{center}
\end{onlineOnly}

\href{https://www.desmos.com/calculator/50jirk9jgg}{163: Approximating ln 2}


\end{question}


\end{document}
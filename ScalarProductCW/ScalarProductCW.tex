\documentclass{ximera}
\title{The Scalar Product, CW}

\newcommand{\pskip}{\vskip 0.1 in}

\begin{document}
\begin{abstract}
The scalar product.
\end{abstract}
\maketitle

\section{Closest Point on a Line}

\begin{question}  \label{ExLMF3r3Er3}

\begin{enumerate}
\item Parameterize the line through the points $A(-2,1,3)$ and $B(-1,3,1)$.

\begin{onlineOnly}
    \begin{center}
\desmosThreeD{pcav8vrwva}{900}{600}
\end{center}
\end{onlineOnly}

\href{https://www.desmos.com/3d/pcav8vrwva}{163: Closest Point on a Line 2}

\item Use the chain rule to find an expression for the derivative
\[
 \frac{d}{dx}\left( (5x-3)^2) \right) .
\]

\item Use calculus and the chain rule as in part (b) to find the coordinates of the point on the line in part (a) closest to the point $P(6,8,5)$.

\item Instead of using calculus, use the scalar prodcut to find the coordinates of the point on the ine in part (a) closest to the point $P(6,8,5)$. \emph{Hint:} The key idea is that a point $Q$ on the line is closest to $P$ if and only if $\overrightarrow{QP}$ is perpendicular to the line.

\item Compare the algebra of your solutions in parts (c) and (d).

\end{enumerate}
\end{question}


\section{Lines of Sight}

\begin{question} \label{QPFLdfeEfe}
A sensor at the point $A(0,2,0)$ measures the line of sight to an object to be in the direction of the vector ${\bf v} = \langle 1, 1, 2 \rangle$. Another sensor at the point $B(3,0,0)$ measures the line of sight to the same object to be in the direction of the vector ${\bf w} = \langle -2,1,1 \rangle$.

\begin{onlineOnly}
    \begin{center}
\desmosThreeD{qzu8ssds5l}{900}{600}
\end{center}
\end{onlineOnly}

\href{https://www.desmos.com/3d/qzu8ssds5l}{163: Lines of Sight 2}

\begin{enumerate}
\item Verify algebraically that the lines of sight do not intersect. Start by parameterizing each line. Be sure to use different parameters. Why?

\item Experiment with the sliders $u_A$ and $u_B$ in the worksheet above to approximate the most likely position of the object.

\begin{hint}
There is one segment with its endpoints on the lines that is perpendicular to both lines of sight. The best guess is that the object is at the midpoint of this segment. 
\end{hint}

\item Use the scalar product to find the exact coordinates of the best approximation to the object's position. Click on the \emph{Reveal Hint} button at the top of this question and activate the folder \emph{Solution} in Line 14 for help.
\end{enumerate}

\end{question}

\section{Drawing a Circle}




\end{document}

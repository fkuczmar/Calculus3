\documentclass{ximera}
\title{The Scalar Product, CW}

\newcommand{\pskip}{\vskip 0.1 in}

\begin{document}
\begin{abstract}
The scalar product.
\end{abstract}
\maketitle

\section{Closest Point on a Line}

\begin{question}  \label{ExLMF3r3Er3}

\begin{enumerate}
\item Parameterize the line through the points $A(-2,1,3)$ and $B(-1,3,1)$.

\begin{onlineOnly}
    \begin{center}
\desmosThreeD{pcav8vrwva}{900}{600}
\end{center}
\end{onlineOnly}

\href{https://www.desmos.com/3d/pcav8vrwva}{163: Closest Point on a Line 2}

\item Use the chain rule to find an expression for the derivative
\[
 \frac{d}{dx}\left( (5x-3)^2) \right) .
\]

\item Use calculus and the chain rule as in part (b) to find the coordinates of the point on the line in part (a) closest to the point $P(6,8,5)$. \emph{Hint:} Treat this as an optimization problem.

\item Instead of calculus, use the scalar product to find the coordinates of the point on the ine in part (a) closest to the point $P(6,8,5)$. \emph{Hint:} The key idea is that a point $Q$ on the line is closest to $P$ if and only if $\overrightarrow{QP}$ is orthogonal to the line.

\item Compare the algebra of your solutions in parts (c) and (d).

\end{enumerate}
\end{question}


\section{Lines of Sight}

\begin{question} \label{QPFLdfeEfe}
A sensor at the point $A(0,2,0)$ measures the line of sight to an object to be in the direction of the vector ${\bf v} = \langle 1, 1, 2 \rangle$. Another sensor at the point $B(3,0,0)$ measures the line of sight to the same object to be in the direction of the vector ${\bf w} = \langle -2,1,1 \rangle$.

\begin{onlineOnly}
    \begin{center}
\desmosThreeD{qzu8ssds5l}{900}{600}
\end{center}
\end{onlineOnly}

\href{https://www.desmos.com/3d/qzu8ssds5l}{163: Lines of Sight 2}

\begin{enumerate}
\item Verify algebraically that the lines of sight do not intersect. Start by parameterizing each line. Be sure to use different parameters. Why?

\item Experiment with the sliders $u_A$ and $u_B$ in the worksheet above to approximate the most likely position of the object.

\begin{hint}
There is one segment with its endpoints on the lines that is perpendicular to both lines of sight. The best guess is that the object is at the midpoint of this segment. 
\end{hint}

\item Use the scalar product to find the exact coordinates of the best approximation to the object's position. Click on the \emph{Reveal Hint} button at the top of this question and activate the folder \emph{Solution} in Line 14 for help.
\end{enumerate}

\end{question}

\section{Drawing a Circle}


\begin{question} \label{Qdf6:CircleandTSquare}

We can draw a circle with a T-square or a rectangle, as illustrated in the animation below. The idea is to slide the rectangle $PQRS$ so that the adjacent sides $PQ$ and $PS$ respectively pass through the two fixed points $A$ and $B$. Then the vertex $P$ traces out a circle.

The  aim of this question is to use the scalar product to prove this statement.

\href{https://www.desmos.com/calculator/mgihe5dswd}{163: Drawing a Circle With a Rectangle}

 
\begin{onlineOnly}
    \begin{center}
\desmos{mgihe5dswd}{900}{600}
\end{center}
\end{onlineOnly}

We'll start by letting ${\bf a}$, ${\bf b}$, and ${\bf p}$ be the position vectors (with respect to some origin) of the respective points $A$, $B$, and $P$. The points $A$ and $B$ are fixed, but the point $P$ varies as we slide the rectangle. 

The curve traced by $P$ is described by the geometric condition that the vectors $\overrightarrow{AP}$ and $\overrightarrow{BP}$ are orthogonal. So a vector equation of the curve traced by $P$ is
\[
       \overrightarrow{AP}   \cdot \overrightarrow{BP} = \answer{0} ,
\]
or equivalently  
\[
     (  {\bf p} -  {\bf a} ) \cdot ({\bf p} - {\bf b}  ) = \answer{0} .
\]

Note. While this is a vector equation, the curve is still a set of points; namely, the set of points $P$ satstifying the above equation, where ${\bf p} = \overrightarrow{OP}$ is the postion of $P$ relative to the origin.

The idea now is to use algebra to show the curve is a circle in two dimensions or a sphere in three dimensions (or a hypersphere in higher dimensions). Using vectors gives a coordinate-free approach that works for any number of dimensions. 

For the algebra, we complete the square in the usual way by first writing the equation in the form
\begin{align*}
0 &= (  {\bf p} -  {\bf a} ) \cdot ({\bf p} - {\bf b}  ) \\
   &= {\bf p}\cdot {\bf p} - ({\bf a} + {\bf b})\cdot {\bf p} + {\bf a}\cdot {\bf b} .\\
\end{align*}

Now rewrite this as
\[
          {\bf p}\cdot {\bf p} - ({\bf a} + {\bf b})\cdot {\bf p}  = - {\bf a}\cdot {\bf b} .
\]

Addint 
\[
  \frac{1}{4}( {\bf a} + {\bf b} ) \cdot  ( {\bf a} + {\bf b} )
\]
to each side gives
\[
  {\bf p}\cdot {\bf p} - ({\bf a} + {\bf b})\cdot {\bf p} + \frac{1}{4}( {\bf a} + {\bf b} ) \cdot  ( {\bf a} + {\bf b} ) = \frac{1}{4}( {\bf a} + {\bf b} ) \cdot  ( {\bf a} + {\bf b} ) -  {\bf a}\cdot {\bf b}
\]
or
\[
     \left( {\bf p} - \frac{1}{2}\left( {\bf a} + {\bf b}  \right)  \right) \cdot \left( {\bf p} - \frac{1}{2}\left( {\bf a} + {\bf b}  \right) \right) = \frac{1}{4}( {\bf a} + {\bf b} ) \cdot  ( {\bf a} + {\bf b} ) -  {\bf a}\cdot {\bf b} .
\]

And finally,
\[
     \left| {\bf p} - \frac{1}{2}\left( {\bf a} + {\bf b} \right) \right| = \frac{1}{2}\sqrt{ ({\bf a} + {\bf b} ) \cdot  ( {\bf a} + {\bf b} ) -  4{\bf a}\cdot {\bf b}} 
\]
or
\[
  \left| {\bf p} - \frac{1}{2}\left( {\bf a} + {\bf b} \right) \right| = \frac{1}{2}\left| {\bf a} - {\bf b}  \right| .
\]

\end{question}


\end{document}

\documentclass{ximera}
\title{Linear Approximation}

\newcommand{\pskip}{\vskip 0.1 in}

\begin{document}
\begin{abstract}
Linear approximations of functions of two or more independent variables.
\end{abstract}
\maketitle


\section{Partial Derivatives}

\begin{example}  \label{Edgtr543}

The function
\[
    T = f(x,y) \, , \, x^2 + y^2 \leq 3
\]
expresses the temperature (in Celsius degrees) of a point on a hot plate in terms of its coordinates, measured in meters. 

(a) Use the graph of the level curves of this function shown below to approximate the partials 
\[
    \frac{\partial T}{\partial x}\Big|_{(x,y)=(2,-1)} \text{ and } \frac{\partial T}{\partial y}\Big|_{(x,y)=(2,-1)} .
\]

\begin{itemize}
\item{Do this by turning off the Level Curve Folder in Line 8 and first turning on the Partial x Folder in Line 14. Use the slider $T$ to adjust the level curve $f(x,y)=T$.} 

\item{Then turn off the Partial x Folder, activate the Partial y Folder, and repeat.}

\item{Follow the directions in Lines 21 and 23 to check your approximations.}

\item{Include units for the partials and explain their meanings.}

\end{itemize}

\pdfOnly{
Access Desmos interactives through the online version of this text at
 
\href{https://www.desmos.com/calculator/mrbwyydrcm}.
}
 
\begin{onlineOnly}
    \begin{center}
\desmos{mrbwyydrcm}{900}{600}
\end{center}
\end{onlineOnly}

Desmos activity available at

\href{https://www.desmos.com/calculator/mrbwyydrcm}{163: Level Curves Hemisphere}

\pskip \pskip


(b) Write an expression for the average rate of change in temperature with respect to the change $w$ in the $x$-coordinate relative to the point $(2,-1)$. Enter this expression on Line 22 of the above worksheet and follow the directions there to approximate 
\[
   \frac{\partial T}{\partial x}\Big|_{(x,y)=(2,-1)} .
\]
Explain the idea.

(c) Write an expression for the average rate of change in temperature with respect to the change $w$ in the $y$-coordinate relative to the point $(2,-1)$. Enter this expression on Line 24 of the above worksheet and follow the directions there to approximate 
\[
   \frac{\partial T}{\partial y}\Big|_{(x,y)=(2,-1)} .
\]
Explain the idea.

(d) Suppose now that 
\[
    T = f(x,y) = \sqrt{9-x^2-y^2} \, , \, x^2 + y^2 \leq 3 .
\]
Find the exact values of the two partials above and compare them with your estimates.

\end{example}

\section{Coordinate Curves and Tangent Planes}


\begin{example}  \label{Esdtr435r}
This is a continuation of Example 1. 

\pskip \pskip

(c) Describe the surface
\[
  T = f(x,y) = \sqrt{9-x^2-y^2} \, , \, x^2 + y^2 \leq 3 .
\]

\pskip

(d) Write expressions  for the partials
\[
    \frac{\partial T}{\partial x}\Big|_{(x,y)=(2,-1)} \text{ and } \frac{\partial T}{\partial y}\Big|_{(x,y)=(2,-1)} .
\]
as limits of average rates of change. Then use the demonstration below to give a geometric interpretation of the average rates of change $\Delta T/\Delta x$ and $\Delta T/\Delta y$ and their respective limits as $\Delta x\to 0$ and $\Delta y\to 0$.

\pdfOnly{
Access Geogebra interactives through the online version of this text at
 
\href{https://www.geogebra.org/classic/egnkjkqw}.
}
 
\begin{onlineOnly}
    \begin{center}
\geogebra{egnkjkqw}{900}{600}
\end{center}
\end{onlineOnly}

Geogebra activity available at

\href{https://www.geogebra.org/classic/egnkjkqw}{163: Level Curves and Sphere}




(e) Open the desmos demonstration at

\href{https://www.desmos.com/3d/967c10cbdc}{163: Tangent Planes to Sphere}

\pskip

(f) Parameterize the $x$-coordinate curve cut from the surface by the plane $y=-1$.

(g) Use part (f) to parameterize the tangent line to the $x$-coordinate curve at the point $P(2,-1,2)$ on the surface.

(h) Parameterize the $y$-coordinate curve cut from the surface by the plane $x=2$.

(i) Use part (h) to parameterize the tangent line to the $y$-coordinate curve at the point $P(2,-1,2)$ on the surface.

(j) Use parts (g) and (i) to find an equation of the tangent plane to the surface at the point $P(2,-1,2)$.

(k) Answer parts (b) and (j) geometrically without using calculus.

(l) Use the result of part (j) to find the linear approximation to the function $T=f(x,y)$ at the point $(2,-1)$.

(M) Use your linear approximation from part (l) to estimate the temperatures at the points with coordinates $(2.1, -1.1)$ and $(1.99, -0.99)$. Without computing the actual temperatures, determine whether your estimates are greater or less than the actual values. Explain your reasoning.

\end{example}



\begin{example} \label{E4rt4t42}
For this entire problem let
\[
   z = f(x,y) = \arctan(y/x) 
\]
and let $P$ be the point on the graph of the function $f$ with coordinates $(\sqrt{3},1,f(\sqrt{3},1))$.

(a) Parameterize the $x$-coordinate curve cut from the surface $z=f(x,y)$ by the plane $y=1$. Use this parameterization to parameterize the tangent line to the curve at $P$.

(b) Parameterize the $y$-coordinate curve cut from the surface $z=f(x,y)$ by the plane $x=\sqrt{3}$. Use this parameterization to parameterize the tangent line to the curve at $P$.

(c) Use parts (a) and (b) to find an equation of the tangent plane to the surface $z=f(x,y)$ at $P$.

(d) Parameterize the level curve of the function $z=f(x,y)$ through the point $(\sqrt{3},1,0)$. Use this parameterization to parameterize the curve on the surface $z=f(x,y)$ lying directly above the level curve (ie. lift the level curve to the surface).

(e) Use the desmos link below to check your work by doing the following:

\begin{itemize}
\item{Input equations of the coordinate planes $x=\sqrt{3}$ and $y=1$.}

\item{Input parameterizations of the two coordinate curves. Use $t$ for the parameter and choose the domain $t\in[-10,10]$.}

\item{Input parameterizations of the two tangent lines. Use $t$ for the parameter and choose the domain $t\in[-10,10]$.}

\item{Input an equation of the tangent plane.}

\item{Input a parameterization for the lift of the level curve. Use $t$ for the parameter and choose the domain $t\in[-10,10]$.}

\item{Include both one or several screenshots of your work and a link to your saved file.}

\end{itemize}


\pskip 

\href{https://www.desmos.com/3d/ade9317b51}{163:Coordinate Curves Student Version}




\end{example}







\section{Tangent Planes}

\begin{exploration} \label{Edgcgtg}
The purpose of this exercise is to explore one method of finding an equation of the tagent plane to the differentiable surface $z=f(x,y)$ at the point $B(a,b,f(a,b))$.

Here are the steps. Use the geogebra exploration to help visualize the ideas, but do \emph{not} use the particular function $f$ graphed there. Work with the general function $z=f(x,y)$ instead.

\pskip

\pskip

(a) Parameterize the $x$-coordinate curve cut from the surface $z=f(x,y)$ by the plane $y=b$.

(b) Use your parameterization from part (a) to parameterize the tangent line to the the $x$-coordinate curve at $B$.

(c) Parameterize the $y$-coordinate curve cut from the surface $z=f(x,y)$ by the plane $x=a$.

(d) Use your parameterization from part (c) to parameterize the tangent line to the the $y$-coordinate curve at $B$.

(e) Use parts (b) and (d) to find a vector normal to the surface $z=f(x,y)$ at $B$.

(f) Use part (e) to write an equation of the tangent plane to the surface $z=f(x,y)$ at the point $B$.


\begin{onlineOnly}
    \begin{center}
\geogebra{fgdjhdjm}{900}{600}
\end{center}
\end{onlineOnly}


\href{https://www.geogebra.org/classic/fgdjhdjm}{163: Coordinate Curves and Tangent Plane}




\end{exploration}




\begin{exploration}  \label{Ede5fhj4665}
\href{https://www.geogebra.org/m/Hud6Hnpk}{Partial Derivatives and Tangent Planes}
\end{exploration}


\section{Linear Approximation}
\begin{exploration}  \label{Ecgnbhyrt}
The demonstration below shows some level curves of a differentiable function
\[
   z = f(x,y) .
\]

(a) Zoom in toward the point $P(a,b)$ and adjust the slider $n$ as appropriate. What do you notice if you zoom in close enough to the point $P$?

(b) Turn on the Linear Approximation folder in Line 11. What do you notice?

(c) What does this demonstration suggest about how to find the linear approximation to the differentiable function $f$ at the point $(a,b)$?


\pdfOnly{
Access Desmos interactives through the online version of this text at
 
\href{https://www.desmos.com/calculator/8hphbdetoj}.
}
 
\begin{onlineOnly}
    \begin{center}
\desmos{8hphbdetoj}{900}{600}
\end{center}
\end{onlineOnly}


\href{https://www.desmos.com/calculator/8hphbdetoj}{163: Level Curves Zoom In}

\end{exploration}


\section{Error Analysis}
When you do an experiment, you typically make measurements and use these measurements to compute something. But measurements have errors and these errors lead to errors in the computed values. We can use linear approximation to see how these errors are related.

\begin{example}  \label{Edstrt4th}
You measure the length and width of a rectangle and use these measurements to compute the area. Use a linear approximation to estimate the error and the relative error in computing the area. Assume the errors in the measurements are small.

\begin{explanation}
Let $L$ and $W$ be the measured length and width, in meters, and let
\[
  A =f(L,W) = LW ,
\]
be the computed area, in square meters. 

Also let $\Delta L$, $\Delta W$ be the errors (measured in meters) in the measurements of the dimensions and let $\Delta A$ be the error (in square meters) in the computed area.

Then
\begin{align*}
   \Delta A  &\sim  \left( \frac{\partial A}{\partial L} \right) \Delta L + \left( \frac{\partial A}{\partial W} \right)  \Delta W \\ \\
                &=  W\Delta L +  L \Delta W .
\end{align*}

\begin{question}  \label{Qegffg545}
Compare the above approximation to $\Delta A$ with the exact error
\[
    \Delta A = (L+\Delta L)(W + \Delta W) 
\]
and sketch a picture to illustrate the two errors (exact and approximate) geometrically.
\end{question}


So, for example, if we measured the dimesions of the rectangle as $L=20$, $W=10$, with respective errors $\Delta L = 0.1$ and $\Delta W = -0.2$, then the approximate error in the computed area would be
\[
    \Delta A \sim (10\text{ ft})(0.1 \text{ ft}) + (20 \text{ ft})(-0.2 \text{ ft}) = - 3 \text{ ft}^2 .
\]

This kind of analysis is typically not very useful because we usually do not know if our measurements are greater or less than the actual values. Rather, we might have some idea about the maximum possible errors in our measurements. We might know, for example, that
\[
      | \Delta L | \leq M \text{  and  }  |\Delta W| \leq N
\]
for some positive constants $M$ and $N$. We can then use these bounds on the errors in our measurements to approximate an upper bound for the error in the computed area. The key is to assume the worst possible case, that the errors in the measurements do not work in opposite directions (as they did in the last example). And for this we need the triangle inequality, that for real numbers $x$ and $y$.
\[
   | x \pm y | \leq |x| + |y| .
\] 
Actually, we've seen this inequality before in its more general, vector form, namely that for vectors ${\bf v}, {\bf w}\in \mathbb{R}^n$,
\[
      | {\bf v} \pm {\bf w}   |  \leq |{\bf v}| +  | {\bf w} | .
\]

\begin{question}  \label{Qdgbbhu7}
(a) Draw two pictures that illustrate why the last inequality holds. 

(b) Explain in words why that inequality is true.
\end{question}

Now continuing with our example, since
\[
  \Delta A \sim W\Delta L + L \Delta W ,
\]
the triangle inequality tells us that

\begin{align*}
    | \Delta A |  &\sim |   W\Delta L +  L \Delta W | \\ \\
                      & \leq W|\Delta L| + L |\Delta W| .
\end{align*}

\begin{question}  \label{Q34gnyyt}
 Suppose $L=20$, $W=10$, with respective errors $|\Delta L| \leq  0.1$ and $|\Delta W| \leq  0.2$. Approximate an upper bound for $|\Delta A|$.
\end{question}


It is often more meaningful to think about relative errors instead of absolute errors. For example, I might approximate the distance from Seattle to Longmire to be $100$ miles with an absolute error of at most $5$ miles. Or the distance from Seattle to Spokane to be $250$ miles with an error of at most $10$ miles. In the first case, the relative error is
\[
          \frac{5 \text{ miles}}{100 \text{ miles}} = 5\% ,
\]
and in the second
\[
          \frac{10\text{ miles}}{250 \text{ miles}} = 4\% .
\]
So in some sense, the second estimate is more accurate than the first.


Now suppose the relative errors in measuring the length and width of our rectangle are at most $p\%$ and $q\%$, respectively, and that $p,q \sim 0$.  We wish to approximate the maximum relative error
\[
      \frac{|\Delta A|}{A}
\]
in the computed area.

We are given that 
\[
      \frac{|\Delta L|}{L} \leq p\% \text{ and } \frac{|\Delta W|}{W} \leq q\% .
\]

Then since
\[
  | \Delta A | \leq W|\Delta L| + L |\Delta W| ,
\]
dividing both sides by $A = LW$ gives
\begin{align*}
      \frac{|\Delta A|}{A} & \leq \frac{|\Delta L|}{L} + \frac{|\Delta W|}{W} \\ \\
                                   & \leq p\% + q\% .
\end{align*}

So the relative error in the product is at most the sum of the relative errors in the factors (if the relative errors in the factors are small).

\end{explanation}

\end{example}


\begin{question}   \label{Qefg547ji8}
(a) What are the units in the relative errors above?

(b) Draw a picture that illustrates the statement about the relative error in a product.
\end{question}
 
\begin{question}  \label{Qert5ty7}
You measure the mass and volume of a sample of gold with respective relative errors of at most $p\%$ and $q\%$. Use the appropriate linear approximation to estimate the maximum relative error in the computed density. Assume $p,q\sim 0$.

\end{question}



\begin{question}  \label{Qerfgghym}
You measure your distance from the base of a tree and the angle of elevation to the top of the tree. You then use these measurements to compute the height of the tree above eye level. Suppose the measured distance to be $s$ feet and the computed height to be $h$ feet.

(a) Use the appropriate linear approximation to estimate the error in the computed height of the tree above eye level. Express the error in terms of $s$, $h$, the error $\Delta s$ (in feet) in measuring $s$, and the error $\Delta \theta$ (in radians) in measuring the angle of elevation. Your expression should not involve the measured angle of elevation $\theta$.

(b) Suppose the relative error in measuring the distance is at most $p\%$ and the absolute error in measuring the angle is at most $M$ radians (ie. that $|\Delta \theta|\leq M$). Approximate an upper bound for the relative error in the computed height of the tree above eye level. Your expression should be in terms of  $p\%$, $M$, $s$, and $h$.            %the relative error in the measured distance and the (absolute) error in the measured angle. 

(c) Suppose $s=100$, $h=200$, and $p=2$. Use part (b) to approximate the maximum error in measuring the angle of elevation to make the relative error in the computed height at most $5\%$.
\end{question}


\begin{question}  \label{QE4gb4ttb}
You approximate a small island to be $6$ miles due east and $8$ miles due west of your current position. You use these approximations to compute both the distance and bearing in sailing directly to the island. 

(a) Use the appropriate linear approximations to estimate the errors in the computed distance and bearing.

(b) Interpret the expressions for these errors geometrically.

(c) Where in this class have we done similar problems before?

\end{question}



\end{document}
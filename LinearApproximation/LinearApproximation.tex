\documentclass{ximera}
\title{Linear Approximation}

\newcommand{\pskip}{\vskip 0.1 in}

\begin{document}
\begin{abstract}
Linear approximations of functions of two or more independent variables.
\end{abstract}
\maketitle


\section{Partial Derivatives}

\begin{example}  \label{Edgtr543}

The function
\[
    T = f(x,y) \, , \, x^2 + y^2 \leq 3
\]
expresses the temperature (in Celsius degrees) of a point on a hot plate in terms of its coordinates, measured in meters. 

(a) Use the graph of the level curves of this function shown below to approximate the partials 
\[
    \frac{\partial T}{\partial x}\Big|_{(x,y)=(2,-1)} \text{ and } \frac{\partial T}{\partial y}\Big|_{(x,y)=(2,-1)} .
\]

\begin{itemize}
\item{Do this by turning off the Level Curve Folder in Line 8 and first turning on the Partial x Folder in Line 14. Use the slider $T$ to adjust the level curve $f(x,y)=T$.} 

\item{Then turn off the Partial x Folder, activate the Partial y Folder, and repeat.}

\item{Follow the directions in Lines 21 and 23 to check your approximations.}

\item{Include units.}

\end{itemize}

\pdfOnly{
Access Desmos interactives through the online version of this text at
 
\href{https://www.desmos.com/calculator/et6f10fnti}.
}
 
\begin{onlineOnly}
    \begin{center}
\desmos{et6f10fnti}{900}{600}
\end{center}
\end{onlineOnly}

\href{https://www.desmos.com/calculator/et6f10fnti}{Level Curves Hemisphere}

\pskip \pskip

(b) Suppose now that 
\[
    T = f(x,y) = \sqrt{9-x^2-y^2} \, , \, x^2 + y^2 \leq 3 .
\]
Find the exact values of the two partials above and compare them with your estimates.

\end{example}

\section{Tangent Planes and Linear Approximations}


\begin{example}  \label{Esdtr435r}
This is a continuation of Example 1. 

\pskip \pskip

(c) Graph the surface
\[
  T = f(x,y) = \sqrt{9-x^2-y^2} \, , \, x^2 + y^2 \leq 3 .
\]

\pskip

(d) Open the desmos deomonstration at

\href{https://www.desmos.com/3d/d78d5a3138}{Tangent Planes to Sphere}

\pskip

(e) Parameterize the $x$-coordinate curve cut from the surface by the plane $x=2$.

(f) Parameterize the tangent line to the $x$-coordinate curve at the point $P(2,-1,2)$ on the surface.

(g) Parameterize the $y$-coordinate curve cut from the surface by the plane $y=-1$.

(h) Parameterize the tangent line to the $y$-coordinate curve at the point $P(2,-1,2)$ on the surface.

(i) Find an equation of the tangent plane to the surface at the point $P(2,-1,2)$.

(j) Verify parts (b)-(h) geometrically.

(k) Use the result of part (h) to approximate the temperature at the points with coordinates $(2.1, -1.1)$ and $(1.99, -0.99)$. Compare these approximations with the actual temperatures.

\end{example}



\section{Tangent Planes}
\begin{exploration}  \label{Edf754665}

\pdfOnly{
Access Desmos interactives through the online version of this text at
 
\href{https://www.desmos.com/calculator/y0h5kuvmbt}.
}
 
\begin{onlineOnly}
    \begin{center}
\desmos{y0h5kuvmbt}{900}{600}
\end{center}
\end{onlineOnly}


\href{https://www.desmos.com/calculator/y0h5kuvmbt}{The Partial Derivative}

\end{exploration}



\begin{exploration}  \label{Edf5khj4665}

\href{https://www.desmos.com/3d/d78d5a3138}{Tangent Planes to Sphere}

\end{exploration}

\begin{exploration}  \label{Ede5fhj4665}
\href{https://www.geogebra.org/m/Hud6Hnpk}{Partial Derivatives and Tangent Planes}
\end{exploration}


\section{Linear Approximation}


\end{document}
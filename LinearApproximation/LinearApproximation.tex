\documentclass{ximera}
\title{Linear Approximation}

\newcommand{\pskip}{\vskip 0.1 in}

\begin{document}
\begin{abstract}
Linear approximations of functions of two or more independent variables.
\end{abstract}
\maketitle


\section{Partial Derivatives}

\begin{example}  \label{Edgtr543}

The function
\[
    T = f(x,y) \, , \, x^2 + y^2 \leq 3
\]
expresses the temperature (in Celsius degrees) of a point on a hot plate in terms of its coordinates, measured in meters. 

(a) Use the graph of the level curves of this function shown below to approximate the partials 
\[
    \frac{\partial T}{\partial x}\Big|_{(x,y)=(2,-1)} \text{ and } \frac{\partial T}{\partial y}\Big|_{(x,y)=(2,-1)} .
\]

\begin{itemize}
\item{Do this by turning off the Level Curve Folder in Line 8 and first turning on the Partial x Folder in Line 14. Use the slider $T$ to adjust the level curve $f(x,y)=T$.} 

\item{Then turn off the Partial x Folder, activate the Partial y Folder, and repeat.}

\item{Follow the directions in Lines 21 and 23 to check your approximations.}

\item{Include units for the partials and explain their meanings.}

\end{itemize}

\pdfOnly{
Access Desmos interactives through the online version of this text at
 
\href{https://www.desmos.com/calculator/mrbwyydrcm}.
}
 
\begin{onlineOnly}
    \begin{center}
\desmos{mrbwyydrcm}{900}{600}
\end{center}
\end{onlineOnly}

Desmos activity available at

\href{https://www.desmos.com/calculator/mrbwyydrcm}{163: Level Curves Hemisphere}

\pskip \pskip


(b) Write an expression for the average rate of change in temperature with respect to the change $w$ in the $x$-coordinate relative to the point $(2,-1)$. Enter this expression on Line 22 of the above worksheet and follow the directions there to approximate 
\[
   \frac{\partial T}{\partial x}\Big|_{(x,y)=(2,-1)} .
\]
Explain the idea.

(c) Write an expression for the average rate of change in temperature with respect to the change $w$ in the $y$-coordinate relative to the point $(2,-1)$. Enter this expression on Line 24 of the above worksheet and follow the directions there to approximate 
\[
   \frac{\partial T}{\partial y}\Big|_{(x,y)=(2,-1)} .
\]
Explain the idea.

(d) Suppose now that 
\[
    T = f(x,y) = \sqrt{9-x^2-y^2} \, , \, x^2 + y^2 \leq 3 .
\]
Find the exact values of the two partials above and compare them with your estimates.

\end{example}

\section{Tangent Planes and Linear Approximations}


\begin{example}  \label{Esdtr435r}
This is a continuation of Example 1. 

\pskip \pskip

(c) Describe the surface
\[
  T = f(x,y) = \sqrt{9-x^2-y^2} \, , \, x^2 + y^2 \leq 3 .
\]

\pskip

(d) Use the demonstration below to interpret the meanings of the average rates of change $\Delta T/\Delta x$ and $\Delta T/\Delta y$ and their respective limits as $\Delta x\to 0$ and $\Delta y\to 0$ geometrically.

\pdfOnly{
Access Geogebra interactives through the online version of this text at
 
\href{https://www.geogebra.org/classic/egnkjkqw}.
}
 
\begin{onlineOnly}
    \begin{center}
\geogebra{egnkjkqw}{900}{600}
\end{center}
\end{onlineOnly}

Geogebra activity available at

\href{https://www.geogebra.org/classic/egnkjkqw}{163: Level Curves and Sphere}




(e) Open the desmos demonstration at

\href{https://www.desmos.com/3d/967c10cbdc}{163: Tangent Planes to Sphere}

\pskip

(f) Parameterize the $x$-coordinate curve cut from the surface by the plane $y=-1$.

(g) Use part (f) to parameterize the tangent line to the $x$-coordinate curve at the point $P(2,-1,2)$ on the surface.

(h) Parameterize the $y$-coordinate curve cut from the surface by the plane $x=2$.

(i) Use part (h) to parameterize the tangent line to the $y$-coordinate curve at the point $P(2,-1,2)$ on the surface.

(j) Use parts (g) and (i) to find an equation of the tangent plane to the surface at the point $P(2,-1,2)$.

(k) Answer parts (b) and (j) geometrically without using calculus.

(l) Use the result of part (j) to find the linear approximation to the function $T=f(x,y)$ at the point $(2,-1)$.

(M) Use your linear approximation from part (l) to estimate the temperatures at the points with coordinates $(2.1, -1.1)$ and $(1.99, -0.99)$. Without computing the actual temperatures, determine whether your estimates are greater or less than the actual values. Explain your reasoning.

\end{example}



\section{Tangent Planes}
\begin{exploration}  \label{Edf754665}

\pdfOnly{
Access Desmos interactives through the online version of this text at
 
\href{https://www.desmos.com/calculator/y0h5kuvmbt}.
}
 
\begin{onlineOnly}
    \begin{center}
\desmos{y0h5kuvmbt}{900}{600}
\end{center}
\end{onlineOnly}


\href{https://www.desmos.com/calculator/y0h5kuvmbt}{The Partial Derivative}

\end{exploration}




\begin{exploration}  \label{Ede5fhj4665}
\href{https://www.geogebra.org/m/Hud6Hnpk}{Partial Derivatives and Tangent Planes}
\end{exploration}


\section{Linear Approximation}


\end{document}
\documentclass{ximera}
\title{Linear Approximation}

\newcommand{\pskip}{\vskip 0.1 in}

\begin{document}
\begin{abstract}
Linear approximations of functions of two or more independent variables.
\end{abstract}
\maketitle


\section{Partial Derivatives}

\begin{example}  \label{Edgtr543}

The function
\[
    T = f(x,y) \, , \, x^2 + y^2 \leq 3
\]
expresses the temperature (in Celsius degrees) of a point on a hot plate in terms of its coordinates, measured in meters. 

(a) Use the graph of the level curves of this function shown below to approximate the partials 
\[
    \frac{\partial T}{\partial x}\Big|_{(x,y)=(2,-1)} \text{ and } \frac{\partial T}{\partial y}\Big|_{(x,y)=(2,-1)} .
\]

\begin{itemize}
\item{Do this by turning off the Level Curve Folder in Line 8 and first turning on the Partial x Folder in Line 14. Use the slider $T$ to adjust the level curve $f(x,y)=T$.} 

\item{Then turn off the Partial x Folder, activate the Partial y Folder, and repeat.}

\item{Follow the directions in Lines 21 and 23 to check your approximations.}

\item{Include units for the partials and explain their meanings.}

\end{itemize}

\pdfOnly{
Access Desmos interactives through the online version of this text at
 
\href{https://www.desmos.com/calculator/mrbwyydrcm}.
}
 
\begin{onlineOnly}
    \begin{center}
\desmos{mrbwyydrcm}{900}{600}
\end{center}
\end{onlineOnly}

Desmos activity available at

\href{https://www.desmos.com/calculator/mrbwyydrcm}{163: Level Curves Hemisphere}

\pskip \pskip


(b) Write an expression for the average rate of change in temperature with respect to the change $w$ in the $x$-coordinate relative to the point $(2,-1)$. Enter this expression on Line 22 of the above worksheet and follow the directions there to approximate 
\[
   \frac{\partial T}{\partial x}\Big|_{(x,y)=(2,-1)} .
\]
Explain the idea.

(c) Write an expression for the average rate of change in temperature with respect to the change $w$ in the $y$-coordinate relative to the point $(2,-1)$. Enter this expression on Line 24 of the above worksheet and follow the directions there to approximate 
\[
   \frac{\partial T}{\partial y}\Big|_{(x,y)=(2,-1)} .
\]
Explain the idea.

(d) Suppose now that 
\[
    T = f(x,y) = \sqrt{9-x^2-y^2} \, , \, x^2 + y^2 \leq 3 .
\]
Find the exact values of the two partials above and compare them with your estimates.

\end{example}

\section{Coordinate Curves and Tangent Planes}


\begin{example}  \label{Esdtr435r}
This is a continuation of Example 1. 

\pskip \pskip

(c) Describe the surface
\[
  T = f(x,y) = \sqrt{9-x^2-y^2} \, , \, x^2 + y^2 \leq 3 .
\]

\pskip

(d) Write expressions  for the partials
\[
    \frac{\partial T}{\partial x}\Big|_{(x,y)=(2,-1)} \text{ and } \frac{\partial T}{\partial y}\Big|_{(x,y)=(2,-1)} .
\]
as limits of average rates of change. Then use the demonstration below to give a geometric interpretation of the average rates of change $\Delta T/\Delta x$ and $\Delta T/\Delta y$ and their respective limits as $\Delta x\to 0$ and $\Delta y\to 0$.

\pdfOnly{
Access Geogebra interactives through the online version of this text at
 
\href{https://www.geogebra.org/classic/egnkjkqw}.
}
 
\begin{onlineOnly}
    \begin{center}
\geogebra{egnkjkqw}{900}{600}
\end{center}
\end{onlineOnly}

Geogebra activity available at

\href{https://www.geogebra.org/classic/egnkjkqw}{163: Level Curves and Sphere}




(e) Open the desmos demonstration at

\href{https://www.desmos.com/3d/967c10cbdc}{163: Tangent Planes to Sphere}

\pskip

(f) Parameterize the $x$-coordinate curve cut from the surface by the plane $y=-1$.

(g) Use part (f) to parameterize the tangent line to the $x$-coordinate curve at the point $P(2,-1,2)$ on the surface.

(h) Parameterize the $y$-coordinate curve cut from the surface by the plane $x=2$.

(i) Use part (h) to parameterize the tangent line to the $y$-coordinate curve at the point $P(2,-1,2)$ on the surface.

(j) Use parts (g) and (i) to find an equation of the tangent plane to the surface at the point $P(2,-1,2)$.

(k) Answer parts (b) and (j) geometrically without using calculus.

(l) Use the result of part (j) to find the linear approximation to the function $T=f(x,y)$ at the point $(2,-1)$.

(M) Use your linear approximation from part (l) to estimate the temperatures at the points with coordinates $(2.1, -1.1)$ and $(1.99, -0.99)$. Without computing the actual temperatures, determine whether your estimates are greater or less than the actual values. Explain your reasoning.

\end{example}



\begin{example} \label{E4rt4t42}
For this entire problem let
\[
   z = f(x,y) = \arctan(y/x) 
\]
and let $P$ be the point on the graph of the function $f$ with coordinates $(\sqrt{3},1,f(\sqrt{3},1))$.

(a) Parameterize the $x$-coordinate curve cut from the surface $z=f(x,y)$ by the plane $y=1$. Use this parameterization to parameterize the tangent line to the curve at $P$.

(b) Parameterize the $y$-coordinate curve cut from the surface $z=f(x,y)$ by the plane $x=\sqrt{3}$. Use this parameterization to parameterize the tangent line to the curve at $P$.

(c) Use parts (a) and (b) to find an equation of the tangent plane to the surface $z=f(x,y)$ at $P$.

(d) Parameterize the level curve of the function $z=f(x,y)$ through the point $(\sqrt{3},1,0)$. Use this parameterization to parameterize the curve on the surface $z=f(x,y)$ lying directly above the level curve (ie. lift the level curve to the surface).

(e) Use the desmos link below to check your work by doing the following:

\begin{itemize}
\item{Input equations of the coordinate planes $x=\sqrt{3}$ and $y=1$.}

\item{Input parameterizations of the two coordinate curves. Use $t$ for the parameter and choose the domain $t\in[-10,10]$.}

\item{Input parameterizations of the two tangent lines. Use $t$ for the parameter and choose the domain $t\in[-10,10]$.}

\item{Input an equation of the tangent plane.}

\item{Input a parameterization for the lift of the level curve. Use $t$ for the parameter and choose the domain $t\in[-10,10]$.}

\item{Include both one or several screenshots of your work and a link to your saved file.}

\end{itemize}


\pskip 

\href{https://www.desmos.com/3d/ade9317b51}{163:Coordinate Curves Student Version}




\end{example}







\section{Tangent Planes}
\begin{exploration}  \label{Edf754665}

\pdfOnly{
Access Desmos interactives through the online version of this text at
 
\href{https://www.desmos.com/calculator/y0h5kuvmbt}.
}
 
\begin{onlineOnly}
    \begin{center}
\desmos{y0h5kuvmbt}{900}{600}
\end{center}
\end{onlineOnly}


\href{https://www.desmos.com/calculator/y0h5kuvmbt}{The Partial Derivative}

\end{exploration}




\begin{exploration}  \label{Ede5fhj4665}
\href{https://www.geogebra.org/m/Hud6Hnpk}{Partial Derivatives and Tangent Planes}
\end{exploration}


\section{Linear Approximation}
The demonstration below shows some level curves of a function
\[
   z = f(x,y) .
\]

(a) Zoom in toward the point $P(a,b)$ and adjust the slider $n$ as appropriate. What do you notice if you zoom in close enough to the point $P$?

(b) Turn on the Linear Approximation folder in Line 11. What do you notice?

(c) What does this demonstration suggest about how to find the linear approximation to the differentiable function $f$ at the point $(a,b)$?

\begin{exploration}   \label{Edsfg907}
\pdfOnly{
Access Desmos interactives through the online version of this text at
 
\href{https://www.desmos.com/calculator/8hphbdetoj}.
}
 
\begin{onlineOnly}
    \begin{center}
\desmos{8hphbdetoj}{900}{600}
\end{center}
\end{onlineOnly}
\end{exploration}

\href{https://www.desmos.com/calculator/8hphbdetoj}{163: Level Curves Zoom In}


\section{Error Analysis}
When you do an experiment, you typically make measurements and use these measurements to compute something else. But measurements have errors and these errors lead to errors in the computations. We can use linear approximation to get an idea of the errors in the computations.

\begin{example}  \label{Edstrt4th}
You measure the length and width of a rectangle and use these measurements to compute the area. Use a linear approximation to estimate the error and the relative error in computing the area. Assume the errors in the measurements are small.

\begin{explanation}
Let $L$ and $W$ be the measured lenght and width, in meters, and let
\[
  A =f(L,W) = LW ,
\]
be the computed area, in square meters. 

Also let $\Delta L$, $\Delta W$ be the errors (measured in meters) in the measurements of the dimensions and let $\Delta A$ be the error (in square meters) in the computed area.

Then
\begin{align*}
   \Delta A  &\sim  \left( \frac{\partial A}{\partial L} \right) \Delta L + \left( \frac{\partial A}{\partial W} \right)  \Delta W \\ \\
                &=  W\Delta L +  L \Delta W .
\end{align*}

So
\begin{align*}
    | \Delta A |  &\sim |   W\Delta L +  L \Delta W | \\ \\
                      & \leq W|\Delta L| + L |\Delta W| .
\end{align*}

Now suppose the relative errors in measuring the length and width are at most $p\%$ and $q\%$, respectively, and that $p,q \sim 0$.  We wish to approximate the maximum relative error
\[
      \frac{|\Delta A|}{A}
\]
in the computed area.

We are given that 
\[
      \frac{|\Delta L|}{L} \leq p\% \text{ and } \frac{|\Delta W|}{W} \leq q\% .
\]

Then since
\[
  | \Delta A | \leq W|\Delta L| + L |\Delta W| ,
\]
dividing both sides by $A = LW$ gives
\begin{align*}
      \frac{|\Delta A|}{A} & \leq \frac{|\Delta L|}{L} + \frac{|\Delta W|}{W} \\ \\
                                   & \leq p\% + q\% .
\end{align*}

So the relative error in the product is at most the sum of the relative errors in the factors (if the relative errors in the factors are small).

\end{explanation}

\end{example}


\begin{question}   \label{Qefg547ji8}
(a) What are the units in the relative errors above?

(b) Draw a picture that illustrates the statement about the relative error in a product.
\end{question}
 
\begin{question}  \label{Qert5ty7}
You measure the mass and volume of a sample of gold with respective relative errors of at most $p\%$ and $q\%$. Use the appropriate linear approximation to estimate the maximum relative error in the computed density. Assume $p,q\sim 0$.

\end{question}



\begin{question}  \label{Qerfgghym}
You measure your distance from the base of a tree to be $s$ feet and the angle of elevation to the top of the tree to be $\theta$ radians.

(a) Approximate the error in the computed height of the tree above eye level (in feet) in terms of the errors in measuring the distance and angle.

(b) Approximate the relative error in the computed height of the tree above eye leve in terms of the relative error in the distance and the (absolute) error in the angle. 

(c) Simplify the relative error in part (b) if $\theta=\pi/12$. Do not use a calculator.

(d) Continuing with part (c), suppose you measure the distance to the tree with a relative error of at most $3\%$. Approximate the maximum error in measuring the angle of elevation to make the relative error in the computed height at most $5\%$.
\end{question}


\begin{question}  \label{QE4gb4ttb}
You approximate a small island to be $6$ miles due east and $8$ miles due west of your current position. You use these approximations to compute both your distance and bearing to the island. 

(a) Use the appropriate linear approximations to estimate the errors in the computed distance and bearing.

(b) Interpret the expressions for these errors geometrically.

(c) Where in this class have we done similar problems before?

\end{question}



\end{document}
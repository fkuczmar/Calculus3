\documentclass{ximera}
\title{Level Curves and Surfaces}

\newcommand{\pskip}{\vskip 0.1 in}

\begin{document}
\begin{abstract}
Introduction to level curves and surfaces.
\end{abstract}
\maketitle



\begin{example}  \label{Edfhpgdf45}
Move the slider $k$ to see the level curves 
\[
    f(x,y)=k
\]
of the function $z=f(x,y)$. Use the level curves to imagine or draw the surface $z=f(x,y)$. Then check the \emph{surface} box in the upper left corner to see how you did. 


\pdfOnly{
Access Geogebra interactives through the online version of this text at
 
\href{https://www.geogebra.org/classic/nk4ap6kj}.
}
 
\begin{onlineOnly}
    \begin{center}
\geogebra{nk4ap6kj}{900}{600}
\end{center}
\end{onlineOnly}

Access Geogebra interactives through the online version of this text at
 
\href{https://www.geogebra.org/classic/nk4ap6kj}{163: Level Curves and Surface 1}.
\end{example}



\begin{example}  \label{Edfhpgdf45}
Move the slider $k$ to see the level curves 
\[
    f(x,y)=k
\]
of the function $z=f(x,y)$. Use the level curves to imagine or draw the surface $z=f(x,y)$. Then check the \emph{surface} box in the upper left corner to see how you did. 


\pdfOnly{
Access Geogebra interactives through the online version of this text at
 
\href{https://www.geogebra.org/classic/vk7zt92m}.
}
 
\begin{onlineOnly}
    \begin{center}
\geogebra{vk7zt92m}{900}{600}
\end{center}
\end{onlineOnly}

Access Geogebra interactives through the online version of this text at
 
\href{https://www.geogebra.org/classic/vk7zt92m}{163: Level Curves and Surface 2}.
\end{example}


\begin{example}  \label{Ed5465f45}
Move the slider $k$ to see the level curves 
\[
    f(x,y)=k
\]
of the function $z=f(x,y)$. Use the level curves to imagine or draw the surface $z=f(x,y)$. Then check the \emph{surface} box in the upper left corner to see how you did. 


\pdfOnly{
Access Geogebra interactives through the online version of this text at
 
\href{https://www.geogebra.org/classic/xgj5cv9x}.
}
 
\begin{onlineOnly}
    \begin{center}
\geogebra{xgj5cv9x}{900}{600}
\end{center}
\end{onlineOnly}

Access Geogebra interactives through the online version of this text at
 
\href{https://www.geogebra.org/classic/xgj5cv9x}{163: Level Curves and Surface 5}.
\end{example}



\begin{question}  \label{E45hh6f45}
(a) Find the domain of the function
\[
   z = f(x,y) 
\]
by moving the slider $k_2$ below to see the family of level curves $f(x,y)=k_2$. Write the domain as a set in the form
\[
   \{  (x,y,z) \in \mathbb{R}^3 \, | \, \ldots    \} \text{ or } \{  (x,y) \in \mathbb{R}^2 \, | \, \ldots    \} \text{ or } \{  x \in \mathbb{R} \, | \, \ldots    \}
\]


(b) Write the range of the function $z=f(x,y)$ as a set.


(c) Use the slider $k_2$ to estimate the partial derivatives
\[
    \frac{\partial z}{\partial x}\Big|_{(2,0.2)} \,\,\, , \,\,\,   \frac{\partial z}{\partial y}\Big|_{(2,0.2)}
\]
of the function $z=f(x,y)$. Explain your reasoning.

(d) Use you approximations from part (c) to approximate the partial derivatives
\[
    \frac{\partial z}{\partial x}\Big|_{(0.01,0.001)} \,\,\, , \,\,\,   \frac{\partial z}{\partial y}\Big|_{(0.01,0.001)} .
\]
Explain your reasoning.

(e) What is the value of the partial derivative
\[
   \frac{\partial z}{\partial y}\Big|_{(0,-2)} ?
\]
Explain your reasoning.

\pdfOnly{
Access Geogebra interactives through the online version of this text at
 
\href{https://www.geogebra.org/classic/ehscekb6}.
}
 
\begin{onlineOnly}
    \begin{center}
\geogebra{ehscekb6}{900}{600}
\end{center}
\end{onlineOnly}

Access Geogebra interactives through the online version of this text at
 
\href{https://www.geogebra.org/classic/ehscekb6}{163: Level Curves and Surface 7}.
\end{question}



\end{document}
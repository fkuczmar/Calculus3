\documentclass{ximera}
\title{Hyperboloids of One Sheet}

\newcommand{\pskip}{\vskip 0.1 in}

\begin{document}
\begin{abstract}
Writing equations of hypberboloids.
\end{abstract}
\maketitle


\section{Twisting a Cylinder}

\begin{exploration} \label{ExKerdfeMdfR}

Drag the sider $\beta$ in Line 1 below to twist the cylinder.

\begin{onlineOnly}
    \begin{center}
\desmosThreeD{5luiw4hvve}{900}{600}
\end{center}
\end{onlineOnly}

\href{https://www.desmos.com/3d/5luiw4hvve}{163: Hyperboloid One Sheet}
\end{exploration}



\section{Hypberboloids of One Sheet}

Start with two skew lines in $\mathbb{R}^3$. Rotate one about the other and you get a surface called a \emph{hyperboloid of one sheet}. The tower of a nuclear power plant is like this. It is reinforced with straight rods.


\begin{question} \label{QOdferefr}
We'll start with a line ${\cal L}$ in the plane $x = a$ through the point $A(a,0,0)$ and making the acute angle $\alpha$ with the $z$-axis as shown below.

\begin{onlineOnly}
    \begin{center}
\desmosThreeD{pmhatuke3f}{900}{600}
\end{center}
\end{onlineOnly}

\href{https://www.desmos.com/3d/pmhatuke3f}{163: Hyperboloid 2}

\begin{enumerate}
\item Parameterize the line ${\cal L}$ in terms of the signed distance from the point $A$.


\end{enumerate}


\end{question}



\end{document}
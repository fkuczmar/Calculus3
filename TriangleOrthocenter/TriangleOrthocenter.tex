\documentclass{ximera}
\title{Altitudes of a Triangle}

\newcommand{\pskip}{\vskip 0.1 in}

\begin{document}
\begin{abstract}
Constructing the altitudes of a triangle.
\end{abstract}
\maketitle





\begin{question}   \label{Exstd67g:Line}
This problem is about parameterizing the alitudes of a triangle $\Delta ABC$ in the plane. The altitudes are the segments $\overline{AA_1}$, $\overline{BB_1}$, $\overline{CC_1}$ shown below. They are respectively perpendicular to the sides $\overline{BC}$, $\overline{CA}$, and $\overline{AB}$.

As usual, point $O$ is the origin.
 
\begin{onlineOnly}
    \begin{center}
\desmos{cmazvewzml}{900}{600}
\end{center}
\end{onlineOnly}

\href{https://www.desmos.com/calculator/cmazvewzml}{Altitudes of a Triangle}

\begin{enumerate}

\item Express the position of the point $A^1$ (the foot of the perpendicular from $A$ to $\overline{BC}$ relative to the origin in terms of the position vectors $\overrightarrow{OA}$, $\overrightarrow{OB}$, $\overrightarrow{OC}$.

\item Enter your expression for $A_1$ in Line 2 of the worksheet above. Remember in desmos to omit all the $O$'s. To write the vector $\overrightarrow{OA}$ in desmos, you would write $A$.

\item Find similar expressions for the positions of $B_1$ and $C_1$ relative to $O$ and enter these in the worksheet.

\item Parameterize the three altitudes. Be sure to include a domain for each parameterization. Enter these in the worksheet.

\item Earlier you showed the altitudes interect in a common point $H$. Express the vector $\overrightarrow{OH}$ in terms of the vectors $\overrightarrow{OA}$, $\overrightarrow{OB}$, $\overrightarrow{OC}$. Then enter your expression in the worksheet above.
\end{enumerate}

\end{question}

\end{document}
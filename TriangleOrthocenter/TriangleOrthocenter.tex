\documentclass{ximera}
\title{Altitudes of a Triangle}

\newcommand{\pskip}{\vskip 0.1 in}

\begin{document}
\begin{abstract}
Constructing the altitudes of a triangle.
\end{abstract}
\maketitle





\begin{question}   \label{Exstd67g:Line}
Let ${\bf a} = \overrightarrow{OA}$ and  ${\bf b} = \overrightarrow{OB}$.

The parameterization 
\[
    {\bf r} = {\bf a} + t({\bf b} - {\bf a}) = t{\bf b} + (1-t){\bf a} , t\in \mathbb{R}
\]
is shown below. Slide $t$ and interpret the geometric meaning of the parameter $t$ in terms of the number line through $A$ and $B$ with its zero at $A$. You may drag points $A$ and $B$ in the demonstration below.
\begin{freeResponse}
\end{freeResponse}

\pdfOnly{
Access Geogebra interactives through the online version of this text at
 
\href{https://www.geogebra.org/classic/xbrvnrtp}.
}
 
\begin{onlineOnly}
    \begin{center}
\desmos{cmazvewzml}{900}{600}
\end{center}
\end{onlineOnly}

\href{https://www.desmos.com/calculator/cmazvewzml}{Altitudes of a Triangle}

\end{question}

\end{document}
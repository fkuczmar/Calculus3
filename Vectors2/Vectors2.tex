\documentclass{ximera}
\title{Vectors, Part 2}

\newcommand{\pskip}{\vskip 0.1 in}
%\usepackage{esvect}

\begin{document}
\begin{abstract}
Circles, spheres, and vectors.
\end{abstract}
\maketitle


\section{Vectors in Desmos}
In the worksheet below, point $O$ is the origin and point $A$ has coordinates $(1,2,3)$. To draw the vector $\overrightarrow{OA}$, we use the \emph{vector} command and write $\text{vector}(O,A)$ as in Line 4.

\begin{onlineOnly}
    \begin{center}
\desmosThreeD{cizkv51anx}{900}{600}
\end{center}
\end{onlineOnly}

\href{https://www.desmos.com/3d/cizkv51anx}{163: Vectors and Desmos}

But desmos does not distinguish between points and vectors arithmetically. For example, in this class we would write the vector that expresses the position of point $B$ relative to point $A$ as
\[
  \overrightarrow{AB} =  \overrightarrow{OB} - \overrightarrow{OA} .
\]

But in desmos, we do vector arithmetic with the points directly. So we would write the above equation as % and define a third point, call it $C$, as
\[
    C = B - A,
\]
where $\overrightarrow{OC}$ is equal to the vector $\overrightarrow{AB}$. To see the \emph{point} $C$, activate Line 7 above.

\emph{Note:} We should \emph{not} write $C = B-A$ in place of the vector equation above other than in desmos. %Instead we should write the vector equation above.

To draw the vector $\overrightarrow{OC} = \overrightarrow{AB}$ from the origin to $C$ activate Line 8.

But to draw the vector $\overrightarrow{AB}$ in what we might think of as its proper position, with its tail at $A$, activate Line 9.

\begin{question} \label{Q9KDEk3er}
\begin{enumerate}
\item Turn off the folder \emph{subtraction} in Line 6 of the worksheet above.

\item Activate the folder \emph{addition} in Line 10.

\item Complete the correct expression in Line 13 to draw the vector $\overrightarrow{OB}$ with its tail at $A$.
\end{enumerate}
\end{question}


\section{A Parallelogram Problem}

\begin{question}  \label{Q:df9GG435r34}

\begin{enumerate}

\item  Let $A$, $B$, and $C$ be points in space. Let $D$ be the point in $\mathbb{R}^3$ that makes quadrilateral $ABCD$ a parallelogram. Express the vector $\overrightarrow{OD}$ in terms of the vectors $\overrightarrow{OA}$, $\overrightarrow{OB}$, and $\overrightarrow{OC}$. Do \emph{not}use the specific coordinates of the points in the worksheet below. Work in general instead. \emph{Note:} Saying that $ABCD$ is a parallelogram means that the vertices $A$, $B$, $C$, $D$ occur in that order and that the opposite sides are parallel.
\[
    \overrightarrow{OD} = \overrightarrow{OA} + \overrightarrow{\answer{OC}} - \overrightarrow{\answer{OB}} .
\]

\item Enter the correct expression for $D$ in Line 2 of the worksheet below. Then activate the folder \emph{Parallelogram} to see if you are correct.

\item Now suppose the points $A$, $B$, $C$, $D$ have respective coordinates $A(1,2,-2)$, $B(3,-4,-5)$, and $C(-2,1,-4)$. Use the result of part (a) to find the coordinates of $D$.

\end{enumerate}

\begin{onlineOnly}
    \begin{center}
\desmosThreeD{9b606e8a14}{900}{600}
\end{center}
\end{onlineOnly}

\href{https://www.desmos.com/3d/9b606e8a14}{163: Parallelogram}
\end{question}


\section{Unit Vectors}

When we multiply a vector by a dimensionless scalar (ie. a real number), we multiply its length by the \emph{absolute value of that number}. For example, the vector 
\[
   {\bf v} = \langle 1 , 2 \rangle \text{cm}
\]
has length 
\[
    | {\bf v} | = \sqrt{(1\text{ cm})^2 + (2\text{ cm}^2} = \sqrt{5} \text{ cm}.
\]
And the vector
\[
   -3 {\bf v} = \langle -3 , -6 \rangle \text{cm}
\]
has length
\[
   |  -3 {\bf v} | = |-3||{\bf v}| = 3\sqrt{5}\text{cm} .
\]


By definition, A \emph{unit} vector has unit length (ie. length $1$).

To find, for example, a unit vector $\hat{{\bf v}}$ pointing in the same direction as the vector ${\bf v} = \langle 1, 2\rangle$cm, we multiply ${\bf v}$ by the reciprocal of its length. The resulting unit vector is 
\begin{align*}
  \hat{{\bf v}} &= \frac{{\bf v}}{|{\bf v}|} \\
                      &= \left( \frac{1}{\sqrt{5}\text{cm}} \right)\langle 1, 2  \rangle\text{cm} \\
                      &= \left\langle \frac{1}{\sqrt{5}} , \frac{2}{\sqrt{5}} \right\rangle . 
\end{align*}

Note this key fact. \emph{Unit vectors are dimensionless}.

\begin{example} \label{ExLFDfefr}
Find the components of the vector ${\bf w}$ with length $10$ cm that points in the same direction as the vector ${\bf v} = \langle 1, 2\rangle$ cm.

\begin{explanation}
All we need to do is scale the dimensionless unit vector
\[
   \hat{{\bf v}} = \left\langle \frac{1}{\sqrt{5}} , \frac{2}{\sqrt{5}} \right\rangle
\]
parallel to ${\bf v}$ by the factor $10$ cm. So 
\begin{align*}
  {\bf w} &= (10 \text{ cm}) \hat{{\bf v}} \\
             & = 10 \left\langle \frac{1}{\sqrt{5}} , \frac{2}{\sqrt{5}} \right\rangle \text{cm} \\
             &= \left\langle \answer{2\sqrt{5}} , \answer{4\sqrt{5}} \right\rangle \text{cm} .
\end{align*}

\end{explanation}
\end{example}


\section{Circles}
A circle is the set of points in a plane a fixed distance from a given point (the circle's center). 

We typically do not think about vectors when writing an equation of a circle, but we can. Here's how.

Suppose, for example, we want to find an equation of the circle of radius $5$ centered at the point $A$ with coordinates $(4,1)$. Then with $O$ as the origin, the point $P$ with coordinates $(x,y)$ lies on this circle if and only if the length of the vector 
\begin{align*}
   \overrightarrow{AP} &= \overrightarrow{OP} - \overrightarrow{OA}  \\
                                 &= \langle x,y \rangle - \langle 4, 1 \rangle \\
                                 & = \langle x -4 , y-1 \rangle 
\end{align*}
is $5$. 

So we can write an equation of the circle as
\[
        \left|   \langle x,y \rangle - \langle 4, 1 \rangle   \right| = 5
\]
or equivalently as
\[
   \left| \langle x -4 , y-1 \rangle  \right|= 5 .
\]

It's important to realize that even though we're thinking of vectors when we write this so-called \emph{vector equation}, the circle is \emph{not} a set of vectors. It is still a \emph{set of points}, namely the set
\[
   \{ (x,y) \in \mathbb{R}^2\, | ,   \left|   \langle x,y \rangle - \langle 4, 1 \rangle   \right| = 5 \}.
\]

Of course, since
\[
  \left| \langle x -4 , y-1 \rangle  \right| = \sqrt{(x-4)^2 + (y-1)^2} ,
\]
either of the above equations are just disguised forms of the more familiar equation
\[
        \sqrt{(x-4)^2 + (y-1)^2} =5
\]
of the same circle.

This way of thinking about vectors when writing an equation of a circle is especially handy in desmos. With $A=(4,1)$ as the center of the circle, Line 2 of the worksheet below shows how to write the vector equation of the circle of radius $5$ centered at $A$.

\begin{onlineOnly}
    \begin{center}
\desmos{hraab6na4k}{900}{600}
\end{center}
\end{onlineOnly}

\href{https://www.desmos.com/calculator/hraab6na4k}{163: Circle Equation}


\section{Spheres}
A sphere is the set of points in space a fixed distance from a given point (the circle's center).

The logic behind writing an equation of a sphere in $\mathbb{R}^3$ is the same as writing an equation of a circle in $\mathbb{R}^2$.

Take, for example, the sphere of radius $4$ centered at the point $A$ with coordinates $(-2,-1,1)$. Then a point $P$ with coordinates $(x,y,z)$ lies on this sphere if and only if
\[
       \left| \langle x,y,z \rangle - \langle 2, -1, 1 \rangle \right| = 4.  
\]

\begin{enumerate}
\item Write the above vector equation in its more familiar (but longer) form as
\[
  \sqrt{\answer{(x-2)^2 + (y+1)^ + (z-1)^2}} = \answer{4}.
\]

\item Does the point $Q$ with coordinates $(-2,-1,0)$ lie on this sphere? If not, is it inside or outside the sphere? Justify your response. Then activate Line 3 in the worksheet below to check..
\begin{freeResponse}
\end{freeResponse}

\end{enumerate}

\begin{onlineOnly}
    \begin{center}
\desmosThreeD{ck3i6fhpyx}{900}{600}
\end{center}
\end{onlineOnly}

\href{https://www.desmos.com/3d/ck3i6fhpyx}{163: Equation of Sphere}

https://www.desmos.com/3d/ck3i6fhpyx


\end{document}
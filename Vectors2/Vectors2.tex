\documentclass{ximera}
\title{Vectors, Part 2}

\newcommand{\pskip}{\vskip 0.1 in}
%\usepackage{esvect}

\begin{document}
\begin{abstract}
Circles, spheres, and vectors.
\end{abstract}
\maketitle


\section{Vectors in Desmos}
In the worksheet below, point $O$ is the origin and point $A$ has coordinates $(1,2,3)$. To draw the vector $\overrightarrow{OA}$, we use the \emph{vector} command and write $\text{vector}(O,A)$ as in Line 4.

\begin{onlineOnly}
    \begin{center}
\desmosThreeD{cizkv51anx}{900}{600}
\end{center}
\end{onlineOnly}

\href{https://www.desmos.com/3d/cizkv51anx}{163: Vectors and Desmos}

But arithmetically, desmos does not distinguish between points and vectors. For example, we would write the vector that expresses the position of point $B$ relative to point $A$ is
\[
  \overrightarrow{AB} =  \overrightarrow{OB} - \overrightarrow{OA} .
\]

But in desmos, we do arithmetic with the points directly and define a third point, call it $C$, as
\[
    C = B - A.
\]
To see point $C$, activate Line 6 above.

\emph{Note:} We should not write $C = B-A$ in this class, except in desmos. Instead we should write the vector equation above.

To draw the vector $\overrightarrow{OC} = \overrightarrow{AB}$ from the origin to $C$ activate Line 7.

But to draw the vector $\overrightarrow{AB}$ in what we might think of as its proper position, with its tail at $A$, activate Line 8.



\section{A Parallelogram Problem}

\begin{question}  \label{Q:df9GG435r34}

\begin{enumerate}

\item  Let $A(1,2,2)$, $B(3,-4,5)$, and $C(-2,1,4)$ be points in space. Let $D$ be the point in $\mathbb{R}^3$ that makes quadrilateral $ABCD$ a parallelogram. Express the vector $\overrightarrow{OD}$ in terms of the vectors $\overrightarrow{OA}$, $\overrightarrow{OB}$, and $\overrightarrow{OC}$. Draw the relevant vectors in the desmos worksheet below.

\item  Use the result of part (b) to determine the coordinates of $D$.

\end{enumerate}


\begin{onlineOnly}
    \begin{center}
\desmosThreeD{hxegp3eszd}{900}{600}
\end{center}
\end{onlineOnly}

\href{https://www.desmos.com/3d/hxegp3eszd}{163: Parallelogram}
\end{question}



\end{document}
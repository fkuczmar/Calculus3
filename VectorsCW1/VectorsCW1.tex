\documentclass{ximera}
\title{Vectors, CW1}

\newcommand{\pskip}{\vskip 0.1 in}
%\usepackage{esvect}

\begin{document}
\begin{abstract}
Exercises with vectors.
\end{abstract}
\maketitle


\emph{Why Vectors?}

One of the main advantages of using vectors is to give a coordinate-free description of the laws of physics. Describing these laws is not part of our class, but we will use vectors to give coordinate-free solutions to geometric problems. 

For the problems below, try to give general solutions and not ones that depend on any specific coordinates that might be given.

Use the ideas in the previous two chapters

\href{https://ximera.osu.edu/calc3/Calculus3/Vectors1/Vectors1}{Introduction to Vectors}

\href{https://ximera.osu.edu/calc3/Calculus3/Vectors2/Vectors2}{Vectors, Part 2}

for your solutions.


\section{Closest Point on a Circle}

\begin{exercise} \label{EDrR3r3rdfdll}
For this problem, we are given points $A$, $B$, and $Q$ in the plane. Point $O$ is the origin.

\begin{onlineOnly}
    \begin{center}
\desmos{rzjkslabok}{900}{600}
\end{center}
\end{onlineOnly}

\href{https://www.desmos.com/calculator/rzjkslabok}{163: Circle Nearest Point}

\begin{enumerate}
\item Find a vector equation for the circle through $B$ centered at $A$. 

\item Let $C$ and $F$ be respectively the points on the circle nearest and farthest from $Q$.

\begin{enumerate}
\item Express the position $\overrightarrow{OC}$ of $C$ relative to the origin in terms of the vectors $\overrightarrow{OA}$, $\overrightarrow{OB}$, and $\overrightarrow{OQ}$.

\item Express the position $\overrightarrow{OF}$ of $F$ relative to the origin in terms of the vectors $\overrightarrow{OA}$, $\overrightarrow{OB}$, and $\overrightarrow{OQ}$.
\end{enumerate}

\item Check your answers in the worksheet above. Be sure to read the instructions there.

\end{enumerate}
\end{exercise}



\section{Exercises}

\begin{exercise} \label{EKDFEf3r933}
\begin{enumerate}
\item Use vector arithmetic to find the centers of all circles of radius $r$ that pass through the points $A$ and $B$ with respective coordinates $(-2,1)$ and $(4,5)$. Are there any restrictions on $r$?

\item Write vector-equations of the circles in part (a). 

\item Enter your equations from part (a) in Lines 3 and 4 of the worksheet below. Then drag the slider $r$ in Line ?? to check  your work..




\end{enumerate}
\end{exercise} 

\begin{exercise} \label{EMDF3r3r3rfD}
\begin{enumerate}
\item Find equations of the two lines through the origin that bisect the angles formed by the lines $y=x$ and $y=3x$.

\item Enter your equations in Lines 3 and 4 of the worksheet below to check your equations.

\end{enumerate}

\begin{onlineOnly}
    \begin{center}
\desmos{tynsxo6gwt}{900}{600}
\end{center}
\end{onlineOnly}

\href{https://www.desmos.com/calculator/tynsxo6gwt}{163: Angle Bisectors}


\begin{hint}
Start by finding two vectors parallel to the lines that have the same length.
\end{hint}
\end{exercise}

\end{document}
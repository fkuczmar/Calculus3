\documentclass{ximera}
\title{Vectors, CW1}

\newcommand{\pskip}{\vskip 0.1 in}
%\usepackage{esvect}

\begin{document}
\begin{abstract}
Exercises with vectors.
\end{abstract}
\maketitle


\emph{Why Vectors?}

One of the main advantages of using vectors is to give a coordinate-free description of the laws of physics. Describing these laws is not part of our class, but we will use vectors to give coordinate-free solutions to geometric problems. 

For the problems below, try to give general solutions and not ones that depend on any specific coordinates that might be given.

Use the ideas in the previous two chapters

\href{https://ximera.osu.edu/calc3/Calculus3/Vectors1/Vectors1}{Introduction to Vectors}

\href{https://ximera.osu.edu/calc3/Calculus3/Vectors2/Vectors2}{Vectors, Part 2}

for your solutions.


\section{Closest Point on a Circle/Sphere}

\begin{exercise} \label{EDrR3r3rdfdll}
For this problem, we are given points $A$, $B$, and $Q$ in the plane. Point $O$ is the origin.

\begin{onlineOnly}
    \begin{center}
\desmos{rzjkslabok}{900}{600}
\end{center}
\end{onlineOnly}

\href{https://www.desmos.com/calculator/rzjkslabok}{163: Circle Nearest Point}

\begin{enumerate}
\item Find a vector equation for the circle through $B$ centered at $A$. 

\item Let $C$ and $F$ be respectively the points on the circle nearest and farthest from $Q$.

\begin{enumerate}
\item Express the position $\overrightarrow{OC}$ of $C$ relative to the origin in terms of the vectors $\overrightarrow{OA}$, $\overrightarrow{OB}$, and $\overrightarrow{OQ}$.

\item Express the position $\overrightarrow{OF}$ of $F$ relative to the origin in terms of the vectors $\overrightarrow{OA}$, $\overrightarrow{OB}$, and $\overrightarrow{OQ}$.
\end{enumerate}

\item Check your answers in the worksheet above by typing the coordinates of $C$ and $F$ in Lines 44 and 45 for the particular points $A$, $B$, and $Q$ given there . Also, you can activate the folders in Lines 32 and 38 (click the open circles in the line numbers) and reveal the solutions by clicking on the right arrows at the start of each line). 

\end{enumerate}
\end{exercise}

\begin{exercise} \label{EMDEFNeefre8877}
This problem is a three-dimensional version of the previous one.

We are given points $A$, $B$, and $Q$ in $\mathbb{R}^3$. Point $O$ is the origin.

\begin{onlineOnly}
    \begin{center}
\desmosThreeD{fqh346gbl9}{900}{600}
\end{center}
\end{onlineOnly}

\href{https://www.desmos.com/3d/fqh346gbl9}{163: Sphere Nearest Point}


\begin{enumerate}
\item Express the postions (relative to the origin) of the points nearest and farthest from $Q$ that lie on the sphere through $B$ centered at $A$ in terms of the vectors $\overrightarrow{OA}$, $\overrightarrow{OB}$, and $\overrightarrow{OQ}$.
\end{enumerate}

\item Use the worksheet above to check your work. You can make the sphere translucent by clicking the wrench in the upper right. Then choose \emph{More Options} and then \emph{Translucent surfaces}.

\end{exercise}

\section{Spheres Tangent to Each Other}

\begin{exercise}  \label{EQdf4r4r5443}
\begin{enumerate}
\item  Drag slider $r_1$ below to see how many spheres centered at the point $B(4,1,5)$ are tangent to the sphere with equation
\[
        | \langle x,y,z\rangle - {\bf a} | = 3 ,
\]
where
\[
   {\bf a} = \overrightarrow{OA}
\]
is the vector from the origin to the point $A$ with coordinates $(1,-1,2)$.


\begin{onlineOnly}
    \begin{center}
\desmosThreeD{bvowkb4jl2}{900}{600}
\end{center}
\end{onlineOnly}

\href{https://www.desmos.com/3d/bvowkb4jl2}{163:Vectors Tangent Spheres 2}

\item Hide the spheres by turning off Lines 2 and 4. Then use vector arithmetic to find the coordinates of the points of tangency. Do this by expressing the vector $\overrightarrow{OP}$ from the origin to a point of tangency in terms of the vectors $\overrightarrow{OA}$ and $\overrightarrow{OB}$. Use the desmos worksheet to check your work.

Work with general points $A$ and $B$ (not the specific ones here) and a given sphere with radius $r_2$ (not $3$ as above). 

Activate and open the folder \emph{Solution} in Line 23 to see the solutions.

\end{enumerate}
\end{exercise}




\section{Angle Bisectors}

%\begin{exercise} \label{EKDFEf3r933}
%\begin{enumerate}
%\item Use vector arithmetic to find the centers of all circles of radius $r$ that pass through the points $A$ and $B$ with respective coordinates $(-2,1)$ and $(4,5)$. Are there any restrictions on $r$?

%\item Write vector-equations of the circles in part (a). 

%\item Enter your equations from part (a) in Lines 3 and 4 of the worksheet below. Then drag the slider $r$ in Line ?? to check  your work..

%\end{enumerate}
%\end{exercise} 

\begin{exercise} \label{EMDF3r3r3rfD}
\begin{enumerate}

\item There are two lines that bisect the angles formed by the lines $y=x$ and $y=3x$. Do you think the one that lies in the first and third quadrants has slope

\begin{enumerate}
\item equal to two,

\item greater than two,

\item or less than two?
\end{enumerate}
Explain your thinking. If you're not sure, that's ok. Say so.


\item Find equations of the two lines through the origin that bisect the angles formed by the lines $y=x$ and $y=3x$.

\item Enter your equations in Lines 8 and 9 of the worksheet below to check your equations. Or activate and open the folder \emph{Solutions} in Line 3.

\end{enumerate}

\begin{onlineOnly}
    \begin{center}
\desmos{tynsxo6gwt}{900}{600}
\end{center}
\end{onlineOnly}

\href{https://www.desmos.com/calculator/tynsxo6gwt}{163: Angle Bisectors}


\begin{hint}
Start by finding two vectors parallel to the lines that have the same length. Then use vector arithmetic to find the directions of the two angle bisectors.
\end{hint}
\end{exercise}


\section{Circles Tangent to Curves}

The point $P$ with coordinates $(0,3)$ lies on the curve defined by the property that the sum of the distances from any point on the curve to the points $A(4,0)$, $B(0,2)$ and the origin $O$ is a constant.

\begin{onlineOnly}
    \begin{center}
\desmos{f5rv0a9ie8}{900}{600}
\end{center}
\end{onlineOnly}

\href{https://www.desmos.com/calculator/f5rv0a9ie8}{163: Oval and Tangent Circles}

\begin{enumerate}
\item Write a vector equation of the curve.

\item Write the equation without vector notation.

\item Use implicit differentiation to find the slope of the tangent to the curve at $P$. %Then find an equation of the normal line to the curve at $P$.

\item Use vector arithmetic to find the vectors that give the positions (relative to the origin) of the centers of the two circles of radius $r$ tangent to the curve at $P$.

\item Find vector equations of these circles.

\item Use the worksheet above to check your work.

\item Find a vector equation of the circle that looks like it best approximates the curve near $P$.
\end{enumerate}



\end{document}
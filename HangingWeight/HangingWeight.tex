\documentclass{ximera}
\title{The Hanging Weight}

\newcommand{\pskip}{\vskip 0.1 in}

\begin{document}
\begin{abstract}
The scalar product.
\end{abstract}
\maketitle



\begin{question}  \label{ExLMsdf3Er3}

A block $B$ with weight $w$ pounds hangs suspended from two ropes attached to the ceiling at points $A_1$, $A_2$ and making respective angles $\theta_1$ and $\theta_2$ with the vertical as shown below. 

\begin{onlineOnly}
    \begin{center}
\desmos{18pdjhgnrm}{900}{600}
\end{center}
\end{onlineOnly}

\href{https://www.desmos.com/calculator/18pdjhgnrm}{163: Hanging Weight and Tension 2}


\begin{enumerate}
\item Our first task is to visually approximate the magnitudes of the tensions $\overrightarrow{T_1}$ and $\overrightarrow{T_2}$ in the ropes.

\begin{enumerate}
\item Drag sliders $u_1$ and $u_2$ in Lines 2 and 3 to adjust these magnitudes until they look right to you. Explain your thinking.

\item Then activate the folder \emph{Vector Triangle} on Line 10 to see the triangle of vectors. Then readjust the sliders $u_1$ and $u_2$ to get better approximations to the tensions. Explain your thinking.

\item Activate the folder \emph{Solution} in Line 19 to see how you did. 
\end{enumerate}

\item Our next task is to find an expression for the magnitude $T_1 = |\overrightarrow{T_1}|$ of the tension in rope $BA_1$.

\begin{enumerate}
\item First turn off the folders \emph{Tension Vectors} and \emph{Vector Triangle} in Lines 4 and 10 above.

\item Use triangle trigonometry to find an expression for the function
\[
   T_1 = f(\theta_1, \theta_2) \, , \, 0 <  \theta_1 \leq \pi/2 \, , \,  0 <  \theta_2 \leq \pi/2.  
\]
that expresses $T_1$ in terms of the angles $\theta_1$ and $\theta_2$. Your expression should also include $w$ (the weight of the block in pounds), but for the remainder of the question we'll regard the weight as constant.

%Work in general, \emph{not} with the specific values of the parameters shown above.

\item Check your expression by substituting $w=6$, $\theta_1 = 1$, and $\theta_2=0.6$.
\end{enumerate}

\item Activate the folder \emph{Level Curves 2} on Line 45. 

\end{enumerate}


\end{question}

\end{document}
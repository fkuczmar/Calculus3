\documentclass{ximera}
\title{Series}

\newcommand{\pskip}{\vskip 0.1 in}

\begin{document}
\begin{abstract}
Convergent and divergent series.
\end{abstract}
\maketitle


\section{Converge of Diverge}


\begin{question} \label{Q9d5nvfbg}
\begin{enumerate}
\item Explain what it means for the series
\[
  \sum_{i=1}^\infty a_n
\]
to 
\begin{enumerate}
\item converge

\item diverge.
\end{enumerate}

\item Give a specific example of a series that converges.

\item Give a specif example of a series that diverges.

\end{enumerate}
\end{question}


\begin{question} \label{QPfef912321e}
Determine whether each series converges or diverges. Find the value of those that converge if you can.

\begin{enumerate}
\item 
\[
      \sum_{i=1}^\infty (-1)^i
\]

\item 
\[
      \sum_{i=2}^\infty \left( -1/3 \right)^i 
\]

\item 
\[
      \sum_{i=2}^\infty \left( \frac{10}{9} \right)^i 
\]

\item 
\[
      \sum_{i=1}^\infty 9 \left( \frac{1}{10} \right)^i 
\]

\end{enumerate}

\end{question}

\section{Monotonic Series}

\begin{question} \label{QLdfef}
Determine whether each of the following sequences converges or diverges by comparing each with an improper integral. 

\begin{enumerate}
\item Draw  pictures to help with your explanations.

\item Find lower and upper bounds for each convergent series. 

\end{enumerate}

\begin{enumerate}
\item 
\[
      \sum_{j=1}^\infty \frac{1}{j}
\]

\item 
\[
      \sum_{k=1}^\infty \frac{2}{k^2}
\]

\item 
\[
      \sum_{k=1}^\infty \frac{2}{k^{1.01}}
\]

\item 
\[
      \sum_{k=0}^\infty \frac{1}{k^2 + 1}
\]

\item 
\[
      \sum_{k=2}^\infty \frac{1}{k^2 - 1} \text{   No integral here.}
\]


\end{enumerate}
\end{question}


\section{Alternating Series}

\begin{question} \label{Q8df3fdm3r3r}
Do  you think the series
\[
   \sum_{n=1}^\infty \frac{(-1)^{n+1}}{n}
\]
converges or diverges? Use the worksheet below to help explain your reasoning (Drag the slider $n$ in Line 2).

\begin{onlineOnly}
    \begin{center}
\desmos{rc8ladgvek}{900}{600}
\end{center}
\end{onlineOnly}

\href{https://www.desmos.com/calculator/rc8ladgvek}{163: Alternating Series 1}

\end{question}

\section{A Taylor Series}

\begin{question} \label{QPD33Lrdfr}
Let's look at the function
\[
    f(x) = \ln (1+x) \, , \, x>-1,
\]
near $x=0$.

\begin{onlineOnly}
    \begin{center}
\desmos{7royyumxbz}{900}{600}
\end{center}
\end{onlineOnly}

\href{https://www.desmos.com/calculator/7royyumxbz}{163: Motion and Taylor Series 5}


\begin{enumerate}
\item Find expressions for the polynomial approximations from degree 1 to degree 4, or until you notice a pattern.

\item Compare the graphs of the approximations with the graph of the function in the worksheet above by dragging the sldier $n$ (the degree of the approximation) in Line 4. 
What do you notice?
\begin{freeResponse}
\end{freeResponse}

\item Use the worksheet above to find successive approximations to $\ln 2$ with Taylor polynomials $p_i(x)$, $i=1,...,9$.

\item Bound the error in the estimate for $\ln 2$ from $p_9(x)$.

\item Use the worksheet above find successive approximations to $\ln(0.5)$ with Taylor polynomials $p_i(x)$, $i=1,...,9$.

\item Use ideas about geometric series to bound the error in the estimate for $\ln (0.5)$ from $p_9(x)$.

\end{enumerate}
\end{question}

\end{document}
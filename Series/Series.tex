\documentclass{ximera}
\title{Series}

\newcommand{\pskip}{\vskip 0.1 in}

\begin{document}
\begin{abstract}
Convergent and divergent series.
\end{abstract}
\maketitle


\section{Converge or Diverge?}


\begin{question} \label{Q9d5nvfbg}
\begin{enumerate}
\item Explain what it means for the series
\[
  \sum_{i=1}^\infty a_n
\]
to converge.

\item Explain what it means for the series in (a) to diverge.

\item Give a specific example of a series that converges.

\item Give a specific example of a series that diverges.

\end{enumerate}
\end{question}


\begin{question} \label{QPfef912321e}
Determine whether each series converges or diverges. Find the values of those that converge.

\begin{enumerate}
\item 
\[
      \sum_{i=1}^\infty (-1)^i
\]

\item 
\[
      \sum_{i=2}^\infty \left( -1/2 \right)^i 
\]

\begin{onlineOnly}
    \begin{center}
\desmos{04oadbivfi}{900}{600}
\end{center}
\end{onlineOnly}

\href{https://www.desmos.com/calculator/04oadbivfi}{163:Geometric Series 4}



\item 
\[
      \sum_{i=2}^\infty \left( \frac{10}{9} \right)^i 
\]

\item 
\[
      \sum_{i=1}^\infty 9 \left( \frac{1}{10} \right)^i 
\]

\end{enumerate}

\end{question}




\section{The Integral Test}

\begin{question} \label{QLdfef}
Determine whether each of the following sequences converges or diverges by comparing each with an improper integral. 

\begin{enumerate}
\item Draw  pictures to help with your explanations.

\item Find lower and upper bounds for each convergent series. 

\end{enumerate}

\begin{enumerate}
\item 
\[
      \sum_{j=1}^\infty \frac{1}{j}
\]

\begin{onlineOnly}
    \begin{center}
\desmos{uolrrejyvr}{900}{600}
\end{center}
\end{onlineOnly}

\href{https://www.desmos.com/calculator/uolrrejyvr}{163: Harmonic Series}


\item 
\[
      \sum_{k=1}^\infty \frac{1}{k^2}
\]

\begin{onlineOnly}
    \begin{center}
\desmos{6celhg9pbw}{900}{600}
\end{center}
\end{onlineOnly}

\href{https://www.desmos.com/calculator/6celhg9pbw}{163: Euler sum of squares}



\item 
\[
      \sum_{k=1}^\infty \frac{1}{k^{1.01}}
\]

\item 
\[
      \sum_{k=0}^\infty \frac{1}{k^2 + 1}
\]

\item 
\[
      \sum_{k=2}^\infty \frac{1}{k^2 - 1} \text{   No integral here.}
\]


\end{enumerate}
\end{question}


\section{Alternating Series}

\begin{question} \label{Q8df3fdm3r3r}
Do  you think the series
\[
   \sum_{n=1}^\infty \frac{(-1)^{n+1}}{n}
\]
converges or diverges? Use the worksheet below to help explain your reasoning (Drag the slider $n$ in Line 2).

\begin{onlineOnly}
    \begin{center}
\desmos{rc8ladgvek}{900}{600}
\end{center}
\end{onlineOnly}

\href{https://www.desmos.com/calculator/rc8ladgvek}{163: Alternating Series 1}

\end{question}


\section{The Monotone Convergence Theorem}

Suppose we start with a \emph{sequence} $a_n$, $n=1,2,3,...$ of \emph{non-negative numbers} and wish to determine if the \emph{series}
\[
 \sum_{i=1}^\infty a_n
\]
converges or diverges. Because all the terms of the sequence $a_n$ are at least zero, the \emph{sequence}
\[
   s_n = \sum_{i=1}^n a_n
\]
of partial sums is increasing (ie. $s_{n+1} \geq s_n$ for $n=1, 2, 3, ...$). Then there are two possibilities.
\begin{enumerate}
\item The sequence of partial sums is \emph{bounded}. This means all the partial sums are less than some positive number $M$. In this case the sequence of partial sums, and therefore the series, converges.

\item The sequence of partial sums is \emph{not} bounded. This means the partial sums eventually get larger than any number. In this case 
\[
     \lim_{n \to \infty} s_n = \infty.
\]
So the sequence of partial sums, and therefore the series, diverges. 
 
\end{enumerate}

Keep this idea in mind when answering the following questions.

\begin{question} \label{Q8dfMFe4131}
This problem is about the series
\[
    \sum_{k=0}^\infty \frac{1}{k!} .
\]

\begin{onlineOnly}
    \begin{center}
\desmos{z9vi7g79xs}{900}{600}
\end{center}
\end{onlineOnly}

\href{https://www.desmos.com/calculator/z9vi7g79xs}{163: Series for e}


\begin{enumerate}

\item Show that $s_n< 3$ for $n=1,2,3,...$. Then use the motonone convergence theorem to prove the series converges.

\item Bound the error in approximating the sum
\[
   L = \sum_{k=0}^\infty \frac{1}{k!}
\]
with the partial sum
\[
     s_9 = \sum_{k=0}^9 \frac{1}{k!} . 
\]
Do this by showing the error
\[
     \sum_{k=10}^\infty \frac{1}{k!}
\]
in the approxiation is less than
\[
   \frac{1}{10!}\left( \frac{11}{10}  \right) .
\]
\emph{Hint: Use a geometric series.}

\item Which partial sums approximate $L$ with an error of at most $10^{-12}$? Use desmos to compute the smallest of these partial sums. 

\end{enumerate}
\end{question}


\begin{question} \label{QpeFKERR31xzc}
This question is about the series
\[
  x  - \frac{x^2}{2} + \frac{x^3}{3} - \frac{x^4}{4} + \ldots ,
\]
when $x=-1/2$.

\begin{onlineOnly}
    \begin{center}
\desmos{nbk8mqqb9h}{900}{600}
\end{center}
\end{onlineOnly}

\href{https://www.desmos.com/calculator/nbk8mqqb9h}{163: Log one half}


\begin{enumerate}
\item Use summation notation to write an expression for the series when $x=-1/2$.

\item Show the series converges by finding a lower bound.

\item Bound the error in approximating the series with the tenth partial sum. 

\item Which partial sums approximate the series  with an error of at most $10^{-12}$? Use desmos to compute the smallest of these partial sums. 

\end{enumerate}
\end{question}


\section{A Taylor Series}

\begin{question} \label{QPD33Lrdfr}
Let's look at the function
\[
    f(x) = \ln (1+x) \, , \, x>-1,
\]
near $x=0$.

\begin{onlineOnly}
    \begin{center}
\desmos{7royyumxbz}{900}{600}
\end{center}
\end{onlineOnly}

\href{https://www.desmos.com/calculator/7royyumxbz}{163: Motion and Taylor Series 5}


\begin{enumerate}
\item Find expressions for the polynomial approximations from degree 1 to degree 4, or until you notice a pattern.

\item Compare the graphs of the approximations with the graph of the function in the worksheet above by dragging the sldier $n$ (the degree of the approximation) in Line 4. 
What do you notice?
\begin{freeResponse}
\end{freeResponse}

\item Use the worksheet above to find successive approximations to $\ln 2$ with Taylor polynomials $p_i(x)$, $i=1,...,9$.

\item Bound the error in the estimate for $\ln 2$ from $p_9(x)$.

\item Use the worksheet above find successive approximations to $\ln(0.5)$ with Taylor polynomials $p_i(x)$, $i=1,...,9$.

\item Use ideas about geometric series to bound the error in the estimate for $\ln (0.5)$ from $p_9(x)$.

\end{enumerate}
\end{question}

\end{document}
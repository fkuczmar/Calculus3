\documentclass{ximera}
\title{The Cross Product}

\newcommand{\pskip}{\vskip 0.1 in}

\begin{document}
\begin{abstract}
Applications of the cross product in $\mathbb{R}^3$.
\end{abstract}
\maketitle


\section{Cross-Product Exercise}
The animation below shows a point $A$ moving around a circle at a constant speed. The vector ${\bf r} = \overrightarrow{BA}$ is the position of $A$ (in meters) relative to the origin. The vector ${\bf v}$ (in meters/sec) is the velocity of $A$. 

(a) Describe how the magnitude and direction of the vector ${\bf u} = {\bf r}\times {\bf v}$ changes as $A$ moves around its path.

(b) Slide the toggle switch to "On" to see if you were correct.

(c) What are the units of ${\bf u}$? How do you know?

(d) What are the units of ${\bf u} \cdot {\bf k}$? How do you know.

(e) What do its units suggest about the meanings of $|{\bf u}|$ and ${\bf u} \cdot {\bf k}$?


\begin{exploration}   \label{Exsd67g:Cross}
\pdfOnly{
Access Geogebra interactives through the online version of this text at
 
\href{https://www.geogebra.org/classic/qmzanr8h}.
}
 
\begin{onlineOnly}
    \begin{center}
\geogebra{v9hnzmee}{900}{600}
\end{center}
\end{onlineOnly}

\end{exploration}

\section{Rotations of the Plane}

Imagine rotating three-dimensional space about an axis, as if space were a block of ice with its points fixed in position relative to one another. Our problem is this: to determine the velocity of a point in the frozen block at some instant in time given the position of the point, the axis of rotation, and the rate and sense of the rotation.

\pdfOnly{
Access Desmos interactives through the online version of this text at
 
\href{https://www.geogebra.org/classic/qmzanr8h}.
}
 
\begin{onlineOnly}
    \begin{center}
\geogebra{qmzanr8h}{900}{600}
\end{center}
\end{onlineOnly}





You may have seen a version of this problem in two-dimensions and we look at this first.


\begin{exploration}  \label{Qdstsdfgf:Cross}

\begin{question}   \label{Qdfg4bk:Cross}
Which of the following is a reasonable approximation of the rate at which point $P$ rotates about the origin in the animation below?
\begin{multipleChoice}  
\choice{$2\pi$ rad/sec}
\choice{$\pi/2$ rad/sec}  
\choice[correct]{$3$ rad/sec} 
\choice{$0.5$ rad/sec} 
\choice{$12$ rad/sec}
\end{multipleChoice}  
\end{question}


(b) Using the rate of rotation from part (a), compute the speed of $P$ as it moves around the circle of radius $20$ meters. Explain your reasoning.

(c) A point $P$ rotates about the center of a circle of radius $r$ meters at the constant rate of $\omega$ rad/sec. Express the speed of $P$ in terms of $r$ and $\omega$.

\pdfOnly{
Access Desmos interactives through the online version of this text at
 
\href{https://www.desmos.com/calculator/azxm3dowby}.
}
 
\begin{onlineOnly}
    \begin{center}
\desmos{azxm3dowby}{900}{600}
\end{center}
\end{onlineOnly}


Suppose instead we wish to find the velocity of the point $P$ in the animation above. Velocity is a vector; it is the rate at which the position of an object changes with respect to time. The (instantaneous) velocity vector points in the direction of motion and is  tangent to the path. The magnitude of the velocity vector gives the speed of the motion. While the speed of point $P$ in the above animation is constant, its velocity is not. That's because the direction of motion keeps changing.

\begin{question}   \label{Q3456457l:Cross}
Activate the {\it Velocity and Position Vectors} folder Line 3 of the above animation by clicking the circle at the left of the line. Use the rotation rate from part (a) to find the velocity of $P$ as it passes the point $Q$ with coordinates $(12,16)$. Note: The velocity vector in the demonstration is scaled by a factor of $0.5$.
\end{question}

\end{exploration}

 %And the sense of the rotation depends on the viewpoint of the observor. For example, a ferris wheel that rotates clockwise from your perspective will rotate counterclockwise for me if I'm standing on the opposite side.  


\section{Rotating About an Axis in Space}
A rotation in space is about an axis, not a point. To describe both the sense and the rate of a rotation about an axis, we give a vector ${\bf \omega}$ parallel to the axis. The vector ${\bf\omega}$ is called the \emph{angular velocity} of the rotation. The components have units of rad/sec and the magnitude $|{\bf \omega}|$ gives the rotation rate. The direction of ${\bf \omega}$ tells us the sense of rotation - point your right thumb in the direction of ${\bf \omega}$ and curl your fingers about the axis to see the sense of rotation.

Check that the sense of rotation is correct in the animation below, where ${\bf \omega} = \overrightarrow{OP}$. Drag points $P$ and $A$ to see if the sense of rotation remains correct.


\pdfOnly{
Access Desmos interactives through the online version of this text at
 
\href{https://www.geogebra.org/classic/qmzanr8h}.
}
 
\begin{onlineOnly}
    \begin{center}
\geogebra{qmzanr8h}{900}{600}
\end{center}
\end{onlineOnly}





Now imagine the $xy$-coordinate system in the animation of Exploration 2 as being the $xy$-plane in a three-dimensional coordinate system with the positive $z$-axis pointing directly out of the page. Imagine also rotating the frozen-block of three-dimensional space about the $z$-axis at the constant rate of $3$ rad/sec. Input the components of the angular velocity vector for this rotation of ${\bf R}^3$. There are two possibilities for ${\bf \omega}$. Give the one that rotates the $xy$-plane as in the above animation. 

\begin{question}  \label{Qhfnbyt:Cross}
\[
{\bf \omega} = \answer{0}{\bf i} + \answer{0}{\bf j} + \answer{3}{\bf k}
\]
\end{question}

\begin{question}  \label{Qnjy55:Cross}
Considering the animation in Exploration 2 as a slice of three-dimensional space rotating about the $z$-axis, let ${\bf r} = \overrightarrow{OP}$ be the position vector of any point $P$ in the $xy$-plane relative to the origin.

(a) Express the velocity ${\bf v}$ of $P$ in terms of the vectors ${\bf\omega}$ and ${\bf r}$. \it{Hint:} Think about how the direction of ${\bf v}$ is related to the directions of ${\bf \omega}$ and ${\bf r}$. Think also about the magnitude of ${\bf v}$, ie. the speed of $P$.

(c) Check if your expression works when $P$ has coordinates $(12, 16,0)$.

(d) Choose a specific point $O_1$on the $z$-axis other than the origin and let ${\bf r}_1 = \overrightarrow{O_1P}$ give the position of $P$ relative to $O_1$. Express the velocity ${\bf v}$ of $P$ in terms of the vectors ${\bf\omega}$ and ${\bf r}_1$. 


(e) Do your expressions in parts (c) and (d) give the correct velocity of a point in the plane $z=7$ as it passes $(-20,0,7)$?


\end{question}


\begin{question}  \label{Qdstjigvgf:Cross}
Imagine now a rotation of space about an axis through the points $Q_1Q_2$ with angular velocity ${\bf \omega}$. Let $P$ be a point in the rotating space. The figure below lies in the plane through $P$ and the axis of rotation. Point $Q$ lies on the axis.

(a) In what direction is the velocity of $P$ at the instant shown below?

(b) Express the speed of $P$ in terms of the distance $d$ from $P$ to the axis of rotation and the rotation rate $|{\bf w}|$.

(c) Express the speed of $P$ in terms of the rotation rate $|{\bf w}|$, the distance $|{\bf r}| and $the angle $\theta$ between the vectors ${\bf \omega}$ and ${\bf r} = \overrightarrow{QP}$.

(d) Express the velocity of $P$ in terms of the vectors ${\bf \omega}$ and ${\bf r}$.


(e) Drag $Q$ and describe how the direction and magnitude of the vector ${\bf v} = {\bf \omega}\times {\overrightarrow{QP}}$ change as you move $Q$ from $Q_1$ to $Q_2$. Explain this two ways:

i) by thinking about the magnitude and direction of ${\bf v}$

ii) by simplifying the difference ${\bf \omega}\times {\overrightarrow{QP}} - {\bf \omega}\times {\overrightarrow{Q_1P}}$

%(b) Use vector algebra to justify your assertion.  {\it Hint:}

\pdfOnly{
Access Geogebra interactives through the online version of this text at
 
\href{https://www.geogebra.org/classic/aasfttnr}.
}
 
\begin{onlineOnly}
    \begin{center}
\geogebra{aasfttnr}{900}{600}
\end{center}
\end{onlineOnly}

\end{question}


\begin{question} \label{Qgdsnt:Cross}
Three-dimensional space rotates about an axis through the points $A(4,2,1)$ and $B(3,0,-1)$ at the constant rate of $6$ radians/sec (the coordinates are measured in meters). The angular velocity of the rotation is in the direction of the vector $\overrightarrow{AB}$. 

(a) Which of the following could {\bf not} be the velocity of a point in the rotating space? Select all that apply.

\begin{selectAll}  
    \choice[correct]{$\langle 3,-1,-1\rangle$}  
    \choice{$\langle -4,1,1\rangle$}  
    \choice[correct]{$\langle 5,1,-3\rangle$}  
    \choice[correct]{$\langle 0,0,0\rangle$}  
  \end{selectAll}  


\pskip

(b) Find the velocity of the point $P(1,1,1)$.

(c) Describe geometrically the set of points in space with velocity ${\bf v} =\langle -4,1,1 \rangle$.

(d) Use vector arithmetic to find the coordinates of a point with velocity ${\bf v} =\langle -4,1,1 \rangle$. Then make your description in part (c) more precise.

\end{question}


\begin{exploration}



\href{https://www.desmos.com/3d/31f1fd3ffd}{Rotating about an axis}


\href{https://www.desmos.com/3d/a82ef238da}{Rotating about an axis, slide show}


\end{exploration}



\begin{question}  \label{Q3254hv:Cross}
We started this chapter with a rotation of space about the $z$-axis having an angular velocity ${\bf \omega} = \langle 0,0,3 \rangle$rad/sec and saw that a rotating point passed $(12,16,0)$m with velocity ${\bf v}= 3\langle -16,12,0\rangle$m/sec. Describe other rotations that have the same action at the point $(12,16,0)$ with their axes

(a) perpendicular to the $xy$-plane

(b) parallel to the $xy$-plane

(c) neither parallel nor perpendicular to the $xy$-plane
 
\end{question}


\begin{question}  \label{Qdsftr54hv:Cross}
(a) Describe a rotation of space through the angle $\pi$ radians that carries the point $P(3,4,7)$ to the point $Q(-1,0,1)$.

(b) Describe a rotation of space through the angle $\pi/2$ radians that carries the point $P(3,4,7)$ to the point $Q(-1,0,1)$.

(c) Describe a rotation of space through the angle $\theta$ radians that carries the point $P(3,4,7)$ to the point $Q(-1,0,1)$.
\end{question}



\end{document}


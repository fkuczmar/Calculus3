\documentclass{ximera}
\title{The Cross Product}

\newcommand{\pskip}{\vskip 0.1 in}

\begin{document}
\begin{abstract}
Applications of the cross product in $\mathbb{R}^3$.
\end{abstract}
\maketitle

\section{Rotating About an Axis}

Imagine rotating three-dimensional space about an axis, as if space were a block of ice with its points fixed in position relative to one another. How can we determine the velocity velocity of a point in the frozen block at some instant in time?

You may have seen a version of this problem in two-dimensions.


\begin{exploration}  \label{Qdstsdfgf:Cross}

\begin{question}   \label{Qdfg4bk:Cross}
Which of the following is a reasonable approximation of the rate at which point $P$ rotates about the origin in the animation below?
\begin{multipleChoice}  
\choice{$2\pi$ rad/sec}
\choice{$\pi/2$ rad/sec}  
\choice[correct]{$3$ rad/sec} 
\choice{$0.5$ rad/sec} 
\choice{$12$ rad/sec}
\end{multipleChoice}  
\end{question}


(b) Using the rate of rotation from part (a), compute the speed of $P$ as it moves around the circle of radius $20$ meters. Explain your reasoning.

(c) What is the speed $v$ (in meters/sec) of a point rotating about the center of a circle of radius $r$ meters at the constant rate of $\omega$ rad/sec.

\pdfOnly{
Access Desmos interactives through the online version of this text at
 
\href{https://www.desmos.com/calculator/mu8mj82dvr}.
}
 
\begin{onlineOnly}
    \begin{center}
\desmos{mu8mj82dvr}{900}{600}
\end{center}
\end{onlineOnly}




\end{exploration}



https://www.desmos.com/calculator/mu8mj82dvr


\begin{question}  \label{Qdstjigvgf:Cross}
(a) Drag $Q$ and describe how the direction and magnitude of the vector ${\bf w}\times {\overrightarrow{QP}}$ change as you move $Q$ from $Q_1$ to $Q_2$ along the line.

(b) Use vector algebra to justify your assertion.  {\it Hint:}

\pdfOnly{
Access Geogebra interactives through the online version of this text at
 
\href{https://www.geogebra.org/classic/aasfttnr}.
}
 
\begin{onlineOnly}
    \begin{center}
\geogebra{aasfttnr}{900}{600}
\end{center}
\end{onlineOnly}



\end{question}



\begin{exploration}



\href{https://www.desmos.com/3d/31f1fd3ffd}{Rotating about an axis}


\href{https://www.desmos.com/3d/a82ef238da}{Rotating about an axis, slide show}


\end{exploration}


\end{document}


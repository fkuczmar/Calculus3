\documentclass{ximera}
\title{The Cross Product}

\newcommand{\pskip}{\vskip 0.1 in}

\begin{document}
\begin{abstract}
Rotating Space: An application of the cross product in $\mathbb{R}^3$.
\end{abstract}
\maketitle


\section{Cross-Product Exercise}
The animation below shows a point $A$ moving around a circle at a constant speed. The vector ${\bf r} = \overrightarrow{BA}$ is the position of $A$ (in meters) relative to the origin. The vector ${\bf v}$ (in meters/sec) is the velocity of $A$. 

(a) Describe how the magnitude and direction of the vector ${\bf u} = {\bf r}\times {\bf v}$ changes as $A$ moves around its path.

(b) Slide the toggle switch to "On" to see if you were correct.

(c) What are the units of ${\bf u}$? How do you know?

(d) What are the units of ${\bf u} \cdot {\bf k}$? How do you know.

(e) What do the units suggest about the meanings of $|{\bf u}|$ and ${\bf u} \cdot {\bf k}$?

\pskip

{\bf Takeaway:} It pays to pay attention to units. In applications (this one is really about angular momentum), it is usually best to interpret the magnitude of the cross product in some way other than as an area.


\begin{exploration}   \label{Exsd67g:Cross}
\pdfOnly{
Access Geogebra interactives through the online version of this text at
 
\href{https://www.geogebra.org/classic/qmzanr8h}.
}
 
\begin{onlineOnly}
    \begin{center}
\geogebra{v9hnzmee}{900}{600}
\end{center}
\end{onlineOnly}

\end{exploration}

\section{Rotations of the Plane}

Imagine rotating three-dimensional space about an axis, as if space were a block of ice with its points fixed in position relative to one another. Our problem is this: to determine the velocity of a point in the frozen block at some instant in time given the position of the point, the axis of rotation, and the rate and sense of the rotation.

\pdfOnly{
Access Desmos interactives through the online version of this text at
 
\href{https://www.geogebra.org/classic/qmzanr8h}.
}
 
\begin{onlineOnly}
    \begin{center}
\geogebra{qmzanr8h}{900}{600}
\end{center}
\end{onlineOnly}





You may have seen a version of this problem in two-dimensions and we'll look at this shortly. But first we'll review what it means to measure an angle in radians.

\begin{exploration} \label{Esaghp:Cross}
The tick marks on the number line below are spaced at intervals equal to the radius of the circle.

Drag the slider $\phi$ to wrap the number line around the circle. The angles that appear each have a measure of one radian.

\pdfOnly{
Access Desmos interactives through the online version of this text at
 
\href{https://www.desmos.com/calculator/a4yeivf5id}.
}
 
\begin{onlineOnly}
    \begin{center}
\desmos{a4yeivf5id}{900}{600}
\end{center}
\end{onlineOnly}


\begin{question}   \label{Qsatd4th:Cross}
What does it mean for an angle to have a measure of one radian? Check all that apply. In the choices below, point $O$ is the vertex of the angle. 

\begin{selectAll}  
    \choice{The angle cuts out an arclength of 1 meter from a circle of any radius centered at $O$.}  
    \choice[correct]{The angle cuts out an arclength of 5 meters from a circle of radius $5$ meters centered at $O$.}  
    \choice[correct]{The angle cuts out an arclength of r meters from a circle of radius $r$ meters centered at $O$}  
    \choice{The angle cuts out an arclength equal to $1/6$ the circumference of any circle centered at $O$}  
  \end{selectAll}  

\end{question}


\begin{question}  \label{Qt467743:Cross}
As you run around a track of radius $40$ meters, you turn about the track's center at the constant rate of $0.1$ rad/sec.

(a) How far do you run in one second?
\[
   \answer{4} \; \text{meters}
\]

(b) What is your speed as you run around the track?
\[
     \answer{4} \; \text{meters/sec} 
\]

(c) Suppose you run around a circlular track of radius $r$ meters while turning about the track's center at the constant rate of $\omega$. Express your speed $v$ (measured in meters/sec) in terms of $r$ and $\omega$.
\[
      v = \answer{\omega r} \; \text{meters/sec}
\]

(d) Check that your expression for the speed in part (c) has the correct units.


{\bf Takeaway:} Measuring a rotation rate in radians/sec gives us an easy way to express our speed in terms of the rotation rate about the center and the radius of our circular path.

\end{question}
\end{exploration}






%\begin{exploration}  \label{Qdstsdfgf:Cross}

\begin{question}   \label{Qdfg4bk:Cross}
Which of the following is a reasonable approximation of the rate at which point $P$ rotates about the origin in the animation below?
\begin{multipleChoice}  
\choice{$2\pi$ rad/sec}
\choice{$\pi/2$ rad/sec}  
\choice[correct]{$3$ rad/sec} 
\choice{$0.5$ rad/sec} 
\choice{$12$ rad/sec}
\end{multipleChoice}  

(b) Using the rate of rotation from part (a), compute the speed of $P$ as it moves around the circle of radius $20$ meters. 
\[
   \answer{60} \text{ meters/sec}
\]

\end{question}


\pdfOnly{
Access Desmos interactives through the online version of this text at
 
\href{https://www.desmos.com/calculator/cofbphrshd}.
}
 
\begin{onlineOnly}
    \begin{center}
\desmos{cofbphrshd}{900}{600}
\end{center}
\end{onlineOnly}


(c) Suppose instead we wish to find the velocity of the point $P$ in the animation above. Velocity is a vector; it is the rate at which the position of an object changes with respect to time. The (instantaneous) velocity vector points in the direction of motion and is  tangent to the path. The magnitude of the velocity vector gives the speed of the motion. While the speed of point $P$ in the above animation is constant, its velocity is not. That's because the direction of motion keeps changing.

\begin{question}   \label{Q3456457l:Cross}
Activate the {\it Velocity and Position Vectors} folder Line 3 of the above animation by clicking the circle at the left of the line. We'll use the speed from part (b) to find the velocity ${\bf v}$ of $P$ as it passes the point $Q$ with coordinates $(12,16)$.

(i) First find the components of the vector ${\bf r}=\overrightarrow{OQ}$ that gives the position of $Q$ relative to the origin.
\[
    {\bf r} = \answer{12}{\bf i} + \answer{16}{\bf j} \text{ meters}
\]

(ii) Next find the components of a vector ${\bf u}$ that points in the direction of the velocity vector ${\bf v}$ and has the same magnitude as ${\bf r}$. {\it Hint:} Because the path is a circle, ${\bf v}$ is perpendicular to ${\bf r}$. Think about a dot product (instead of slope) to find the two possibilities for ${\bf u}$. Then choose the one with the correct direction, pointing in the (forward) direction of motion.
\[
    {\bf u} =  \answer{-16}{\bf i} + \answer{12}{\bf j} \text{ meters}
\]

(iii) Finally, find the components of the velocity vector ${\bf v}$ by scaling ${\bf u}$ so that it has the correct length.
\[
     {\bf v} = \answer{3} {\bf u} = \answer{-48}{\bf i} + \answer{36}{\bf j} \text{ m/s}
\]

(iv) What are the units of the scalar in the above expression for ${\bf v}$? What is the meaning of the scalar?

\end{question}  

%\end{exploration}

 

\section{Rotating Space About an Axis}
There is a better way to compute the velocity of a point rotating in a plane, by considering the plane as being part of three-dimensional space and thinking about a rotation of that space.

We rotate space about about an axis, not a point. To describe both the sense and the rate of a rotation, we give a vector $\boldsymbol{\omega}$ parallel to the axis. The vector $\boldsymbol{\omega}$ is called the \emph{angular velocity} of the rotation. Its components have units of rad/sec and the magnitude $|\boldsymbol{\omega}|$ gives the rotation rate. The direction of $\boldsymbol{\omega}$ tells us the sense of rotation - point your right thumb in the direction of $\boldsymbol{\omega}$ and curl your fingers about the axis; your fingers give you the sense of rotation.

\begin{exploration}  \label{Ex4gb3e3:Cross}

(a) Check that the sense of rotation is correct in the animation below where the angular velocity is the vector $\boldsymbol{\omega} = \overrightarrow{OA}$. Drag points $P$ and $A$ to see if the sense of rotation remains correct.


\pdfOnly{
Access Desmos interactives through the online version of this text at
 
\href{https://www.geogebra.org/classic/qmzanr8h}.
}
 
\begin{onlineOnly}
    \begin{center}
\geogebra{qmzanr8h}{900}{600}
\end{center}
\end{onlineOnly}

\begin{question} \label{Qdsbhjj}
What angle does the velocity vector of point $E$ make with the angular velocity vector $\boldsymbol{\omega}$? 
\[
   \answer{\pi/2} \text{ radians}
\]
\end{question}


\begin{question} \label{Qddsghhhjj}
What angle does the velocity vector of point $E$ make with the vector ${\overrightarrow{OE}}$? This one might be harder to see. It always is for me. If so, a physical model might help (or wait a bit for an easier way to visualize this angle).

\[
   \answer{\pi/2} \text{ radians}
\]
\end{question}




\end{exploration}



\begin{question}  \label{Qhfnbyt:Cross}
Now imagine the $xy$-coordinate system in the animation of Question 5  as being in a three-dimensional coordinate system with the positive $z$-axis pointing directly out of the page. Imagine also rotating the frozen-block of three-dimensional space about the $z$-axis at the constant rate of $3$ rad/sec. 


(a) Input the components of the angular velocity vector for this rotation of ${\bf R}^3$. There are two possibilities for $\boldsymbol{\omega}$. Give the one that rotates the $xy$-plane counterclockwise as in the animation of Question 5. 
\[
\boldsymbol{\omega} = \answer{0}{\bf i} + \answer{0}{\bf j} + \answer{3}{\bf k}
\]

(b) Considering the animation in Question 5 as a slice of three-dimensional space rotating about the $z$-axis, let ${\bf r} = \overrightarrow{OP}$ be the position vector of any point $P$ in the $xy$-plane relative to the origin. Let ${\bf v}$ be the velocity of {\bf any} point $P$ in the plane and ${\bf r} = \overrightarrow{OP}$ the position of $P$ relative to the origin.

(c) What angle do the vectors ${\bf v}$,  ${\bf r}$, and $\boldsymbol{\omega}$ make with each other?
\[
     \answer{\pi/2}\text{ radians}
\]

(d) Express the speed $v$ (in meters/sec) of $P$ in terms of the distance $|{\bf r}|$ and the rotation rate $|\boldsymbol{\omega}|$.

(e) Parts (c) and (d) suggest a way to express the velocity of $P$ in terms of the vectors ${\bf r}$ and $\boldsymbol{\omega}$.
There are two possible choices. To choose the correct one, think about how the direction of ${\bf v}$ is related to the directions of $\boldsymbol{\omega}$ and ${\bf r}$. 


sdfsaaaaaaaaaaaaaaaaaaaaaaaaaaa


(d) Check if your expression works when $P$ has coordinates $(12, 16,0)$.

(e) Choose a specific point $O_1$on the $z$-axis other than the origin and let ${\bf r}_1 = \overrightarrow{O_1P}$ give the position of $P$ relative to $O_1$. Express the velocity ${\bf v}$ of $P$ in terms of the vectors ${\boldsymbol{\omega}}$ and ${\bf r}_1$. 


(f) Do your expressions in parts (d) and (e) give the correct velocity of a point in the plane $z=7$ as it passes $(-20,0,7)$?


\end{question}


\begin{question}  \label{Qdstjigvgf:Cross}
Imagine now a rotation of space about an axis through the points $Q_1Q_2$ with angular velocity $\boldsymbol{\omega}$. Let $P$ be a point in the rotating space. The figure below lies in the plane through $P$ and the axis of rotation. Point $Q$ lies on the axis.

(a) In what direction is the velocity of $P$ at the instant shown below?

(b) Express the speed of $P$ in terms of the distance $d$ from $P$ to the axis of rotation and the rotation rate $|\boldsymbol{\omega}|$.

(c) Express the speed of $P$ in terms of the rotation rate $|\boldsymbol{\omega}|$, the distance $|{\bf r}|$, and the angle $\theta$ between the vectors $\boldsymbol{\omega}$ and ${\bf r} = \overrightarrow{QP}$.

(d) Express the velocity of $P$ in terms of the vectors $\boldsymbol{\omega}$ and ${\bf r}$.


(e) Drag $Q$ and describe how the direction and magnitude of the vector ${\bf v} = \boldsymbol{\omega}\times {\overrightarrow{QP}}$ change as you move $Q$ from $Q_1$ to $Q_2$. Explain this two ways:

i) by thinking about the magnitude and direction of ${\bf v}$

ii) by simplifying the difference $\boldsymbol{\omega}\times {\overrightarrow{QP}} - \boldsymbol{\omega}\times {\overrightarrow{Q_1P}}$

%(b) Use vector algebra to justify your assertion.  {\it Hint:}

\pdfOnly{
Access Geogebra interactives through the online version of this text at
 
\href{https://www.geogebra.org/classic/aasfttnr}.
}
 
\begin{onlineOnly}
    \begin{center}
\geogebra{aasfttnr}{900}{600}
\end{center}
\end{onlineOnly}

\end{question}


\begin{question} \label{Qgdsnt:Cross}
Three-dimensional space rotates about an axis through the points $A(4,2,1)$ and $B(3,0,-1)$ at the constant rate of $6$ radians/sec (the coordinates are measured in meters). The angular velocity of the rotation is in the direction of the vector $\overrightarrow{AB}$. 

(a) Which of the following could {\bf not} be the velocity of a point in the rotating space? Select all that apply.

\begin{selectAll}  
    \choice[correct]{$\langle 3,-1,-1\rangle$}  
    \choice{$\langle -4,1,1\rangle$}  
    \choice[correct]{$\langle 5,1,-3\rangle$}  
    \choice[correct]{$\langle 0,0,0\rangle$}  
  \end{selectAll}  


\pskip

(b) Find the velocity of the point $P(1,1,1)$.

(c) Describe geometrically the set of points in space with velocity ${\bf v} =\langle -4,1,1 \rangle$.

(d) Use vector arithmetic to find the coordinates of a point with velocity ${\bf v} =\langle -4,1,1 \rangle$. Then make your description in part (c) more precise.

\end{question}


\begin{exploration}



\href{https://www.desmos.com/3d/31f1fd3ffd}{Rotating about an axis}


\href{https://www.desmos.com/3d/a82ef238da}{Rotating about an axis, slide show}


\end{exploration}



\begin{question}  \label{Q3254hv:Cross}
We started this chapter with a rotation of space about the $z$-axis having an angular velocity $\boldsymbol{\omega} = \langle 0,0,3 \rangle$rad/sec and saw that a rotating point passed $(12,16,0)$m with velocity ${\bf v}= 3\langle -16,12,0\rangle$m/sec. Describe other rotations that have the same action at the point $(12,16,0)$ with their axes

(a) perpendicular to the $xy$-plane

(b) parallel to the $xy$-plane

(c) neither parallel nor perpendicular to the $xy$-plane
 
\end{question}


%\begin{question}  \label{Qdsftr54hv:Cross}
%(a) Describe a rotation of space through the angle $\pi$ radians that carries the point $P(3,4,7)$ to the point $Q(-1,0,1)$.

%(b) Describe a rotation of space through the angle $\pi/2$ radians that carries the point $P(3,4,7)$ to the point $Q(-1,0,1)$.

%(c) Describe a rotation of space through the angle $\theta$ radians that carries the point $P(3,4,7)$ to the point $Q(-1,0,1)$.
%\end{question}



\end{document}


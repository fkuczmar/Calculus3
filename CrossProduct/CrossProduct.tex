\documentclass{ximera}
\title{The Cross Product}

\newcommand{\pskip}{\vskip 0.1 in}

\begin{document}
\begin{abstract}
Applications of the cross product in $\mathbb{R}^3$.
\end{abstract}
\maketitle

\section{Rotating About a Point in a Plane}

Imagine rotating three-dimensional space about an axis, as if space were a block of ice with its points fixed in position relative to one another. Our problem is this: to determine the velocity of a point in the frozen block at some instant in time given the position of the point, the axis of rotation, and the rate and sense of the rotation.

You may have seen a version of this problem in two-dimensions.


\begin{exploration}  \label{Qdstsdfgf:Cross}

\begin{question}   \label{Qdfg4bk:Cross}
Which of the following is a reasonable approximation of the rate at which point $P$ rotates about the origin in the animation below?
\begin{multipleChoice}  
\choice{$2\pi$ rad/sec}
\choice{$\pi/2$ rad/sec}  
\choice[correct]{$3$ rad/sec} 
\choice{$0.5$ rad/sec} 
\choice{$12$ rad/sec}
\end{multipleChoice}  
\end{question}


(b) Using the rate of rotation from part (a), compute the speed of $P$ as it moves around the circle of radius $20$ meters. Explain your reasoning.

(c) What is the speed $v$ (in meters/sec) of a point rotating about the center of a circle of radius $r$ meters at the constant rate of $\omega$ rad/sec.

\pdfOnly{
Access Desmos interactives through the online version of this text at
 
\href{https://www.desmos.com/calculator/azxm3dowby}.
}
 
\begin{onlineOnly}
    \begin{center}
\desmos{azxm3dowby}{900}{600}
\end{center}
\end{onlineOnly}


Suppose instead we wanted to find the velocity of the point $P$ in the animation above. Velocity is a vector; it is the rate at which the position of an object changes with respect to time. The (instantaneous) velocity vector points in the direction of motion and is  tangent to the path. The magnitude of the vector gives the speed of the motion. While the speed of point $P$ in the above animation is constant, its velocity is not. That's because the direction of the motion keeps changing.

\begin{question}   \label{Q3456457l:Cross}
Activate the {\it Velocity and Position Vectors} folder Line 3 of the above animation by clicking the circle at the left of the line. Use the rotation rate from part (a) to find the velocity of $P$ as it passes the point $Q$ with coordinates $(12,16)$. Note: The velocity vector in the demonstration is scaled by a factor of $0.5$.
\end{question}

\end{exploration}

 %And the sense of the rotation depends on the viewpoint of the observor. For example, a ferris wheel that rotates clockwise from your perspective will rotate counterclockwise for me if I'm standing on the opposite side.  


\section{Rotating About an Axis in Space}
A rotation in space is about an axis, not a point. To describe both the sense and the rate of a rotation about the axis, we give a vector ${\bf \omega}$ parallel to the axis. The vector ${\bf\omega}$ is called the \emph{angular velocity} of the rotation. The components have units of rad/sec and the magnitude $|{\bf \omega}|$ gives the rotation rate. The direction of ${\bf \omega}$ gives the sense of rotation - point your right thumb in the direction of ${\bf \omega}$ and curl your fingers about the axis to see the sense of rotation.


Now imagine the $xy$-coordinate system in the animation above as being the $xy$-plane in a three-dimensional coordinate system with the positive $z$-axis pointing directly out of the page. Imagine also rotating the frozen-block of three-dimensional space about the $z$-axis at the constant rate of $3$ rad/sec. Input the components of the angular velocity vector for this rotation of ${\bf R}^3$ that restricts to the rotation of the $xy$-plane in the above animation. 

\begin{question}  \label{Qhfnbyt:Cross}
\[
{\bf \omega} = \answer{0}{\bf i} + \answer{0}{\bf j} + \answer{3}{\bf k}
\]
\end{question}

\begin{question}  \label{Qnjy55:Cross}
Considering the animation above as a slice of three-dimensional space rotating about the $z$-axis, let ${\bf r} = \overrightarrow{OP}$ be the position vector of the point $P$ in the $x$-plane relative to the origin.

(a) Express the velocity ${\bf v}$ of $P$ in terms of the vectors ${\bf\omega}$ and ${\bf r}$. \it{Hint:} Think about how the direction of ${\bf v}$ is related to the directions of ${\bf \omega}$ and ${\bf r}$. Think also about the magnitude of ${\bf v}$, ie. the speed of $P$.

(c) Check if your expression works when $P$ has coordinates $(12, 16)$ as above.

(d) Does your expression give the correct velocity of a point in the plane $z=7$ as it passes the point $(20,0,7)$?. Or the point $(12,16,7)$?


\end{question}


\begin{question}  \label{Qdstjigvgf:Cross}
(a) Drag $Q$ and describe how the direction and magnitude of the vector ${\bf w}\times {\overrightarrow{QP}}$ change as you move $Q$ from $Q_1$ to $Q_2$ along the line.

(b) Use vector algebra to justify your assertion.  {\it Hint:}

\pdfOnly{
Access Geogebra interactives through the online version of this text at
 
\href{https://www.geogebra.org/classic/aasfttnr}.
}
 
\begin{onlineOnly}
    \begin{center}
\geogebra{aasfttnr}{900}{600}
\end{center}
\end{onlineOnly}



\end{question}



\begin{exploration}



\href{https://www.desmos.com/3d/31f1fd3ffd}{Rotating about an axis}


\href{https://www.desmos.com/3d/a82ef238da}{Rotating about an axis, slide show}


\end{exploration}


\end{document}


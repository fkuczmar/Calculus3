\documentclass{ximera}
\title{The Scalar Product}

\newcommand{\pskip}{\vskip 0.1 in}

\begin{document}
\begin{abstract}
An introduction to vector and scalar projections
\end{abstract}
\maketitle

\section{Vector Projections}

It often helps in applications to resolve a vector ${\bf v}$ into two components, one parallel to a given vector ${\bf w}$ and the other perpendicular to ${\bf w}$. For example, resolving the acceleration vector into components parallel and perpendicular to the velocity vector gives information about the external forces acting on an object and how they affect the object's speed and the rate at which it changes direction.


\begin{exploration}
Drag points $A$, $B$, $D$, and $E$ to see how the vector projection of ${\bf v}$ in the direction of ${\bf w}$ changes. Watch also the vector component of ${\bf v}$ perpendicular to ${\bf w}$.
\pdfOnly{
Access Geogebra interactives through the online version of this text at
 
\href{https://www.geogebra.org/calculator/dfwwd9dg}.
}
 
\begin{onlineOnly}
    \begin{center}
\geogebra{dfwwd9dg}{900}{600}
\end{center}
\end{onlineOnly}
\end{exploration}





The scalar projection of the vector ${\bf v}$ in the direction of the vector ${\bf w}$ is the vector ${\bf v}_{par}$ parallel to ${\bf w}$ that make the difference
\[
      {\bf v}_{perp} = {\bf v} - {\bf v}_{par}
\]
perpendicular to ${\bf w}$.

In two dimensions slope makes it easy to determine if two vectors are parallel. But we cannot really talk about the slope of a vector in $\mathbb{R}^3$ or higher dimensions. So to resolve a vector into two components as just described, we need a way to determine whether two vectors in the same dimension are perpendicular. It you think about this problem from scratch (ie. ignoring what you may have read in another source), the solution is not immediately obvious.

\begin{question}  \label{Q235r74:Scalar}
Which of the following are equivalent to the condition that the non-zero vectors ${\bf a}, {\bf b} \in \mathbb{R}^n$ are perpendicular? Check all that apply. Draw pictures.

\begin{selectAll}  
    \choice{The vectors ${\bf a}+{\bf b}$ and ${\bf a}-{\bf b}$ are perpendicular.}  
    \choice{The vector ${\bf a}+{\bf b}$ bisects the angle between ${\bf a}$ and ${\bf b}$.}  
    \choice[correct]{$|{\bf a}+{\bf b}| = |{\bf a}-{\bf b}|$}  
    \choice[correct]{$|{\bf a}+{\bf b}|^2 = |{\bf a}|^2 + |{\bf b}|^2$}
      \choice{$|{\bf a}-{\bf b}|^2 = |{\bf a}|^2-|{\bf b}|^2$}
    \choice[correct]{$|{\bf a}-{\bf b}|^2 = |{\bf a}|^2+|{\bf b}|^2$} 
  \end{selectAll}  
 \end{question}


\begin{question}   \label{Q32erdf:Scalar}
Use one of the correct conditions above to find an algebraic condition for the vectors 
\[
    {\bf v}_1= \langle x_1, y_1, z_1 \rangle, {\bf v}_2 = \langle  x_2, y_2, z_2\rangle \in \mathbb{R}^3 
\]
to be perpendicular.
\end{question}

\begin{question}  \label{Qgsdfbe34:Scalar}
Find three vectors in $\mathbb{R}^3$, all pointing in different directions, that are perpendicular to the vector ${\bf v}= \langle -4,3,11\rangle$. 
\end{question}



\begin{question} \label{Qdf6:Scalar}
\pdfOnly{
Access Geogebra interactives through the online version of this text at
 
\href{https://www.desmos.com/calculator/mgihe5dswd}.
}
 
\begin{onlineOnly}
    \begin{center}
\desmos{mgihe5dswd}{900}{600}
\end{center}
\end{onlineOnly}


\end{question}


\begin{question}   \label{Qrasd5r:Scalar}
(a) Given vectors ${\bf v}, {\bf w} \in \mathbb{R}^n$, use the scalar product to determine the value of the scalar $\lambda\in \mathbb{R}$ that makes the vector
\[
     {\bf v}_{perp} = {\bf v} - \lambda {\bf w}
\]
perpendicular to ${\bf w}$.

(b) Use part (a) to find an expression for the component of ${\bf v}$ in the direction of ${\bf w}$.
\end{question}


\begin{question}  \label{Qfgsadge:Scalar}
Let ${\cal L}$ be the line through the origin and the point $A(3,1,4)$. Let $Q$ be the point with coordinates $(11,5,10)$.

(a) Use a vector projection to find the coordinates of the point on ${\cal L}$ closest to $Q$.

(b) Use calculus to find the coordinates of the point on ${\cal L}$ closest to $Q$.

(c) Compare your solutions to (a) and (b).

(d) Find an equation of the circular cylinder through $Q$ that is symmetric about ${\cal L}$. {\it Hint:} Start by descrbing the cylinder as a set of points satisfying some condition involving distance. Then translate this description into an equation.

(e) Find an equation of the circular cone through $Q$ with its vertex at the origin that is symmetric about ${\cal L}$. Start by descrbing the cylinder as a set of points satisfying some condition involving distance. Then translate this description into an equation.
\end{question}


\begin{question}  \label{Qfgs643r6e:Scalar}
Let ${\cal L}$ be the line through the points $A(3,1,4)$ and $B(1,-1,0)$. Let $Q$ be the point with coordinates $(7,-3,-2)$.

(a) Use a vector projection to find the coordinates of the point on ${\cal L}$ closest to $Q$.

(b) Find an equation of the circular cylinder through $Q$ that is symmetric about ${\cal L}$. 

(c) Find an equation of the circular cone through $Q$ with its vertex at $A$ that is symmetric about ${\cal L}$. Start by descrbing the cone as a set of points satisfying some condition involving distance. Then translate this description into an equation.
\end{question}



\section{Light Reflecting off a Mirror}
\begin{exploration}  \label{E32fvsav:Scalar}
Light hits the mirror below moving parallel to the vector ${\bf v}_{i}$. It refects off the mirror heading in the direction parallel to the vector ${\bf v}_o$. The vector ${\bf w}$ is parallel to the mirror.

Express the vector ${\bf v}_o$ in terms of the vectors ${\bf v}_{i}$ and ${\bf w}$.

\pdfOnly{
Access Geogebra interactives through the online version of this text at
 
\href{https://www.geogebra.org/classic/xvxbt43k}.
}
 
\begin{onlineOnly}
    \begin{center}
\geogebra{xvxbt43k}{900}{600}
\end{center}
\end{onlineOnly}


\end{exploration}



\begin{exploration}  \label{E32fvsdf9v:Scalar}

\pdfOnly{
Access Geogebra interactives through the online version of this text at
 
\href{https://www.desmos.com/calculator/1xrumhkojh}.
}
 
\begin{onlineOnly}
    \begin{center}
\desmos{1xrumhkojh}{900}{600}
\end{center}
\end{onlineOnly}


\end{exploration}



\section{Billiards Anyone?}

\begin{exploration}\label{exp:pc1c}
Play the slider $s$. Control the direction of the cue ball with the slider $\phi$. 
%Enter the motions of the balls in the desmos activity in Lines ??-?? below.

\pdfOnly{
Access Desmos interactives through the online version of this text at
 
\href{https://www.desmos.com/calculator/8p4pp5ri6k}.
}
 
\begin{onlineOnly}
    \begin{center}
\desmos{8p4pp5ri6k}{900}{600}
\end{center}
\end{onlineOnly}
\end{exploration}


\section{Scalar Projections}
The easiest way for me to think about scalar projections is via a change in coordinates. Given the coordinates of a point $P$ with respect to the $xy$-coordinate system, we wish to determine its coordinates with respect to some other coordinate system with the same origin but with its axes pointing in some other directions.

\begin{exploration}
\pdfOnly{
Access Geogebra interactives through the online version of this text at
 
\href{https://www.geogebra.org/classic/fppzrurp}.
}
 
\begin{onlineOnly}
    \begin{center}
\geogebra{fppzrurp}{900}{600}
\end{center}
\end{onlineOnly}
\end{exploration}


https://www.geogebra.org/classic/fppzrurp


\end{document}
\documentclass{ximera}
\title{Motion, Part 1}

\newcommand{\pskip}{\vskip 0.1 in}

\begin{document}
\begin{abstract}
Motion
\end{abstract}
\maketitle


\begin{exploration}
Suppose  you're driving your car along a winding road. The instantaneous velocity of your car at some moment is the rate of change in its position with repsect to time. Since position is a vector, so is velocity. It has a direction and magnitude. The direction points in the direction you are moving  and the magnitude gives your speed.

But just like all other rates of change, we can speak about average velocity. And we really should speak about this first. Your average velocity over a time interval is the change in your position divided by the length of the interval. This is also a vector.

The geogebra worksheet below suggest a projectile motion in a uniform graviational field. The position of the projectile is plotted at equally-spaced time intervals along its trajectory. The time $u$ is measured in seconds and the position vectors are measured in meters.

The vector ${\bf v}_{avg}$ represents the average velocity between times $u=-1.4$ and $u=-1.4+\Delta t = 1.6$. 

\begin{question}  \label{Qfdhgyuj:Motion}
(a) Which vector gives the change in position going forward in time from time $u=-1.4$ to time $u=1.6$ seconds. 
\begin{multipleChoice}  
\choice[correct]{${\bf r}_Q - {\bf r}_P$}  
\choice{${\bf r}_P - {\bf r}_Q$}  
\choice{$0.5({\bf r}_P + {\bf r}_Q)$}  
\end{multipleChoice}  


(b) (a) Which vector gives the average velocity ${\bf v}_{avg}$ between times $u=-1.4$ to time $u=1.6$ seconds. 
\begin{multipleChoice}  
\choice[correct]{$\frac{1}{3}\left({\bf r}_Q - {\bf r}_P\right)$}  
\choice{$\frac{1}{3}\left({\bf r}_P - {\bf r}_Q\right)$}  
\choice{$0.5({\bf r}_P + {\bf r}_Q)$}  
\end{multipleChoice}  


(c) True or False? 
The magnitude $| {\bf v}_{avg} |$ of the average velocity vector measures the average speed of the projectile between times $u=-1.4$ to time $u=1.6$ seconds.
\begin{multipleChoice}  
\choice[correct]{False}  
\choice{True}  
\end{multipleChoice}  




(d) Slide $\Delta t$ toward zero and watch how the average velocity vector changes. What can you say about the direction of this vector as $\Delta t \to 0$? What does this tell you about the direction of the instanstaneous velocity vector? 

(e) In light of part (c), is it true that the magnitude of the instantaneous velocity vector equals the speed of the motion? Explain your reasoning.

(f) Move point $P$ with the slider $u$ and repeat part (c). How do the speeds of the motion compare at times $u=-1.4$ and your other time? 
\end{question}


\pdfOnly{
Access Geogebra interactives through the online version of this text at
 
\href{https://www.geogebra.org/classic/egzphw3q}.
}
 
\begin{onlineOnly}
    \begin{center}
\geogebra{p2y4j2ap}{900}{600}
\end{center}
\end{onlineOnly}


Access Geogebra interactives through the online version of this text at
 
\href{https://www.geogebra.org/classic/egzphw3q}{Projectile Motion}.

\end{exploration}



\begin{exploration}
\begin{onlineOnly}
    \begin{center}
\geogebra{jdsf9ttr}{900}{600}
\end{center}
\end{onlineOnly}


Access Geogebra interactives through the online version of this text at

 
\href{https://www.geogebra.org/classic/jdsf9ttr}{Hodograph 1}.

\end{exploration}




\begin{exploration}
\pdfOnly{
Access Geogebra interactives through the online version of this text at
 
\href{https://www.geogebra.org/classic/c7gttegp}.
}
 
\begin{onlineOnly}
    \begin{center}
\geogebra{c7gttegp}{900}{600}
\end{center}
\end{onlineOnly}
\end{exploration}



\section{Curves and Velocity Questions}


\begin{question}  \label{Qvhsdsdgt:Motion}
Let $r,k_1, k_2\in\mathbb{R}$ be positive constants and consider the cylindrical helix
\[
  {\bf p}(t) =  \langle a \cos (k_1 \phi) , r \sin (k_1 \phi) , k_2 \phi \rangle \, , \, \phi \in \mathbb{R}, 
\]
where the components of the vector ${\bf p}(t)$ (giving the position of a point on the helix relative to the origin) are measured in meters.

(a) What are the units of $a$, $k_1$, and $k_2$? Explain your reasoning.

(b) Graph the helix in desmos with $a$, $k_1$, $k_2$ as sliders. Experiment with the sliders and describe now each parameter affects the geometry of the helix. See the link to a desmos worksheet below.

(c) Parameterize the tangent line to the helix at the point corresponding to the parameter $\phi = \phi_0$.

(d) Find the coordinates of the point where the tangent line in part (c) intersects the $yz$-plane. 

\pskip

Here is a linke to a desmos worksheet for you to check your work. Please include this as part of your solution.

\href{https://www.desmos.com/3d/9b85b075ac}{163: Cylindrical Helix}

\begin{itemize}
\item{Input your parameterization of the tangent line on Line 12.}

\item{Input the coordinates of the point where the tangent line intersects the $yz$-plane in Line 13.}

\item{Move the slider $\phi_0$ to make sure your tangent line and $yz$-intercept are correct.}

\end{itemize}

\end{question}


\begin{question}  \label{Qvhyu55:Motion}
Let $r,k\in\mathbb{R}$ be positive constants. 

(a) Show that all the tangent lines to the cylindrical helix
\[
  {\bf p}(t) =  \langle r \cos (\theta) , r \sin (\theta) , k \theta \rangle \, , \, \theta \in \mathbb{R}, 
\]
make the same angle with the $z$-axis. Find this angle both with and without calculus. 

(b) Express the angle $\phi$ the tangent lines make with the $xy$-plane in terms of $r$ and $k$.

(c) Express the $z$-coordinate of the parameterization above in terms of $r$, $\phi$, and $\theta$.
\end{question}


\begin{question}  \label{Qdgtbyt5:Motion}
Between times $t=0$ and $t=5$ seconds past noon, a gnat flies in a smooth path. The function 
\[
       {\bf r}(t) =  \langle  \frac{18 }{1+t} , 9\ln (1+4t) , 6\arctan(t/2) \rangle \, , \, 0\leq t \leq 2 ,
\]
expresses the position (in meters) of the gnat relative to the origin for the first two seconds of its motion. Between times $t=2$ and $t=5$ seconds past noon, the gnat flies with a constant velocity. Find a multi-part function that gives the gnat's position (in meters) relative to the origin between times $t=0$ and $t=5$. 

\end{question}


\begin{question} \label{Qcbh65y6:Motion}
At a certain instant, a gnat has position
\[
   {\bf r}_0 = \langle -2, 5, 4\rangle \text{ meters}
\]
relative to the origin and is moving with velocity
\[
  {\bf v}_0 = \langle  2, 1,-1  \rangle \text{ m/sec} .
\]

Is the distance between the gnat and the origin increasing or decreasing at this instant? At what rate? Justify your answer.

\begin{exploration} \label{Edsss4tDE}
Drag the slider $u$. When is the distance between points $A$ and $C$ increasing? Decreasing? How can you tell from the vectors ${\bf r}=\overrightarrow{AC}$ and ${\bf v}$ (the velocity of $A$)?

Access Desmos interactive at
 
\href{https://www.geogebra.org/classic/x4nun5yb}{163: Distance to Origin}

 
\begin{onlineOnly}
    \begin{center}
\geogebra{x4nun5yb}{900}{600}
\end{center}
\end{onlineOnly}


\end{exploration}

https://www.geogebra.org/classic/x4nun5yb


\end{question}

\section{Bicycle Tracks}


\begin{exploration}

The animation below shows the motion
\[
   {\bf p}_b(t) = \langle a\cos (\omega t), b \sin (\omega t)  \rangle \, , \, 0\leq t \leq 2\pi 
\]
of the rear wheel of a bicycle around an ellipse, where the components are measured in meters and time $t$ in seconds.
Here $a$, $b$, and $\omega$ are positive constants.

(a) What are the units of $a$, $b$, and $\omega$? How do you know?

(b) What is the meaning of $\omega$?

(c) Suppose the bicyle's frame has length $L$ meters ($L$ is the distance between the front and back axles.) Parameterize the motion of the front wheel. Work in general, not with the specific motion given above. That is, express the position ${\bf p}_f(t)$
of the front wheel (really the point where the wheel touches the ground) at time $t$ in terms of the vector ${\bf p}_b(t)$ and $L$.

(d) Compare the speeds of the two wheels as the bicycle moves along its path.

(e) More questions to follow.

\begin{onlineOnly}
    \begin{center}
\desmos{cuo1lynesx}{900}{600}
\end{center}
\end{onlineOnly}


Access Desmos interactives through the online version of this desmos activity at
 
\href{https://www.desmos.com/calculator/cuo1lynesx}{163: Bicycle Tracks 1}.


\end{exploration}


\section{Hodograph to Motion}

\begin{exploration}
\begin{onlineOnly}
    \begin{center}
\desmos{ljnsc6jsvt}{900}{600}
\end{center}
\end{onlineOnly}


Access Desmos interactives through the online version of this text at

 
\href{https://www.desmos.com/calculator/ljnsc6jsvt}{163: Hodograph to Motion}.

\end{exploration}



\section{Ball on an Inclined Plane}

\begin{exploration}
\begin{onlineOnly}
    \begin{center}
\geogebra{rcmapxuh}{900}{600}
\end{center}
\end{onlineOnly}


Access Geogebra interactives through the online version of this text at

 
\href{https://www.geogebra.org/classic/rcmapxuh}{163: Ball on a Plane}.

\end{exploration}





\end{document}
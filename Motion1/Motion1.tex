\documentclass{ximera}
\title{Motion, Part 1}

\newcommand{\pskip}{\vskip 0.1 in}

\begin{document}
\begin{abstract}
Motion
\end{abstract}
\maketitle


\begin{exploration}
Suppose  you're driving your car along a winding road. The instantaneous velocity of your car at some moment is the rate of change in its position with repsect to time. Since position is a vector, so is velocity. It has a direction and magnitude. The direction points in the direction you are moving  and the magnitude gives your speed.

But just like all other rates of change, we can speak about average velocity. And we really should speak about this first. Your average velocity over a time interval is the change in your position divided by the length of the interval. This is also a vector.

The geogebra worksheet below suggest a projectile motion in a uniform graviational field. The position of the projectile is plotted at equally-spaced time intervals along its trajectory. The time $u$ is measured in seconds and the position vectors are measured in meters.

The vector ${\bf v}_{avg}$ represents the average velocity between times $u=-1.4$ and $u=-1.4+\Delta t = 1.6$. 

\begin{question}  \label{Qfdhgyuj:Motion}
(a) Which vector gives the change in position going forward in time from time $u=-1.4$ to time $u=1.6$ seconds. 
\begin{multipleChoice}  
\choice[correct]{${\bf r}_Q - {\bf r}_P$}  
\choice{${\bf r}_P - {\bf r}_Q$}  
\choice{$0.5({\bf r}_P + {\bf r}_Q)$}  
\end{multipleChoice}  


(b) (a) Which vector gives the average velocity ${\bf v}_{avg}$ between times $u=-1.4$ to time $u=1.6$ seconds. 
\begin{multipleChoice}  
\choice[correct]{$\frac{1}{3}\left({\bf r}_Q - {\bf r}_P\right)$}  
\choice{$\frac{1}{3}\left({\bf r}_P - {\bf r}_Q\right)$}  
\choice{$0.5({\bf r}_P + {\bf r}_Q)$}  
\end{multipleChoice}  

(c) Slide $\Delta T$ toward zero and watch how the average velocity vector between times $u=-1.4$ and $t=-1.4+ \Delta t$ changes. How does the direction of the average velocity vector change? What does this tell you about the direction of the instanstaneous velocity vector? 

(d) Keeping the animation of part (c) in mind, try to explain why the magnitude of the instantaneous velocity vector gives the speed of the motion. There is something to think about here.

(e) Move point $P$ with the slider $u$ and repeat part (c). How do the speeds of the motion compare at times $u=-1.4$ and your other time?
\end{question}


\pdfOnly{
Access Geogebra interactives through the online version of this text at
 
\href{https://www.geogebra.org/classic/egzphw3q}.
}
 
\begin{onlineOnly}
    \begin{center}
\geogebra{p2y4j2ap}{900}{600}
\end{center}
\end{onlineOnly}


Access Geogebra interactives through the online version of this text at
 
\href{https://www.geogebra.org/classic/egzphw3q}.

\end{exploration}


\begin{exploration}
\pdfOnly{
Access Geogebra interactives through the online version of this text at
 
\href{https://www.geogebra.org/classic/c7gttegp}.
}
 
\begin{onlineOnly}
    \begin{center}
\geogebra{c7gttegp}{900}{600}
\end{center}
\end{onlineOnly}
\end{exploration}



\section{Bicycle Tracks}


\begin{exploration}

 
\begin{onlineOnly}
    \begin{center}
\desmos{cuo1lynesx}{900}{600}
\end{center}
\end{onlineOnly}


Access Desmos interactives through the online version of this desmos activity at
 
\href{https://www.desmos.com/calculator/cuo1lynesx}{163: Bicycle Tracks 1}.


\end{exploration}


\end{document}
\documentclass{ximera}
\title{Motion, Part 1}

\newcommand{\pskip}{\vskip 0.1 in}

\begin{document}
\begin{abstract}
Parameterizing motion in the plane and in space. 
\end{abstract}
\maketitle


\section{The Velocity Vector}

\begin{exploration}
\emph{Velocity} is the rate of change of position with respect to time. Since position is a vector (position is measured relative to some point, usually the origin), so is velocity. 

Just like any other rate of change, we can speak about average velocity or instantaneous velocity. These are both vectors. Average velocity is a change in position (a vector) divided by the time interval (a scalar) over which that change occurs. Instantaneous velocity is a limit of average velocities as the time interval approaches zero. It is the derivative of position with respect to time.

For example, the instantaneous velocity of a car at some moment is a vector that points in the direction you're driving. The magnitude of this vector gives the speed. Both the vector and its magnitude have units, for a car typically miles/hour or kilometers/hour. But usually in physics velocity has units of meters/sec.
 
%Suppose  you're driving your car along a winding road. The instantaneous velocity of your car at some moment is the rate of change in its position with repsect to time. Since position is a vector, so is velocity. It has a direction and magnitude. The direction points in the direction you are moving  and the magnitude gives your speed.

%But just like all other rates of change, we can speak about average velocity. And we really should speak about this first. Your average velocity over a time interval is the change in your position divided by the length of the interval. This is also a vector.



The geogebra worksheet below suggest a projectile motion in a uniform graviational field. The position of the projectile is plotted at equally-spaced time intervals along its trajectory. The time $u$ is measured in seconds and the position vectors are measured in meters.

The vector ${\bf v}_{avg}$ represents the average velocity over a time interval of length $\Delta t=3$ seconds that starts at time  $u=-1.4$ seconds. 

\begin{question}  \label{Qfdhgyuj:Motion}

\begin{enumerate}
\item Which vector gives the change in position going forward in time from time $u=-1.4$ to time $u=1.6$ seconds. 
\begin{multipleChoice}  
\choice[correct]{${\bf r}_Q - {\bf r}_P$}  
\choice{${\bf r}_P - {\bf r}_Q$}  
\choice{$0.5({\bf r}_P + {\bf r}_Q)$}  
\end{multipleChoice}  


\item Which vector gives the average velocity ${\bf v}_{avg}$ between times $u=-1.4$ to time $u=1.6$ seconds. 
\begin{multipleChoice}  
\choice[correct]{$\frac{1}{3}\left({\bf r}_Q - {\bf r}_P\right)$}  
\choice{$\frac{1}{3}\left({\bf r}_P - {\bf r}_Q\right)$}  
\choice{$0.5({\bf r}_P + {\bf r}_Q)$}  
\end{multipleChoice}  


\item True or False? 
The magnitude $| {\bf v}_{avg} |$ of the average velocity vector measures the average speed of the projectile between times $u=-1.4$ to time $u=1.6$ seconds.
\begin{multipleChoice}  
\choice[correct]{False}  
\choice{True}  
\end{multipleChoice}  

\begin{freeResponse}
Which is greater, the average speed between times $u=-1.4$ and $u=1.6$ seconds or the magnitude $| {\bf v}_{avg} |$ of the average velocity over the same time interval? Explain your reasoning. %Explain the logic behind the answer to the previous True-False question.
\end{freeResponse}


\item Slide $\Delta t$ toward zero and watch how the average velocity vector changes. 
\begin{freeResponse}
What can you say about the direction of this vector as $\Delta t \to 0$? What does this tell you about the direction of the instanstaneous velocity vector? 
\end{freeResponse}

%\item We said that the magnitude of the instantaneous velocity vector gives the speed of the motion. 
%\begin{freeResponse}
%Try to reconcile this with part (c).
%\end{freeResponse}

%In light of part (c), is it true that the magnitude of the instantaneous velocity vector equals the speed of the motion? Explain your reasoning.

\item Compare the speeds of the motion at times $u=1.4$ seconds and $u=0$ seconds. Be as specific as possible.
\begin{freeResponse}
Briefly explain your reasoning.
\end{freeResponse}

%Move point $P$ with the slider $u$ and repeat part (c). How do the speeds of the motion compare at times $u=-1.4$ and your other time? 

\end{enumerate}
\end{question}


\pdfOnly{
Access Geogebra interactives through the online version of this text at
 
\href{https://www.geogebra.org/classic/egzphw3q}.
}
 
\begin{onlineOnly}
    \begin{center}
\geogebra{p2y4j2ap}{900}{600}
\end{center}
\end{onlineOnly}


\section{More Velocity Vectors}
\begin{question}  \label{Qd34trgetgyt}
The function
\[
 {\bf p}(t) = \langle t, t^2, t^3 \rangle \, , \, 0\leq t \leq 2 .
\]
gives the position (measured in meters) of a fly relative to the origin in terms of the number of seconds past noon.  

\begin{enumerate}
\item Compute the fly's average velocity  between times $t=1$ and $t=1.5$ seconds.

\item Compute the fly's average velocity  between times $t=1$ and $t=1.1$ seconds.

\item Compute the fly's average velocity  between times $t=1$ and $t=1.01$ seconds.

\item What does all this suggest about the fly's (instantaneous) velocity at time $t=1$ second?

\item Compute and evaluate the appropriate derivative to find the fly's velocity at time $t=1$ second.


\end{enumerate}
\end{question}


\section{Graphing the Velocity Function}
One way to graph the velocity function, or the average velocity function computed over some constant time interval, is to pin the velocity vectors with their tails at a common point. The tips of these clustered velocity vectors (instantaneous or average) trace out a curve. The next two demonstrations show examples, the first for projectile motion in a uniform gravitational field, the second for motion along cycloids.

Access Geogebra interactives through the online version of this text at
 
\href{https://www.geogebra.org/classic/egzphw3q}{Projectile Motion}.

\end{exploration}



\begin{exploration}
\begin{onlineOnly}
    \begin{center}
\geogebra{jdsf9ttr}{900}{600}
\end{center}
\end{onlineOnly}


Access Geogebra interactives through the online version of this text at

 
\href{https://www.geogebra.org/classic/jdsf9ttr}{Hodograph 1}.

\end{exploration}




\begin{exploration}
\pdfOnly{
Access Geogebra interactives through the online version of this text at
 
\href{https://www.geogebra.org/classic/c7gttegp}.
}
 
\begin{onlineOnly}
    \begin{center}
\geogebra{c7gttegp}{900}{600}
\end{center}
\end{onlineOnly}
\end{exploration}


\section{From Velocity to Position}


\begin{question}  \label{Qdst65nvcy}
Given the velocity function ${\bf v}(t)$ of a motion, which of the following are also necessray to determine the position function? Choose all that apply.
\begin{selectAll}  
    \choice{The acceleration function}  
    \choice{The velocity at a particular time}  
    \choice[correct]{The position at a particular time}  
    \choice{The velocity function is sufficient. Nothing else is needed.}  
  \end{selectAll}
\end{question}


Backing up for a moment, let's review a key idea from Integral Calculus.
\begin{question}  \label{Qddggt5t6vcy}
Suppose the continuous function 
\[
    r=g(t) , 0\leq t \leq 6 ,
\]
expresses the rate (in gal/min) at which water flows into a tank in terms of the number of minutes past noon. Suppose also that the tank holds $20$ gallons at time $t=t_0$ minutes past noon, for some $t_0\in [0,6]$. Which of the following functions gives the volume of water (in gallons in the tank) at time $t$ minutes past noon? Choose all that apply. Explain your reasoning.
\begin{selectAll}  
    \choice{$f(t) = 20 + r(t)$}  
    \choice{$f(t) = \int r(t)\, dt + C$}  
    \choice{$f(t) =\int r(t)\, dt + C$}
    \choice{$f(t) =\int_{t_0}^t r(u)\, du$}  
    \choice[correct]{$f(t) =20 + \int_{t_0}^t r(u)\, du$}  
  \end{selectAll}
\end{question}

One way to approximte the volume function is to use a step function to approximate the rate function. That is, we partition the interval $t\in [0,6]$ into subintervals, usually of equal lengths, and assume that the rate of flow is constant over each interval. The resulting volume function is then piecewise linear. 

Experiment with the Desmos demonstration below to see this.

Access Desmos interactive at
 
\href{https://https://www.desmos.com/calculator/zswxfftnnk}{163: Riemann Sums and Integration}

 
\begin{onlineOnly}
    \begin{center}
\desmos{zswxfftnnk}{900}{600}
\end{center}
\end{onlineOnly}




\begin{question}    \label{Qedtrg78}
We can apply these same ideas to find the position function of a motion given its (continuous) velocity function ${\bf v}(t)$ (as a function of the number of seconds past noon and measured in say meters/sec) and the position ${\bf p}_0$ (in meters) at time $t=t_0$ seconds past noon.  

Which of the following functions gives the position (in meters) at time $t$ seconds past noon? Choose all that apply.
\begin{selectAll}  
      \choice{${\bf p}(t) = \int {\bf v}(t)\, dt + C$}  
       \choice{${\bf p}(t) = \int_{t_0}^t r(u)\, du$} 
    \choice{${\bf p}(t) = {\bf p}_0 + \int_{t_0}^u {\bf v}(t)\, dt$}  
    \choice[correct]{${\bf p}(t) = {\bf p}_0 + \int_{t_0}^t {\bf v}(u)\, du$}  
  \end{selectAll}
\end{question}


\begin{question}  \label{Qdeggbfr4}
Suppose now that 
\[
    {\bf v}(t) = \langle \cos\left( \frac{t^2}{2\pi} \right) , \sin\left( \frac{t^2}{2\pi} \right) \rangle , t\geq 0 ,
\]
and that 
\[
{\bf p}(0) = \langle 3, -5 \rangle .
\]

(a) Find an expression for the position function.

(b) Change the first slider to $n=4$ in the demonstration below. Then drag slider $u$ to see the velocity hodograph. Use the four vectors in the hodograph to sketch a piece-wise linear approximation to the trajectory of the motion. Explain your reasoning.

(c) Drag slider $u$ back to $u=0$. Then click the ``segments'' box in the upper left. Then drag the slider $u$. Explain what you see. How does the path compare with your approximation in part (b)?

(d) Drag slider $u$ back to $u=0$. Increase the value of $n$ and drag slider $u$. Explain what you see. What happens as you increase $n$? 

\begin{exploration}

Access Geogebra interactives through the online version of this desmos activity at

\href{https://www.geogebra.org/classic/pdefd7mb}{163:Hodograph to Motion}

\begin{onlineOnly}
    \begin{center}
\geogebra{pdefd7mb}{900}{600}
\end{center}
\end{onlineOnly}

\end{exploration}

\end{question}


\section{From Velocity to Position and Acceleration}
\begin{question} \label{Qder44t4tr4tr4}
The function
\[
  {\bf v} = \langle \cos \left( \frac{t^2}{2\pi}  \right) ,  \sin \left( \frac{t^2}{2\pi} , 0 \right)\rangle \, , \, 0\leq t \leq 2\pi ,
\]
expresses the velocity of a bug (in m/sec) in terms of time, measured in seconds. The bug has position
\[
   {\bf p}_0 = \langle 3, -1, 0  \rangle ,
\]
measured relative to the origin in meters at time $t_0 = 2$ seconds.

\begin{itemize}
\item Try to sketch the path of the motion. Sketch velocity and acceleration vectors along the path at equally-spaced time intervals.

\item Find an expression for the position function ${\bf p}(t)$.

\item Find an expression for the acceleration function ${\bf a}(t)$.

\item Play the slider in the animation below to check how you did.

\item What do you notice about the velocity and acceleration vectors. What does the hodograph tell you?

\begin{onlineOnly}
    \begin{center}
\desmosThreeD{xhuiyzkueq}{900}{600}
\end{center}
\end{onlineOnly}


\end{itemize}
\href{https://www.desmos.com/3d/xhuiyzkueq}{163: Motion From Velocity to Position and Acceleration}

\end{question}





\section{Projectile Motion}
\begin{question}  \label{Qerwer}
Suppose a projectile is lauched at time $t=0$ seconds with an inital velocity ${\bf v}_0$ (measured in m/s) from position ${\bf p}_0$ (relative to the origin) in the $xy$-plane in a uniform gravitational field. Suppose the field has acceleration vector ${\bf g}$ (measured in (m/s)/s).

\begin{enumerate}

\item Use a definite integral to find a function ${\bf v}(t)$ that expresses the velocity of the projectile in terms of the number of seconds since it was launched. Evaluate the integral to simplify the expression.

\item Use a definite integral to find a function ${\bf p}(t)$ that expresses the position of the projectile (relative to the origin) in terms of the number of seconds since it was launched. Evaluate the integral to simplify the expression.

\item Find an expression in terms of the vectors ${\bf v_0}$, ${\bf g}$, and ${\bf k}=\langle 0,0,1 \rangle$ for the time when the projectile hits the $xy$-plane.

\item Follow the directions in the worksheet below to check your work.

\begin{onlineOnly}
    \begin{center}
\desmosThreeD{csovv2cwn4}{900}{600}
\end{center}
\end{onlineOnly}

\href{https://www.desmos.com/3d/csovv2cwn4}{Projectile 1}

\end{enumerate} 

\end{question}


\section{Conservation of Mechanical Energy}
\begin{question} \label{Q4re4r4rfefe}
Show the mechanical energy of a projectile moving only under the influence of gravity in a uniform gravitational field is constant. The mechanical energy
\[
   E = \frac{1}{2}mv^2 + mgh
\]
is the sum of the kinetic and potential energies. Here $m$ is the projectile's mass (in kg), $v$ is its speed (in m/s) as a function of time, $g$ is the magnitude of the gravitaional acceleration (in (m/s)/s), and $h$ is the projectile's height (in meters) as a function ot time. 

Do this by showing that the derivative $dE/dt$ vanishes, where $t$ is time, mesaured in seconds. But first write the mechanical energy in terms of the velocity vector ${\bf v}$, the acceleration vector ${\bf g}$, the position vector ${\bf p}$ (relative to some point) and the vector ${\bf k}$ that points ``upward" (in the direction opposite ${\bf g}$).
\end{question}



\section{Ball on an Inclined Plane}

\begin{question}  \label{Qefd46365}
A small mass slides on an inclined plane without friction in a uniform graviational field with gravitational acceleration $g \;  m/sec^2$. The mass passes the point $P_0$ with coordinates $(p_1, p_2, p_3 )$ in the plane at time $t=t_0$ seconds with velocity ${\bf v}_0 = \langle v_1, v_2, v_3 \rangle$m/sec.  The vector ${\bf n} =\langle n_1, n_2, n_3\rangle$ is normal to the plane.

Find a function ${\bf p}(t)$ that expresses the position of the mass relative to the origin in terms of time, measured in seconds. Work with the vectors ${\bf p}_0$ from the origin to $P_0$, ${\bf g} = \langle 0, 0, -g \rangle$, ${\bf v}_0$, and ${\bf n}$. Avoid working with their components.

{\it Hint:}  The mass slides with constant acceleration equal to the vector component of ${\bf g}$ parallel to the plane.




\begin{exploration}
\begin{onlineOnly}
    \begin{center}
\geogebra{rcmapxuh}{900}{600}
\end{center}
\end{onlineOnly}


Access Geogebra interactives through the online version of this text at

 
\href{https://www.geogebra.org/classic/rcmapxuh}{163: Ball on a Plane}.

Access Desmos interactives through the online version of this desmos activity at

\href{https://www.desmos.com/3d/07057c1762}{163: Ball on Plane}


\end{exploration}

\end{question}


\section{Problems}


\begin{question}  \label{Qvhsdsdgt:Motion}
Let $r,k_1, k_2\in\mathbb{R}$ be positive constants and consider the cylindrical helix
\[
  {\bf p}(\phi) =  \langle a \cos (k_1 \phi) , a \sin (k_1 \phi) , k_2 \phi \rangle \, , \, \phi \in \mathbb{R}, 
\]
where the components of the vector ${\bf p}(t)$ (giving the position of a point on the helix relative to the origin) are measured in meters.

(a) What are the units of $a$, $k_1$, and $k_2$? Explain your reasoning.

(b) Graph the helix in desmos with $a$, $k_1$, $k_2$ as sliders. Experiment with the sliders and describe now each parameter affects the geometry of the helix. See the link to a desmos worksheet below.

(c) Parameterize the tangent line to the helix at the point corresponding to the parameter $\phi = \phi_0$.

(d) Find the coordinates of the point where the tangent line in part (c) intersects the $yz$-plane. 

\pskip

Here is a linke to a desmos worksheet for you to check your work. Please include this as part of your solution.

\href{https://www.desmos.com/3d/9b85b075ac}{163: Cylindrical Helix}

\begin{itemize}
\item{Input your parameterization of the tangent line on Line 12.}

\item{Input the coordinates of the point where the tangent line intersects the $yz$-plane in Line 13.}

\item{Move the slider $\phi_0$ to make sure your tangent line and $yz$-intercept are correct.}

\end{itemize}

\end{question}


\begin{question}  \label{Qvhyu55:Motion}
Let $r,k\in\mathbb{R}$ be positive constants. 

(a) Show that all the tangent lines to the cylindrical helix
\[
  {\bf p}(\theta) =  \langle r \cos (\theta) , r \sin (\theta) , k \theta \rangle \, , \, \theta \in \mathbb{R}, 
\]
make the same angle with the $z$-axis. Find this angle both with and without calculus. 

(b) Express the angle $\phi$ the tangent lines make with the $xy$-plane in terms of $r$ and $k$.

(c) Express the $z$-coordinate of the parameterization above in terms of $r$, $\phi$, and $\theta$.
\end{question}


\begin{question}  \label{Qdgtbyt5:Motion}
Between times $t=0$ and $t=5$ seconds past noon, a gnat flies in a smooth path. The function 
\[
       {\bf r}(t) =  \langle  \frac{18 }{1+t} , 9\ln (1+4t) , 6\arctan(t/2) \rangle \, , \, 0\leq t \leq 2 ,
\]
expresses the position (in meters) of the gnat relative to the origin for the first two seconds of its motion. Between times $t=2$ and $t=5$ seconds past noon, the gnat flies with a constant velocity. Find a multi-part function that gives the gnat's position (in meters) relative to the origin between times $t=0$ and $t=5$. 

\end{question}


\begin{question} \label{Qcbh65y6:Motion}
At a certain instant, a gnat has position
\[
   {\bf r}_0 = \langle -2, 5, 4\rangle \text{ meters}
\]
relative to the origin and is moving with velocity
\[
  {\bf v}_0 = \langle  2, 1,-1  \rangle \text{ m/sec} .
\]

(a) Is the distance between the gnat and the origin increasing or decreasing at this instant? At what rate? Justify your answer.

(b) Is the distance between the gnat and the point $(1,2,1)$ increasing or decreasing at this instant? At what rate? 

(c) At this same instant, the distance from the gnat to a point $P$ is neither increasing or decreasing. Find possible coordinates of $P$.

(d) Estimate the distance between the gnat and the origin $0.01$ seconds before the instant in question.


\begin{exploration} \label{Edsss4tDE}
Drag the slider $u$. When is the distance between points $A$ and $C$ increasing? Decreasing? How can you tell from the vectors ${\bf r}=\overrightarrow{AC}$ and ${\bf v}$ (the velocity of $A$)?

Access Desmos interactive at
 
\href{https://www.geogebra.org/classic/x4nun5yb}{163: Distance to Origin}

 
\begin{onlineOnly}
    \begin{center}
\geogebra{x4nun5yb}{900}{600}
\end{center}
\end{onlineOnly}


\end{exploration}

https://www.geogebra.org/classic/x4nun5yb


\end{question}

\section{The Escape Lemma}





\section{Bicycle Tracks}


\begin{exploration}

The animation below shows the motion
\[
   {\bf p}_b(t) = \langle a\cos (\omega t), b \sin (\omega t)  \rangle \, , \, 0\leq t \leq 2\pi 
\]
of the rear wheel of a bicycle around an ellipse, where the components are measured in meters and time $t$ in seconds.
Here $a$, $b$, and $\omega$ are positive constants.

(a) What are the units of $a$, $b$, and $\omega$? How do you know?

(b) What is the meaning of $\omega$?

(c) Suppose the bicyle's frame has length $L$ meters ($L$ is the distance between the front and back axles.) Parameterize the motion of the front wheel. Work in general, not with the specific motion given above. That is, express the position ${\bf p}_f(t)$
of the front wheel (really the point where the wheel touches the ground) at time $t$ in terms of the vector ${\bf p}_b(t)$ and $L$.

(d) Compare the speeds of the two wheels as the bicycle moves along its path.

(e) More questions to follow.

\begin{onlineOnly}
    \begin{center}
\desmos{cuo1lynesx}{900}{600}
\end{center}
\end{onlineOnly}


Access Desmos interactives through the online version of this desmos activity at
 
\href{https://www.desmos.com/calculator/cuo1lynesx}{163: Bicycle Tracks 1}.


\end{exploration}




\section{The Hodograph}

\begin{exploration}   \label{E67ugfdgtrt}
\begin{onlineOnly}
    \begin{center}
\desmos{cuo1lynesx}{900}{600}
\end{center}
\end{onlineOnly}



\end{exploration}
 


\begin{exploration}

Access Geogebra interactives through the online version of this desmos activity at

\href{https://www.geogebra.org/classic/pdefd7mb}{163:Hodograph to Motion}

\begin{onlineOnly}
    \begin{center}
\geogebra{pdefd7mb}{900}{600}
\end{center}
\end{onlineOnly}



Access Geogebra interactives through the online version of this desmos activity at

\href{https://www.geogebra.org/classic/z4r5quj4}{163:Hodograph to Motion Cycloid}
 
\begin{onlineOnly}
    \begin{center}
\geogebra{z4r5quj4}{900}{600}
\end{center}
\end{onlineOnly}




\end{exploration}








\begin{exploration}
\begin{onlineOnly}
    \begin{center}
\desmos{ljnsc6jsvt}{900}{600}
\end{center}
\end{onlineOnly}


Access Desmos interactives through the online version of this text at

 
\href{https://www.desmos.com/calculator/ljnsc6jsvt}{163: Hodograph to Motion}.

\end{exploration}








\end{document}
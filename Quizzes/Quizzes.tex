\documentclass{ximera}
\title{Chapter Quizzes}


\newcommand{\pskip}{\vskip 0.1 in}

\begin{document}
\begin{abstract}
Quiz solutions.
\end{abstract}
\maketitle


\pskip

\section*{Quiz 1: Coordinates in Space}

\begin{question}  \label{Q:943r34r34}
A regulation limits the maximum dimension (length, width, or height) of a box taken onto a city bus to be at most $3.5$ feet. Can you find a way to carry a six-foot long fishing pole onto a bus? Explain.

\begin{explanation}
You can fit the pole in a box measuring $3.5'\times 3.5'\times 3.5'$ since the main diagonal of this box measures
\[
      \sqrt{3 (3.5)^2} \text{ ft} = \sqrt{36.75} \text{ ft} > 6 \text{ ft}.
\]
\end{explanation}


\end{question}


\begin{question}  \label{Q:9df33334}
Find an equation of the sphere centered at the point $A(3,-2,4)$ that passes through the point $B(1,1,-3)$. Explain the \emph{logic} behind the equation.

\begin{explanation}
By the Pythagorean theorem, the sphere has radius 
\[
       \text{dist}(A,B) = \sqrt{(3-1)^2 + (-2-1)^2 + (4 - (-3))^2} = \sqrt{62} .
\]

The sphere is the set of points exactly $\sqrt{62}$ units from its center $A(3,-2,4)$. So a point $P$ with coordinates $(x,y,z)$ lies on the sphere if and only if 
\[
  \text{dist} (A,P) = \sqrt{62} .
\]
Using the Pythagorean theorem again, the distance $AP$ is
\[
        \text{dist} (A,P) = \sqrt{(x-3)^2 + (y+2)^2 + (z-4)^2} .
\]
So the point $P(x,y,z)$ lies on the sphere if and only if 
\[
   \sqrt{(x-3)^2 + (y+2)^2 + (z-4)^2} = \sqrt{62}.
\]

That's an equation of the sphere centered at $A(3,-2,4)$ passing through the point $B(1,1,-3)$.

\end{explanation}

\end{question}

\begin{question}  \label{Q44435313}
Describe and sketch each of the following sets of points.

(a) $\{ (x,y) \, | \, y=x^2 \}$

(b) $\{ (x,y,z) \, | \, y=x^2 \}$

\begin{explanation}
(a) The curve is a parabola with vertex at the origin and symmetric about the $y$-axis. Sketch omitted.

(b) The surface is a parabolic cylinder (sketch omitted). The cylinder is composed of lines parallel to the $z$-axis through the point of the parabola
\[
         \{  (t,t^2,0) , t \in \mathbb{R}     \} .
\]
\end{explanation}

\end{question}


\begin{question}  \label{Qpo8873032}
Let ${\cal L}_1$ be the line through the point $(0,0,1)$ parallel to the $y$-axis. Let ${\cal L}_2$ be the line through the point $(0,0,-1)$ parallel to the $x$-axis.

(a) Try to visualize or sketch the set of points in $\mathbb{R}^3$ equidistant from ${\cal L}_1$ and ${\cal L_2}$.

(b) Find an equation of the set of points in $\mathbb{R}^3$ equidistant from ${\cal L}_1$ and ${\cal L_2}$. Explain your logic.

\begin{hint}
Start by finding an expression for the distance between the point with coordinates $(x,y,z)$ and the line ${\cal L}_1$.
\end{hint}

(c) Enter  your equation in the worksheet below. Do the points on the surface look like they satisfy the condition? Explain.

\begin{onlineOnly}
    \begin{center}
\desmosThreeD{pzksnds2pe}{900}{600}
\end{center}
\end{onlineOnly}

\href{https://www.desmos.com/3d/pzksnds2pe}{163: Quiz 1 Question 4}


\begin{explanation}

(a) I find it impossible to visualize such a surface. I can see that the lines
\[
      \{ (t,t,0) , t \in \mathbb{R}   \}
\]
and
\[
      \{ (t,-t,0) , t \in \mathbb{R}   \}
\]
lie on the surface, but that's about it.   %the minimum distance between $P$ and the points of ${\cal L}_1$. % 

(b) We'll start by finding an expression for the distance from the point $P$ with coordinates $(x,y,z)$ to the line ${\cal L}_1$. To do this we need to find the coordinates of the point on the line closest to $P$.

A point $Q$ on ${\cal L}_1$ has coordinates $(0,t,1)$ for some $t\in \mathbb{R}$. So the distance between $P(x,y,z)$ and $Q(0,t,1)$ is 
\[
       \text{dist}(P,Q) = \sqrt{x^2 + (y-t)^2 + (z-1)^2} .
\] 
This distance is minimized when we choose $t=y$. So the distance from $P$ to ${\cal L}_1$ is
\[
        \text{dist}(P,{\cal L}_1) = \sqrt{x^2 + (t-t)^2 + (z-1)^2} = \sqrt{x^2 + (z-1)^2} .
\]

Similarly, the distance from $P(x,y,z)$ to the line ${\cal L}_2$ is
\[
        \text{dist}(P,{\cal L}_2) = \sqrt{(t-t)^2 + y^2 + (z+1)^2} = \sqrt{y^2 + (z+1)^2} .
\]

So the point $P(x,y,z)$ lies on the set of points equidistant from the lines ${\cal L}_1$ and ${\cal L}_2$ if and only if
\[
      \sqrt{x^2 + (z-1)^2} = \sqrt{y^2 + (z+1)^2}  .
\] 

Simplifying, we get
\[
      z = \left( x^2 - y^2 \right)/4
\]
for an equation of this surface.
\end{explanation}



\end{question}

\begin{question}  \label{Qpdfdsfe032}
(a) Try to visualize or sketch the set of points in $\mathbb{R}^3$  twice as far from the plane $z=3$ as from the $y$-axis. 

(b) Find an equation of the set of points in part (a). Explain your logic.

(c) Enter  your equation in the worksheet below. Do the points on the surface look like they satisfy the condition? Explain.

\begin{onlineOnly}
    \begin{center}
\desmosThreeD{qvin80o05t}{900}{600}
\end{center}
\end{onlineOnly}

\href{https://www.desmos.com/3d/qvin80o05t}{163: Quiz 1 Question 5}


\begin{explanation}
(a) This surface is also hard for me to visualize. Because of symmetry, I can see that the surface is a cylinder made up of lines parallel to the $y$-axis. But maybe not much more than that.

(b) We need to find expressions for the distance from a point $P(x,y,z)$ to the $y$-axis and the plane $z=3$. 

Reasoning similar to the previous question, the distance from $P(x,y,z)$ to the $y$-axis (where a point has coordinates $(0,t,0)$ for some $t\in \mathbb{R})$ is
\[
         \sqrt{(x-0)^2 + (t-t)^2 + (z-0)^2} = \sqrt{x^2 + z^2} .
\] 

Now the point on the plane $z=3$ nearest the point $P(x,y,z)$ has coordinates $(x,y,3)$. So the distance from $P$ to the plane $z=3$ is
\[
   \sqrt{(x-x)^2 + (y-y)^2 + (z-3)^2} = \sqrt{(z-3)^2}  = |  z - 3 |.
\]

So the point $P(x,y,z)$ is twice as far from the plane $z=3$ as it is from the $y$-axis if and only if 
\[
     \text{dist}(P, \text{plane } z= 3) = 2\, \text{dist}(P, y\text{-axis}) 
\]
and an equation of the set of all such points is
\[
         |  z - 3 |    = 2  \sqrt{x^2 + z^2}.
\]
This turns out to be an elliptic cylinder (ie. cross-sections perpendicular to the $y$-axis are ellipses).
\end{explanation}


\end{question}

\begin{question}  \label{Q43532432423432}
Let ${\cal L}$ be the line through the origin and the point $(2,-1,1)$.

(a) Try to visualize or sketch the set of points in $\mathbb{R}^3$  twice as far from the origin as from the line ${\cal L}$.

(b) Find an equation of the set of points in part (a). Explain your logic.

(c) Enter  your equation in the worksheet below. Do the points on the surface look like they satisfy the condition? Explain.

\begin{onlineOnly}
    \begin{center}
\desmosThreeD{wzv6plrbe6}{900}{600}
\end{center}
\end{onlineOnly}

\href{https://www.desmos.com/3d/wzv6plrbe6}{163: Quiz 1 Question 6}
\end{question}





\end{document}

\documentclass{ximera}
\title{Intersecting Spheres, Part 1}

\newcommand{\pskip}{\vskip 0.1 in}

\begin{document}
\begin{abstract}
Constructing the center of the circle of intersection of two intersecting spheres..
\end{abstract}
\maketitle





\begin{question}   \label{ExsSpheres}
This problem is about two intersecting spheres and their circle of intersection. We'll use vectors to find the center of this circle.

The first sphere has center $A_1$ and radius $r_1$. The second has center $A_2$ and radius $r_2$.

\begin{enumerate}

\item We'll start by writing vector equations for each sphere. 

\begin{enumerate}

\item The key idea is that a sphere is the set of points in $\mathbb{R}^3$ a fixed distance from a given point.

So a point $P$ with coordinates $(x,y,z)$ lies on the sphere with center $A_1$ and radius $r_1$ if and only if
\[
  \left|  \overrightarrow{A_1P} \right| = \answer{r_1} .
\]
Now with $O$ as the origin, 
\[
    \overrightarrow{A_1P} = \overrightarrow{\answer{OP}} - \overrightarrow{\answer{OA_1}}
\]
and we can write the equation of the sphere as
\[
     \left|  \overrightarrow{OP} - \overrightarrow{OA_1}\right| = \answer{r_1} .
\]
But since $\overrightarrow{OP}$ has components
\[
    \overrightarrow{OP} = \langle \answer{x}, \answer{y}, \answer{z}  \rangle ,
\]
we can write the vector equation of the sphere as
\[
     \left| \langle \answer{x},\answer{y},\answer{z} \rangle - \overrightarrow{OA_1}  \right| = \answer{r_1} .
\]

\emph{Key Point:} Even though this is an equation of vectors, the sphere is still a set of \emph{points}. Namely, those points $(x,y,z)$ that satify the above equation and are therefore exactly $r_1$ units from $A_1$.

%\pskip

To write this equation in desmos, we think about points instead of vectors and replace the vector $\langle x, y ,z \rangle$ with the point $(x,y,z)$ and the vector $\overrightarrow{OA_1}$ with the point $A_1$.

Enter this equation in Line 4 of the worksheet below.

\begin{onlineOnly}
    \begin{center}
\desmosThreeD{3tvdaj6do2}{900}{600}
\end{center}
\end{onlineOnly}

\href{https://www.desmos.com/3d/3tvdaj6do2}{Intersecting Spheres}

\item Enter an equation for the sphere of radius $r_2$ centered at the point $A_2$ in Line 6 above.

\end{enumerate}

\item The next step is to imagine the circle of intersection and a random point on this circle.
\end{enumerate}


We'll let 
\[
    {\bf p} = \langle x, y ,z \rangle
\]
be the position of the point $P$ with coordinates $(x,y,z)$ relative to the origin $O$.

Now suppose the two spheres have equations
\[
      |{\bf p} - {\bf a}_1| = r_1
\]
 and
\[
     |{\bf p} - {\bf a}_2| = r_2,
\]
where the vectors 
\[
   {\bf a}_i = \overrightarrow{OA_i} \, , \, i=1,2,
\]
are the positions of the sphere's centers relative to the origin.

Assuming the spheres intersect (in a circle), our goal is to express the position vector
\[
      {\bf c} = \overrightarrow{OC}
\]
of the circle's center (relative to $O$) in terms of the vectors ${\bf a}_i$ and the radii $r_i$ of the spheres.



\href{https://www.desmos.com/3d/3tvdaj6do2}{Intersecting Spheres}

The idea is to imagine a point $Q$ (orangle above) on the circle of intersection and consider $\Delta A_1QA_2$. To see this triangle turn off the folder \emph{Spheres} in Line 6 above.

\begin{enumerate}

\item First use the law of cosines in $\Delta A_1QA_2$ to express $\cos (\angle QA_1A_2)$ in terms of the vectors ${\bf a}_i$ and the radii $r_i$.

\item Then find an expression for the vector $\overrightarrow{A_1C}$.

\item Finally, find an expression for the vector $\overrightarrow{OC}$.
\end{enumerate}

\end{question}

\end{document}
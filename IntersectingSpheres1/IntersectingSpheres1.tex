\documentclass{ximera}
\title{Intersecting Spheres, Part 1}

\newcommand{\pskip}{\vskip 0.1 in}

\begin{document}
\begin{abstract}
Constructing the center of the circle of intersection of two intersecting spheres..
\end{abstract}
\maketitle





\begin{question}   \label{ExsSpheres}
This problem is about two intersecting spheres and their circle of intersection. We'll use vectors to find the center of this circle.

The first sphere has center $A$ and radius $a$. The second has center $B$ and radius $b$.

\begin{enumerate}

\item We'll start by writing a vector equation for each sphere. 

\begin{enumerate}

\item The key idea is that a sphere is the set of points in $\mathbb{R}^3$ a given distance from a given point.

So a point $P$ with coordinates $(x,y,z)$ lies on the sphere with center $A$ and radius $a$ if and only if
\[
  \left|  \overrightarrow{AP} \right| = \answer{a} .
\]
Now with $O$ as the origin, 
\[
    \overrightarrow{AP} = \overrightarrow{\answer{OP}} - \overrightarrow{\answer{OA}}
\]
and we can write an equation of the sphere as
\[
     \left|  \overrightarrow{OP} - \overrightarrow{OA}\right| = \answer{a} .
\]
But since $\overrightarrow{OP}$ has components
\[
    \overrightarrow{OP} = \langle \answer{x}, \answer{y}, \answer{z}  \rangle ,
\]
we can write the vector equation of the sphere as
\[
     \left| \langle \answer{x},\answer{y},\answer{z} \rangle - \overrightarrow{OA}  \right| = \answer{a} .
\]

\emph{Key Point:} Even though this is an equation of vectors, the sphere is still a set of \emph{points}. Namely, those points $(x,y,z)$ that satify the above equation and are therefore exactly $a$ units from $A$.

%\pskip

To write this equation in desmos, we think about points instead of vectors and replace the vector $\langle x, y ,z \rangle$ with the point $(x,y,z)$ and the vector $\overrightarrow{OA}$ with the point $A$.

Enter this equation in Line 4 of the worksheet below.

\begin{onlineOnly}
    \begin{center}
\desmosThreeD{s0luszbdmc}{900}{600}
\end{center}
\end{onlineOnly}

\href{https://www.desmos.com/3d/s0luszbdmc}{Intersecting Spheres}

\item Enter an equation for the sphere of radius $b$ centered at the point $B$ in Line 6 above.

\end{enumerate}

\item The next step is to imagine the circle of intersection and a random point $Q$ on this circle.

\begin{enumerate}
\item Activate the folder \emph{Circle of Inersection} on LIne 7 to see this circle and the point $Q$.

\item To find the center $C$ of the circle of intersection, we'll work with triangle $\Delta AQB$. To see this triangle hide your spheres in Lines 4 and 6. Then activate the folder \emph{Triangle} on Line 13.

The key idea is this:

The line $AB$ through the centers of the spheres is perpendicular to the plane of the circle of intersection. This implies $\overline{QC}$ is perpendicular to $\overline{AB}$ for any point $Q$ on the circle of intersection. 

Our focus now will be on finding an expression for the vector $\overrightarrow{AC}$. Remember, a vector is determined by its length and direction.  We know the direction of $\overrightarrow{AC}$. We just need to find its length. Since $\Delta ACQ$ is a right triangle and we know the length of $\overline{AQ}$, if we knew $\angle QAB$, or rather just the cosine of that angle, that would be enough to find the length of $\overline{AC}$.

\item First use the law of cosines in $\Delta AQB$ to express $\cos (\angle QAB)$ in terms of the vectors $\overrightarrow{OA}$, $\overrightarrow{OB}$ and the radii $a$, $b$.

\item Second, find an expression for the vector $\overrightarrow{AC}$ from $A$ to the center of the circle of intersection in terms of the vectors $\overrightarrow{OA}$, $\overrightarrow{OB}$ and the radii $a$, $b$.

\item Third, find an expression for the vector $\overrightarrow{OC}$  in terms of the vectors $\overrightarrow{OA}$, $\overrightarrow{OB}$ and the radii $a$, $b$.

\item Finally, enter this expression in Line 22 of the worksheet. Remember to omit all the $O$'s since desmos works with points instead of vectors.

\end{enumerate}

\end{enumerate}


\end{question}

\end{document}
\documentclass{ximera}
\title{Geometric Sequences and Series}

\newcommand{\pskip}{\vskip 0.1 in}

\begin{document}
\begin{abstract}
Geometric sequences and series
\end{abstract}
\maketitle


\section{Geometric Series}
\begin{question} \label{Q9d98fbg}
Let
\[
     s_n = \sum_{i=0}^n 2^i .
\]

\begin{enumerate}
\item  Compute the values of $s_1$, $s_2$, $s_3$, and $s_4$.

\item Guess an expression for $s_n$.

\item Here's another way to see that
\[
     s_n = 2^{n+1} - 1.
\]

\begin{enumerate}
\item Simplify the expression
\[
         2^{i+1} - 2^i .
\]

\item Use the result of part (i) to find an expression for $s_n$.

\end{enumerate}

\item Does $\lim_{n \to \infty} s_n$ exist?
\end{enumerate}
\end{question}


\begin{question} \label{Qpferr3r3r}
Repeat Question 1 for the sums
\[
     s_n = \sum_{i=0}^n 3^i .
\]
\end{question}




\begin{question} \label{OrerR8vmZ}
Let
\[
    s_n = \sum_{i=1}^n \left( \frac{1}{2} \right)^i .
\]

\begin{enumerate}
\item Compute the values of $s_1$, $s_2$, $s_3$, and $s_4$. Leave as fractions, not decimals.

\item Do you see a pattern? Can you guess an expression for $s_n$?

\item Take a piece of paper. 

\begin{enumerate}
\item Cut it in half. Keep half and give the other half to your friend.

\item Cut what you have left in half. Keep half of that and give the other half to your friend. What fraction of the original do you have? Your friend?

\item Keep repeating this process. After $n$ cuts, what fraction of the original do you have? Your friend?

\end{enumerate}


\item Here's another way to prove that
\[
     \sum_{i=1}^n \left( \frac{1}{2} \right)^i = 1 - 2^{-n} .
\]

\begin{enumerate}
\item Simplify the expression
\[
    \left( \frac{1}{2} \right)^{i-1} - \left( \frac{1}{2} \right)^{i} .
\]

\item Use the result of part (i) to express

\item Use the result of (i) to find a simplified expression for
\[
    s_n = \sum_{i=1}^n \left( \frac{1}{2} \right)^i .
\]
 
\end{enumerate}


\item Evaluate $\lim_{n \to \infty}s_n$.

\item Evaluate the sum
\[
  \sum_{i=1}^\infty  \left( \frac{1}{2} \right)^i  = \lim_{n\to \infty} \sum_{i=1}^n  \left( \frac{1}{2} \right)^i
\]

\item Evaluate the sum
\[
  \sum_{i=0}^\infty  \left( \frac{1}{2} \right)^i  = \lim_{n\to \infty} \sum_{i=0}^n  \left( \frac{1}{2} \right)^i
\]

\item Evaluate the sum
\[
  \sum_{i=0}^\infty 12 \left( \frac{1}{2} \right)^i  = \lim_{n\to \infty} \sum_{i=0}^n 12  \left( \frac{1}{2} \right)^i
\]

\end{enumerate}
\end{question}


\begin{question} \label{Ore33R8vmZ}
Let
\[
    s_n = \sum_{i=1}^n \left( \frac{1}{3} \right)^i .
\]

\begin{enumerate}
\item Compute the values of $s_1$, $s_2$, $s_3$, and $s_4$. Leave as fractions, not decimals.

\item Do you see a pattern? Can you guess an expression for $s_n$?

\item Take a piece of paper. 

\begin{enumerate}
\item Cut it in half. Put one half aside. Keep the other half.

\item Now take $2/3$ of your half and give it to your friend. What fraction of the original do you have? Your friend?

\item Keep repeating this process, always giving $2/3$ of what you have left to your friend. After $n$ cuts, what fraction of the original do you have? Your friend? Keep in mind the $1/2$ fraction you set aside at the start.

\end{enumerate}



\item Here's another way to prove that
\[
     \sum_{i=1}^n \left( \frac{1}{3} \right)^i = \frac{1}{2}\left( 1 - 3^{-n}\right) .
\]

\begin{enumerate}
\item Simplify the expression
\[
    \left( \frac{1}{3} \right)^{i-1} - \left( \frac{1}{3} \right)^{i} .
\]

\item Use the result of (i) to find a simplified expression for
\[
    s_n = \sum_{i=1}^n \left( \frac{1}{3} \right)^i .
\]
 
\end{enumerate}

\item Evaluate $\lim_{n \to \infty}s_n$.

\item Evaluate the sum
\[
  \sum_{i=1}^\infty  \left( \frac{1}{3} \right)^i  = \lim_{n\to \infty} \sum_{i=1}^n  \left( \frac{1}{3} \right)^i
\]

\item Evaluate the sum
\[
  \sum_{i=0}^\infty  \left( \frac{1}{3} \right)^i  = \lim_{n\to \infty} \sum_{i=0}^n  \left( \frac{1}{3} \right)^i
\]

\item Evaluate the sum
\[
  \sum_{i=0}^\infty 5 \left( \frac{1}{3} \right)^i  = \lim_{n\to \infty} \sum_{i=0}^n 5  \left( \frac{1}{3} \right)^i
\]

\end{enumerate}
\end{question}


\section{A Taylor Series}

\begin{question} \label{QPD33Lrdfr}
Let's look at the function
\[
    f(x) = \ln (1+x) \, , \, x>-1,
\]
near $x=0$.

\begin{onlineOnly}
    \begin{center}
\desmos{7royyumxbz}{900}{600}
\end{center}
\end{onlineOnly}

\href{https://www.desmos.com/calculator/7royyumxbz}{163: Motion and Taylor Series 5}


\begin{enumerate}
\item Find expressions for the polynomial approximations from degree 1 to degree 4, or until you notice a pattern.

\item Compare the graphs of the approximations with the graph of the function in the worksheet above by dragging the sldier $n$ (the degree of the approximation) in Line 4. 
What do you notice?
\begin{freeResponse}
\end{freeResponse}

\item Use the worksheet above to find successive approximations to $\ln 2$ with Taylor polynomials $p_i(x)$, $i=1,...,9$.

\item Bound the error in the estimate for $\ln 2$ from $p_9(x)$.

\item Use the worksheet above find successive approximations to $\ln(0.5)$ with Taylor polynomials $p_i(x)$, $i=1,...,9$.

\item Use ideas about geometric series to bound the error in the estimate for $\ln (0.5)$ from $p_9(x)$.

\end{enumerate}
\end{question}

\end{document}
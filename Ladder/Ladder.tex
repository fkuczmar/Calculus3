\documentclass{ximera}
\title{Motion: The Sliding Ladder}

\newcommand{\pskip}{\vskip 0.1 in}

\newtheorem{theorem9}{Robinson's Arclength Theorem for Cycloids}
\newtheorem{theorem10}{An Arclength Theorem for Epi and Hypocycloids}

\begin{document}
\begin{abstract}
The sliding ladder problem.
\end{abstract}
\maketitle

\begin{question} \label{QodfREEVCe}

A ladder of length $L$ meters rests vertically in a precarious position against a wall. The top of the ladder then slides down the wall as the bottom end moves along a horizontal floor.

\begin{enumerate}

\item Sketch the path of the point $P$ shown below at rest in the reference frame of the sliding ladder.

\begin{onlineOnly}
    \begin{center}
\desmos{wsqq0znsnp}{900}{600}
\end{center}
\end{onlineOnly}

\href{https://www.desmos.com/calculator/wsqq0znsnp}{Sliding Ladder 2}

\item Play the slider $\phi$ in Line 1 to slide the ladder and activate the folder \emph{Path} in Line 2 to see the path traced by the point $P$.

\item Now we'll parameterize the path traced by the point $P$ located $p$ meters from the bottom end ($A$) of the ladder. We'll use the angle $\phi$ the ladder makes with the ground as the parameter. And 

 \begin{enumerate}
\item Express the position $\overline{OA}$ of the point $A$ relative to the origin in terms of $\phi$ and the length $L$ of the ladder.
\[
    \overrightarrow{OP} = \langle \answer{L \cos \phi} , \answer{0} \rangle 
\]

\item Express the position $\overrightarrow{AP}$ of $P$ relative to $A$ in terms of the distance $p = AP$ and the angle $\phi$.
\[
   \overrightarrow{OP} = \langle \answer{-p \cos \phi}, \answer{p \sin \phi} \rangle
\]

\item Finally, express the position ${\bf p} = \overrightarrow{OP}$ of $P$ relative to $O$ in terms of $L$, $p$, and the parameter $\phi$.
\[
     {\bf p} = \langle \answer{L\cos\phi -p \cos \phi}, \answer{p \sin \phi} \rangle \, , \, 0\leq \phi \leq \pi/2.
\]

\end{enumerate}

\item Now express the velocity of point $P$ in terms of $L$, $p$, $\phi$, and the rotation rate $\omega$ (measured in rad/sec) of the ladder. Do not assume the rotation rate is constant.

\item Write the velocity vector ${\bf v}$ as the sum of two vectors, one parallel to the ladder, the other perpendicular to the ladder.

\item Express the speed of the ladder in terms of  $L$, $p$, $\phi$, and $\omega$.

\item Assuming the ladder rotates at a constant rate, what point of the ladder is moving parallel to the ladder when the ladder makes the angle $\phi$ with the ground?

\item Assuming a constant rotation rate, what point of the ladder is moving the slowest when the ladder makes the angle $\phi$ with the ground? The fastest?

\begin{onlineOnly}
    \begin{center}
\desmos{m932trb8pd}{900}{600}
\end{center}
\end{onlineOnly}

\href{https://www.desmos.com/calculator/m932trb8pd}{Sliding Ladder}


\end{enumerate}


\end{question}


\end{document}
 

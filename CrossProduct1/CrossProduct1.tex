\documentclass{ximera}
\title{The Cross Product, Part 1}

\newcommand{\pskip}{\vskip 0.1 in}

\begin{document}
\begin{abstract}
An introduction to the cross product.
\end{abstract}
\maketitle

\begin{question}  \label{Q34325rhgt}
(a) How many vectors in $\mathbb{R}^3$ are orthogonal to both ${\bf a}= \langle 1, 3,2 \rangle$ and ${\bf b}=\langle -4,3,-1 \rangle$?
 \begin{multipleChoice}
\choice{none}
\choice{one}
\choice{two}
\choice[correct]{infinitely many} 
\end{multipleChoice}

(b) How many directions in $\mathbb{R}^3$ are orthogonal to both ${\bf a}= \langle 1, 3,2 \rangle$ and ${\bf b}=\langle -4,3,-1 \rangle$?
 \begin{multipleChoice}
\choice{none}
\choice{one}
\choice[correct]{two}
\choice{infinitely many} 
\end{multipleChoice}


We wish to find the components of a vector ${\bf v}=\langle x, y , z \rangle \in \mathbb{R}^3$ perpendicular to the vectors ${\bf a}=\langle a_1, a_2, a_3 \rangle, {\bf b}=\langle b_1, b_2, b_3 \rangle \in \mathbb{R}^3$.  Necessary and sufficient conditions are that
\[
 \begin{cases}
 {\bf v}\cdot {\bf a} = a_1 x + a_2 y +a_3 z  = \answer{0}  \\
 {\bf v}\cdot {\bf b} = b_1 x + b_2 y +b_3 z  = \answer{0} .
\end{cases}
\]

Eliminating $x$ from this system of equations leads to the \emph{equation}
\[
    (a_1 b_2 - a_2 b_1)y + (\answer{a_1 b_3 - a_3 b_1})z = \answer{0}  .
\]
And one simple solution to this equation is
\[
     y = a_1 b_3 - a_3 b_1  \text{ and } z = \answer{a_2 b_1 - a_1 b_2} .
\]
The solve for $x$ to get
\[
       (x, y, z) = (   \answer{a_2 b_1 - a_1 b_2}     ,    \answer{ a_1 b_3 - a_3 b_1}        ,    \answer{a_2 b_1 - a_1 b_2}    )
\]
as a solution to the system.

Another equally simple solution is
\[ 
         (x, y, z) = (   a_1 b_2 - a_2 b_1     ,     a_3 b_1 - a_1 b_3        ,   a_1 b_2 - a_2 b_1    ) .
\]

It's a minor miracle that of the infinitely many solutions to our original system, it's the two simplest ones above that are the most useful. By convention, the latter
\[ 
         (x, y, z) = (   a_1 b_2 - a_2 b_1     ,     a_3 b_1 - a_1 b_3        ,   a_1 b_2 - a_2 b_1    ) = {\bf a}\times {\bf b}
\]
is the cross-product ${\bf a}\times {\bf b}$.

The cross product ${\bf a}\times {\bf b}$ of the vectors ${\bf a},{\bf b}\in \mathbb{R}^3$ is defined by the following properties.

\begin{enumerate}
\item{The vector ${\bf a}\times {\bf b}$ is orthogonal to ${\bf a}$ and ${\bf b}$}.

\item{The vectors ${\bf a}$, ${\bf b}$,${\bf a}\times {\bf b}$, in that order, are positively-oriented.}

\item{$|{\bf a}\times {\bf b}| = |{\bf a}||{\bf b}| \sin\theta$, where $\theta$, $0\leq \theta \leq \pi$, is the angle between ${\bf a}$ and ${\bf b}$.}
\end{enumerate}

\begin{question} \label{Q98erg4323}
Let ${\bf a} = \langle 1, -1, 3 \rangle$ and ${\bf b}=\langle -2,2,1 \rangle$.

(a) Compute by hand the vector ${\bf c} = {\bf a}\times {\bf b}$.

(b) Use the arithmetic of vectors to verify properties (a), (c) above.

\begin{hint}
For property (c), first use the scalar product to compute $\cos\theta$.
\end{hint}

(c) Use the desmos worksheet below to verify property (b) visually by entering the components of the vector ${\bf c}$ in Line 6. Show a screenshot and explain your reasoning.

\begin{onlineOnly}
    \begin{center}
\desmosThreeD{jla3lweqrp}{900}{600}
\end{center}
\end{onlineOnly}

\href{https://www.desmos.com/3d/jla3lweqrp}{163: Cross Product 1}
\end{question}

%To better understand this, suppose ${\bf a}$ and ${\bf b}$ are parallel to the $xy$-plane. Then the $\answer{z}$-

\end{question}



\end{document}

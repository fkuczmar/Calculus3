\documentclass{ximera}
\title{The Cross Product, Part 1}

\newcommand{\pskip}{\vskip 0.1 in}

\begin{document}
\begin{abstract}
An introduction to the cross product.
\end{abstract}
\maketitle

\begin{question}  \label{Q34325rhgt}
(a) How many vectors in $\mathbb{R}^3$ are orthogonal to both ${\bf a}= \langle 1, 3,2 \rangle$ and ${\bf b}=\langle -4,3,-1 \rangle$?
 \begin{multipleChoice}
\choice{none}
\choice{one}
\choice{two}
\choice[correct]{infinitely many} 
\end{multipleChoice}

(b) How many directions in $\mathbb{R}^3$ are orthogonal to both ${\bf a}= \langle 1, 3,2 \rangle$ and ${\bf b}=\langle -4,3,-1 \rangle$?
 \begin{multipleChoice}
\choice{none}
\choice{one}
\choice[correct]{two}
\choice{infinitely many} 
\end{multipleChoice}



We wish to find the components of a vector ${\bf v}=\langle x, y , z \rangle \in \mathbb{R}^3$ perpendicular to the vectors ${\bf a}=\langle a_1, a_2, a_3 \rangle, {\bf b}=\langle b_1, b_2, b_3 \rangle \in \mathbb{R}^3$.  Necessary and sufficient conditions are that
\[
 \begin{cases}
 {\bf v}\cdot {\bf a} = a_1 x + a_2 y +a_3 z  = \answer{0}  \\
 {\bf v}\cdot {\bf b} = b_1 x + b_2 y +b_3 z  = \answer{0} .
\end{cases}
\]

Eliminating $x$ from this system of equations leads to the \emph{equation}
\[
    (a_1 b_2 - a_2 b_1)y + (\answer{a_1 b_3 - a_3 b_1})z = \answer{0}  .
\]
And one simple solution to this equation is
\[
     y = a_1 b_3 - a_3 b_1  \text{ and } z = \answer{a_2 b_1 - a_1 b_2} .
\]
The solve for $x$ to get
\[
       (x, y, z) = (   \answer{a_2 b_1 - a_1 b_2}     ,    \answer{ a_1 b_3 - a_3 b_1}        ,    \answer{a_2 b_1 - a_1 b_2}    )
\]
as a solution to the system.

Another equally simple solution is
\[ 
         (x, y, z) = (   a_1 b_2 - a_2 b_1     ,     a_3 b_1 - a_1 b_3        ,   a_1 b_2 - a_2 b_1    ) .
\]

It's a minor miracle that of the infinitely many solutions to our original system, it's the two simplest ones above that are the most useful. By convention, the latter
\[ 
         (x, y, z) = (   a_1 b_2 - a_2 b_1     ,     a_3 b_1 - a_1 b_3        ,   a_1 b_2 - a_2 b_1    ) = {\bf a}\times {\bf b}
\]
is the cross-product ${\bf a}\times {\bf b}$.

\end{question}

The cross product ${\bf a}\times {\bf b}$ of the vectors ${\bf a},{\bf b}\in \mathbb{R}^3$ is defined by the following properties.

\begin{enumerate}
\item{The vector ${\bf a}\times {\bf b}$ is orthogonal to ${\bf a}$ and ${\bf b}$}.

\item{The vectors ${\bf a}$, ${\bf b}$,${\bf a}\times {\bf b}$, in that order, are positively-oriented.}

\item{$|{\bf a}\times {\bf b}| = |{\bf a}||{\bf b}| \sin\theta$, where $\theta$, $0\leq \theta \leq \pi$, is the angle between ${\bf a}$ and ${\bf b}$.}
\end{enumerate}


\begin{question} \label{Q98erg4323}
Let ${\bf a} = \langle 1, -1, 3 \rangle$ and ${\bf b}=\langle -2,2,1 \rangle$.

(a) Compute by hand the vector ${\bf c} = {\bf a}\times {\bf b}$.

(b) Use the arithmetic of vectors to verify properties (a), (c) above.

\begin{hint}
For property (c), first use the scalar product to compute $\cos\theta$.
\end{hint}

(c) Use the desmos worksheet below to verify property (b) visually by entering the components of the vector ${\bf c}$ in Line 6. Show a screenshot and explain your reasoning.



\begin{onlineOnly}
    \begin{center}
\desmosThreeD{jla3lweqrp}{900}{600}
\end{center}
\end{onlineOnly}

\href{https://www.desmos.com/3d/jla3lweqrp}{163: Cross Product 1}

\end{question}

%To better understand this, suppose ${\bf a}$ and ${\bf b}$ are parallel to the $xy$-plane. Then the $\answer{z}$-


\begin{exploration}  \label{E73830233}
(a) In the animation below, the green vector $\overrightarrow{OC}$ is equal to ${\bf u}\times {\bf w}$ or ${\bf w}\times {\bf u}$. Which one? 

(b) Try to predict how the cross product $\overrightarrow{OA}\times \overrightarrow{OG}$ will change as point $G$ rotates around the circle. Then drag the slider $\theta$ (the angle from ${\bf u}$ to ${\bf w}$) to see if you were correct.

(c) Summarize your observations about how the vector $\overrightarrow{OA}\times \overrightarrow{OG}$ changes magnitude and direction as $G$ moves around the circle. Be precise.

(d) Approximate the lengths of the vectors ${\bf u}$ and ${\bf w}$. Explain your reasoning.

\begin{onlineOnly}
    \begin{center}
\geogebra{atvydddh}{900}{600}
\end{center}
\end{onlineOnly}

\href{https://www.geogebra.org/classic/atvydddh}{163: Cross Product Animation}
\end{exploration}


\begin{question} \label{Q344343222111}
Find vectors ${\bf a}, {\bf b}\in \mathbb{R}^3$ such that
\[
       {\bf a} \times {\bf b} = \langle -2, 5, -3 \rangle .
\]
Explain your method.
\end{question}


\section{Squares and Cubes in Space}

\begin{question}  \label{Q354624ggtg}
The idea of this problem is to construct a square $ABCD$ in $\mathbb{R}^3$ given the vertices $A$, $B$ and a third point $P$ in the plane of the square. We assume that $A$, $B$, and $P$ are not colinear so that they determine a unique plane.

%We'll work in general with the vectors from the origin to the vertices of the square. So given the points $A$, $B$ and the vector ${\bf n}$, our aim is to express the vectors ${\bf c} = \overrightarrow{OC}$ and ${\bf d} =\overrightarrow{OD}$ from the origin to $C$, $D$ in terms of the vectors ${\bf a} = \overrightarrow{OA}$, ${\bf b} = \overrightarrow{OB}$, and ${\bf n}$. 

We'll work in general with the vectors from the origin to the vertices of the square. So given the points $A$, $B$ $P$, our aim is to express the vectors ${\bf c} = \overrightarrow{OC}$ and ${\bf d} =\overrightarrow{OD}$ from the origin to $C$, $D$ in terms of the vectors ${\bf a} = \overrightarrow{OA}$, ${\bf b} = \overrightarrow{OB}$, and ${\bf p} = \overrightarrow{OP}$. 


\pskip

(a) How many possibilites are there for the vertex $C$ of our square? Remember that $A$, $B$, $C$, $D$ are consecutiv vertices of square $ABCD$.
\begin{multipleChoice}
\choice{none}
\choice{one}
\choice[correct]{two}
\choice{infinitely many}
\end{multipleChoice}

(b) The first step is to find a vector ${\bf n}$ normal to the plane through $A$, $B$, and $C$. Do this. Click the Hint tab above if you need a hint.
\begin{hint}
A vector ${\bf n}$ is normal to the plane if and only if ${\bf n}$ is normal to the vectors ${\bf b}-{\bf a}$ and ${\bf p}-{\bf a}$. So we can choose
\[
   {\bf n} = ( {\bf b}-{\bf a}  ) \answer{\times} ({\bf p}-{\bf a})  .
\] 
\end{hint}

(c) Next express the vector ${\bf c} = \overrightarrow{OC}$ in terms of ${\bf a}$, ${\bf b}$, and ${\bf n}$. There are two possibilities. Choose one and enter your expression for ${\bf c}$ in Line 14 of the worksheet below. 

\begin{onlineOnly}
    \begin{center}
\desmosThreeD{uule7dqdl4}{900}{600}
\end{center}
\end{onlineOnly}

\href{https://www.desmos.com/3d/uule7dqdl4}{163: Square}

(d) Finally, express the vector ${\bf d} = \overrightarrow{OD}$ in terms of the vectors ${\bf a}$, ${\bf b}$ and ${\bf c}$. Enter your expression for ${\bf d}$ in Line 15. 

(e) Change your expression for ${\bf c} = \overrightarrow{OC}$ in Line 14 to get the other square.

(f) Try to construct the cube $ABCDB_1C_1D_1A_1$ where $A_1$ is the vertex opposite $A$, etc. 

\end{question}

\section{Equilateral Triangles and Regular Tetrahedrons}



\begin{question}  \label{Qdfdsffrr3tg}
The idea of this problem is to construct an equilateral triangle $ABC$ in $\mathbb{R}^3$ given the vertices $A$, $B$ and a third point $P$ in the plane of the triangle. We assume that $A$, $B$, and $P$ are not colinear so that they determine a unique plane.

\begin{enumerate}
\item{Express the vector $\overrightarrow{OC}$ from the origin to the third vertex of equilateral triangle $\Delta ABC$ in terms of the vectors $\overrightarrow{OA}$, $\overrightarrow{OB}$ and 
\[
    {\bf n} = \overrightarrow{PA}  \times \overrightarrow{PB}.
\]
There are two possibilites. Choose one.
}

\item{Enter your expression for $\overrightarrow{OC}$ on Line 13 of the desmos worksheet below to check your work.}

\item{Express the vector $\overrightarrow{OD}$ from the origin to the fourth vertex of the regular tethrahdron $\Delta ABCD$ in terms of the vectors $\overrightarrow{OA}$, $\overrightarrow{OB}$, $\overrightarrow{OC}$, and ${\bf n}$. The four faces of a regular tetrahedron are equilateral triangles).  
}

\end{enumerate}

\begin{onlineOnly}
    \begin{center}
\desmosThreeD{3fktg1c31g}{900}{600}
\end{center}
\end{onlineOnly}

\href{https://www.desmos.com/3d/3fktg1c31g}{163: Equilateral Triangle}

\end{question}


\section{Circles in Space}

\begin{question}  \label{Qdfds45rt54rr3tg}
This problem is a first step toward parameterizing circles in space. 

We are given two orthogonal vectors $\overrightarrow{QA}$ and $\overrightarrow{QB}$ with equal lengths as shown below. Let $P$ be a point on the circle through $A$ and $B$ centered at $Q$. 

\begin{onlineOnly}
    \begin{center}
\desmos{arrfrcusn5}{900}{600}
\end{center}
\end{onlineOnly}

\href{https://www.desmos.com/calculator/arrfrcusn5}{163: Parameterizing Circles in the Plane}

\begin{enumerate}
\item{Our first goal is to express the vector $\overrightarrow{QP}$ in terms of the angle $\phi$ from $\overrightarrow{QA}$ to  $\theta$ from $\overrightarrow{QB}$. This angle is measured continuously and is taken to be positive in the counterclockwise direction. Do this as follows:}

\begin{enumerate}
\item{Find the vector projection of $\overrightarrow{QP}$ onto $\overrightarrow{QA}$.
\begin{multipleChoice}
    \choice[correct]{$(\cos \phi)\overrightarrow{QA}$}
     \choice{$(\sin \phi)\overrightarrow{QA}$}
     \choice{$(\tan \phi)\overrightarrow{QA}$}
\end{multipleChoice}
}
 \item{Find the vector projection of $\overrightarrow{QP}$ onto $\overrightarrow{QB}$.
\begin{multipleChoice}
    \choice{$(\cos \phi)\overrightarrow{QB}$}
     \choice[correct]{$(\sin \phi)\overrightarrow{QB}$}
     \choice{$(\tan \phi)\overrightarrow{QB}$}
\end{multipleChoice}
}

\item{Use the two previous results to write $\overrightarrow{QP}$ as a linear combination of  $\overrightarrow{QA}$ and $\overrightarrow{QB}$.
\begin{multipleChoice}
    \choice[correct]{$\overrightarrow{QP} = (\cos \phi)\overrightarrow{QA} + (\sin\phi)\overrightarrow{QB}$}
     \choice{$\overrightarrow{QP} = (\sin \phi)\overrightarrow{QA} + (\cos\phi)\overrightarrow{QB}$}
\end{multipleChoice}
}

\end{enumerate}

\item{Now express the vector $\overrightarrow{OP}$ in terms of the vectors $\overrightarrow{OA}$, $\overrightarrow{OB}$, $\overrightarrow{OQ}$, and the angle $\phi$. Enter this expression in Line 19 of the worksheet above and play the slider $\phi$ to see if you are correct. Keep in mind that to write the vector $\overrightarrow{OA}$ in desmos, for example, you would just write $O$.
}

\item{With the appropriate domain for $\phi$, we can regard the answer to the previous question as a parameterization of the circle centered at $Q$ through $A$ and $B$. But desmos requires we use the parameter $t$ to parameterize a curve. Enter this parameterization with the appropriate domain in Line 20 as a check.}

\end{enumerate} 

\end{question}



\begin{question}  \label{Qdfrr54r55thlnmvvc}
Let's apply the idea of the previous question to parameterize a circle in $\mathbb{R}^3$. We'll suppose the circle is centered at the point $Q(-1,4,1)$ and has radius $3$. 

\begin{enumerate}

\item{How many such circles are there?
\begin{multipleChoice}
\choice{none}
\choice{one}
\choice{two}
\choice[correct]{infinitely many}
\end{multipleChoice}
}

\item{To pick out one of the infinitely many circles in space with a given center and radius, we need to specify the plane in which the circle lies. To do that, we'll say that our particular circle with center $Q(-1,4,1)$ and radius $3$ lies in the plane parallel to the vectors ${\bf u} = \langle -2, -1, 2 \rangle$ and ${\bf v}= \langle 3/\sqrt{2},0,3/\sqrt{2} \rangle$. These vectors are particularly convenient because
\begin{enumerate}
\item {${\bf u}$ and ${\bf v}$ are orthogonal.}

\item{$|{\bf u}| = |{\bf v}| = 3$.}
\end{enumerate}

\begin{freeResponse}
Verify the two facts above.
\end{freeResponse}
}

\item{We can draw the vectors ${\bf u}$ and ${\bf v}$ anywhere in space, but because our circle has center $Q(-1,4,1)$, we'll draw these vectors with their tails at $Q$ as shown below.

\begin{onlineOnly}
    \begin{center}
\desmosThreeD{ftosqazxfm}{900}{600}
\end{center}
\end{onlineOnly}

\href{https://www.desmos.com/3d/ftosqazxfm}{163: Circle in Space 1}

Now let $P(x,y,z)$ be a point on the circle such that the vector $\overrightarrow{QP}$ makes the angle $\phi_1$ with ${\bf u}$ (the blue vector), where the angle is measured toward ${\bf v}$. Our aim is to express the position $\overrightarrow{OP}$ of $P$ relative to the origin in terms of the vectors $\overrightarrow{OQ}$, ${\bf u}$ and ${\bf v}$. But this is exactly what we did in the previous question. Click the blue arrow (below right) to check your work.

\begin{expandable}
\begin{align*}
   \overrightarrow{OP} &= \overrightarrow{OQ} + \overrightarrow{QP} \\
                                   & = \overrightarrow{OQ} + (\cos\phi_1) {\bf u} + (\sin\phi_1){\bf v} , 0\leq \phi_1 \leq 2\pi .
\end{align*}
\end{expandable}
}

\item{That's it. The above is a \emph{vector} equation of the circle with center $A$ and radius $r=|{\bf u}| = |{\bf v}|$ that lies in the plane parallel to the vectors ${\bf u}$ and ${\bf v}$. To enter this in the desmos worksheet above, remember that desmos treats points as vectors (not graphically, but algebraically), so we can write $U$ for ${\bf u}$, $V$ for ${\bf v}$, and $Q$ for $\overrightarrow{OQ}$. But we must use $t$ for the parameter. Try entering this parameterization on Line 8 in the worksheet above. Drag slider $\phi_1$ to change the angle and move $P$ around the circle.
}
\end{enumerate}

\pskip

{\bf Summary:} The parameterization
\[
      \overrightarrow{OP} = \overrightarrow{OQ} + (\cos t){\bf u} + (\sin t){\bf u}\, , \, 0\leq t < 2\pi ,
\]
expresses the positions relative the origin of the points on the circle of radius $|{\bf u}| = |{\bf v}|=3$ in the plane parallel to the vectors ${\bf u}$ and ${\bf v}$.
\end{question}


\begin{question}  \label{Q98de983234}
This question is much like the last, but with one significant change. We wish to parameterize the circle with center $Q(-1,4,1)$ and radius $3$ that lies in the plane parallel to the vectors ${\bf u} = \langle -2, -1, 2 \rangle$ and ${\bf v}= \langle 3,0,0 \rangle$. 
Now the vectors ${\bf u}$ and ${\bf v}$ are \emph{not} orthogonal. As a result, the same parameterization we used before does not give a circle but an ellipse. See below.

\begin{onlineOnly}
    \begin{center}
\desmosThreeD{k5idcpjrep}{900}{600}
\end{center}
\end{onlineOnly}

\href{https://www.desmos.com/3d/k5idcpjrep}{163: Circle in Space 2}

How can we fix this? The idea is to replace one of the vectors, say ${\bf v}$ with another of length $3$ that is perpendicular to ${\bf u}$ and parallel to the plane of our circle (ie. the plane through $A$ parallel to the vectors ${\bf u}$ and ${\bf v}$). 

\begin{hint}
To do this you will need to use the cross product. Maybe more than once. 
\end{hint}


\end{question}


\begin{question}  \label{Q9df43r23rfddfs}
The idea of this problem is to parameterize a circle in space that lies in the plane determined by the non-colinear points $A$, $B$ and $P$. The circle is centered at $A$ and passes through $B$.

\begin{enumerate}
\item{Use the ideas of the last two problems to express the vector $\overrightarrow{OP}$ from the origin to a point $P$ on the circle in terms of a parameter $t$ and the vectors $\overrightarrow{OA}$, $\overrightarrow{OB}$, and
\[
   {\bf n} = \overrightarrow{PA} \times \overrightarrow{PB} .
\]
}

\item{Interpret the geometric meaning of the parameter above.}

\item{Enter the correct parameterization in Line 13 of the worksheet below.}

\end{enumerate}


\begin{onlineOnly}
    \begin{center}
\desmosThreeD{4qvuqtp4lo}{900}{600}
\end{center}
\end{onlineOnly}

\href{https://www.desmos.com/3d/4qvuqtp4lo}{163: Circle in Space}

\end{question}


\section{Geodesics on a Sphere}

\begin{question}  \label{Q9df7udsftr}
A \emph{great circle} on a sphere is a circle centered at the sphere's center. It looks like a straight line to residents of the sphere. The shorter arc of a great circle through two given points is the shortest path on the sphere between the points.

\begin{enumerate}
\item{Parameterize the great circle through the given points $A$ and $B$ on a sphere of radius $R$ centered at the origin. The parametrization will be in terms of a parameter $t$ and the vectors $\overrightarrow{OA}$ and $\overrightarrow{OB}$. Include an appropriate domain.}

\item{Parameterize the shorter arc of that great circle between $A$ and $B$.}
\item{Compute the length of that arc and compare it to the radius of the sphere.}
\item{Enter your parameterizations in the worksheet below to see if you are correct.}

\end{enumerate}

\begin{onlineOnly}
    \begin{center}
\desmosThreeD{vulzde8gww}{900}{600}
\end{center}
\end{onlineOnly}

\href{https://www.desmos.com/3d/vulzde8gww}{163: Great Circles 1}

\end{question}


\section{Intersecting Spheres}

\begin{question} \label{Q:74dsfretg4444}
Circles in the $xy$-plane centered at the points $A$ and $B$ intesect at point $Q$ as shown below. Our aim is to describe the second point of intersection by expressing the vector $\overrightarrow{OS}$ from the origin to that point in terms of the vectors ${\bf a} = \overrightarrow{OA}$, ${\bf b} = \overrightarrow{OB}$, and ${\bf q} = \overrightarrow{OQ}$. There are two steps.

\begin{enumerate}
\item Express the vector $\overrightarrow{OC}$ from $O$ the origin to the midpoint $C$ of segment $\overline{QS}$ in terms of the vectors ${\bf a}$, ${\bf b}$, and ${\bf q}$.
\begin{hint}
Think about a vector projection.
\end{hint}
Input your expression for $\overrightarrow{OC}$ in Line 8 below.


\item Express the vector $\overrightarrow{OS}$ in terms of the vectors $\overrightarrow{OQ}$ and $\overrightarrow{OC}$. Then input this expression in Line 10 below.

\end{enumerate}

\begin{onlineOnly}
    \begin{center}
\desmosThreeD{nttmyvnge7}{900}{600}
\end{center}
\end{onlineOnly}

\href{https://www.desmos.com/3d/nttmyvnge7}{163: Intersecting Circles}
\end{question}



\begin{question} \label{Q:743rf4rtg4444}

\begin{onlineOnly}
    \begin{center}
\desmosThreeD{wftyxbvczm}{900}{600}
\end{center}
\end{onlineOnly}

\href{https://www.desmos.com/3d/wftyxbvczm}{163: Intersecting Spheres}
\end{question}


\end{document}

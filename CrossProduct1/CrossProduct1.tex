\documentclass{ximera}
\title{The Cross Product, Part 1}

\newcommand{\pskip}{\vskip 0.1 in}

\begin{document}
\begin{abstract}
An introduction to the cross product.
\end{abstract}
\maketitle

\begin{question}  \label{Q34325rhgt}
(a) How many vectors in $\mathbb{R}^3$ are orthogonal to both ${\bf a}= \langle 1, 3,2 \rangle$ and ${\bf b}=\langle -4,3,-1 \rangle$?
 \begin{multipleChoice}
\choice{none}
\choice{one}
\choice{two}
\choice[correct]{infinitely many} 
\end{multipleChoice}

(b) How many directions in $\mathbb{R}^3$ are orthogonal to both ${\bf a}= \langle 1, 3,2 \rangle$ and ${\bf b}=\langle -4,3,-1 \rangle$?
 \begin{multipleChoice}
\choice{none}
\choice{one}
\choice[correct]{two}
\choice{infinitely many} 
\end{multipleChoice}



We wish to find the components of a vector ${\bf v}=\langle x, y , z \rangle \in \mathbb{R}^3$ perpendicular to the vectors ${\bf a}=\langle a_1, a_2, a_3 \rangle, {\bf b}=\langle b_1, b_2, b_3 \rangle \in \mathbb{R}^3$.  Necessary and sufficient conditions are that
\[
 \begin{cases}
 {\bf v}\cdot {\bf a} = a_1 x + a_2 y +a_3 z  = \answer{0}  \\
 {\bf v}\cdot {\bf b} = b_1 x + b_2 y +b_3 z  = \answer{0} .
\end{cases}
\]

Eliminating $x$ from this system of equations leads to the \emph{equation}
\[
    (a_1 b_2 - a_2 b_1)y + (\answer{a_1 b_3 - a_3 b_1})z = \answer{0}  .
\]
And one simple solution to this equation is
\[
     y = a_1 b_3 - a_3 b_1  \text{ and } z = \answer{a_2 b_1 - a_1 b_2} .
\]
The solve for $x$ to get
\[
       (x, y, z) = (   \answer{a_2 b_1 - a_1 b_2}     ,    \answer{ a_1 b_3 - a_3 b_1}        ,    \answer{a_2 b_1 - a_1 b_2}    )
\]
as a solution to the system.

Another equally simple solution is
\[ 
         (x, y, z) = (   a_1 b_2 - a_2 b_1     ,     a_3 b_1 - a_1 b_3        ,   a_1 b_2 - a_2 b_1    ) .
\]

It's a minor miracle that of the infinitely many solutions to our original system, it's the two simplest ones above that are the most useful. By convention, the latter
\[ 
         (x, y, z) = (   a_1 b_2 - a_2 b_1     ,     a_3 b_1 - a_1 b_3        ,   a_1 b_2 - a_2 b_1    ) = {\bf a}\times {\bf b}
\]
is the cross-product ${\bf a}\times {\bf b}$.

\end{question}

The cross product ${\bf a}\times {\bf b}$ of the vectors ${\bf a},{\bf b}\in \mathbb{R}^3$ is defined by the following properties.

\begin{enumerate}
\item{The vector ${\bf a}\times {\bf b}$ is orthogonal to ${\bf a}$ and ${\bf b}$}.

\item{The vectors ${\bf a}$, ${\bf b}$,${\bf a}\times {\bf b}$, in that order, are positively-oriented.}

\item{$|{\bf a}\times {\bf b}| = |{\bf a}||{\bf b}| \sin\theta$, where $\theta$, $0\leq \theta \leq \pi$, is the angle between ${\bf a}$ and ${\bf b}$.}
\end{enumerate}


\begin{question} \label{Q98erg4323}
Let ${\bf a} = \langle 1, -1, 3 \rangle$ and ${\bf b}=\langle -2,2,1 \rangle$.

(a) Compute by hand the vector ${\bf c} = {\bf a}\times {\bf b}$.

(b) Use the arithmetic of vectors to verify properties (a), (c) above.

\begin{hint}
For property (c), first use the scalar product to compute $\cos\theta$.
\end{hint}

(c) Use the desmos worksheet below to verify property (b) visually by entering the components of the vector ${\bf c}$ in Line 6. Show a screenshot and explain your reasoning.



\begin{onlineOnly}
    \begin{center}
\desmosThreeD{jla3lweqrp}{900}{600}
\end{center}
\end{onlineOnly}

\href{https://www.desmos.com/3d/jla3lweqrp}{163: Cross Product 1}

\end{question}

%To better understand this, suppose ${\bf a}$ and ${\bf b}$ are parallel to the $xy$-plane. Then the $\answer{z}$-


\begin{exploration}  \label{E73830233}
(a) In the animation below, the green vector $\overrightarrow{OC}$ is equal to ${\bf u}\times {\bf w}$ or ${\bf w}\times {\bf u}$. Which one? 

(b) Try to predict how the cross product $\overrightarrow{OA}\times \overrightarrow{OG}$ will change as point $G$ rotates around the circle. Then drag the slider $\theta$ (the angle from ${\bf u}$ to ${\bf w}$) to see if you were correct.

(c) Summarize your observations about how the vector $\overrightarrow{OA}\times \overrightarrow{OG}$ changes magnitude and direction as $G$ moves around the circle. Be precise.

(d) Approximate the lengths of the vectors ${\bf u}$ and ${\bf w}$. Explain your reasoning.

\begin{onlineOnly}
    \begin{center}
\geogebra{atvydddh}{900}{600}
\end{center}
\end{onlineOnly}

\href{https://www.geogebra.org/classic/atvydddh}{163: Cross Product Animation}
\end{exploration}


\begin{question} \label{Q344343222111}
Find vectors ${\bf a}, {\bf b}\in \mathbb{R}^3$ such that
\[
       {\bf a} \times {\bf b} = \langle -2, 5, -3 \rangle .
\]
Explain your method.
\end{question}


\section{Squares and Cubes in Space}

\begin{question}  \label{Q354624ggtg}
The idea of this problem is to construct a square $ABCD$ in $\mathbb{R}^3$ given the vertices $A$, $B$ and a third point $P$ in the plane of the square. We assume that $A$, $B$, and $P$ are not colinear so that they determine a unique plane.

%We'll work in general with the vectors from the origin to the vertices of the square. So given the points $A$, $B$ and the vector ${\bf n}$, our aim is to express the vectors ${\bf c} = \overrightarrow{OC}$ and ${\bf d} =\overrightarrow{OD}$ from the origin to $C$, $D$ in terms of the vectors ${\bf a} = \overrightarrow{OA}$, ${\bf b} = \overrightarrow{OB}$, and ${\bf n}$. 

We'll work in general with the vectors from the origin to the vertices of the square. So given the points $A$, $B$ $P$, our aim is to express the vectors ${\bf c} = \overrightarrow{OC}$ and ${\bf d} =\overrightarrow{OD}$ from the origin to $C$, $D$ in terms of the vectors ${\bf a} = \overrightarrow{OA}$, ${\bf b} = \overrightarrow{OB}$, and ${\bf p} = \overrightarrow{OP}$. 


\pskip

(a) How many possibilites are there for the vertex $C$ of our square? Remember that $A$, $B$, $C$, $D$ are consecutive vertices of square $ABCD$.
\begin{multipleChoice}
\choice{none}
\choice{one}
\choice[correct]{two}
\choice{infinitely many}
\end{multipleChoice}

(b) The first step is to find a vector ${\bf n}$ normal to the plane through $A$, $B$, and $C$. Do this. Click the Hint tab above if you need a hint.
\begin{hint}
A vector ${\bf n}$ is normal to the plane if and only if ${\bf n}$ is normal to the vectors ${\bf b}-{\bf a}$ and ${\bf p}-{\bf a}$. So we can choose
\[
   {\bf n} = ( {\bf b}-{\bf a}  ) \answer{\times} ({\bf p}-{\bf a})  .
\] 
\end{hint}

(c) Next express the vector ${\bf c} = \overrightarrow{OC}$ in terms of ${\bf a}$, ${\bf b}$, and ${\bf n}$. There are two possibilities. Choose one and enter your expression for ${\bf c}$ in Line 14 of the worksheet below. 

\begin{onlineOnly}
    \begin{center}
\desmosThreeD{uule7dqdl4}{900}{600}
\end{center}
\end{onlineOnly}

\href{https://www.desmos.com/3d/uule7dqdl4}{163: Square}

(d) Finally, express the vector ${\bf d} = \overrightarrow{OD}$ in terms of the vectors ${\bf a}$, ${\bf b}$ and ${\bf c}$. Enter your expression for ${\bf d}$ in Line 15. 

(e) Change your expression for ${\bf c} = \overrightarrow{OC}$ in Line 14 to get the other square.

(f) Try to construct the cube $ABCDB_1C_1D_1A_1$ where $A_1$ is the vertex opposite $A$, etc. 

\end{question}


\begin{question} \label{Q346563211}
This problem is almost identical to the previous one. The only difference is that now we are given a pair of \emph{opposite} vertices $A$ and $C$ of square $ABCD$ lying in the plane through the non-colinear points $A$, $B$, and $P$.

\begin{enumerate}
\item{ How many possibilites are there for vertex $A$ of our square? 
\begin{multipleChoice}
\choice{none}
\choice[correct]{one}
\choice{two}
\choice{infinitely many}
\end{multipleChoice}
}

\item{Express the vector $\overrightarrow{OC}$ from the origin to $C$ in terms of the vectors $\overrightarrow{OA}$, $\overrightarrow{OC}$, and 
\[
   {\bf n} = ( {\bf b}-{\bf a}  ) \times ({\bf p}-{\bf a})  .
\] 
Click the Hint tab above for help if necessary. 
\begin{hint}
Start by expressing the vector $\overrightarrow{OM}$ from the origin to the midpoint $M$ of segment $\overline{AC}$ in terms of $\overrightarrow{OA}$ and $\overrightarrow{OC}$. 
\end{hint}
}

\item{Find a similar expression for the vector $\overrightarrow{OD}$.}

\item{Check your work in the desmos activity below by entering your expressions for $\overrightarrow{OB}$ and $\overrightarrow{OD}$ in Lines 14 and 15.

\begin{onlineOnly}
    \begin{center}
\desmosThreeD{n7pzbo9hum}{900}{600}
\end{center}
\end{onlineOnly}

\href{https://www.desmos.com/3d/n7pzbo9hum}{163: Square2}
}

\end{enumerate}

\end{question}

\section{Equilateral Triangles and Regular Tetrahedrons}



\begin{question}  \label{Qdfdsffrr3tg}
The idea of this problem is to construct an equilateral triangle $ABC$ in $\mathbb{R}^3$ given the vertices $A$, $B$ and a third point $P$ in the plane of the triangle. We assume that $A$, $B$, and $P$ are not colinear so that they determine a unique plane.

\begin{enumerate}
\item{Express the vector $\overrightarrow{OC}$ from the origin to the third vertex of equilateral triangle $\Delta ABC$ in terms of the vectors $\overrightarrow{OA}$, $\overrightarrow{OB}$ and 
\[
    {\bf n} = \overrightarrow{PA}  \times \overrightarrow{PB}.
\]
There are two possibilites. Choose one.
}

\item{Enter your expression for $\overrightarrow{OC}$ on Line 13 of the desmos worksheet below to check your work.}

\item{Express the vector $\overrightarrow{OD}$ from the origin to the fourth vertex of the regular tethrahdron $\Delta ABCD$ in terms of the vectors $\overrightarrow{OA}$, $\overrightarrow{OB}$, $\overrightarrow{OC}$, and ${\bf n}$. The four faces of a regular tetrahedron are equilateral triangles).  
}

\end{enumerate}

\begin{onlineOnly}
    \begin{center}
\desmosThreeD{3fktg1c31g}{900}{600}
\end{center}
\end{onlineOnly}

\href{https://www.desmos.com/3d/3fktg1c31g}{163: Equilateral Triangle}

\end{question}





\end{document}

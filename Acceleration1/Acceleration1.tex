\documentclass{ximera}
\title{The Acceleration Vector, Part 1}

\newcommand{\pskip}{\vskip 0.1 in}

\begin{document}
\begin{abstract}
Acceleration and speed. 
\end{abstract}
\maketitle

In everday English we might say something like ``a car is accelerating'' to mean that the car is speeding up. But this is not really correct. Keep in mind that acceleration measures the rate of change of velocity with respect to time. And since velocity is a vector, so is acceleration. Velocity encodes both speed and direction, so acceleration encodes the rates of change of both speed and direction.

This chapter focuses on determining the rate of change (with respect to time) in speed. For this, we'll need both the acceleration and velocity vectors, or really just the angle between these vectors and the magnitude of the acceleration vector.

Before addressing this question, we'll jump up one level and consider an analogous question about position and velocity. But first, try the following question.

\begin{question}  \label{Q34r05r34gtt}
\begin{enumerate}
\item Interpret the meanings of the expressions 
\[ 
      \Big| \frac{d}{dt} \left( {\bf p}(t) \right) \Big|
\]
and
\[
   \frac{d}{dt} \left( |{\bf p}(t)| \right) .
\]
for a motion with position function ${\bf p}(t)$ measured in meters (and $t$ in seconds). Inlude units in your interpretations. Be sure to state clearly whether each is a vector or a scalar.

\item One of the expressions 
\[ 
      \Big| \frac{d}{dt} \left( {\bf v}(t) \right) \Big|
\]
and
\[
   \frac{d}{dt} \left( |{\bf v}(t)| \right) .
\]
is the magnitude (in $m/s^2)$ of the acceleration vector for a motion with  velocity function ${\bf v}(t)$ measured in meters/sec (and $t$ in seconds). Which one? Interpret the meaning of the other expression.
 
\end{enumerate}
\begin{freeResponse}
\end{freeResponse}
\end{question}



\section{Escaping from the Origin} 
This section under construction. Ignore it for now.

\begin{question}  \label{QErerttg}
\begin{align*}
\frac{d}{dt}\Big| {\bf p}(t)  \Big| &=  \frac{d}{dt} \left(\sqrt{x^2+y^2}\right)    &&   \frac{d}{dt}\Big| {\bf p}(t)  \Big| = \frac{d}{dt} \left(  \sqrt{{\bf p} \cdot {\bf p}} \right) \\
 &= \frac{2x \answer{\frac{dx}{dt}} + 2y \answer{\frac{dy}{dt}}}{2\sqrt{x^2 + y^2}}  && \hskip 0.5 in = \frac{{\bf v}\cdot {\bf p} + {\bf p}\cdot {\bf v}}{2\sqrt{{\bf p}\cdot {\bf p}}} \\
&= \frac{x \frac{dx}{dt} + y \frac{dy}{dt}}{\sqrt{x^2+y^2}}  &&  \hskip 0.5 in = \frac{{\bf v}\cdot {\bf p}}{|{\bf p}|} 
\end{align*}
\end{question}



\section{Speed and Acceleration}
We saw in the last chapter that for a circular motion, we can compute the rate of change in speed with respect to time as the scalar projection of the acceleration vector in the direction of the velocity vector. The same is true for any motion and the proof in this more general case turns out to be very quick.

The idea is to forget about the position function. Instead we'll imagine the hodograph and write the velocity function of a motion
in polar form as
\[
    {\bf v} = v \langle \cos \phi, \sin \phi \rangle .
\]
Here $v = | {\bf v}|$ is the speed and $\phi$ the angle from the positive $x$-axis to the velocity vector, measured counterclockwise. 

Now differentiate with respect to time, keeping in mind that both $v$ and $\phi$ are functions of time. Using the product rule, we find that the acceleration vector is
\[
    {\bf a} = \frac{d{\bf v}}{dt} = \frac{dv}{dt} \langle \cos\phi, \sin\phi \rangle + v \frac{d\phi}{dt} \langle -\sin\phi, \cos \phi \rangle .
\] 

The reason for doing this is that the derivative $dv/dt$ is exactly what we want to compute. It is the rate of change of speed with respect to time. So we just take the scalar product with the unit vector
\[
   \frac{{\bf v}}{|{\bf v}|} = \langle \cos\phi, \sin \phi \rangle
\]
in the direction of motion. This tells us
\begin{align*}
{\bf a} \cdot \frac{{\bf v}}{|{\bf v}|} &= {\bf a} \cdot \langle \cos\phi, \sin \phi \rangle \\
 & = \left(     \frac{dv}{dt} \langle \cos\phi, \sin\phi \rangle + v \frac{d\phi}{dt} \langle -\sin\phi, \cos \phi \rangle      \right) \cdot \langle \cos\phi, \sin \phi \rangle \\
                    & = \frac{dv}{dt} .
\end{align*} 
And that's what we wanted to show.

\emph{The rate of change in the speed of a motion with respect to time is the scalar projection of the acceleration vector in the direction of the velocity vector.}

The advantage of this proof is that it is simple. A disadvantage is that it works only for motions in the plane. The same fact holds for motions in space, but we'll skip the proof for now. Let's look at some examples first.

\section{Projectile Motion}
\begin{question} \label{Q:34re45r45r}
 
\begin{enumerate}
\item Identify the two graphs in the animation. One is speed as a function of time. The other is the derivative $dv(t)/dt$ of speed with respect to time.

\item Use the animation below to find the initial position, the initial velocity, and the acceleration for the projectile motion. Assume SI units. 

\item Use part (b) to find expressions for the position, velocity, and acceleration functions.

\item Use the result of part (c) to determine the speed of the motion at time $t=2$ seconds.

\item Is the projectile speeding up or slowing down at time $t=2$ seconds? At what rate. Use part (c) and then check your results with the graphs below. 

\item Use part (c) to evaluate the limit
\[
    \lim_{t\to \infty} v^\prime (t) .
\]

\item How can you visualize the above limit in the two graphs below?

\item How is the above limit related to the acceleration vector? Explain.

\item Identify the vector ${\bf w}$ in the demonstration above as a vector projection. What role does it play in some of the questions above?



\end{enumerate}

\begin{onlineOnly}
    \begin{center}
\geogebra{vbnga9rj}{900}{600}
\end{center}
\end{onlineOnly}

\href{https://www.geogebra.org/classic/vbnga9rj}{163: Projectile Motion Plane}


\end{question}





\end{document}
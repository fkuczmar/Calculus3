\documentclass{ximera}
\title{Quiz Solutions Winter 2025}

\newcommand{\pskip}{\vskip 0.1 in}

\begin{document}
\begin{abstract}
Quiz solutions.
\end{abstract}
\maketitle

\section{Quiz 2A}

\begin{question} \label{QDfdf3r0}

Let $O$ be the origin in $\mathbb{R}^3$. 

Let $A$ and $B$ be distinct points in $\mathbb{R}^3$.

Let $P$ be a point on the line through $A$ and $B$ that is four times as far from $A$ as $B$ is from $O$.

Express the position of $P$ relative to the origin in terms of the vectors $\overrightarrow{OA}$ and $\overrightarrow{OB}$. Give all possibilities.
\end{question}

\begin{explanation}

There are two points on the line a given distance from $A$, one on the same side of the line as $B$, the other on the opposite side.

Let $P_1$ be the point on the same side of the line as $B$ that is four times as far from $A$ as $B$ is from $O$.

To get from the origin to $P_1$, we can first move from the origin to $A$ and then from $A$ to $P_1$. This tells us that
\[
      \overrightarrow{OP_1} = \overrightarrow{OA} +\overrightarrow{AP_1}.
\]

We since $P_1$ is on the same side of the line as $B$, the vector $\overrightarrow{AP_1}$ is some positive multiple of the vector $\overrightarrow{AB}$. Since 
\[
         \left|  \overrightarrow{AP_1}   \right| = 4 \left| \overrightarrow{OB}  \right| ,
\]
and the vector
\[
      {\bf u} =  \frac{1}{|\overrightarrow{AB}|} \cdot \overrightarrow{AB}
\]
has unit length,
\begin{align*}
       \overrightarrow{AP_1} &= 4\left| \overrightarrow{OB}  \right|  {\bf u}  \\
                                         & = 4 \frac{\left| \overrightarrow{OB}\right|}{\left| \overrightarrow{AB}\right|}\cdot \overrightarrow{AB} \\
                                        &= \frac{4 \left| \overrightarrow{OB}\right|}{\left| \overrightarrow{OB} -\overrightarrow{OA} \right|}\cdot \left( \overrightarrow{OB} - \overrightarrow{OA} \right) .
\end{align*}

Therefore,
\begin{align*}
                 \overrightarrow{OP_1} &= \overrightarrow{OA} +\overrightarrow{AP_1} \\
                                                  &= \overrightarrow{OA} +  \frac{4 \left| \overrightarrow{OB}\right|}{\left| \overrightarrow{OB} -\overrightarrow{OA} \right|}\cdot \left( \overrightarrow{OB} - \overrightarrow{OA} \right) .
\end{align*}

For the second point $P_2$ on the \emph{opposite side} of $A$ as $B$, 
\begin{align*}
                 \overrightarrow{OP_2} &= \overrightarrow{OA} -\overrightarrow{AP_1} \\
                                                   &= \overrightarrow{OA} - \frac{4 \left| \overrightarrow{OB}\right|}{\left| \overrightarrow{OB} -\overrightarrow{OA} \right|}\cdot \left( \overrightarrow{OB} - \overrightarrow{OA} \right) .
\end{align*}

\end{explanation}



%\begin{onlineOnly}
%    \begin{center}
%\desmosThreeD{pmhatuke3f}{900}{600}
%\end{center}
%\end{onlineOnly}

%\href{https://www.desmos.com/3d/pmhatuke3f}{163: Hyperboloid 2}

\end{document}
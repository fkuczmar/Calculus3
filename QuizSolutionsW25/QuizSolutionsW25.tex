\documentclass{ximera}
\title{Quiz Solutions Winter 2026}

\newcommand{\pskip}{\vskip 0.1 in}

\begin{document}
\begin{abstract}
Quiz solutions.
\end{abstract}
\maketitle

\section{Quiz 2A}

\begin{question} \label{QDfdf3r0}

Let $O$ be the origin in $\mathbb{R}^3$. 

Let $A$ and $B$ be distinct points in $\mathbb{R}^3$.

Let $P$ be a point on the line through $A$ and $B$ that is four times as far from $A$ as $B$ is from $O$.

Express the position of $P$ relative to the origin in terms of the vectors $\overrightarrow{OA}$ and $\overrightarrow{OB}$. Give all possibilities.
\end{question}

\begin{explanation}

There are two points on the line a given distance from $A$, one on the same side of the line as $B$, the other on the opposite side.

Let $P_1$ be the point on the same side of the line as $B$ that is four times as far from $A$ as $B$ is from $O$.

To get from the origin to $P_1$, we can first move from the origin to $A$ and then from $A$ to $P_1$. This tells us that
\[
      \overrightarrow{OP_1} = \overrightarrow{OA} +\overrightarrow{AP_1}.
\]

Since $P_1$ is on the same side of the line as $B$, the vector $\overrightarrow{AP_1}$ is some positive multiple of the vector $\overrightarrow{AB}$. Since 
\[
         \left|  \overrightarrow{AP_1}   \right| = 4 \left| \overrightarrow{OB}  \right| ,
\]
and the vector
\[
      {\bf u} =  \frac{1}{|\overrightarrow{AB}|} \cdot \overrightarrow{AB}
\]
has unit length,
\begin{align*}
       \overrightarrow{AP_1} &= 4\left| \overrightarrow{OB}  \right|  {\bf u}  \\
                                         & = 4 \frac{\left| \overrightarrow{OB}\right|}{\left| \overrightarrow{AB}\right|}\cdot \overrightarrow{AB} \\
                                        &= \frac{4 \left| \overrightarrow{OB}\right|}{\left| \overrightarrow{OB} -\overrightarrow{OA} \right|}\cdot \left( \overrightarrow{OB} - \overrightarrow{OA} \right) .
\end{align*}

Therefore,
\begin{align*}
                 \overrightarrow{OP_1} &= \overrightarrow{OA} +\overrightarrow{AP_1} \\
                                                  &= \overrightarrow{OA} +  \frac{4 \left| \overrightarrow{OB}\right|}{\left| \overrightarrow{OB} -\overrightarrow{OA} \right|}\cdot \left( \overrightarrow{OB} - \overrightarrow{OA} \right) .
\end{align*}

For the second point $P_2$ on the \emph{opposite side} of $A$ as $B$, 
\begin{align*}
                 \overrightarrow{OP_2} &= \overrightarrow{OA} -\overrightarrow{AP_1} \\
                                                   &= \overrightarrow{OA} - \frac{4 \left| \overrightarrow{OB}\right|}{\left| \overrightarrow{OB} -\overrightarrow{OA} \right|}\cdot \left( \overrightarrow{OB} - \overrightarrow{OA} \right) .
\end{align*}

\end{explanation}

\section{Quiz 2B}

\begin{question} \label{Q2Bdfd}

\item Let $O$ be the origin in $\mathbb{R}^3$.

Let $A$ and $B$ be distinct points in $\mathbb{R}^3$.

Let $P$ be the point on the sphere with radius $r$ centered at $A$ that is farthest from $B$.

Use vector arithmetic to express the position of $P$ relative to the origin in terms of $r$, $\overrightarrow{OA}$, and $\overrightarrow{OB}$.

\begin{enumerate}

\item Give a thorough explanation of your thinking in complete sentences.

\item Draw a picture to help with your explanation.

\item Preface each computation block with a brief explanation of what you are about to compute.

\end{enumerate}

\begin{explanation}
The farthest and closest points from a point $B$ that lie on a sphere centered at $A$ lie on the line $AB$. %The farthest point $P$ lies on the opposite side of the line as $B$.

To get from the origin to $P$, we can first move from the origin to $A$ and then from $A$ to $P$. This tells us that
\[
      \overrightarrow{OP} = \overrightarrow{OA} +\overrightarrow{AP}.
\]

Since $P$ is the farthest point on the sphere from $B$, the vector $\overrightarrow{AP}$ is a positive multiple of the vector $\overrightarrow{BA}$. And since 
\[
   \left| \overrightarrow{AP} \right| = r,
\]
we know
\begin{align*}
        \overrightarrow{AP} &=  r \left( \frac{\overrightarrow{BA}}{|\overrightarrow{BA}|}  \right)  \\ \\
                                      &=  r \left( \frac{\overrightarrow{OA} - \overrightarrow{OB}}{|\overrightarrow{OA} - \overrightarrow{OB}|}  \right) .
\end{align*}

So the vector
\begin{align*}
   \overrightarrow{OP} &= \overrightarrow{OA} +\overrightarrow{AP} \\
                                 &= \overrightarrow{OA} +  r \left( \frac{\overrightarrow{OA} - \overrightarrow{OB}}{|\overrightarrow{OA} - \overrightarrow{OB}|}  \right)
\end{align*}
gives the position (relative to the origin) of the point  $P$ on the sphere with radius $r$ centered at $A$ that is farthest from $B$.

\end{explanation}
\end{question}

\section{Quiz 3A}

\begin{question} \label{Q99e8e}
\begin{enumerate}

\item Let $O$ be the origin in $\mathbb{R}^3$.

Let $A$ and $B$ be distinct points in $\mathbb{R}^3$.

Let $P$ be the point on the line through $A$ and $B$ \emph{not} between $A$ and $B$ such that
\[
     \frac{\left| \overrightarrow{AP} \right|}{\left| \overrightarrow{PB} \right|} = \frac{3}{8} . 
\]

Use vector arithmetic to express the position of $P$ relative to the origin in terms of $\overrightarrow{OA}$, and $\overrightarrow{OB}$.

\begin{enumerate}

\item Give a thorough explanation of your thinking in complete sentences.

\item Draw a reasonably accurate picture to help with your explanation.

\item Preface each computation block with a brief explanation of what you are about to compute.


\end{enumerate}
\end{enumerate}

\begin{explanation}
Since
\[
   \frac{\left| \overrightarrow{AP} \right|}{\left| \overrightarrow{PB} \right|} = \frac{3}{8}  < 1 ,
\]
$P$ is closer to $A$ than it is to $B$. And since $P$ is \emph{not} between $A$ and $B$, it must be that $A$ is between $B$ and $P$ as in the figure below.

\begin{onlineOnly}
    \begin{center}
\desmosThreeD{mefaqb0bvv}{900}{600}
\end{center}
\end{onlineOnly}

\href{https://www.desmos.com/3d/mefaqb0bvv}{163: Quiz 3A}

To get from the origin to $P$, we can first move from the origin to $A$ and then from $A$ to $P$. This tells us that
\[
      \overrightarrow{OP} = \overrightarrow{OA} +\overrightarrow{AP}.
\]

Because $A$ is between $B$ and $P$, 
\[
    \overrightarrow{AP} = t \overrightarrow{BA}
\]
for some positive scalar $t$. 

But because $A$ is between $B$ and $P$,
and
\[
  \frac{\left| \overrightarrow{AP} \right|}{\left| \overrightarrow{PB} \right|} = \frac{3}{8} ,
\]
we know that
\[
    \frac{\left| \overrightarrow{AP} \right|}{\left| \overrightarrow{BA} \right|} = \frac{3}{5}.
\]
So
\[
  \overrightarrow{AP} = \frac{3}{5} \overrightarrow{BA}
\]
and
\begin{align*}
   \overrightarrow{OP} &= \overrightarrow{OA} +\overrightarrow{AP}  \\
                                 &=  \overrightarrow{OA} +\frac{3}{5} \overrightarrow{BA} \\
                                 &=  \overrightarrow{OA} +\frac{3}{5} \left(    \overrightarrow{OA} - \overrightarrow{OB}  \right).
\end{align*}

\end{explanation}

\end{question}



\section{Quiz 3B}

\begin{question} \label{QPdre4340}

\begin{enumerate}
\item The four hydrogen atoms in the methane molecule $CH_4$ form the vertices of a regular tetrahedron surrounding the carbon atom. 

In a rectangular coordinate system the hydrogen atoms have coordinates $H_1(0,0,0)$, $H_2(2,2,0)$, $H_3(2,0,2)$, and $H_4(0,2,2)$. The carbon atom has coordinates $(1, 1, 1)$.

\emph{Use the methods of our textbook (ie. vectors and the scalar product)} to find the exact measure of the bond angle $\angle H_1 C H_2$ formed by two hydrogen atoms and the carbon atom.

No credit for other methods. This problem is meant to test if you did the homework. Its purpose is \emph{not} to see if you can solve the problem another way.

\begin{explanation}

\begin{onlineOnly}
    \begin{center}
\desmosThreeD{uk1fgzyccs}{900}{600}
\end{center}
\end{onlineOnly}

\href{https://www.desmos.com/3d/v}{163: Quiz 3B}

Let the measure of the bond angle be $\theta$. This is the angle between the vectors
\begin{align*}
   \overrightarrow{CH_1} &= \langle 0,0,0\rangle - \langle 1,1,1 \rangle \\
                                    &= \langle  -1,-1,-1 \rangle 
\end{align*}
and
\begin{align*}
   \overrightarrow{CH_2} &= \langle 2,2,0\rangle - \langle 1,1,1 \rangle \\
                                    &= \langle  1,1,-1 \rangle ,
\end{align*}
or any other similar such pair.

Then since
\begin{align*}
 \cos \theta &= \frac{\overrightarrow{CH_1}\cdot \overrightarrow{CH_2}}{\left| \overrightarrow{CH_1} \right| \left| \overrightarrow{CH_2} \right|} \\
                  & = \frac{\langle  -1,-1,-1 \rangle \cdot \langle  1,1,-1 \rangle}{|\langle  -1,-1,-1 \rangle ||\langle  1,1,-1 \rangle |} \\
                  &= -\frac{1}{3},
\end{align*}
the bond angle has measure
\[
    \theta = \arccos(-1/3).
\]

\end{explanation}
\end{enumerate}
\end{question}


\section{Quiz 4A}
\begin{question} \label{Qldfer33r}
\begin{enumerate}

\item Let ${\bf v}$ be a non-zero vector in $\mathbb{R}^3$ not parallel to the vector ${\bf w} = \langle 1,2,3\rangle$.

Use vector arithmetic (not algebra) to express the vector ${\bf v}$ as the cross-product of two other vectors.

Explain your reasoning thoroughly. Include pictures to help with your explanation.

Your explanation should include properties of the vector product, concerning both the magnitude and direction of this product.

Work in general with the given vectors ${\bf v}$ and ${\bf w}$, \emph{not} with their components.
\end{enumerate}


\begin{explanation}

\begin{onlineOnly}
    \begin{center}
\desmosThreeD{drkauraaje}{900}{600}
\end{center}
\end{onlineOnly}

\href{https://www.desmos.com/3d/drkauraaje}{163: Quiz 4A}

The solution is outlined in the worksheet above. Activate the folders in Lines 8 and 11 to see the steps. 

Here is a detailed explanation.

One key idea, to avoid repeatedly using the right-hand rule, is this:

If the vectors ${\bf a}$ and ${\bf b}$ are perpendicular and 
\[
    {\bf c} = {\bf a}\times {\bf b},
\]
then the vectors ${\bf a}$, ${\bf b}$, ${\bf c}$, in that order, form a \emph{right-handed} system in the sense that 
\[
      {\bf b} \times {\bf c} = \lambda_1 {\bf a}
\]
and
\[
       {\bf c} \times {\bf a} = \lambda_2 {\bf b}
\]
for \emph{positive} scalars $\lambda_1$ and $\lambda_2$. This says that ${\bf b} \times {\bf c}$ is parallel (not anti-parallel) to ${\bf a}$ and ${\bf c} \times {\bf a}$ is parallel to ${\bf b}$.

The simplest example is the triple ${\bf a} = \langle 1, 0, 0 \rangle$, ${\bf b} = \langle 0,1,0 \rangle$, and ${\bf c} = \langle 0,0,1\rangle$.

Moving on to the construction, the first step is to construct a unit vector ${\bf u}$ perpendicular to ${\bf v}$. One choice is
\[
      {\bf u} =  \frac{{\bf v} \times {\bf w}}{\left| {\bf v} \times {\bf w}\right|} .
\]
By the definition of the vector product, we know ${\bf v}\times {\bf w}$, and hence ${\bf u}$, is perpendicular to ${\bf v}$.

To avoid having to draw in three-dimensions, let's picture the vectors ${\bf v}$ and ${\bf u}$ as shown below.

\begin{onlineOnly}
    \begin{center}
\desmos{89y8r8q2ut}{900}{600}
\end{center}
\end{onlineOnly}

\href{https://www.desmos.com/calculator/89y8r8q2ut}{163: Quiz 4A Solution}

The next step in the construction is to define a vector, call it ${\bf a}$, as
\[
 {\bf a} = {\bf v} \times {\bf u}.
\]
 By the definition of the vector-product, we know 
\begin{enumerate}
\item ${\bf a}$ is perpendicular to ${\bf v}$ and ${\bf u}$,

\item ${\bf a}$ points out of the page, directly toward us (the right-hand rule), and

\item 
\begin{align*}
   |{\bf a}|  &= |{\bf v} | |{\bf u} | \sin(\pi/2) \\
                 &= |{\bf v}| .
\end{align*}
\end{enumerate}

From this, it follows that 
\[
  {\bf v} = {\bf u}\times {\bf a} .
\]

To justify this claim, we need to check that ${\bf u}\times {\bf a}$ has the same direction and magnitude as ${\bf v}$.

\begin{enumerate}
\item As for the magnitude, since ${\bf u}$ and ${\bf a}$ are perpendicular,
\[
       | {\bf u}\times {\bf a} |  = |  {\bf u}  | | {\bf a}  | \sin (\pi/2) = | {\bf a}| =  |{\bf v}| .
\]

\item To check ${\bf u}\times {\bf a}$ has the same direction as ${\bf v}$, we can return to the picture above and imagine the vector
\[
 {\bf a} = {\bf v} \times {\bf u}.
\]
pointing directly out of the page. Then we can check with the right-hand rule that the vector ${\bf u}\times {\bf a}$ points in the direction of ${\bf v}$.

\end{enumerate}

\end{explanation}
\end{question}

\section{Quiz 5B}

\begin{question} \label{Qkfdser}

Find a vector-parameterization of the tangent line to the curve
\[
   {\bf p}(t) = \Biggr \langle  \, 3\cos\left( \frac{\pi}{4}t \right), 4 \ln \left(\frac{t}{2} \right) , e^{5(t-2)} \Biggr \rangle , \, 1\leq t \leq 4 ,
\] 
at the point $P$ with coordinates $(0,0,1)$.

\begin{enumerate}
\item Explain your reasoning thoroughly.

\item Annotate the picture below to help explain the logic behind the parameterization of the tangent line by labeling the coordinates of the three points and drawing the appropriate vectors.

\begin{onlineOnly}
    \begin{center}
\desmosThreeD{gycfdbogol}{900}{600}
\end{center}
\end{onlineOnly}

\href{https://www.desmos.com/3d/gycfdbogol}{163: Quiz 5B}

\end{enumerate}

\begin{explanation}
At the point $P(0,0,1)$, since the $y$-component of the position vector is 
\[
     y = 4\ln (t/2) = 0,
\]
we know $t=2$ at the point $P$ (you should check that the $x$ and $z$ components are correct).

To parameterize the tangent line to the curve at $P$, we need a vector that gives the direction of the tangent line. One such vector is the derivative ${\bf p}^\prime(2)$. 

Using the chain rule when necessary, we differentiate the three components of the position vector.

\begin{enumerate}
\item
\begin{align*}
    \frac{d}{dt}\left( 3\cos\left( \frac{\pi}{4}t \right) \right)  &= -3 \sin  \left( \frac{\pi}{4}t \right) \frac{d}{dt}\left(  \frac{\pi}{4}t \right) \\
        &= -\frac{3\pi}{4} \sin  \left( \frac{\pi}{4}t \right)
\end{align*}

\item 
\begin{align*}
    \frac{d}{dt}\left( 4 \ln \left( \frac{t}{2}  \right) \right)  &=  4 \frac{d}{dt}\left( \ln t - \ln 2 \right)   \\
        &= \frac{4}{t} .
\end{align*}

\begin{align*}
    \frac{d}{dt}\left(  e^{5(t-2)} \right)  &= e^{5(t-2)} \frac{d}{dt}\left( 5(t-2) \right) \\
        &= 5 e^{5(t-2)} .
\end{align*}

\end{enumerate}

So the tangent line to the curve at the point $P$ is parallel to the vector
\[
 {\bf v} =  \frac{d}{dt}\left( {\bf p}(t) \right)\Big|_{t=2} = \biggr \langle -\frac{3\pi}{4} , 2, 5  \biggr \rangle . 
\] 

A

To annotate the figure,  label the coordinates of the points as $O(0,0,0)$, $P(0,0,1)$, and $Q(x,y,z)$. We should also draw vectors $\overrightarrow{OP}$, $\overrightarrow{PQ}$, and  $\overrightarrow{OQ}$.

Now a point $Q$ with coordinates $(x,y,z)$ lies on the line through $P$ parallel to the vector ${\bf v}$ if and only if
\[
   \overrightarrow{OQ} = \overrightarrow{OP} + t {\bf v} 
\]
for some real number $t$.

So a vector equation of the tangent line at $P$ is
\[
 \langle x, y, z\rangle  = \langle 0,0,1 \rangle + t \biggr \langle -\frac{3\pi}{4} , 2, 5  \biggr \rangle,  t\in \mathbb{R}.
\]


\end{explanation}

\end{question}



%\begin{onlineOnly}
%    \begin{center}
%\desmosThreeD{pmhatuke3f}{900}{600}
%\end{center}
%\end{onlineOnly}

%\href{https://www.desmos.com/3d/pmhatuke3f}{163: Hyperboloid 2}

\end{document}
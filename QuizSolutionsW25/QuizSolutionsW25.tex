\documentclass{ximera}
\title{Quiz Solutions Winter 2026}

\newcommand{\pskip}{\vskip 0.1 in}

\begin{document}
\begin{abstract}
Quiz solutions.
\end{abstract}
\maketitle

\section{Quiz 2A}

\begin{question} \label{QDfdf3r0}

Let $O$ be the origin in $\mathbb{R}^3$. 

Let $A$ and $B$ be distinct points in $\mathbb{R}^3$.

Let $P$ be a point on the line through $A$ and $B$ that is four times as far from $A$ as $B$ is from $O$.

Express the position of $P$ relative to the origin in terms of the vectors $\overrightarrow{OA}$ and $\overrightarrow{OB}$. Give all possibilities.
\end{question}

\begin{explanation}

There are two points on the line a given distance from $A$, one on the same side of the line as $B$, the other on the opposite side.

Let $P_1$ be the point on the same side of the line as $B$ that is four times as far from $A$ as $B$ is from $O$.

To get from the origin to $P_1$, we can first move from the origin to $A$ and then from $A$ to $P_1$. This tells us that
\[
      \overrightarrow{OP_1} = \overrightarrow{OA} +\overrightarrow{AP_1}.
\]

Since $P_1$ is on the same side of the line as $B$, the vector $\overrightarrow{AP_1}$ is some positive multiple of the vector $\overrightarrow{AB}$. Since 
\[
         \left|  \overrightarrow{AP_1}   \right| = 4 \left| \overrightarrow{OB}  \right| ,
\]
and the vector
\[
      {\bf u} =  \frac{1}{|\overrightarrow{AB}|} \cdot \overrightarrow{AB}
\]
has unit length,
\begin{align*}
       \overrightarrow{AP_1} &= 4\left| \overrightarrow{OB}  \right|  {\bf u}  \\
                                         & = 4 \frac{\left| \overrightarrow{OB}\right|}{\left| \overrightarrow{AB}\right|}\cdot \overrightarrow{AB} \\
                                        &= \frac{4 \left| \overrightarrow{OB}\right|}{\left| \overrightarrow{OB} -\overrightarrow{OA} \right|}\cdot \left( \overrightarrow{OB} - \overrightarrow{OA} \right) .
\end{align*}

Therefore,
\begin{align*}
                 \overrightarrow{OP_1} &= \overrightarrow{OA} +\overrightarrow{AP_1} \\
                                                  &= \overrightarrow{OA} +  \frac{4 \left| \overrightarrow{OB}\right|}{\left| \overrightarrow{OB} -\overrightarrow{OA} \right|}\cdot \left( \overrightarrow{OB} - \overrightarrow{OA} \right) .
\end{align*}

For the second point $P_2$ on the \emph{opposite side} of $A$ as $B$, 
\begin{align*}
                 \overrightarrow{OP_2} &= \overrightarrow{OA} -\overrightarrow{AP_1} \\
                                                   &= \overrightarrow{OA} - \frac{4 \left| \overrightarrow{OB}\right|}{\left| \overrightarrow{OB} -\overrightarrow{OA} \right|}\cdot \left( \overrightarrow{OB} - \overrightarrow{OA} \right) .
\end{align*}

\end{explanation}

\section{Quiz 2B}

\begin{question} \label{Q2Bdfd}

\item Let $O$ be the origin in $\mathbb{R}^3$.

Let $A$ and $B$ be distinct points in $\mathbb{R}^3$.

Let $P$ be the point on the sphere with radius $r$ centered at $A$ that is farthest from $B$.

Use vector arithmetic to express the position of $P$ relative to the origin in terms of $r$, $\overrightarrow{OA}$, and $\overrightarrow{OB}$.

\begin{enumerate}

\item Give a thorough explanation of your thinking in complete sentences.

\item Draw a picture to help with your explanation.

\item Preface each computation block with a brief explanation of what you are about to compute.

\end{enumerate}

\begin{explanation}
The farthest and closest points from a point $B$ that lie on a sphere centered at $A$ lie on the line $AB$. %The farthest point $P$ lies on the opposite side of the line as $B$.

To get from the origin to $P$, we can first move from the origin to $A$ and then from $A$ to $P$. This tells us that
\[
      \overrightarrow{OP} = \overrightarrow{OA} +\overrightarrow{AP}.
\]

Since $P$ is the farthest point on the sphere from $B$, the vector $\overrightarrow{AP}$ is a positive multiple of the vector $\overrightarrow{BA}$. And since 
\[
   \left| \overrightarrow{AP} \right| = r,
\]
we know
\begin{align*}
        \overrightarrow{AP} &=  r \left( \frac{\overrightarrow{BA}}{|\overrightarrow{BA}|}  \right)  \\ \\
                                      &=  r \left( \frac{\overrightarrow{OA} - \overrightarrow{OB}}{|\overrightarrow{OA} - \overrightarrow{OB}|}  \right) .
\end{align*}

So the vector
\begin{align*}
   \overrightarrow{OP} &= \overrightarrow{OA} +\overrightarrow{AP} \\
                                 &= \overrightarrow{OA} +  r \left( \frac{\overrightarrow{OA} - \overrightarrow{OB}}{|\overrightarrow{OA} - \overrightarrow{OB}|}  \right)
\end{align*}
gives the position (relative to the origin) of the point  $P$ on the sphere with radius $r$ centered at $A$ that is farthest from $B$.

\end{explanation}
\end{question}

\section{Quiz 3A}

\begin{question} \label{Q99e8e}
\begin{enumerate}

\item Let $O$ be the origin in $\mathbb{R}^3$.

Let $A$ and $B$ be distinct points in $\mathbb{R}^3$.

Let $P$ be the point on the line through $A$ and $B$ \emph{not} between $A$ and $B$ such that
\[
     \frac{\left| \overrightarrow{AP} \right|}{\left| \overrightarrow{PB} \right|} = \frac{3}{8} . 
\]

Use vector arithmetic to express the position of $P$ relative to the origin in terms of $\overrightarrow{OA}$, and $\overrightarrow{OB}$.

\begin{enumerate}

\item Give a thorough explanation of your thinking in complete sentences.

\item Draw a reasonably accurate picture to help with your explanation.

\item Preface each computation block with a brief explanation of what you are about to compute.


\end{enumerate}
\end{enumerate}

\begin{explanation}
Since
\[
   \frac{\left| \overrightarrow{AP} \right|}{\left| \overrightarrow{PB} \right|} = \frac{3}{8}  < 1 ,
\]
$P$ is closer to $A$ than it is to $B$. And since $P$ is \emph{not} between $A$ and $B$, it must be that $A$ is between $B$ and $P$ as in the figure below.

\begin{onlineOnly}
    \begin{center}
\desmosThreeD{mefaqb0bvv}{900}{600}
\end{center}
\end{onlineOnly}

\href{https://www.desmos.com/3d/mefaqb0bvv}{163: Quiz 3A}

To get from the origin to $P$, we can first move from the origin to $A$ and then from $A$ to $P$. This tells us that
\[
      \overrightarrow{OP} = \overrightarrow{OA} +\overrightarrow{AP}.
\]

Because $A$ is between $B$ and $P$, 
\[
    \overrightarrow{AP} = t \overrightarrow{BA}
\]
for some positive scalar $t$. 

But because $A$ is between $B$ and $P$,
and
\[
  \frac{\left| \overrightarrow{AP} \right|}{\left| \overrightarrow{PB} \right|} = \frac{3}{8} ,
\]
we know that
\[
    \frac{\left| \overrightarrow{AP} \right|}{\left| \overrightarrow{BA} \right|} = \frac{3}{5}.
\]
So
\[
  \overrightarrow{AP} = \frac{3}{5} \overrightarrow{BA}
\]
and
\begin{align*}
   \overrightarrow{OP} &= \overrightarrow{OA} +\overrightarrow{AP}  \\
                                 &=  \overrightarrow{OA} +\frac{3}{5} \overrightarrow{BA} \\
                                 &=  \overrightarrow{OA} +\frac{3}{5} \left(    \overrightarrow{OA} - \overrightarrow{OB}  \right).
\end{align*}

\end{explanation}

\end{question}



\section{Quiz 3B}

\begin{question} \label{QPdre4340}

\begin{enumerate}
\item The four hydrogen atoms in the methane molecule $CH_4$ form the vertices of a regular tetrahedron surrounding the carbon atom. 

In a rectangular coordinate system the hydrogen atoms have coordinates $H_1(0,0,0)$, $H_2(2,2,0)$, $H_3(2,0,2)$, and $H_4(0,2,2)$. The carbon atom has coordinates $(1, 1, 1)$.

\emph{Use the methods of our textbook (ie. vectors and the scalar product)} to find the exact measure of the bond angle $\angle H_1 C H_2$ formed by two hydrogen atoms and the carbon atom.

No credit for other methods. This problem is meant to test if you did the homework. Its purpose is \emph{not} to see if you can solve the problem another way.

\begin{explanation}

\begin{onlineOnly}
    \begin{center}
\desmosThreeD{uk1fgzyccs}{900}{600}
\end{center}
\end{onlineOnly}

\href{https://www.desmos.com/3d/v}{163: Quiz 3A}

Let the measure of the bond angle be $\theta$. This is the angle between the vectors
\begin{align*}
   \overrightarrow{CH_1} &= \langle 0,0,0\rangle - \langle 1,1,1 \rangle \\
                                    &= \langle  -1,-1,-1 \rangle 
\end{align*}
and
\begin{align*}
   \overrightarrow{CH_2} &= \langle 2,2,0\rangle - \langle 1,1,1 \rangle \\
                                    &= \langle  1,1,-1 \rangle ,
\end{align*}
or any other similar such pair.

Then since
\begin{align*}
 \cos \theta &= \frac{\overrightarrow{CH_1}\cdot \overrightarrow{CH_2}}{\left| \overrightarrow{CH_1} \right| \left| \overrightarrow{CH_2} \right|} \\
                  & = \frac{\langle  -1,-1,-1 \rangle \cdot \langle  1,1,-1 \rangle}{|\langle  -1,-1,-1 \rangle ||\langle  1,1,-1 \rangle |} \\
                  &= -\frac{1}{3},
\end{align*}
the bond angle has measure
\[
    \theta = \arccos(-1/3).
\]

\end{explanation}

\end{enumerate}

\end{question}




%\begin{onlineOnly}
%    \begin{center}
%\desmosThreeD{pmhatuke3f}{900}{600}
%\end{center}
%\end{onlineOnly}

%\href{https://www.desmos.com/3d/pmhatuke3f}{163: Hyperboloid 2}

\end{document}
\documentclass{ximera}
\title{Coordinates in Space}

\newcommand{\pskip}{\vskip 0.1 in}

\begin{document}
\begin{abstract}
Describing sets of points in $\mathbb{R}^3$.
\end{abstract}
\maketitle


\begin{question} \label{Q1:Coordinates}
For each of the following sets, do the following:

\begin{itemize}
\item{Describe the set geometrically.}

\item{Sketch the set.}

\end{itemize}

(a) $\{  (x,y)\in \mathbb{R}^2 \, | \, y=3  \}$

(b) $\{  (x,y,z)\in \mathbb{R}^3 \, | \, y=3  \}$

(c) $\{  (x,y)\in \mathbb{R}^2 \, | \, y=2x  \}$

(d) $\{  (x,y,z)\in \mathbb{R}^3 \, | \, y=2x  \}$

(e) $\{  (x,y)\in \mathbb{R}^2 \, | \,  x^2 + y^2 = 6x - 8y  \}$

(f)  $\{  (x,y,z)\in \mathbb{R}^3 \, | \, x^2 + y^2 = 6x - 8y  \}$

\end{question}


\begin{question} \label{Q2:Coordinates}
Describe the following sets of points algebraically and geometrically. Then graph each set. Use desmos to help with your sketches as necessary.

Complete the square for (b), (c), (d), (f).

(a) The set of points equidistant from the $xy$-plane and the point $F(0,0,3)$.

(b) The set of points twice as far from the $xy$-plane as from the point $F(0,0,3)$.

(c) The set of points twice as far from the point $F(0,0,3)$ as from the $xy$-plane.

%(b) The set of points twice as far from $A(0,0,0)$ as from $B(2,4,6)$.

(d) The set of points twice as far from the origin as from the $xy$-plane.

(e) The set of points equidistant from the $x$ and $z$ axes.

(f) The set of points $k$ times as far from the $x$-axis as from the $z$-axis. Explore how the graph changes with $k$. Assume $k>0$.

\end{question}


\begin{question} \label{Q3:Coordinates}
(a) Find an equation of a circle of radius $r$ in the first quadrant that is tangent to the $x$ and $y$ axes.

(b) 
\end{question}





\pdfOnly{
Access Desmos interactives through the online version of this text at
 
\href{https://www.desmos.com/3d/fbaa38f918}.
}
 
\begin{onlineOnly}
    \begin{center}
\geogebra{d2xv3dwc}{900}{600}
\end{center}
\end{onlineOnly}




\pdfOnly{
Access Desmos interactives through the online version of this text at
 
\href{https://www.desmos.com/3d/fbaa38f918}.
}
 
\begin{onlineOnly}
    \begin{center}
\geogebra{jsqe5guf}{900}{600}
\end{center}
\end{onlineOnly}


A try with vectors.

\pdfOnly{
Access Desmos interactives through the online version of this text at
 
\href{https://www.desmos.com/3d/fbaa38f918}.
}
 
\begin{onlineOnly}
    \begin{center}
\geogebra{jhgvamja}{900}{600}
\end{center}
\end{onlineOnly}

jhgvamja

\end{document}
\documentclass{ximera}
\title{Coordinates in Space}

\newcommand{\pskip}{\vskip 0.1 in}

\begin{document}
\begin{abstract}
Describing sets of points in $\mathbb{R}^3$.
\end{abstract}
\maketitle

%\pdfOnly{
%Access Desmos interactives through the online version of this text at
 
%\href{https://www.www.geogebra.org/classic/e6rsvnsz}.
%}
 
%\begin{onlineOnly}
%    \begin{center}
%\geogebra{e6rsvnsz}{900}{600}
%\end{center}
%\end{onlineOnly}



%The coordinates of a point on the sphere of radius $r$ with colatitude $\phi$ and longitude $\theta$ are
%\[
%   (x,y,z) = (\cos\theta \sin\phi, \sin\theta \sin \phi, \cos\phi ) \, , \, 0\leq \theta \leq 2\pi, \, 0\leq \phi \leq \pi .
%\]
%\begin{onlineOnly}
%    \begin{center}
%\geogebra{x48cgxbg}{900}{600}
%\end{center}
%\end{onlineOnly}

\section*{Introduction}
The first part of this chapter is about spheres. The following questions adress the key points. I'm asking them at the start to give you a chance to work through the ideas on your own.


\begin{question}   \label{QLMn4947332}
\begin{freeResponse}
(a) Can a thin wooden dowel 15 inches long fit (without bending) in a shoe box with dimensions $4"\times 8"\times 12"$?

(b) What do we mean by a sphere of radius 5 feet centered at origin? Be precise.

(c) In trying to determine whether or not the point $P(4,3,-1)$ lies inside, on, or outside the sphere of part (b), a student reasons that because the three coordinates are each less than $5$ in absolute value, the point $P$ lies inside the sphere. Is that reasoning correct? Explain why or why not. If not, answer the question correctly by giving a complete explanation that uses the ideas of parts (a) and (b). 

%(i) $(4,3,1)$

%(ii) $(4,2,-1)$

(d) The reasoning given in part (c) answers a different question about whether or not the point $P$ lies inside, on, or outside  some other surface. What surface? Explain.

(e) Find all possible values of $b$ that put the point with coordinates $(2,b,1)$ on the sphere of part (b).

(f) Complete the following sentence using everday English without using any mathematical notation or vocabulary:
\pskip

\emph{A point $P$ lies on the sphere of radius $5$ feet centered at the origin if and only if ...}

\pskip

(g)  Now suppose the point $P$ has coordinates $(x,y,z)$ and translate your condition from part (b) into an equation of the sphere.
 
(h) What do we mean by a sphere of radius 4 feet centered at the point $(3,2,1) feet$? Find an equation of such a sphere and explain how the equation encodes the meaning.

\end{freeResponse}

\end{question}


\section*{Computing Distances in Space}
We wish to compute the distance between the points $P_1$ and $P_2$ in space with respective coordinates $(x_1, y_1, z_1)$ and $(x_2, y_2, z_2)$. The idea is to draw a box with its edges parallel to the coordinate axes and with the points $P_1$ and $P_2$ as opposite vertices. The 12 edges of this box come in three sets of four, the four in each set being parallel to one of the three coordinate axes. 

\begin{question} \label{Q324tgtgt}
What is the length of the four edges parallel to the $x$-axis? Choose all that apply. 
\begin{selectAll}
\choice{$x_2-x_1$}
\choice{$y_2-y_1$}
\choice{$z_2-z_1$}
\choice[correct]{$|x_2-x_1|$}
\choice{$|y_2-y_1|$}
\choice{$|z_2-z_1|$}
\choice[correct]{$\sqrt{(x_1-x_2)^2}$}
\end{selectAll}
\end{question}

\begin{question} \label{Q32sdfdsftgt}
What is the length of a diagonal in one of the two faces parallell to the $yz$-plane? Choose all that apply. 
\begin{selectAll}
\choice{$\sqrt{(x_2-x_1)^2 +(y_2-y_1)^2 }$}
\choice[correct]{$\sqrt{(y_2-y_1)^2 +(z_2-z_1)^2 }$}
\choice{$\sqrt{(x_2-x_1)^2 +(z_2-z_1)^2 }$}
\choice{$\sqrt{|x_2-x_1|^2 +|y_2-y_1|^2 }$}
\choice[correct]{$\sqrt{|y_2-y_1|^2 +|z_2-z_1|^2 }$}
\choice{$\sqrt{|x_2-x_1|^2 +|z_2-z_1|^2 }$}
\end{selectAll}
\end{question}

\begin{question} \label{Q32sdfdsftgt}
What is the length of the main diagonal $\overline{P_1P_2}$? Choose all that apply. 
\begin{selectAll}
\choice[correct]{$\sqrt{(x_2-x_1)^2 +(y_2-y_1)^2  + (z_2-z_1)^2}$}
\choice[correct]{$\sqrt{(x_1-x_2)^2 +(y_2-y_1)^2  + (z_1-z_2)^2}$}
\choice[correct]{$\sqrt{|x_2-x_1|^2 +|y_2-y_1|^2  + |z_2-z_1|^2}$}
\end{selectAll}
\end{question}

\begin{question}  \label{Q:Kmm3454433}
\begin{freeResponse}
Draw a picture of a box in a coordinate system to illustrate the above computations. Label the vertices $P_1$ and $P_2$,  three edges with their different lengths, and two diagonals with their different lengths. 
\end{freeResponse}
\end{question}

\section*{Spheres}
A sphere is the set of all points in space that are a fixed distance from a given point. The equation of a sphere is a consequence of this definition and it is important to understand the logic behind this equation.

\begin{question}   \label{QDfsdgfte}
To find an equation of the sphere of radius $5$ centered at the point $A(3,1,-2)$, the key idea is this: a point $P$ lies on this sphere  if and only if the distance $\text{dist}(A,P)$ between $P$ and $A$ is equal to $5$. To translate this geometric condition into an algebraic description, let $P$ have coordinates $(x,y,z)$. Then $P$ lies on the sphere if and only if
\[
   \text{dist}(A,P) = 5 ,
\]
or if and only if 
\[
      \answer{\sqrt{(x-3)^2+ (y-1)^2 + (z+2)^2}} = 5   .
\]
And that's an equation of the sphere of radius $5$ centered at the point $(3,1,-2)$.
\end{question}





\section*{Tangent Spheres}
Two spheres are \emph{tangent} to each other if they intersect in exactly one point. They are said to be \emph{internally} tangent if one sphere lies inside the other and \emph{externally} tangent otherwise.

\begin{question}   \label{Q9sdf43gt4t44}
(a) How many spheres centered at the point $A(2,-3,6)$ are tangent to the sphere $x^2 + y^2 +z^2 = 9$?

(b) Find equations of the sphere(s) in part (a). Explain your reasoning. 

(c) Check your answer to part (b) by entering the equation(s) in the demonstration below.

\begin{exploration}
\begin{onlineOnly}
    \begin{center}
\desmosThreeD{jxgkzdtfuo}{900}{600}
\end{center}
\end{onlineOnly}
\end{exploration}

\end{question}



\section*{Center Sets, Part 1}
We have four degrees of freedom (ie. four choices) in selecting a sphere in three dimensions - one for each of the three coordinates of its center and another for its radius.

In the next question we'll find equations of spheres passing through two given points. An equation of such a sphere must satisfy two conditions. These conditions use up two of our four degrees of freedom, leaving us with just two.  

\begin{question}   \label{Q5453466366}
\begin{freeResponse}
(a) Is the point $Q(0,-2,3)$ the center of a sphere through the points $A(4,-3,1)$ and $B(-2,1,3)$? Explain your reasoning.

(b) Complete the following sentence using everyday English.

\pskip

\emph{A point $P$ is the center of a sphere through the points $A(4,-3,1)$ and $B(-2,1,3)$ if and only if ...}

(c) Use part (b) to describe geometrically the set of points that are centers of spheres through $A$ and $B$.

\end{freeResponse}

(d) Now suppose $P$ has coordinates $(x,y,z)$ and translate your condition from part (b) into an equation of the set of points that are centers of spheres through $A(4,-3,1)$ and $B(-2,1,3)$. 

The necessary and sufficient condition for a point $P$ with coordinates $(x,y,z)$ to be a center of a sphere through $A(4,-3,1)$ and $B(-2,1,3)$ is that
\[
     \sqrt{(x-4)^2 + (y+3)^2 + (z-1)^2} = \answer{\sqrt{(x-2)^2 + (y-1)^2 + (z-3)^2}}
\]

The simplified condition is 
\[
       \answer{-3}x + \answer{2}y + z = \answer{-3}
\]

(e) Enter you equation from part (d) on Line 3 in the desmos activity below to see if your equation from part (d) gives a surface that matches your geometric description from part (c).


\begin{onlineOnly}
    \begin{center}
\desmosThreeD{w6uvayd4oh}{900}{600}
\end{center}
\end{onlineOnly}


(f) Find an equation of the smallest sphere through $A(4,-3,1)$ and $B(-2,1,3)$. Enter the equation on Line 4 in the worksheet above.

(g) Use part (d) to find an equation of another sphere through $A(4,-3,1)$ and $B(-2,1,3)$. Enter the equation on Line 5 in the worksheet above.

\end{question}


\begin{question}  \label{Q2231324ffh}
We just saw that the centers of spheres through two given points $A$ and $B$ are the points of a plane. That plane passes through the midpoint of $\overline{AB}$ and is perpendicular to $\overline{AB}$. Since a plane is two-dimensional, we have two degrees of freedom in choosing the center of such a sphere. After choosing the center, the radius is determined by the condition that the sphere pass through $A$ and $B$.

For example, in the last problem we saw that the center set of the spheres through $A(4,-3,1)$ and $B(-2,1,3)$ is the plane
\[
 -3x + 2y +z = -3 ,
\]
or equivalently the plane
\[
   z = f(x,y) = 3x - 2y - 3.
\]
So we can choose $x$ and $y$ to be arbitrary real numbers (our two degrees of freedom), say 
\[
    x = u  \text{ and } y = v ,
\]
and then
\[
  z = \answer{-3 + 3u - 2v} .
\]

\begin{freeResponse}
(a) Move sliders $u$ and $v$ in Lines 3,4 below to move the green point (call it $P$). What do you notice?
\end{freeResponse}

(b) Find an equation of the sphere centered at $P(u,v,f(u,v))$ that passes through $A(4,-3,1)$ and $B(-2,1,3)$. Enter this equation in Line 7. Drag the sliders $u,v$ to check that your equation is correct.

\begin{onlineOnly}
    \begin{center}
\desmosThreeD{e9vq8rri9k}{900}{600}
\end{center}
\end{onlineOnly}

\href{https://www.desmos.com/3d/e9vq8rri9k}{163: Spheres Through Two Points 2}

\pskip

{\bf Summary} The requirement that a sphere pass through two given points leaves us two degrees of freedom. We get a two-parameter family of spheres passing through the points. Their centers form a plane.

\end{question}

\begin{question} \label{Q:LKdf44g44tr4}
Continuing with the previous question, we'll use up one of our two remaining degrees of freedom by requiring that the sphere pass through the point $C(0,1,-1)$ as well as the points $A(4,-3,1)$ and $B(-2,1,3)$.

\begin{freeResponse}
(a) Use your geometric intuition to describe the centers of all spheres that pass through three given non-colinear points. Explain your reasoning.
\end{freeResponse}

(b) Now for the algebra. Centers of spheres through the points $A(4,-3,1)$ and $C(0,1,-1)$ are the points in the plane with equation
\[
     \sqrt{(x-4)^2 + (y+3)^2 + (z-1)^2} = \answer{\sqrt{x^2 + (y-1)^2 + (z+1)^2}} ,
\]
or equivalently
\[
   \answer{2x -2y +z} = 6 .
\]

(c) Find also an equation of the plane whose points are the centers of spheres through $B(-2,1,3)$ and $C(0,1,-1)$.

(d) Enter the equations of the planes from parts (b), (c) in Lines 8,9 of the worksheet below. What do you notice?

(e) Now to parameterize the line of centers, substitute the parameterization 
\[
      (x,y,z) = (u,v,-3+3u-2v) \, , \, (u,v)\in \mathbb{R}^2 .
\]
of the first plane (points equidistant from $A(4,-3,1)$ and $B(-2,1,-3))$ into the equation of either of the two planes in parts (b),(c). Then solve for $v$ in terms of $u$ to get
\[
   v = \answer{\frac{5u-9}{4}} .
\]
Substitute this expression for $v$ into the expression for $z$ (in part (e)) to get a parameterization of the line of centers
\[
   (x,y,z) = \left( u, \answer{\frac{5u-9}{4}} , \answer{\frac{u+3}{2} }    \right) \, , \, u\in \mathbb{R} .
\]
Enter this parameterization in Line 10 (ie. correct the parameterization there), but use $t$ (as required by desmos) for the  parameter instead of $u$.

(f) Next, copy the parameterization from Line 10 into Line 11, but now use $u$ (one of our sliders) for the parameter. This will give you a point on the line of centers. Drag the slider $u$ to see if this works.

(g) Finally, write an equation of the sphere centered at the point in Line 11 that passes through the points $A(4,-3,1)$, $B(-2,1,3)$, and $C(0,1,-1)$. Enter this equation on Line 12. Drag $u$ and make sure your sphere passes through the three points.


\begin{onlineOnly}
    \begin{center}
\desmosThreeD{zkltopagls}{900}{600}
\end{center}
\end{onlineOnly}

\href{https://www.desmos.com/3d/zkltopagls}{163: Spheres Through Two Points 3}

\pskip

{\bf Summary} The requirement that a sphere pass through three given non-colinear points leaves us just one degree of freedom. We get a one-parameter family of spheres passing through the points. Their centers form a line.
\end{question}


\begin{question}  \label{Q:9df3gyhuuu}
Requiring a sphere to pass through three given non-colinear points ($A$, $B$,$C$) leaves us with $4-3 = 1$ degree of freedom. The centers of all spheres passing through these points form a line. % and choosing a point on this line determines the sphere's radius.

Requiring the sphere to pass through one additional point $D$ not the plane through $A$, $B$, and $C$ uses up our last degree of freedom. There is a unique sphere through the four points. Suppose $A$, $B$, $C$ are as before and that $D$ has coordinates $(2,-3,1)$.

\begin{freeResponse}
(a) Use the result of the previous problem and a bit more algebra to determine an equation of the sphere through the points $A(4,-3,1)$, $B(-2,1,3)$, $C(0,1,-1)$, and $D(2,-3,1)$. Explain your reasoning.
\end{freeResponse}

(b) Enter an equation of the sphere in Line 10 of the worksheet below as a check.

\begin{onlineOnly}
    \begin{center}
\desmosThreeD{islhwrjogm}{900}{600}
\end{center}
\end{onlineOnly}

\href{https://www.desmos.com/3d/islhwrjogm}{163: Spheres Through Four Points}


(c) An equation of the sphere through the points $A(4,-3,1)$, $B(-2,1,3)$, $C(0,1,-1)$, and $D(2,-3,1)$ is
\[
   \sqrt{ \answer{(x-3)^2 + (y-1.5)^2 + (z-3)^2}} = \answer{25.25} .
\]

\end{question}


\section*{Spheres in a Corner}
\begin{question}  \label{Q3442g455t}
(a) How many spheres of radius $3$ are tangent to the planes $x=0$, $y=0$, and $z=0$?

(b) Find equations of two such spheres.

(c) Find an equation of the sphere of radius $a > 0$ in the first octant tangent to the three coordinate planes. Enter this equation in Line 2 of the worksheet below. Check that you are correct by dragging the slider $a$.

\begin{exploration}
\begin{onlineOnly}
    \begin{center}
\desmosThreeD{ljizsai4mu}{900}{600}
\end{center}
\end{onlineOnly}

(d) Drag the slider $a$ above to determine how many spheres tangent to the three coordinate planes pass through the point $(4,3,5)$. Then find equations of all these spheres.

(e) Do you think there are any points with strictly positive coordinates that do not lie on any sphere tangent to the coordinate planes? Explain. Then find the coordinates of one such point and use algebra to prove that you are correct. 

(f) Do you think there are any points with strictly positive coordinates that lie on exactly one sphere tangent to the coordinate planes? Explain.

(g) Find necessary and sufficient conditions on the point $A$ with coordinates $(x_1, y_1, z_1)$ so that 

(i) $A$ lies on exactly two spheres tangent to the coordinate planes

(ii) $A$ lies on exactly one sphere tangent to the coordinate planes

(iii) $A$ lies on no sphere tangent to the coordinate planes.

(h) Find the coordinates of a specific point that lies on exactly one sphere tangent to the coordinate planes. Then use the demonsration above to check you are correct.

(i) Describe geometrically the set of points in the first octant that lie on exactly one sphere tangent to the coordinate planes. Use the demonstration to check you are correct.


\href{https://www.desmos.com/3d/ljizsai4mu}{163:Spheres in a Corner}

\end{exploration}

\end{question}



%\begin{example}  \label{Ex:9090dfsadf0}
%Describe algebraically and geometrically the set of points in space twice as far from the origin as from the point $B(4,2,6)$.
%\end{example}



\section*{Cylinders and Other Surfaces}
We know how to compute the distance between two points in space. But what about the distance from a point to a curve, or the distance from a point to a surface? We really need calculus to answer these questions, but first we should be precise about what we even mean.

By the distance between a point $P$ and a curve, we mean the least distance between $P$ and the points on the curve. Similarly, for a surface. Here are some examples.

\begin{question}  \label{Q:9sdfu3r5g43}
(a) Find the coordinates of the point on the $x$-axis closest to the point $A(3,5,-2)$. Drag the slider $s$ in the demonstration below as a check.

\begin{onlineOnly}
    \begin{center}
\desmosThreeD{xunhxawngj}{900}{600}
\end{center}
\end{onlineOnly}

The point on the $x$-axis closest to $A(3,5,-2)$ has coordinates $\answer{(3,0,0)}$.

(b) So the distance from the  point $A(3,5,-2)$ is 
\[
  s = \sqrt{(3-\answer{3})^2 + (5-\answer{0})^2 + (-2 - \answer{0})^2} = \answer{\sqrt{29}} .
\]

(c) Find an expression for the distance between the $x$-axis and the point $P$ with coordinates $(x,y,z)$.

The distance is given by the function 
\[
      f(x,y,z) = \answer{\sqrt{y^2 + z^2}} .
\]

(d) Find an equation of the right circular cylinder of radius $5$ symmetric about the $x$-axis.

The cylinder is the set of points exactly 5 units from the $x$-axis. So the point $P$ with coordinates $(x,y,z)$ lies on the cylinder if and only if the distance from $P$ to the $x$-axis is 5 units. So $P(x,y,z)$ lies on the cylinder if and only if
\[
     \answer{\sqrt{y^2 + z^2}} = 5 .
\]
This is an equation of the cylinder.
\end{question}


\begin{question}  \label{QLK8df433rrw}
Find an equation of the right circular cylinder of radius $4$ symmetric about the line ${\cal L}$ through $(0,0,0)$ and $(1,1,1)$ as follows:

(a) To get started, use calculus to find the coordinates of the point on ${\cal L}$ closest to the point $(3,1,2)$. Drag slider $s$ in the desmos worksheet below to check if your result is reasonable.

\begin{onlineOnly}
    \begin{center}
\desmosThreeD{4fuxj00cmn}{900}{600}
\end{center}
\end{onlineOnly}


(b) Then find the coordinates of the point on ${\cal L}$ closest to the point $(x,y,z)$.

(c) Then find an expression for the function $f(x,y,z)$ that gives the distance from $(x,y,z)$ to the line ${\cal L}$.

(d) Use the result of part (c) to write an equation of the cylinder. Input the equation on Line 6 of the desmos demonstration to check.
 
\end{question}


\begin{question}  \label{Q:343w4ffer}
First try to visualize or sketch each of the following surfaces \emph{in space}. Then use algebra to describe each set. Then use desmos to help 
visualize each sets.

(a) The set of points equidistant from the $x$-axis and the $z$-axis.

(b) The set of points twice as far from the $x$-axis as from the $z$-axis. Drag the slider $s$ below to help visualize this surface. But first explain what the demonstration is doing.

\begin{onlineOnly}
    \begin{center}
\desmosThreeD{jhtdrue4nx}{900}{600}
\end{center}
\end{onlineOnly}

(c) The set of points twice as far from the origin as from the $y$-axis.

(d) The set of points equidistant from the $y$-axis and the point $(0,0,3)$.

(e) The set of points twice as far from the $x$-axis as from the point $(0,0,3)$. Drag the slider $s$ below to help visualize this surface. But first explain what the demonstration is doing.

\begin{onlineOnly}
    \begin{center}
\desmosThreeD{25zfzyniuc}{900}{600}
\end{center}
\end{onlineOnly}

(f) The set of points twice as far from the point $(0,0,3)$ as from the $x$-axis.

(g) The set of points equidistant from the $xy$-plane and the point $(0,0,3)$.

(h) The set of points twice as far from the $xy$-plane as from the point $(0,0,-3)$.

(i) The set of points twice as far from the the point $(0,0,-3)$ as from the $xy$-plane.

(j) The set of points in space twice as far from the origin as from the point $B(4,2,6)$.

\end{question}


\begin{question} \label{Qdfrr3rf}
Find an equation the elliptical cylinder parallel to the $z$-axis that passes through the circle where the spheres
\[
    x^2 + y^2 + z^2 = 9
\]
and
\[
   (x-1)^2 + (y-2)^2 + (z-3)^2 = 4
\]
intersect.

An equation is
\[
       x^2 + y^2 + \answer{(\frac{-19+2x+4y}{6})^2} = \answer{9}
\]
\end{question}


\section*{Center Sets, Part 2}
A sphere tangent to the $xy$-plane passes through the points $A(3,4,1)$ and $B(1,2,5)$ 

(a) How many degrees of freedom do we have in describing such a sphere?

(b) Describe geometrically and algebraically the set of all possible tangency points.

(c) Parameterize the family of all such spheres.




\section*{Sets of Points Defined by Equations}

It is important to be aware of the ambient space when drawing or describing a set of points. For example, the graph of the set of points satisfying the equation 
\[
    x^2 + y^2 = 1
\]
depends on whether we plot the curve
\[
   \{  (x,y) \in\mathbb{R}^2 \, | \, x^2 + y^2 = 1 \}
\]
in two dimensions or the surface
\[
   \{  (x,y,z) \in\mathbb{R}^3 \, | \, x^2 + y^2 = 1 \}
\]
in three dimensions. The curve in $\mathbb{R}^2$ is a circle of unit radius centered at the origin, but the surface in $\mathbb{R}^3$ is a circular cylinder of units radius symmetric about the $z$-axis.

Keep this idea in mind when answering the following questions.

\begin{question} \label{Q1:Coordinates}
For each of the following sets, do the following:

\begin{itemize}
\item{Describe the set geometrically.}

\item{Sketch the set.}

\end{itemize}

(a) $\{  (x,y)\in \mathbb{R}^2 \, | \, y=3  \}$

(b) $\{  (x,y,z)\in \mathbb{R}^3 \, | \, y=3  \}$

(c) $\{  (x,y)\in \mathbb{R}^2 \, | \, y=2x  \}$

(d) $\{  (x,y,z)\in \mathbb{R}^3 \, | \, y=2x  \}$

(e) $\{  (x,y)\in \mathbb{R}^2 \, | \,  x^2 + y^2 = 6x - 8y  \}$  \hskip 0.2 in {\it Hint:} Complete the square.

(f)  $\{  (x,y,z)\in \mathbb{R}^3 \, | \, x^2 + y^2 = 6x - 8y  \}$

(g) $\{  (x,y,z)\in \mathbb{R}^3 \, | \, x^2 +4z^2 = 6x - 8z  \}$

(h) $\{  (x,y,z)\in \mathbb{R}^3 \, | \, y+z^2= 0  \}$

\end{question}

The surfaces in parts (b), (d), (f), and (g) above are all \emph{cylinders}, even the plane of part (b). A cylinder is a collection of parallel lines and need {\bf not} be circlular.


\section*{More Questions about Spheres}


\begin{question} \label{Q522:Coordinates}
(a) Find equations of two spheres with different radii that do not intersect. Explain your reasoning.

(b) Find an equation of a third sphere that does not intersect either of the two spheres in part (a). Explain your reasoning.


\begin{hint}
Experiment with the sliders and drag the center of the green circle to help find conditions that determine when two circles 

i) do not intersect,

(ii) are externally tangent,

(iii) are internally tangent, or

(iv) intersect in two points.

\pdfOnly{
Access Desmos interactives through the online version of this text at
 
\href{https://www.desmos.com/xzxjw41xxb}.
}
 
\begin{onlineOnly}
    \begin{center}
\desmos{xzxjw41xxb}{900}{600}
\end{center}
\end{onlineOnly}
\end{hint}




\end{question}





\begin{question} \label{Q52:Coordinates}
(a) Find equations of two spheres with different radii that are tangent to each other. Explain your reasoning.

(b) Find an equation of a third sphere that is tangent to both spheres in part (a). Explain your reasoning.
\end{question}


\begin{question} \label{Q53:Coordinates}
(a) Find equations of two spheres with different radii that intersect in a circle that does not lie in a plane parallel to any of the coordinate planes. Explain your reasoning.

(b) Find an equation of a third sphere that passes through the circle in part (a). Explain your reasoning. 

\begin{hint} 
Starting with equations 
\[
      x^2 + y^2 - r_1^2 = 0
\]
and 
\[
    (x-a)^2 +y^2 - r_2^2 = 0 ,
\]
of two circles and two real numbers $m$ and $n$, we can write an equation of a third circle as
\[
          m(x^2 + y^2 - r_1^2) + n((x-a)^2 +y^2 - r_2^2) = 0.
\]

Experiment with the sliders $m$ and $n$ below. What do you notice about the third (blue) circle? What happens if the original two circles (green and red) do not intersect?


\pdfOnly{
Access Desmos interactives through the online version of this text at
 
\href{https://www.desmos.com/5kamulcwua}.
}
 
\begin{onlineOnly}
    \begin{center}
\desmos{5kamulcwua}{900}{600}
\end{center}
\end{onlineOnly}
\end{hint}


\end{question}


\begin{question} \label{Q533:Coordinates}
(a) Explain why the spheres
\[
     x^2 + y^2 + z^2 = 25
\]
and
\[
    (x-2)^2 + (y-1)^2 + (z-2)^2 = 16
\]
intersect in a circle.

(b) Use trigonometry to find the radius of the circle  in part (a). Do {\bf not} use a calculator. Explain your reasoning. \it{Hint:} Draw a triangle with two vertices at the centers of the spheres and the third vertex on the circle.

\end{question}





\section{Spheres and Cylinders}
\begin{question}   \label{Q87:Coordinates}
(a) Find equations of the spheres with radii $r_1$, $r_2>0$, and respective centers $(0,0,r_1)$ and $(a,0,r_2)$

(b) Write an inequality involving $r_1$, $r_2$, and $a$ that is satisfied if and only if the spheres intersect in a circle.

(c) Assuming the spheres intersect in a circle  ${\cal C}$, find 

\hskip 0.3 in (i) an equation of the plane through  ${\cal C}$,

\hskip 0.3 in (ii) an equation of the cylinder through  ${\cal C}$ that is parallel to the $z$-axis, and

\hskip 0.3 in (iii) an equation of the cylinder through  ${\cal C}$ that is parallel to the $x$-axis.

Use the link below and follow the directions there to check if you are correct. Then experiment with the sliders as directed. What do you notice? Anything surprising?

\href{https://www.desmos.com/3d/93aea43a95}{Intersecting Spheres and Cylinders}

\end{question}



\begin{question}   \label{Q88:Coordinates}
(a) Find an equation of the sphere with radius $r_1$ centered at the origin.

(b) Find an equation of the right circular cylinder of radius $r_1/2$ symmetric about the line through the point $(a,0,0)$ parallel to the $z$-axis.

(c) Assuming the sphere and cylinder intesrect in a curve  ${\cal C}$, find 


\hskip 0.3 in (i) an equation of the cylinder through  ${\cal C}$ that is parallel to the $y$-axis, and

\hskip 0.3 in (ii) an equation of the cylinder through  ${\cal C}$ that is parallel to the $x$-axis.

Use the link below and follow the directions there to check if you are correct. Then experiment with the sliders as directed. What do you notice? Anything surprising?

\href{https://www.desmos.com/3d/63f35399e3}{Intersecting Spheres and Cylinders}

\end{question}


\section{Spheres in a Corner}

\begin{question} \label{Q3:Coordinates}
(a) Find an equation of a sphere with radius $r$ in the first octant that is tangent to the coordinate planes $x=0$, $y=0$, and $z=0$.

(b) Find equations of all spheres in the first octant that pass through the point P with coordinates $(a,b,c)$ and are tangent to the  planes $x=0$, $y=0$, and $z=0$. Assume $a,b,c \geq 0$.

(c) Under what conditions on $a,b,c$ are there two such spheres (in part (b))? A unique sphere? No such sphere?

(d) Find the coordinates of three points in the first octant that respectively lie on two spheres tangent to the coordinate planes, on exactly one such sphere, and on no such sphere.

(e) Fix a sphere ${\cal S}$ of radius $r$ in the first octant tangent to the three coordinate planes. Color red all points on ${\cal S}$ that lie on a second sphere with radius less than $r$ that is also tangent to the three coordinate planes. Color blue all points on ${\cal S}$ that lie on a second such sphere with radius greater than $r$.

(i) Are any point of ${\cal S}$ that are left uncolored? If so, which points?

(ii) What fraction of ${\cal S}$ is colored red?



\pdfOnly{
Access Desmos interactives through the online version of this text at
 
\href{https://www.www.geogebra.org/3d/fbaa38f918}.
}
 
\begin{onlineOnly}
    \begin{center}
\geogebra{d2xv3dwc}{900}{600}
\end{center}
\end{onlineOnly}

\end{question}


\pdfOnly{
Access Desmos interactives through the online version of this text at
 
\href{https://www.desmos.com/3d/fbaa38f918}.
}
 
\begin{onlineOnly}
    \begin{center}
\geogebra{jsqe5guf}{900}{600}
\end{center}
\end{onlineOnly}


\section{Discussion Questions}

\begin{question} \label{Q4934:Space}
(a) Can a thin wooden dowel 15 inches long fit (without bending) in a shoe box with dimensions $4"\times 8"\times 12"$?

(b) What do we mean by a sphere of radius 4 feet centered at the point $(3,2,1) feet$? Find an equation of such a sphere and explain how the equation encodes the meaning.

(c) What do we mean by a right circular cylinder of radius 4 feet symmetric about the line parallel to the $y$-axis through the point $(1,2,3)$ feet? Find an equation of such a cylinder and explain how the equation encodes the meaning. Think in terms of distance.

(d) Find equations of two spheres with different radii centered at the origin and the point $(2,3,,6)$ that are externally tangent to each other. Then find an equation of a third sphere tangent to the first two and passing through their point of tangency.

(e) Find equations of two spheres with different radii centered at the origin and the point $(2,3,,6)$ that are internally tangent to each other. Then find an equation of a third sphere tangent to the first two and passing through their point of tangency.

(f) Find equations of two spheres with different radii centered at the origin and the point $(2,3,,6)$ that intersect in a circle of radius 4. Then find an equation of a third sphere through that circle.

(g) Describe the set of centers of all spheres through the points $(2,1,3)$ and $(6,5,-1)$.

(h) Describe the set of centers of all spheres with radius $10$ through the points $(2,1,3)$ and $(6,5,-1)$.

(i) Describe the set of centers of all spheres through the points $(2,1,3)$ , $(6,5,-1)$, and $(0,-4,8)$.

\end{question}





\section{Surfaces Defined by Distance Relations}

\begin{question} \label{Q2:Coordinates}
Describe the following sets of points algebraically and geometrically. Then graph each set. Use desmos to help with your sketches as necessary.

Complete the square for (b), (c), (d), (f).

(a) The set of points equidistant from the $xy$-plane and the point $F(0,0,3)$.

(b) The set of points twice as far from the $xy$-plane as from the point $F(0,0,3)$.

(c) The set of points twice as far from the point $F(0,0,3)$ as from the $xy$-plane.

%(b) The set of points twice as far from $A(0,0,0)$ as from $B(2,4,6)$.

(d) The set of points twice as far from the origin as from the $xy$-plane.

(e) The set of points equidistant from the $x$ and $z$ axes.

(f) The set of points $k$ times as far from the $x$-axis as from the $z$-axis. Explore how the graph changes with $k$. Assume $k>0$.

\end{question}



While this chapter is about coordinates in three-dimensions, we'll start with the number line in one dimension.

\begin{question}  \label{Q0:Coordinates}
What is the distance between the real number $x$ and zero?
\begin{selectAll}
   \choice{$x$}
   \choice{$\pm x$}
   \choice[correct]{$\sqrt{x^2}$}
    \choice[correct]{$|x|$}
\end{selectAll}
\end{question}

\begin{question}  \label{Q00:Coordinates}
What is the distance between the real numbers $x=a$ and $x=b$?
\begin{selectAll}
   \choice[correct]{$|b-a|$}
   \choice{$\pm |b-a|$}
   \choice[correct]{$\sqrt{(a-b)^2}$}
    \choice[correct]{$|b|-|a|$}
\end{selectAll}
\end{question}

\begin{question}  \label{Q50:Coordinates}
Write an equation you would solve to find all real numbers $x$ that are twice as far from $3$ as from $11$.
\begin{selectAll}
   \choice{$2|x-3| = |x-11|$}
   \choice{$x-3 =  2(x-11)$}
   \choice[correct]{$|x-3| = 2|x-11|$}
    \choice[correct]{$\sqrt{(x-3)^2} = 2\sqrt{(x-11)^2}$}
\end{selectAll}
\end{question}


\begin{question}  \label{Q50:Coordinates}
Write an equation that desribes the set of points $(x,y,z)\in \mathbb{R}^3$ that are twice as far from the origin as from the point $(6,0,3)$.
\begin{selectAll}
   \choice{$|x| = 2|x-6|$}
   \choice[correct]{$\sqrt{x^2+y^2+z^2} = 2\sqrt{(x-6)^2 + y^2+ (z-3)^2}$}
   \choice{$2\sqrt{x^2+y^2+z^2} = \sqrt{(x-6)^2 + y^2+ (z-3)^2}$}
\end{selectAll}
\end{question}

\begin{question}  \label{Q51:Coordinates}
Do some algebra and then describe the set of points in the previous question geometrically. Be precise. \it{Hint:} Complete the square.
\end{question}





\begin{question} \label{Q1:Coordinates}
For each of the following sets, do the following:

\begin{itemize}
\item{Describe the set geometrically.}

\item{Sketch the set.}

\end{itemize}

(a) $\{  (x,y)\in \mathbb{R}^2 \, | \, y=3  \}$

(b) $\{  (x,y,z)\in \mathbb{R}^3 \, | \, y=3  \}$

(c) $\{  (x,y)\in \mathbb{R}^2 \, | \, y=2x  \}$

(d) $\{  (x,y,z)\in \mathbb{R}^3 \, | \, y=2x  \}$

(e) $\{  (x,y)\in \mathbb{R}^2 \, | \,  x^2 + y^2 = 6x - 8y  \}$

(f)  $\{  (x,y,z)\in \mathbb{R}^3 \, | \, x^2 + y^2 = 6x - 8y  \}$

\end{question}






A try with vectors.

\pdfOnly{
Access Desmos interactives through the online version of this text at
 
\href{https://www.desmos.com/3d/fbaa38f918}.
}
 
\begin{onlineOnly}
    \begin{center}
\geogebra{jhgvamja}{900}{600}
\end{center}
\end{onlineOnly}

jhgvamja

\end{document}
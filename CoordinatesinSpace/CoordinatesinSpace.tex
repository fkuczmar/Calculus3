\documentclass{ximera}
\title{Coordinates in Space}

\newcommand{\pskip}{\vskip 0.1 in}

\begin{document}
\begin{abstract}
Describing sets of points in $\mathbb{R}^3$.
\end{abstract}
\maketitle

%\pdfOnly{
%Access Desmos interactives through the online version of this text at
 
%\href{https://www.www.geogebra.org/classic/e6rsvnsz}.
%}
 
%\begin{onlineOnly}
%    \begin{center}
%\geogebra{e6rsvnsz}{900}{600}
%\end{center}
%\end{onlineOnly}



%The coordinates of a point on the sphere of radius $r$ with colatitude $\phi$ and longitude $\theta$ are
%\[
%   (x,y,z) = (\cos\theta \sin\phi, \sin\theta \sin \phi, \cos\phi ) \, , \, 0\leq \theta \leq 2\pi, \, 0\leq \phi \leq \pi .
%\]
%\begin{onlineOnly}
%    \begin{center}
%\geogebra{x48cgxbg}{900}{600}
%\end{center}
%\end{onlineOnly}

\section*{Computing Distances in Space}
Suppose we wish to compute the distance between the points $P_1$ and $P_2$ in space with respective coordinates $(x_1, y_1, z_1)$ and $(x_2, y_2, z_2)$. The idea is to draw a box with its edges parallel to the coordinate axes and with the points $P_1$ and $P_2$ as opposite vertices. The 12 edges of this box come in three sets of four, the four in each set being parallel to the coordinate axes. 

\begin{question} \label{Q324tgtgt}
The four edges parallel to the $x$-axis have length $\answer{|x_1-x_2 |}$.
\end{question}


\section{Cylinders}

It is important to be aware of the ambient space when drawing or describing a set of points. For example, the graph of the set of points satisfying the equation 
\[
    x^2 + y^2 = 1
\]
depends on whether we plot the curve
\[
   \{  (x,y) \in\mathbb{R}^2 \, | \, x^2 + y^2 = 1 \}
\]
in two dimensions or the surface
\[
   \{  (x,y,z) \in\mathbb{R}^3 \, | \, x^2 + y^2 = 1 \}
\]
in three dimensions. The curve in $\mathbb{R}^2$ is a circle of unit radius centered at the origin, but the surface in $\mathbb{R}^3$ is a circular cylinder of units radius symmetric about the $z$-axis.

Keep this idea in mind when answering the following questions.

\begin{question} \label{Q1:Coordinates}
For each of the following sets, do the following:

\begin{itemize}
\item{Describe the set geometrically.}

\item{Sketch the set.}

\end{itemize}

(a) $\{  (x,y)\in \mathbb{R}^2 \, | \, y=3  \}$

(b) $\{  (x,y,z)\in \mathbb{R}^3 \, | \, y=3  \}$

(c) $\{  (x,y)\in \mathbb{R}^2 \, | \, y=2x  \}$

(d) $\{  (x,y,z)\in \mathbb{R}^3 \, | \, y=2x  \}$

(e) $\{  (x,y)\in \mathbb{R}^2 \, | \,  x^2 + y^2 = 6x - 8y  \}$  \hskip 0.2 in {\it Hint:} Complete the square.

(f)  $\{  (x,y,z)\in \mathbb{R}^3 \, | \, x^2 + y^2 = 6x - 8y  \}$

(g) $\{  (x,y,z)\in \mathbb{R}^3 \, | \, x^2 +4z^2 = 6x - 8z  \}$

(h) $\{  (x,y,z)\in \mathbb{R}^3 \, | \, y+z^2= 0  \}$

\end{question}

The surfaces in parts (b), (d), (f), and (g) above are all \emph{cylinders}, even the plane of part (b). A cylinder is a collection of parallel lines and need {\bf not} be circlular.


\section{Spheres}
A sphere in $\mathbb{R}^3$ is a set of points a fixed distance from a given point. The fixed distance is the sphere's radius and the given point is the sphere's center.




\begin{question} \label{Q522:Coordinates}
(a) Find equations of two spheres with different radii that do not intersect. Explain your reasoning.

(b) Find an equation of a third sphere that does not intersect either of the two spheres in part (a). Explain your reasoning.


\begin{hint}
Experiment with the sliders and drag the center of the green circle to help find conditions that determine when two circles 

i) do not intersect,

(ii) are externally tangent,

(iii) are internally tangent, or

(iv) intersect in two points.

\pdfOnly{
Access Desmos interactives through the online version of this text at
 
\href{https://www.desmos.com/xzxjw41xxb}.
}
 
\begin{onlineOnly}
    \begin{center}
\desmos{xzxjw41xxb}{900}{600}
\end{center}
\end{onlineOnly}
\end{hint}




\end{question}





\begin{question} \label{Q52:Coordinates}
(a) Find equations of two spheres with different radii that are tangent to each other. Explain your reasoning.

(b) Find an equation of a third sphere that is tangent to both spheres in part (a). Explain your reasoning.
\end{question}


\begin{question} \label{Q53:Coordinates}
(a) Find equations of two spheres with different radii that intersect in a circle that does not lie in a plane parallel to any of the coordinate planes. Explain your reasoning.

(b) Find an equation of a third sphere that passes through the circle in part (a). Explain your reasoning. 

\begin{hint} 
Starting with equations 
\[
      x^2 + y^2 - r_1^2 = 0
\]
and 
\[
    (x-a)^2 +y^2 - r_2^2 = 0 ,
\]
of two circles and two real numbers $m$ and $n$, we can write an equation of a third circle as
\[
          m(x^2 + y^2 - r_1^2) + n((x-a)^2 +y^2 - r_2^2) = 0.
\]

Experiment with the sliders $m$ and $n$ below. What do you notice about the third (blue) circle? What happens if the original two circles (green and red) do not intersect?


\pdfOnly{
Access Desmos interactives through the online version of this text at
 
\href{https://www.desmos.com/5kamulcwua}.
}
 
\begin{onlineOnly}
    \begin{center}
\desmos{5kamulcwua}{900}{600}
\end{center}
\end{onlineOnly}
\end{hint}


\end{question}


\begin{question} \label{Q533:Coordinates}
(a) Explain why the spheres
\[
     x^2 + y^2 + z^2 = 25
\]
and
\[
    (x-2)^2 + (y-1)^2 + (z-2)^2 = 16
\]
intersect in a circle.

(b) Use trigonometry to find the radius of the circle  in part (a). Do {\bf not} use a calculator. Explain your reasoning. \it{Hint:} Draw a triangle with two vertices at the centers of the spheres and the third vertex on the circle.

\end{question}





\section{Spheres and Cylinders}
\begin{question}   \label{Q87:Coordinates}
(a) Find equations of the spheres with radii $r_1$, $r_2>0$, and respective centers $(0,0,r_1)$ and $(a,0,r_2)$

(b) Write an inequality involving $r_1$, $r_2$, and $a$ that is satisfied if and only if the spheres intersect in a circle.

(c) Assuming the spheres intersect in a circle  ${\cal C}$, find 

\hskip 0.3 in (i) an equation of the plane through  ${\cal C}$,

\hskip 0.3 in (ii) an equation of the cylinder through  ${\cal C}$ that is parallel to the $z$-axis, and

\hskip 0.3 in (iii) an equation of the cylinder through  ${\cal C}$ that is parallel to the $x$-axis.

Use the link below and follow the directions there to check if you are correct. Then experiment with the sliders as directed. What do you notice? Anything surprising?

\href{https://www.desmos.com/3d/93aea43a95}{Intersecting Spheres and Cylinders}

\end{question}



\begin{question}   \label{Q88:Coordinates}
(a) Find an equation of the sphere with radius $r_1$ centered at the origin.

(b) Find an equation of the right circular cylinder of radius $r_1/2$ symmetric about the line through the point $(a,0,0)$ parallel to the $z$-axis.

(c) Assuming the sphere and cylinder intesrect in a curve  ${\cal C}$, find 


\hskip 0.3 in (i) an equation of the cylinder through  ${\cal C}$ that is parallel to the $y$-axis, and

\hskip 0.3 in (ii) an equation of the cylinder through  ${\cal C}$ that is parallel to the $x$-axis.

Use the link below and follow the directions there to check if you are correct. Then experiment with the sliders as directed. What do you notice? Anything surprising?

\href{https://www.desmos.com/3d/63f35399e3}{Intersecting Spheres and Cylinders}

\end{question}


\section{Spheres in a Corner}

\begin{question} \label{Q3:Coordinates}
(a) Find an equation of a sphere with radius $r$ in the first octant that is tangent to the coordinate planes $x=0$, $y=0$, and $z=0$.

(b) Find equations of all spheres in the first octant that pass through the point P with coordinates $(a,b,c)$ and are tangent to the  planes $x=0$, $y=0$, and $z=0$. Assume $a,b,c \geq 0$.

(c) Under what conditions on $a,b,c$ are there two such spheres (in part (b))? A unique sphere? No such sphere?

(d) Find the coordinates of three points in the first octant that respectively lie on two spheres tangent to the coordinate planes, on exactly one such sphere, and on no such sphere.

(e) Fix a sphere ${\cal S}$ of radius $r$ in the first octant tangent to the three coordinate planes. Color red all points on ${\cal S}$ that lie on a second sphere with radius less than $r$ that is also tangent to the three coordinate planes. Color blue all points on ${\cal S}$ that lie on a second such sphere with radius greater than $r$.

(i) Are any point of ${\cal S}$ that are left uncolored? If so, which points?

(ii) What fraction of ${\cal S}$ is colored red?



\pdfOnly{
Access Desmos interactives through the online version of this text at
 
\href{https://www.www.geogebra.org/3d/fbaa38f918}.
}
 
\begin{onlineOnly}
    \begin{center}
\geogebra{d2xv3dwc}{900}{600}
\end{center}
\end{onlineOnly}

\end{question}


\pdfOnly{
Access Desmos interactives through the online version of this text at
 
\href{https://www.desmos.com/3d/fbaa38f918}.
}
 
\begin{onlineOnly}
    \begin{center}
\geogebra{jsqe5guf}{900}{600}
\end{center}
\end{onlineOnly}


\section{Discussion Questions}

\begin{question} \label{Q4934:Space}
(a) Can a thin wooden dowel 15 inches long fit (without bending) in a shoe box with dimensions $4"\times 8"\times 12"$?

(b) What do we mean by a sphere of radius 4 feet centered at the point $(3,2,1) feet$? Find an equation of such a sphere and explain how the equation encodes the meaning.

(c) What do we mean by a right circular cylinder of radius 4 feet symmetric about the line parallel to the $y$-axis through the point $(1,2,3)$ feet? Find an equation of such a cylinder and explain how the equation encodes the meaning. Think in terms of distance.

(d) Find equations of two spheres with different radii centered at the origin and the point $(2,3,,6)$ that are externally tangent to each other. Then find an equation of a third sphere tangent to the first two and passing through their point of tangency.

(e) Find equations of two spheres with different radii centered at the origin and the point $(2,3,,6)$ that are internally tangent to each other. Then find an equation of a third sphere tangent to the first two and passing through their point of tangency.

(f) Find equations of two spheres with different radii centered at the origin and the point $(2,3,,6)$ that intersect in a circle of radius 4. Then find an equation of a third sphere through that circle.

(g) Describe the set of centers of all spheres through the points $(2,1,3)$ and $(6,5,-1)$.

(h) Describe the set of centers of all spheres with radius $10$ through the points $(2,1,3)$ and $(6,5,-1)$.

(i) Describe the set of centers of all spheres through the points $(2,1,3)$ , $(6,5,-1)$, and $(0,-4,8)$.

\end{question}





\section{Surfaces Defined by Distance Relations}

\begin{question} \label{Q2:Coordinates}
Describe the following sets of points algebraically and geometrically. Then graph each set. Use desmos to help with your sketches as necessary.

Complete the square for (b), (c), (d), (f).

(a) The set of points equidistant from the $xy$-plane and the point $F(0,0,3)$.

(b) The set of points twice as far from the $xy$-plane as from the point $F(0,0,3)$.

(c) The set of points twice as far from the point $F(0,0,3)$ as from the $xy$-plane.

%(b) The set of points twice as far from $A(0,0,0)$ as from $B(2,4,6)$.

(d) The set of points twice as far from the origin as from the $xy$-plane.

(e) The set of points equidistant from the $x$ and $z$ axes.

(f) The set of points $k$ times as far from the $x$-axis as from the $z$-axis. Explore how the graph changes with $k$. Assume $k>0$.

\end{question}



While this chapter is about coordinates in three-dimensions, we'll start with the number line in one dimension.

\begin{question}  \label{Q0:Coordinates}
What is the distance between the real number $x$ and zero?
\begin{selectAll}
   \choice{$x$}
   \choice{$\pm x$}
   \choice[correct]{$\sqrt{x^2}$}
    \choice[correct]{$|x|$}
\end{selectAll}
\end{question}

\begin{question}  \label{Q00:Coordinates}
What is the distance between the real numbers $x=a$ and $x=b$?
\begin{selectAll}
   \choice[correct]{$|b-a|$}
   \choice{$\pm |b-a|$}
   \choice[correct]{$\sqrt{(a-b)^2}$}
    \choice[correct]{$|b|-|a|$}
\end{selectAll}
\end{question}

\begin{question}  \label{Q50:Coordinates}
Write an equation you would solve to find all real numbers $x$ that are twice as far from $3$ as from $11$.
\begin{selectAll}
   \choice{$2|x-3| = |x-11|$}
   \choice{$x-3 =  2(x-11)$}
   \choice[correct]{$|x-3| = 2|x-11|$}
    \choice[correct]{$\sqrt{(x-3)^2} = 2\sqrt{(x-11)^2}$}
\end{selectAll}
\end{question}


\begin{question}  \label{Q50:Coordinates}
Write an equation that desribes the set of points $(x,y,z)\in \mathbb{R}^3$ that are twice as far from the origin as from the point $(6,0,3)$.
\begin{selectAll}
   \choice{$|x| = 2|x-6|$}
   \choice[correct]{$\sqrt{x^2+y^2+z^2} = 2\sqrt{(x-6)^2 + y^2+ (z-3)^2}$}
   \choice{$2\sqrt{x^2+y^2+z^2} = \sqrt{(x-6)^2 + y^2+ (z-3)^2}$}
\end{selectAll}
\end{question}

\begin{question}  \label{Q51:Coordinates}
Do some algebra and then describe the set of points in the previous question geometrically. Be precise. \it{Hint:} Complete the square.
\end{question}





\begin{question} \label{Q1:Coordinates}
For each of the following sets, do the following:

\begin{itemize}
\item{Describe the set geometrically.}

\item{Sketch the set.}

\end{itemize}

(a) $\{  (x,y)\in \mathbb{R}^2 \, | \, y=3  \}$

(b) $\{  (x,y,z)\in \mathbb{R}^3 \, | \, y=3  \}$

(c) $\{  (x,y)\in \mathbb{R}^2 \, | \, y=2x  \}$

(d) $\{  (x,y,z)\in \mathbb{R}^3 \, | \, y=2x  \}$

(e) $\{  (x,y)\in \mathbb{R}^2 \, | \,  x^2 + y^2 = 6x - 8y  \}$

(f)  $\{  (x,y,z)\in \mathbb{R}^3 \, | \, x^2 + y^2 = 6x - 8y  \}$

\end{question}






A try with vectors.

\pdfOnly{
Access Desmos interactives through the online version of this text at
 
\href{https://www.desmos.com/3d/fbaa38f918}.
}
 
\begin{onlineOnly}
    \begin{center}
\geogebra{jhgvamja}{900}{600}
\end{center}
\end{onlineOnly}

jhgvamja

\end{document}
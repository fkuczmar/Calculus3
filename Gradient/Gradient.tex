\documentclass{ximera}
\title{The Gradient Vector}

\newcommand{\pskip}{\vskip 0.1 in}

\begin{document}
\begin{abstract}
The gradient vector and its properties
\end{abstract}
\maketitle

\section{Working in Two Dimensions}

\begin{question}  \label{Qert4thbhg}
Suppose $z=f(x,y)$ is a differentiable function that gives the temperature (in Celsius degrees) at the point with coordinates $(x,y)$, measured in meters.

Suppose that 
\[
    \Big|  \nabla f(3,5) \Big| = 2 .
\]

and that the vector ${\bf v}$ makes the angle $\theta$ with $\nabla f(3,5)$.

(a) Find an expression for the rate at which the temperature changes at $P$ in the direction of ${\bf v}$. Give a geometric explanation for this expression. No computations.

(b) Suppose the temperature at another point $Q$ increases at the rate of $3^\circ$C/m in the direction of the vector $\langle 2,1\rangle$ and decreases at the rate of $3^\circ$C/m in the direction of the vector $\langle -3,4\rangle$. In what direction does the temperature at $Q$ increase at the fastest rate? What is this maximum rate?

\end{question}


\begin{question}  \label{Qdr4356}
The figure below shows two level curves of the differentiable function $z=f(x,y)$ and the gradient of that function at point $C$. Sketch the gradient at point $D$. Explain your reasoning.

\pdfOnly{
Access Geogebra interactives through the online version of this text at
 
\href{https://www.geogebra.org/classic/zau4grcm}.
}
 
\begin{onlineOnly}
    \begin{center}
\geogebra{zau4grcm}{900}{600}
\end{center}
\end{onlineOnly}

Geogebra activity available at

\href{https://www.geogebra.org/classic/zau4grcm}{163: Gradient Exercise}
\end{question}


\begin{question} \label{Q34gvbty}
The function 
\[
 \sigma = f(x,y)
\]
gives the charge density (in $\mu C/m^2$) at the point with $(x,y)$ (measured in meters) on the surface of a thin conducting plate.

Two level curves of the function $f$ are shown below.

\pskip

\pdfOnly{
Access Geogebra interactives through the online version of this text at
 
\href{https://www.geogebra.org/classic/xnqmhyvd}.
}
 
\begin{onlineOnly}
    \begin{center}
\geogebra{xnqmhyvd}{900}{600}
\end{center}
\end{onlineOnly}

Geogebra activity available at

\href{https://www.geogebra.org/classic/xnqmhyvd}{163: Gradient Exercise 2}


Suppose point $C$ has coordinates $(2,5)$,
\[
    f(2,5) = 4,
\]
and that
\[
    \Big|  \langle \frac{\partial z}{\partial x}, \frac{\partial z}{\partial y} \rangle \Big|_{(x,y) =(2,5)} = 30
\]

\pskip

The gradient vector at $C$ is shown in the figure above. Consecutive marks along the line $CD$ are at intervals of length $1$cm. Answer the following questions \emph{without} imposing a coordinate system. Work with the vectors themselves, not with their components.

(a) What are the units of the above gradient? Explain its meaning.

(b) Approximate the charge density at $D$.

(c) Approximate the charge desity at $F$.



\end{question}


\section{Avoiding Implicit Differentiation}

\pskip

Set $z$ free

when wishing to find

a vector normal

to the surface

$f(x,y,z)=9$.

Don't differentiate implicitly,

kiss the chain rule goodbye,

just let 

$w=f(x,y,z)$

and forget about the nine.

$z$ is now free,

it's so simple you see,

just take the partials of $w$

with respect to $x$, $y$, and $z$.

Then you'll have found,

without spending much time,

a vector normal to the surface

$f(x,y,z) = 9$.

What if 

$f(x,y,z) = 10$?

Then do the same thing

all over again.


\pskip\pskip

\begin{question} \label{Qdfbtt44}
Find an equation of the tangent plane to the surface
\[
    xyz - yz^2 +3x^2 -y^2 = 2
\]
at the point $(1,1,1)$.
\end{question}


\begin{question}  \label{Qweregtt5}
A light at the point $(a,b,c)$ illuminates part of the paraboloid
\[
   z = x^2 + y^2.
\]
Parameterize the terminator. This is the curve on the surface that forms the boundary separating the illuminated and dark parts of the paraboloid. 

\end{question}


\end{document}
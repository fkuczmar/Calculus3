\documentclass{ximera}
\title{The Gradient Vector}

\newcommand{\pskip}{\vskip 0.1 in}

\begin{document}
\begin{abstract}
The gradient vector and its properties
\end{abstract}
\maketitle

\section{Working in Two Dimensions}


\begin{question}  \label{Qdr4356}
The figure below shows two level curves of the differentiable function $z=f(x,y)$ and the gradient of that function at point $C$. Sketch the gradient at point $D$. Explain your reasoning.

\pdfOnly{
Access Geogebra interactives through the online version of this text at
 
\href{https://www.geogebra.org/classic/zau4grcm}.
}
 
\begin{onlineOnly}
    \begin{center}
\geogebra{zau4grcm}{900}{600}
\end{center}
\end{onlineOnly}

Geogebra activity available at

\href{https://www.geogebra.org/classic/zau4grcm}{163: Gradient Exercise}
\end{question}


\begin{question} \label{Q34gvbty}
The function 
\[
 \sigma = f(x,y)
\]
gives the charge density (in $\mu C/m^2$) at the point with $(x,y)$ (measured in meters) on the surface of a thin conducting plate.

Two level curves of the function $f$ are shown below.

\pskip

\pdfOnly{
Access Geogebra interactives through the online version of this text at
 
\href{https://www.geogebra.org/classic/xnqmhyvd}.
}
 
\begin{onlineOnly}
    \begin{center}
\geogebra{xnqmhyvd}{900}{600}
\end{center}
\end{onlineOnly}

Geogebra activity available at

\href{https://www.geogebra.org/classic/xnqmhyvd}{163: Gradient Exercise 2}


Suppose in some coordinate system, point $C$ has coordinates $(2,5)$,
\[
    f(2,5) = 4,
\]
and
\[
    \Big|  \langle \frac{\partial z}{\partial x}, \frac{\partial z}{\partial y} \rangle \Big|_{(x,y) =(2,5)} = 30
\]

\pskip

The gradient vector at $C$ is shown in the figure above. Consecutive marks along the line $CD$ are at intervals of length $1$cm. Answer the following questions \emph{without} imposing a coordinate system. Work with the vectors themselves, not with their components.

(a) What are the units of the above gradient? Explain its meaning.

(b) Approximate the charge density at $D$.

(c) Approximate the charge desity at $F$.



\end{question}


\section{Avoiding Implicit Differentiation}


\end{document}
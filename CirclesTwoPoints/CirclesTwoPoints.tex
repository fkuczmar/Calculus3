\documentclass{ximera}
\title{Circles Through Two Points}


\newcommand{\pskip}{\vskip 0.1 in}

\begin{document}
\begin{abstract}
Using vectors to find equations of circles through two given points.
\end{abstract}
\maketitle




\begin{question} \label{QKDkfder33r}
Let $O$ be the origin and $A$ and $B$ distinct pionts in $\mathbb{R}^3$.

Let $P$ be the point on the segment $\overline{AB}$ such that 
\[
       \frac{AP}{PB} = \frac{3}{8} .
\]

\begin{enumerate}

\item Express the vector $\overrightarrow{OP}$ in terms of the vectors $\overrightarrow{OA}$ and $\overrightarrow{OB}$.

\item Check your work by entering the expression in Line 7 of the worksheet below. Rember to omit the $O$'s.

\end{enumerate}

\begin{onlineOnly}
    \begin{center}
\desmos{cjgf3k2a4k}{900}{600}
\end{center}
\end{onlineOnly}

\href{https://www.desmos.com/3d/cjgf3k2a4k}{163: Ratios}


\end{question}


\begin{question} \label{QPdfeff9vsz}
Find the components of all vectors perpendicular to the vector ${\bf v}=\langle 4,6 \rangle$ and having the same magnitude.
\end{question}


\begin{question} \label{QLDfmms}

This question is about using vectors to find equations of all circles in the plane through two given points having a given radius.

To be able to solve this problem in general, we need a way of finding a vector perpendicular to a given vector. We saw in the previous question, that an easy way to do this is to swap the components and negate one of them. Later in the class we'll learn how to do this with vector operations, but for now we'll use desmos to define a function $j$ that acts on a vector.


In desmos, we can do this by defining



\begin{onlineOnly}
    \begin{center}
\desmos{6tpc3cpshx}{900}{600}
\end{center}
\end{onlineOnly}

\href{https://www.desmos.com/calculator/6tpc3cpshx}{163: Circle Through Two Points}


\end{question}


\end{document}
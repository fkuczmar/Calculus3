\documentclass{ximera}
\title{Circles Through Two Points}


\newcommand{\pskip}{\vskip 0.1 in}

\begin{document}
\begin{abstract}
Using vectors to find equations of circles through two given points.
\end{abstract}
\maketitle




\begin{question} \label{QKDkfder33r}
Let $O$ be the origin and $A$ and $B$ distinct pionts in $\mathbb{R}^3$.

Let $P$ be the point on the segment $\overline{AB}$ such that 
\[
       \frac{|\overrightarrow{AP}|}{|\overrightarrow{PB}|} = \frac{3}{8} .
\]

\begin{enumerate}

\item Express the vector $\overrightarrow{OP}$ in terms of the vectors $\overrightarrow{OA}$ and $\overrightarrow{OB}$.

\item Check your work by entering the expression in Line 7 of the worksheet below. Rember to omit the $O$'s.

\end{enumerate}

\begin{onlineOnly}
    \begin{center}
\desmosThreeD{cjgf3k2a4k}{900}{600}
\end{center}
\end{onlineOnly}

\href{https://www.desmos.com/3d/cjgf3k2a4k}{163: Ratios}


\end{question}


\begin{question} \label{QPdfeff9vsz}
Find the components of all vectors perpendicular to the vector ${\bf v}=\langle 4,6 \rangle$ with the same magnitude.
\end{question}


\begin{question} \label{QLDfmms}

This question is about using vectors to find equations of all circles in the plane through two given points with a given radius.

To be able to solve this problem in general, we need a way of finding a vector perpendicular to a given vector. We saw one way to do this in the previous question; just swap the components and negate one of them. Later in the class we'll learn how to do the same with vector operations, but for now we'll use desmos to define a function $j$ that acts on a vector ${\bf w}$ and returns the vector we get by rotating ${\bf w}$ counterclockwise $\pi/2$ radians.


\begin{onlineOnly}
    \begin{center}
\desmos{6tpc3cpshx}{900}{600}
\end{center}
\end{onlineOnly}

\href{https://www.desmos.com/calculator/6tpc3cpshx}{163: Circle Through Two Points}

Here is an outline of the method to find equations of the two circles through $A$ and $B$.

\begin{enumerate}
\item The first step is to think about the centers of all circles through the points $A$ and $B$. 

A point $P$ with coordinates $(x,y)$ is the center of a circle through $A$ and $B$ if and only if $P$ is equidistant (ie. equal distances) from $A$ and $B$, that is if and only if 
\[
 \left| \overrightarrow{AP} \right| =  \left| \overrightarrow{BP} \right|.
\]

So a vector equation of the center set of the circles through $A$ and $B$ is
\[
            \left|   \langle x,y \rangle - \overrightarrow{OA}   \right| =  \left|   \langle x,y \rangle - \overrightarrow{OB}  \right| .
\]

We'll show later in the course that this set is the perpendicular bisector of segment $\overrightarrow{AB}$. This is the line through the midpoint $M$ of $\overrightarrow{AB}$ that is perpendicular to $\overrightarrow{AB}$.

\item So the first step is to express the vector $\overrightarrow{OM}$ from the origin to $M$ in terms of the vectors $\overrightarrow{OA}$ and $\overrightarrow{OB}$.

Find such an expression and enter its demos-form in Line 13 above.

\item Next find an expression for the vector $\overrightarrow{OC_1}$ from the origin to center of one circle of radius $r$ through $A$ and $B$. This expression should be in terms of the vectors $\overrightarrow{OA}$, $\overrightarrow{OB}$, and ${\bf v}$. Here ${\bf v} = j(\overrightarrow{OB} - \overrightarrow{OA})$ is the vector we get by rotating $\overrightarrow{AB}$ counterclockwise $\pi/2$ radians.

Then enter the desmos version of this expression in Line 18. 

\item Next write a vector equation for the circle of radius $r$ (through $A$ and $B$) centered at $C_1$. Enter the desmos version in Line 20.

\item Repeat the last two steps for the second circle.

\item Drag the slider $r$ in Line 3 to check your work. 

\end{enumerate}


\end{question}


\end{document}
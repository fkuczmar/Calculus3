\documentclass{ximera}
\title{Introduction to Taylor Series}

\newcommand{\pskip}{\vskip 0.1 in}

\begin{document}
\begin{abstract}
Approximating functions with polynomials. Bounding the errors in the approximations. 
\end{abstract}
\maketitle


\section{Introduction: Two Motions and Their Polynomial Approximations}
\begin{question}  \label{QPfKKDC433445r}

\begin{enumerate}

\item Use the worksheet below to explore the polynomial approximations to the motion
\[
       {\bf p}(t) = \langle 0 ,  f(t)  \rangle \, , \, t\in \mathbb{R} ,
\]
near time $t=0$ seconds for the function
\[
    f(t) = 4 \sin \left( \frac{1}{2}t \right).
\]
The position is measured in meters and time $t$ in seconds.


\begin{onlineOnly}
    \begin{center}
\desmos{s23pohegip}{900}{600}
\end{center}
\end{onlineOnly}

\href{https://www.desmos.com/calculator/s23pohegip}{163: Motion and Taylor Series 1}


\item  Use the worksheet below to explore the polynomial approximations to the motion
\[
       {\bf p}(t) = \langle 0 ,  f(t)  \rangle \, , \, t \in \mathbb{R},
\]
near time $t=0$ seconds for the function
\[
    f(t) = 4 \arctan \left( \frac{1}{2}t \right).
\]
The position is measured in meters and time $t$ in seconds.

\begin{onlineOnly}
    \begin{center}
\desmos{wxbfrcfqch}{900}{600}
\end{center}
\end{onlineOnly}

\href{https://www.desmos.com/calculator/wxbfrcfqch}{163: Motion and Taylor Series 2}

\item Compare the behavior of the successive approximations to the motions in parts (a) and (b).
\begin{freeResponse}
\end{freeResponse}

\end{enumerate}
\end{question}


\section{Bounding the Error in Polynomial Approximations}



\end{document}
\documentclass{ximera}
\title{Vectors}

\newcommand{\pskip}{\vskip 0.1 in}

\begin{document}
\begin{abstract}
Vectors in $\mathbb{R}^2$ and  $\mathbb{R}^3$.
\end{abstract}
\maketitle

\section{Vectors as Position}

Position is relative. You describe where something is not in absolute terms but only in relation to something else.

We might say, for example, that Bellevue is 4 miles east and 3 miles north of Akron. Then in a rectangular coordinate system with the $x$-axis pointing due east, the $y$-axis pointing due north, and the coordinates measured in miles, we can represent the position of Bellvue (at point $B$) relative to Akron (at point $A$) by the vector
\[
   \overrightarrow{AB} = \langle 4, 3 \rangle .
\]
This vector tells how to get from point $A$ to point $B$. 


Suppose that in our rectangular coordinate sytstem Charleston (point $C$) has coordinates $(-5,-2)$ and Dayton (point $D$) has coordinates $(-1,1)$. Then to get from Charleston to Dayton we would travel east
\[
     ( -1 - (-5) ) \text{ miles} = 4 \text{ miles}
\]
and north
\[
     ( 1 - (-2)) \text{ miles} = 3\text{ miles} .
\]
So we can represent the position of Dayton relative to Charleston by the vector
\[
  \overrightarrow{CD} =       \langle -1 - (-5) , 1- (-2)  \rangle   =  \langle 4, 3 \rangle .
\] 

So the directions that tell us how to get from Akron to Bellevue also tell us how to get from Charleston to Dayton. The
vectors $\overrightarrow{AB}$ and $\overrightarrow{CD}$ are equal and we write 
\[
      \overrightarrow{AB} = \overrightarrow{CD} .
\]

To go the other way and get from Bellevue to Akron or from Dayton to Charleston, we would go 4 miles west and 3 miles south. So the  position of Akron relative to Bellevue, or of Charleston relative to Dayton, is given by the vector
\[ 
   \overrightarrow{BA} = - \overrightarrow{AB} = - \langle 4,3 \rangle =  \langle -4,-3 \rangle.
\]

Using the Pythagorean theorem, the distance between Akron and Bellvue is just the length of the vector $\overrightarrow{AB}$, written as
\[
    | \overrightarrow{AB}  | =  | \langle 3, 4 \rangle  | \text{ miles} = \sqrt{3^2 + 4^2} \text{miles} = 5 \text{ miles} .
\] 


\section{Adding and Subtracting Vectors}
Continuing with our example above, suppose that Frankfort (point $F$) is $2$ miles east and $5$ miles south of Bellevue. Then the vector
\[
    \overrightarrow{BF}  = \langle 2, -5\rangle
\]
gives the position of Frankfort relative to Bellvue. 


\begin{question} \label{Q233:Vectors}
(a) Give directions to get from Bellevue to Frankfort.

(b) Fill in the vector and its components that gives the position of Frankfort relative to Bellevue.
\[
       \overrightarrow{\answer{A} \answer{F}}    =  \langle  \answer{6} , \answer{-1}  \rangle 
\]

\end{question}


\begin{question}  \label{Q543:Vectors}
Let ${\bf v}$ be the vector that gives the position of point $G$ relative to point $K$ and ${\bf w}$ the vector that gives the position of $M$ relative to $K$. 

Express in terms of ${\bf v}$ and ${\bf w}$ each of the following vectors: 

(a) The position of $M$ relative to $G$:  $\overrightarrow{\bf \answer{v}} +  \overrightarrow{\bf \answer{w}} $

\end{question}






\pdfOnly{
Access Desmos interactives through the online version of this text at
 
\href{https://www.www.geogebra.org/classic/e6rsvnsz}.
}
 
\begin{onlineOnly}
    \begin{center}
\geogebra{e6rsvnsz}{900}{600}
\end{center}
\end{onlineOnly}


\end{document}
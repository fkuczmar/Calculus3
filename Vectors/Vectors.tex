\documentclass{ximera}
\title{Vectors}

\newcommand{\pskip}{\vskip 0.1 in}

\begin{document}
\begin{abstract}
Vectors in $\mathbb{R}^2$ and  $\mathbb{R}^3$.
\end{abstract}
\maketitle

\section{Vectors as Position}

Position is relative. You describe where something is not in absolute terms but only in relation to something else.

We might say, for example, that Bellevue is 4 miles east and 3 miles north of Akron. Then in a rectangular coordinate system with the $x$-axis pointing due east, the $y$-axis pointing due north, and the coordinates measured in miles, we can represent the position of Bellevue (at point $B$) relative to Akron (at point $A$) by the vector
\[
   \overrightarrow{AB} = \langle 4, 3 \rangle .
\]
This vector tells how to get from Akron to Bellevue. 


Suppose that in our rectangular coordinate sytstem Charleston (point $C$) has coordinates $(-5,-2)$ and Dayton (point $D$) has coordinates $(-1,1)$. Then to get from Charleston to Dayton we would travel east
\[
     ( -1 - (-5) ) \text{ miles} = 4 \text{ miles}
\]
and north
\[
     ( 1 - (-2)) \text{ miles} = 3\text{ miles} .
\]
So we can represent the position of Dayton relative to Charleston by the vector
\[
  \overrightarrow{CD} =       \langle -1 - (-5) , 1- (-2)  \rangle   =  \langle 4, 3 \rangle .
\] 

So the directions that tell us how to get from Akron to Bellevue also tell us how to get from Charleston to Dayton. The
vectors $\overrightarrow{AB}$ and $\overrightarrow{CD}$ are equal and we write 
\[
      \overrightarrow{AB} = \overrightarrow{CD} .
\]

To go the other way and get from Bellevue to Akron or from Dayton to Charleston, we would go 4 miles west and 3 miles south. So the  position of Akron relative to Bellevue, or of Charleston relative to Dayton, is given by the vector
\[ 
   \overrightarrow{BA} = - \overrightarrow{AB} = - \langle 4,3 \rangle =  \langle -4,-3 \rangle.
\]

The distance between Akron and Bellvue is just the length of the vector $\overrightarrow{AB}$, written as
\[
    | \overrightarrow{AB}  | =  | \langle 3, 4 \rangle  | \text{ miles} = \sqrt{3^2 + 4^2} \text{miles} = 5 \text{ miles} .
\] 


\section{Adding and Subtracting Vectors}

\begin{question}    \label{Q234234:Vectors}
Continuing with our example above, suppose that Frankfort (point $F$) is $2$ miles east and $5$ miles south of Bellevue. Then the vector
\[
    \overrightarrow{BF}  = \langle \answer{2}, \answer{-5}\rangle
\]
gives the position of Frankfort relative to Bellvue. 
\end{question}

\begin{question} \label{Q233:Vectors}
(a) Give directions to get from Akron to Frankfort.

(b) Fill in the vector and its components that gives the position of Franfort relative to Akron.
\[
       \overrightarrow{\answer{A} \answer{F}}    =  \langle  \answer{6} , \answer{-1}  \rangle 
\]

(c) Which of the following are correct? Choose all that apply.
\begin{selectAll}  
\choice[correct]{$\overrightarrow{AF}   = \overrightarrow{AB} + \overrightarrow{BF}$}  
\choice{$\overrightarrow{AF}   = \overrightarrow{AB} - \overrightarrow{BF} $}  
\choice[correct]{$\overrightarrow{AF}   = \overrightarrow{BF} + \overrightarrow{AB} $}  
\end{selectAll} 


\end{question}


\begin{question}  \label{Q543:Vectors}
Let ${\bf v}$ be the vector that gives the position of point $K$ relative to point $G$ and ${\bf w}$ the vector that gives the position of $M$ relative to $K$. 

(a) Express in terms of ${\bf v}$ and ${\bf w}$ each of the following vectors: 

(i) The position of $M$ relative to $G$:  $\overrightarrow{\bf \answer{v}} +  \overrightarrow{\bf \answer{w}} $

(ii) The position of $G$ relative to $M$:  $\overrightarrow{\bf \answer{-v}} +  \overrightarrow{\bf \answer{-w}} $

\pskip

(b) Draw vectors ${\bf v}$ and ${\bf w}$ tail-to-tail. Any choices will do, just make sure the vectors are not parallel. Then draw the vector ${\bf v} + {\bf w}$

(c) Explain the meaning of each of the following in the context of this scenario. Include units. 

(i) $|  {\bf v} |$

(ii) $|  {\bf w} |$

(iii) $|  {\bf v+w} |$

(iv) $|  {\bf v} | + |  {\bf w} | - |  {\bf v+w} |$

\pskip

(d) When would the expression in part (b)(iv) above be positive? Negative? Zero? Explain your reasoning. 

\end{question}


\begin{question}  \label{Qsdfsdt4r3:Vectors}
Suppose
\[
 \overrightarrow{AB} = \langle  -3,5 \rangle
\]
and
\[
   \overrightarrow{AC} = \langle  4,2 \rangle ,
\]
where the components are measured in miles.

(a) Give directions to get from $B$ to $C$.

(b) What vector gives the position of $C$ relative to $B$? Choose all that apply.
\begin{multipleChoice} 
 \choice[correct]{$\langle  7, -3  \rangle$}  
 \choice{$\langle  -7, 3  \rangle$}
\choice{$\overrightarrow{CB}$}  
\choice[correct]{$\overrightarrow{BC}$} 
\choice{$\overrightarrow{AB} -\overrightarrow{AC} $} 
\choice[correct]{$\overrightarrow{AC} -\overrightarrow{AB} $}
\end{multipleChoice} 

\end{question}





\begin{question}  \label{Q234r3:Vectors}
Let ${\bf v}$ be the vector that gives the position of point $Q$ relative to point $R$ and ${\bf w}$ the vector that gives the position of $S$ relative to $R$. 

(a) Which of the following express the position of $S$ relative to $Q$? Choose all that apply.

\begin{selectAll}  
\choice{${\bf v} + {\bf w}$}  
\choice{${\bf v} - {\bf w}$}  
\choice[correct]{${\bf w} - {\bf v}$}  
\end{selectAll}

(b) Draw a picture illustrating part (a). Include the vectors ${\bf v}$ and ${\bf w}$ tail-to-tail, and the vector that gives the position of $S$ relative to $R$. That's it. Do not draw a fourth vector like $-{\bf v}$ or $-{\bf w}$.


\end{question}


\begin{question}  \label{Qtete4r3:Vectors}
Let $A$, $B$, and $C$ be non-colinear points. Let ${\bf b}$ be the vector giving the position of $B$ relative to $A$ and let ${\bf c}$ be the vector giving the position of $C$ relative to $A$. 

Suppose between times $t=-24$ and $t=24$ seconds you walk with a constant velocity (ie. with a constant speed in a fixed direction), passing point $B$ at time $t=0$ seconds and point $C$ at time $t=10$ seconds.

Express each of the following in terms of ${\bf b}$ and ${\bf c}$. Draw a picture for each and explain your reasoning.

(i) your position relative to $B$ at time $t=5$ seconds.

(ii) your position relative to $A$ at time $t=5$ seconds.

(iii) your position relative to $C$  at time $t=8$ seconds.

(iv) your position relative to $A$  at time $t=8$ seconds.

(v) your position relative to $A$  at time $t=24$ seconds.

(vi) your position relative to $A$  at time $t=-24$ seconds.

(vii)  your position relative to $B$  at time $t$ seconds.

(vii)  your position relative to $C$  at time $t$ seconds.

(viii)  your position relative to $A$  at time $t$ seconds.

\end{question}





IGNORE THIS

\pdfOnly{
Access Desmos interactives through the online version of this text at
 
\href{https://www.www.geogebra.org/classic/e6rsvnsz}.
}
 
\begin{onlineOnly}
    \begin{center}
\geogebra{e6rsvnsz}{900}{600}
\end{center}
\end{onlineOnly}


\end{document}
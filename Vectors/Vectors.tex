\documentclass{ximera}
\title{Vectors}

\newcommand{\pskip}{\vskip 0.1 in}
%\usepackage{esvect}

\begin{document}
\begin{abstract}
Vectors in $\mathbb{R}^2$ and  $\mathbb{R}^3$.
\end{abstract}
\maketitle

\section{Position Vectors}

Position is relative. You describe where something is not in absolute terms but only in relation to something else.

We might say, for example, that Bellevue is 4 miles east and 3 miles north of Akron. Then in a rectangular coordinate system with the $x$-axis pointing due east, the $y$-axis pointing due north, and the coordinates measured in miles, we can represent the position of Bellevue (at point $B$) relative to Akron (at point $A$) by the vector
\[
   \overrightarrow{AB} = \langle 4, 3 \rangle .
\]
This vector tells how to get from Akron to Bellevue. 


Suppose that in our rectangular coordinate sytstem Charleston (point $C$) has coordinates $(-5,-2)$ and Dayton (point $D$) has coordinates $(-1,1)$. Then to get from Charleston to Dayton we would travel east
\[
     ( -1 - (-5) ) \text{ miles} = 4 \text{ miles}
\]
and north
\[
     ( 1 - (-2)) \text{ miles} = 3\text{ miles} .
\]
So we can represent the position of Dayton relative to Charleston by the vector
\[
  \overrightarrow{CD} =       \langle -1 - (-5) , 1- (-2)  \rangle   =  \langle 4, 3 \rangle .
\] 

So the directions that tell us how to get from Akron to Bellevue also tell us how to get from Charleston to Dayton. The
vectors $\overrightarrow{AB}$ and $\overrightarrow{CD}$ are equal and we write 
\[
      \overrightarrow{AB} = \overrightarrow{CD} .
\]

To go the other way and get from Bellevue to Akron or from Dayton to Charleston, we would go 4 miles west and 3 miles south. So the  position of Akron relative to Bellevue, or of Charleston relative to Dayton, is given by the vector
\[ 
   \overrightarrow{BA} = - \overrightarrow{AB} = - \langle 4,3 \rangle =  \langle -4,-3 \rangle.
\]

The distance between Akron and Bellvue is just the length of the vector $\overrightarrow{AB}$, written as
\[
    | \overrightarrow{AB}  | =  | \langle 3, 4 \rangle  | \text{ miles} = \sqrt{3^2 + 4^2} \text{miles} = 5 \text{ miles} .
\] 


\section{Adding and Subtracting Vectors}

\begin{question}    \label{Q234234:Vectors}
Continuing with our example above, suppose that Frankfort (point $F$) is $2$ miles east and $5$ miles south of Bellevue. Then the vector
\[
    \overrightarrow{BF}  = \langle \answer{2}, \answer{-5}\rangle
\]
gives the position of Frankfort relative to Bellvue. 
\end{question}

\begin{question} \label{Q233:Vectors}
(a) Give directions to get from Akron to Frankfort.

(b) Fill in the vector and its components that gives the position of Franfort relative to Akron.
\[
       \overrightarrow{\answer{A} \answer{F}}    =  \langle  \answer{6} , \answer{-2}  \rangle 
\]

(c) Which of the following are correct? Choose all that apply.
\begin{selectAll}  
\choice[correct]{$\overrightarrow{AF}   = \overrightarrow{AB} + \overrightarrow{BF}$}  
\choice{$\overrightarrow{AF}   = \overrightarrow{AB} - \overrightarrow{BF} $}  
\choice[correct]{$\overrightarrow{AF}   = \overrightarrow{BF} + \overrightarrow{AB} $}  
\end{selectAll} 


\end{question}


\begin{question}  \label{Q543:Vectors}
Let ${\bf v}$ be the vector that gives the position of point $K$ relative to point $G$ and ${\bf w}$ the vector that gives the position of $M$ relative to $K$. 

(a) Express in terms of ${\bf v}$ and ${\bf w}$ each of the following vectors: 

(i) The position of $M$ relative to $G$:  $\overrightarrow{\bf \answer{v}} +  \overrightarrow{\bf \answer{w}} $

(ii) The position of $G$ relative to $M$:  $\overrightarrow{\bf \answer{-v}} +  \overrightarrow{\bf \answer{-w}} $

\pskip

(b) Draw vectors ${\bf v}$ and ${\bf w}$ tail-to-tail. Any choices will do, just make sure the vectors are not parallel. Then draw the vector ${\bf v} + {\bf w}$

(c) Explain the meaning of each of the following in the context of this scenario. Include units. 

(i) $|  {\bf v} |$

(ii) $|  {\bf w} |$

(iii) $|  {\bf v+w} |$

(iv) $|  {\bf v} | + |  {\bf w} | - |  {\bf v+w} |$

\pskip

(d) When would the expression in part (b)(iv) above be positive? Negative? Zero? Explain your reasoning. 

\end{question}


\begin{question}  \label{Qsdfsdt4r3:Vectors}
Suppose
\[
 \overrightarrow{AB} = \langle  -3,5 \rangle
\]
and
\[
   \overrightarrow{AC} = \langle  4,2 \rangle ,
\]
where the components are measured in miles.

(a) Give directions to get from $B$ to $C$.

(b) What vector gives the position of $C$ relative to $B$? Choose all that apply.
\begin{multipleChoice} 
 \choice[correct]{$\langle  7, -3  \rangle$}  
 \choice{$\langle  -7, 3  \rangle$}
\choice{$\overrightarrow{CB}$}  
\choice[correct]{$\overrightarrow{BC}$} 
\choice{$\overrightarrow{AB} -\overrightarrow{AC} $} 
\choice[correct]{$\overrightarrow{AC} -\overrightarrow{AB} $}
\end{multipleChoice} 

\end{question}





\begin{question}  \label{Q234r3:Vectors}
Let ${\bf v}$ be the vector that gives the position of point $Q$ relative to point $R$ and ${\bf w}$ the vector that gives the position of $S$ relative to $R$. 

(a) Which of the following express the position of $S$ relative to $Q$? Choose all that apply.

\begin{selectAll}  
\choice{${\bf v} + {\bf w}$}  
\choice{${\bf v} - {\bf w}$}  
\choice[correct]{${\bf w} - {\bf v}$}  
\end{selectAll}

(b) Draw a picture illustrating part (a). Include the vectors ${\bf v}$ and ${\bf w}$ tail-to-tail, and the vector that gives the position of $S$ relative to $R$. That's it. Do not draw a fourth vector like $-{\bf v}$ or $-{\bf w}$.


\end{question}


\begin{question}  \label{Qtete4r3:Vectors}
Let $A$, $B$, and $C$ be non-colinear points. Let ${\bf b}$ be the vector giving the position of $B$ relative to $A$ and let ${\bf c}$ be the vector giving the position of $C$ relative to $A$. 

Suppose between times $t=-24$ and $t=24$ seconds you walk with a constant velocity (ie. with a constant speed in a fixed direction), passing point $B$ at time $t=0$ seconds and point $C$ at time $t=10$ seconds.

Express each of the following in terms of ${\bf b}$ and ${\bf c}$. Draw a picture for each and explain your reasoning.

(i) your position relative to $B$ at time $t=5$ seconds.

(ii) your position relative to $A$ at time $t=5$ seconds.

(iii) your position relative to $C$  at time $t=8$ seconds.

(iv) your position relative to $A$  at time $t=8$ seconds.

(v) your position relative to $A$  at time $t=24$ seconds.

(vi) your position relative to $A$  at time $t=-24$ seconds.

(vii)  your position relative to $B$  at time $t$ seconds.

(vii)  your position relative to $C$  at time $t$ seconds.

(viii)  your position relative to $A$  at time $t$ seconds.

\end{question}





IGNORE THIS

\pdfOnly{
Access Desmos interactives through the online version of this text at
 
\href{https://www.www.geogebra.org/classic/e6rsvnsz}.
}
 
\begin{onlineOnly}
    \begin{center}
\geogebra{e6rsvnsz}{900}{600}
\end{center}
\end{onlineOnly}



\section{Computing Coordinates With Vectors}

\begin{question}  \label{Q:dsfr4rtgt4r}
Let $A(1,2,3)$ and $B(-3,5,-1)$ be points in $\mathbb{R}^3$ and let $O$ be the origin.

(a) Explain the meanings of the vectors $\overrightarrow{OA}$, $\overrightarrow{OB}$, and $\overrightarrow{AB}$.

(b) Express $\overrightarrow{AB}$ in terms of $\overrightarrow{OA}$ and $\overrightarrow{OB}$.

(c) Draw a picture to illustrate part (b).
\end{question}


\begin{question}  \label{Q:df945r435r34}
(a) Let $A(1,2,2)$, $B(3,-4,5)$, and $C(-2,1,4)$ be points in space. Let $D$ be the point in $\mathbb{R}^3$ that makes quadrilateral $ABCD$ a parallelogram. Express the vector $\overrightarrow{OD}$ in terms of the vectors $\overrightarrow{OA}$, $\overrightarrow{OB}$, and $\overrightarrow{OC}$. Draw the relevant vectors in the desmos worksheet below.

(b) Use the result of part (b) to determine the coordinates of $D$.

\begin{onlineOnly}
    \begin{center}
\desmosThreeD{hxegp3eszd}{900}{600}
\end{center}
\end{onlineOnly}

\href{https://www.desmos.com/3d/hxegp3eszd}{163: Parallelogram}
\end{question}


\begin{question} \label{Q98344422}
Let ${\cal L}$ be the line through the points $A(1,2,3)$ and $B(-2,1,1)$ and $O$ the origin.

(a) Let $P$ be the point on ${\cal L}$ between $A$ and $B$ such that
\[
      \frac{|\overrightarrow{AP}|}{|\overrightarrow{PB}|} = \frac{3}{5} . 
\]
First express the vector $\overrightarrow{OP}$ in terms of the vectors $\overrightarrow{OA}$ and $\overrightarrow{OB}$. Then use this expression to find the coordinates of $P$. Check your work in the desmos worksheet below.

\begin{explanation}
The idea is to describe how to get from the origin to point $P$. One path goes from the origin to $A$ and then from $A$ to $P$. In terms of vectors,
\[
   \overrightarrow{OP} = \overrightarrow{OA} + \overrightarrow{\answer{AP}}.
\]
Next we'll express vector $\overrightarrow{AP}$ in terms of $\overrightarrow{AB}$ using the given ratio.
\[
      \overrightarrow{AP} = \answer{3/8} \overrightarrow{AB} .
\]
Then
\begin{align*}
  \overrightarrow{OP} &= \overrightarrow{OA} + \overrightarrow{\answer{AP}}  \\
                                &= \overrightarrow{OA} + \frac{3}{8} \overrightarrow{AB} \\
                                &= \overrightarrow{OA} + \frac{3}{8} \left( \overrightarrow{\answer{OB}} - \overrightarrow{\answer{OA}}   \right) \\
                                 &= \answer{\frac{5}{8}} \overrightarrow{OA} + \answer{\frac{3}{8}} \overrightarrow{OB}
\end{align*}

\begin{freeResponse}
Enter this last expression, writing $A$ for $\overrightarrow{OA}$ and $B$ for $\overrightarrow{OB}$ in Line 10 of the worksheet below. Does this look correct?
\end{freeResponse}

\begin{onlineOnly}
    \begin{center}
\desmosThreeD{kqq5ktp0jw}{900}{600}
\end{center}
\end{onlineOnly}

\href{https://www.desmos.com/3d/kqq5ktp0jw}{163: Harmonic Conjugates}

(b) So the vector $\overrightarrow{OP}$ has components
\[
    \overrightarrow{OP} = \langle \answer{-1/8} , \answer{13/8} , \answer{9/4}   \rangle .
\]
And since $O$ is the origin, $P$ has coordinates
\[
   (x,y,z) = (\answer{-1/8} , \answer{13/8} , \answer{9/4}) .
\]

\end{explanation}


(b) Let $Q$ be the point on ${\cal L}$ \emph{not} between $A$ and $B$ such that
\[
      \frac{|\overrightarrow{AQ}|}{|\overrightarrow{QB}|} = \frac{3}{5} . 
\]
First express the vector $\overrightarrow{OQ}$ in terms of the vectors $\overrightarrow{OA}$ and $\overrightarrow{OB}$. Then use this expression to find the coordinates of $Q$. Check your work in the desmos worksheet.


\end{question}

\begin{question}  \label{Q:98dfer3vv}
Find the components of a vector with length $4$ parallel to the vector ${\bf v}=\langle 3, 2, -1\rangle>$.
\end{question}

\begin{question}  \label{QP9erf333}
Let ${\cal L}$ be the line through the points $A(1,2,3)$ and $B(-2,1,1)$ and $O$ the origin.

(a) Use vector arithmetic to find the coordinates of all points on ${\cal L}$ that are $3$ units from $A$. 

To do this, let $P$ be such a point and first express the vector $\overrightarrow{OP}$ in terms of the vectors $\overrightarrow{OA}$ and $\overrightarrow{OB}$. Check your work in the desmos worksheet below by plotting these points on Lines 7 and 8.

\begin{onlineOnly}
    \begin{center}
\desmosThreeD{dxn6mfy9mq}{900}{600}
\end{center}
\end{onlineOnly}

\href{https://www.desmos.com/3d/dxn6mfy9mq}{163: Sphere of Given Radius}

(b) Find vector equations of the sphere(s) centered on ${\cal L}$ of radius $3$ through $A$. For these equations, let ${\bf p}=\langle x,y,z \rangle$, ${\bf a} = \langle 1,2,3 \rangle$, and ${\bf b} = \langle -2,1,1 \rangle$ be the vectors from the origin to the points $P(x,y.,z)$, $A(1,2,3)$, and $B(-2,1,1)$. 

Check  your work in the worksheet above by writing $(x,y,z)$, $A$, and $B$ in place of ${\bf p}$, ${\bf a}$, and ${\bf b}$. Enter the equations on Lines 9 and 10.



\end{question}





\begin{question}  \label{Qdf4r4r5443}
(a) Drag slider $r_1$ below to see how many spheres centered at the point $B(4,1,5)$ are tangent to the sphere with equation
\[
        | {\bf p} - {\bf b} | = 3 .
\]


Here
\[
     {\bf p} = \overrightarrow{OP}
\]
is the vector from the origin to the point $P$ with coordinates $(x,y,z)$ and 
\[
   {\bf b} = \overrightarrow{OB}
\]
is the vector from the origin to the point $B$.



\begin{onlineOnly}
    \begin{center}
\desmosThreeD{4lmtlrfpms}{900}{600}
\end{center}
\end{onlineOnly}

\href{https://www.desmos.com/3d/4lmtlrfpms}{163:Vectors Tangent Spheres}

(b) Hide the spheres by turning off Lines 2 and 4. Then use vector arithmetic to find the coordinates of the points of tangency. Do this by expressing the vector $\overrightarrow{OP}$ from the origin to a point of tangency in terms of the vectors $\overrightarrow{OA}$ and $\overrightarrow{OB}$. Use the desmos worksheet to check your work.

\end{question}



%\begin{question} \label{Q45340:Vectors}
%(a) Explain why the spheres with equations
%\[
 %  (x-4)^2 + (y+5)^2 + (z-1)^2 = 49
%\]
%and
%\[
%    (x-6)^2 + (y+7)^2 + (z-2)^2 = 100
%\]
%are tangent to each other.

%(b) Use vectors, not algebra, to determine the coordinates of the point of tangency.

%(c) Use algebra to determine the coordinates of the point of tangency.

%\end{question}


\begin{question} \label{Q45323240:Vectors}
Points $A$ and $B$ have respective coordinates $(5,1,-8)$ and $(7,-2,-14)$. 

(a) Use vectors, not algebra, to determine the coordinates of the point $P$ on segment $\overline{AB}$ that is exactly 5 units from $A$.

(b) Use vectors, not algebra, to determine the coordinates of another point $P$ on the line through $A$ and $B$ that is exactly 5 units from $A$.

(c) Solve questions (a) and (b) simultaneously using algebra, not vectors.

\end{question}


\begin{question}  \label{Qder23455243}
Let 
\[
   f(x) = \ln |\sec(x/12)| \, , \, 0<x<6\pi.
\]

(a) Find an expression for the derivative $f^\prime(x)$.
 
(b) Find an equation of the normal line to the curve $y=f(x)$ at the point $P(u,f(u))$. Assume $0<x<6\pi$.

(c) Use vectors to write an equation of the circle of radius $r$ centered at the point $P$ that lies ``above'' the curve.

(d) Use the desmos demonstration below to check your work.

\begin{onlineOnly}
    \begin{center}
\desmos{taku9xvamz}{900}{600}
\end{center}
\end{onlineOnly}

\href{https://www.desmos.com/calculator/taku9xvamz}{163: Osculating Circle}

\end{question}


\begin{question} \label{Q4df8240:Vectors}
(a) Use vectors to determine the coordinates of the centers of all circles with radius 6 that are tangent to the ellipse
\[
    x^2 - xy + y^2 = 19
\]
at the point $(3,-2)$.

(b) Use the method of part (a) to determine the coordinates of the centers of all circles with radius $r$ that are tangent to the ellipse at the point $(3,-2)$.

(c) Follow the directions in the demonstration below to check your work.

\pdfOnly{
Access Desmos interactives through the online version of this text at
 
\href{https://www.desmos.com/calculator/aqpedvdawb}.
}
 
\begin{onlineOnly}
    \begin{center}
\desmos{aqpedvdawb}{900}{600}
\end{center}
\end{onlineOnly}

\end{question}


\begin{question}   \label{Q6547:Vectors}
(a) Use vectors, not trigonometry, to help find equations of the two lines that bisect the angles between the lines $y=x$ and $y=3x$.

(b) Use part (a) to find equations of the two lines that bisect the angles between the lines $y=x+3$ and $y=3x-5$.


(c) Use vectors to help find an algebraic description of the set of points twice as far from the line $y=x$ as from the line $y=3x$. Keep in mind that the distance from a point $P$ to a line is the minimum distance between $P$ and the points of the line. 

\end{question}

\begin{question} \label{Q5704r:Vectors}
The rear wheel of a bicycle moves along the curve
\[
    y = 4 -\int_2^x \frac{2}{\sqrt{5+u^2}}\, du \, , \, -10 \leq x \leq 10 
\]
in the direction of decreasing $x$-coordinates. The coordinates are measure in feet. The front and rear axles are three feet apart.

(a) Find the coordinates of the point where the front wheel touches the $xy$-plane when the rear wheel touches the plane at the point $(2,4)$. 

(b) Enter these coordinates in the desmos demonstration below. 

(c) Find an equation of the segment running between the two contact points and enter it in desmos.

\pdfOnly{
Access Desmos interactives through the online version of this text at
 
\href{https://www.desmos.com/calculator/a20scadjvh}.
}
 
\begin{onlineOnly}
    \begin{center}
\desmos{a20scadjvh}{900}{600}
\end{center}
\end{onlineOnly}

\end{question}


\begin{question} \label{Q5sdf04r:Vectors}
The purpose of this problem is to find equations of several spheres with a given radius that pass through two given points. Later, we will find equations of all these spheres.

So we'll start with two points $A$ and $B$ with respective coordinates $(2,4,6)$ and $(6,-4,10)$. Our problem is to find three spheres of radius $7$ through $A$ and $B$.

(a) We'll first ignore the condition that the spheres have radius $7$ and consider the family of all spheres through $A$ and $B$.

What can you say about the set of centers of all such spheres? Choose all that apply.
\begin{selectAll}
\choice{The set contains exactly one point, the midpoint of segment $\overline{AB}$}.
\choice{The set is a sphere with diameter $\overline{AB}$}.
\choice[correct]{The set consists of all points equidistant (ie. the same distance) from $A$ and $B$.}
\end{selectAll}


 (b) Using the correct description of the relaxed center-set above and without doing any computations, describe the set geometrically. Be precise. Explain your reasoning.

(c) Translate the correct description of the relaxed center-set in the multiple choice question above directly into an equation of the center-set. Then do some algebra and simplify the equation. Enter it in the line below.
\[
     \answer{x-2y+z=6}
\]

(d) Now let's impose the additional condition that the spheres have radius $7$. Describe geometrically the center-set of all spheres through $A$ and $B$ with radius $7$. Be precise.

(e) Find the coordinates of the center $M$ of the smallest sphere through $A$ and $B$. Use these coordinates and vector arithmetic to find the coordinates of three points that are centers of spheres with radius $7$ through $A$ and $B$. Do this by first finding the coordinates of random points in the relaxed center set (parts (a)-(c)). 

(f) Write equations of three sphere with radius $7$ through $A$ and $B$. Check your work in desmos or geogebra.

\end{question}


\begin{question}  \label{Q45367:Vectors}
Given two spheres that intersect, we'll use vector arithmetic to find the center of their circle of interection.

\begin{itemize}

\item{Let the spheres have centers $(a_1, b_1, c_1)$, $(a_2, b_2, c_2)$ with respective radii $r_1$ and $r_2$.} 

\item{Let ${\bf v}_1$ and ${\bf v}_2$ be the vectors from the origin to the respective centers.}

\end{itemize}

Express the vector ${\bf c}$ from the origin to the center of the circle of intersection in terms of the vectors ${\bf v}_1$, ${\bf v}_2$ and the scalars $r_1$, $r_2$, $r_1$, $r_2$, and $|{\bf v}_1-{\bf v}_2|$. 

Do {\bf not} work with the components of ${\bf v}_1$ and ${\bf v}_2$, but just the vectors themselves. That way your expression for ${\bf c}$ should work for any number of dimensions. 

Follow the directions in the demonstration below to check your work in two-dimensions. Desmos does not recognize the absolute value symbol for the magnitude of a vector. So use 
\[
    d = \sqrt{(a_1-a_2)^2 + (b_1-b_2)^2}
\]
in place of 
\[
   |{\bf v}_1-{\bf v}_2|.
\]

What happens if the circles do not intersect? Can you explain this?



\pdfOnly{
Access Desmos interactives through the online version of this text at
 
\href{https://www.desmos.com/calculator/q4lnt9y1h4}.
}
 
\begin{onlineOnly}
    \begin{center}
\desmos{q4lnt9y1h4}{900}{600}
\end{center}
\end{onlineOnly}


\end{question}



\section{Writing Equations with Vectors}

\begin{question}  \label{Eq:32404:Vectors}
In the following questions, we use capital letters like $A$ to denote points and the corresponding boldfaced lower case letter, like ${\bf a}$, to denote the vector from the origin to the point. Also, we use the vector ${\bf r}$ to denote a vector from the origin to the point with coordinates $(x,y)$ in two-dimensions or to the point $(x,y,z)$ in three dimensions.

\pskip

(a) Find a vector equation of the sphere with radius 5 centered at the point $A$.

(b) Find a vector equation of the set of points twice as far from the point $A$ as from $B$.

(c) Find a vector equation of all points the sum of whose distances from $A$ and $B$ is 10 units.


\end{question}








\end{document}
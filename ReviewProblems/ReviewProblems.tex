\documentclass{ximera}
\title{Some Problems for Review}

\newcommand{\pskip}{\vskip 0.1 in}

\begin{document}
\begin{abstract}
Review problems.
\end{abstract}
\maketitle

Here are some sample finals. They are \emph{not} necessarily indicative of the exact types of problems you will see on the final, nor are they meant to be an exhaustive list of all the types of problems. But they should give you some general idea of what to expect.

Neither calculators or notes will be permitted. 

\section{Sample 1}

\begin{question}  \label{Qvftgbr45tg4rt}
The points $A$ and $B$ lie in a plane in $\mathbb{R}^3$ perpendicular to the vector ${\bf n}$.

Find the position(s) (relative to the origin) of all points $C$ in the plane that make $\Delta ABC$ an equilateral triangle. Express this (these) position(s) in terms of ${\bf n}$ and the postions $\overrightarrow{OA}$, $\overrightarrow{OB}$ of $A$ and $B$ relative to the origin.
\end{question}



\begin{question} \label{Qblhlh544tgg}
The function ${\bf p}(t)$ expresses the position (measured in meters) of a point mass relative to the origin in terms of time (measured in seconds).

\begin{enumerate}
\item Interpret the meanings of the following derivatives. Include units in your interpretations.
\begin{enumerate}
\item 
\[
  \frac{d}{dt} \left( {\bf p}(t)  \right).
\]

\item \[
  \frac{d}{dt} \Big| {\bf p}^\prime (t)  \Big|.
\]

\end{enumerate}

\item Suppose that
\[
   \frac{d^2 \left( {\bf p}(t)  \right)}{dt^2} \Big|_{t=3} = \langle 2, -1, 3 \rangle.
\]

\begin{enumerate}
\item Interpret then meaining of this derivative. Include units.

\item What can you say about the possible values of the derivative  
\[
  \frac{d}{dt}\left( \Big| {\bf p}^\prime (t)  \Big| \right) \Big|_{t=3}? 
\]
\end{enumerate}

\end{enumerate}

\end{question}

\begin{question}  \label{Qctr43trtg44}
You measure the mass and volume of a chunk of gold with relative errors at most $p\%$ and $q\%$ respectively.

Approximate the maximum possible relative error when you use your measurements to compute the density of gold. Assume $p,q\sim 0$.
\end{question}

\begin{question}  \label{Q434trgfgfg}
Assume for this problem that the earth is a perfect sphere of radius $R$ centered at the origin.

\begin{enumerate}
\item Fix an angle $\phi$, with $0< \phi < \pi/2$. 

Find an arclength parameterization of the circle ${\cal C}$ cut from the sphere by the plane 
\[
      z = y \tan \phi .
\]
Take the arclength parameter $s$ to be zero at the point $A(R,0,0)$ and the positive direction to head into the northern hemisphere from $A$.

\item Suppose a plane flies along the circle ${\cal C}$ in the direction in which the arclength parameter increases.

Find a function
\[
 \theta = f(s) , 0\leq s \leq 2\pi R,
\]
that expresses the bearing in which the plane is heading in terms of the arclength paramter. The bearing is measured as an angle (in radians) from due north, with the positive sense taken clockwise. Assume that the plane moves in an easterly direction along the circle. 
\end{enumerate}

\begin{onlineOnly}
    \begin{center}
\desmosThreeD{umtip2fclm}{900}{600}
\end{center}
\end{onlineOnly}


\href{https://www.desmos.com/3d/umtip2fclm}{163: NY to Paris}


\end{question}


\end{document}
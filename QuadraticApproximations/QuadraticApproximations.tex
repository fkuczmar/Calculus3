\documentclass{ximera}
\title{Quadratic Approximations}

\newcommand{\pskip}{\vskip 0.1 in}

\begin{document}
\begin{abstract}
Using quadratic power series to approximate curves in the plane and in space.
\end{abstract}
\maketitle

\section{Linear Approximation}
In differential calculus we learned about linear approximation. That's just a fancy term meaning to approximate the graph of a function near a point with the tangent line at that point.

So, for example, to find the linear approximation to the function $f(x) = \sin x$ at the point $P(5\pi/6,0)$, we just find an equation of the tangent line at $P$. And it's best to use the point-slope form of a line.

\begin{question}  \label{Qegbygyyu}
What is the point-slope equation of the tangent line above?
\[
    y = \answer{\frac{1}{2}} + \answer{-\frac{\sqrt{3}}{2}}\left( x - \answer{\frac{5\pi}{6}}  \right) .
\]
\end{question}

\begin{question}   \label{Qt43666}
More generally, what is the linear approximation to the differential function $y=f(x)$ at the point $P(a,f(a))$?
\[
     L(x) = \answer{f(a)} + \answer{f^\prime(a)} (x - \answer{a} ) .
\]
\end{question}


We find the linear approximation to a curve ${\bf p}(t)$ at a point with position ${\bf p}(a)$ relative to the origin in much the same way, by finding a parameterization of the tangent line to the curve at ${\bf p}(a)$. But unlike for a function $y=f(x)$ where there is only one way to write the Cartesian equation of a line, we have many choices in parameterizing a line and we need to choose the {\bf one} that gives the correct approximation. 

\begin{question}   \label{Qggkuthds}
Find the linear approximation to the curve
\[
   {\bf p}(t) = \langle 3 \cos t, 3\sin t, e^t \rangle, -2\leq t \leq 2 ,
\]
at the point $P(0,3,e^{\pi/2})$.

The idea is that we replace the point-slope form of an equation of the tangent line to the graph of a function with the {\bf point-velocity} parameterization of the tangent line (here we regard the parameter $t$ as being time).

For the curve ${\bf p}(t)$ above, what is the linear approximation at $P(0,3,e^{\pi/2})$?
\[
   \boldsymbol{\ell}(t) = \langle  \answer{0}, \answer{3}  , \answer{e^{\pi/2}}  \rangle  + (t - \answer{\pi/2}) \langle \answer{-3}, \answer{0}, \answer{e^{\pi/2}}  \rangle , -2\leq t \leq 2 .
\]
\end{question}

\begin{question}  \label{Qfdgtyy}
Find the linear approximation to the curve ${\bf p}(t)$ at parameter $t=a$.
\[
     {\bf L} (t) = \answer{{\bf p}(t)} +   (t - \answer{a} )\answer{{\bf p}^\prime(a)}  .
\]
\end{question}



\section{Quadratic Approximations}


\begin{question}  \label{Qgtyhtsry}
(a) Find quadratic approximations to the component functions of the space curve
\[
   {\bf p}(t) = \langle  3 \cos t, 3\sin t, e^t \rangle , -2\leq t \leq 2
\]
to find the linear and quadratic approximations to the curve at the point $P(3,0,1)$. Write the approximation as a linear combination of the vectors ${\bf p}^\prime(0)$ and ${\bf p}^{\prime\prime}(0)$.

(b) Explain why the two approximations lie in the same plane. Then find an equation of that plane.

(c) In the desmos activity below vary the slider $u$ on Line 13 to see the family of planes through the tangent line to the curve at $P$. Which of these planes best contains the space curve near $P$?

Desmos activity available at:

\href{https://www.desmos.com/3d/196e85fe1a}{163: Quadratic Approximation to Space Curve 2}


\href{https://www.desmos.com/3d/a263cd2883}{163: Quadratic Approximation to Space Curve}.

 

\end{question}



\end{document}

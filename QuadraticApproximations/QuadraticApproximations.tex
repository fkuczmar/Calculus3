\documentclass{ximera}
\title{Quadratic Approximations}

\newcommand{\pskip}{\vskip 0.1 in}

\begin{document}
\begin{abstract}
Using quadratic power series to approximate curves in the plane and in space.
\end{abstract}
\maketitle

\section{Linear Approximation}
In differential calculus we learned about linear approximation. That's just a fancy term meaning to approximate the graph of a function near a point with the tangent line at that point.

So, for example, to find the linear approximation to the function $f(x) = \sin x$ at the point $P(5\pi/6,0)$, we just find an equation of the tangent line at $P$. And it's best to use the point-slope form of a line.

\begin{question}  \label{Qegbygyyu}
What is the point-slope equation of the tangent line above?
\[
    y = \answer{\frac{1}{2}} + \answer{-\frac{\sqrt{3}}{2}}\left( x - \answer{\frac{5\pi}{6}}  \right) .
\]
\end{question}

\begin{question}   \label{Qt43666}
More generally, what is the linear approximation to the differential function $y=f(x)$ at the point $P(a,f(a))$?
\[
     L(x) = \answer{f(a)} + \answer{f^\prime(a)} (x - \answer{a} ) .
\]
\end{question}


We find the linear approximation to a curve ${\bf p}(t)$ at a point with position ${\bf p}(a)$ relative to the origin in much the same way, by finding a parameterization of the tangent line to the curve at ${\bf p}(a)$. But unlike for a function $y=f(x)$ where there is only one way to write the Cartesian equation of a line, we have many choices in parameterizing a line and we need to choose the {\bf one} that gives the correct approximation. 

\begin{question}   \label{Qggkuthds}
Find the linear approximation to the curve
\[
   {\bf p}(t) = \langle 3 \cos t, 3\sin t, e^t \rangle, -2\leq t \leq 2 ,
\]
at the point $P(0,3,e^{\pi/2})$.

The idea is that we replace the point-slope form of an equation of the tangent line to the graph of a function with the {\bf point-velocity} parameterization of the tangent line (here we regard the parameter $t$ as being time).

For the curve ${\bf p}(t)$ above, what is the linear approximation at $P(0,3,e^{\pi/2})$?
\[
   \boldsymbol{\ell}(t) = \langle  \answer{0}, \answer{3}  , \answer{e^{\pi/2}}  \rangle  + (t - \answer{\pi/2}) \langle \answer{-3}, \answer{0}, \answer{e^{\pi/2}}  \rangle , -2\leq t \leq 2 .
\]
\end{question}

\begin{question}  \label{Qfdgtyy}
Find the linear approximation to the curve ${\bf p}(t)$ at parameter $t=a$.
\[
     {\bf L} (t) = \answer{{\bf p}(t)} +   (t - \answer{a} )\answer{{\bf p}^\prime(a)}  .
\]
\end{question}




\section{A More Careful Look at Linear Approximations}
Let's start again with the aim of addressing some interesting questions that were raised in Thursday's class.

\begin{itemize}

\item{How good is a linear approximation?}

\item{Why do we improve a linear approximation by adding a quadratic term?}

\item{How can we visualize the quadratic term?}

\end{itemize}

It will help to look at the problem in the context of a function $y=f(t)$ that expresses the position (say in meters) of a beetle moving along the $y$-axis in terms of the number of seconds past noon. We'll illustrate the logic with the particular position function
\[
     y = f(t) = 2 + 0.5\ln t \, ,  \, 0.1 \leq t \leq 4 ,
\]
and we'll go about approximating the position near time $a=1$ second where $y=2$ meters.

The first step is to find a linear approximation. For this, we assume the beetle crawls with a constant velocity 
\[
   \frac{dy}{dt} \Big|_{t=1} = 0.5 \text{ m/sec} .
\]
equal to its velocity at time $a=1$ second. Rather fast for a beetle, but we'll ignore that.

Assuming now that the beetle crawls with this \emph{constant} velocity gives a linear approximation to the position function for values of $t$ near $a=1$. This approximation is
\[
    p_1(t) = 2 + 0.5(t-1) .
\]
It should be a fairly accurate approximation to the actual position function $f(t)$ for $t\sim 1$. That is, for $t\sim 1$,
\[
       p_1(t) = 2 + 0.5(t-1) \sim 2 + 0.5 \ln t = f(t).
\]

But what exactly do we mean by approximately? And what can we say about the accuracy of our approxmation?

To answer these questions, let's back up a bit and consider all linear functions that pass through the point $(1,2)$. In what sense does the linear approximation $p_1(t)$ best approximate $f(t)$?

To answer this, we'll let
\[
   f_1(t) = 2 + m(t-1)
\]
be the approximation to the position function near time $t=1$ that assumes a constant velocity of $m$ m/sec. The key idea is to look at the error
\[
    e_1(t) = f(t) - f_1(t) = f(t) - (2 + m(t-1))
\]
in the approximation near $t=1$.

\begin{exploration}\label{Exp3:Comp}

\pdfOnly{
Access Desmos interactives through the online version of this text at
 
\href{https://www.desmos.com/calculator/fhbtcot3ld}.
}
 
\begin{onlineOnly}
    \begin{center}
\desmos{fhbtcot3ld}{900}{600}
\end{center}
\end{onlineOnly}

Access Desmos interactives through the online version of this text at
 
\href{https://www.desmos.com/calculator/ueb0dgixdq}{163: Taylor Series 1}


(a) Turn on the Linear Approximations folder in Line 10. You should see the linear approximation $y=f_1(t)$ (purple) and the error function $e_1(t)$ (green).

(b) Now zoom in close to the point $(1,0)$ and adjust the slider $m$.
\begin{question} \label{Qfdsg677}
(i) Describe how the error function changes as you change the velocity $m$.

(ii) What is special about the local behavior of the error function near $(1,0)$ when $m=0.5$?
\end{question}

(c) It's admittedly difficult to pick out the best linear approximation by looking at just the tangent line or the error function. The best approximation is the one with an error function that approaches zero (as $t\to 1$) faster than all the others. To better see which approximation is best, turn on the relative error function
\[
    r_1 (t) = \frac{e_1(t)}{t-1} .
\]
in Line 17. Then vary the slider $m$.

\begin{question} \label{Q3435rgg}
(i) What do you notice? 

(ii) The graph of the relative error function might be a bit misleading. What is the value of $r_1(1)$?

(iii) Use the graph of the relative error function to approximate the limits
\[
    \lim_{t\to 1}r_1(t) 
\]
for velocities $m=0, 0.3, 0.5, 0.8$
\end{question}

\it{Key Point:} The linear approximation $p_1(t)=2+0.5(t-1)$ with velocity $m=0.5$ is the best among all possible linear approximations to $f(t)=2+0.5\ln t$ in the sense that it is the \emph{only} one with an error function that approaches zero faster than $t-1$ as $t\to 1$. That is,
\[
          \lim_{t\to 1} \frac{f(t) - (2 + m(t-1))}{t-1} = 0
\]
if and only if $m=0.5$

This is actually what it means for a function $f(t)$ to be differentiable at $t=a$; that $f(t)$ can be approximated with a linear function near $t=a$ for which the error in the error function appoaches zero faster than $t-a$. 




\end{exploration}




\section{Quadratic Approximations}
The last activity suggested that of all possible linear approximations to the differentiable function $f(t)$ at $t=a$, the tangent line approximation
\[
 p_1(x) = f(a) + f^\prime(a)(x-a)
\]
is the best. It is the only one that makes the error function
\[
   e_1(t) = f(x) - p_1(x)
\]
approach zero faster than $(t-a)$ as $t\to a$ in the sense that
\[
   \lim_{t\to a} \frac{e_1(t)}{t-a} = 0 .
\]
This is actually what it means for a function $f(t)$ to be differentiable at $t=a$; that $f(t)$ can be approximated with a linear function near $t=a$ for which the error in the error function appoaches zero faster than $t-a$. 

But how fast? The answer to that question depends upon whether $f(t)$ is twice differentiable at $t=a$. If so, and we will assume it is, then near $t=a$ the error function $e_1(t)$ approaches zero at the same rate as $(t-a)^2$. In other words, for $t\sim a$,
\[
  e_1(t) \sim k (t-a)^2 ,
\]
for some constant $k$. Or more precisely, 
\[
   \lim_{t\to a} \frac{e_1(t)}{(t-a)^2} = k .
\]

We can see this

\begin{exploration}\label{Exp3:Comp}

\pdfOnly{
Access Desmos interactives through the online version of this text at
 
\href{https://www.desmos.com/calculator/oswisy36tn}.
}
 
\begin{onlineOnly}
    \begin{center}
\desmos{fhbtcot3ld}{900}{600}
\end{center}
\end{onlineOnly}

Access Desmos interactives through the online version of this text at
 
\href{https://www.desmos.com/calculator/oswisy36tn}{163: Taylor Series 2}


\end{exploration}





\section{Curves in Space}


\begin{question}  \label{Qgtyhtsry}
(a) Find quadratic approximations to the component functions of the space curve
\[
   {\bf p}(t) = \langle  3 \cos t, 3\sin t, e^t \rangle , -2\leq t \leq 2
\]
to find the linear and quadratic approximations to the curve at the point $P(3,0,1)$. Write the approximation as a linear combination of the vectors ${\bf p}^\prime(0)$ and ${\bf p}^{\prime\prime}(0)$.

(b) Explain why the two approximations lie in the same plane. Then find an equation of that plane.

(c) In the desmos activity below vary the slider $u$ on Line 13 to see the family of planes through the tangent line to the curve at $P$. Which of these planes best contains the space curve near $P$?

Desmos activity available at:

\href{https://www.desmos.com/3d/196e85fe1a}{163: Quadratic Approximation to Space Curve 2}


\href{https://www.desmos.com/3d/a263cd2883}{163: Quadratic Approximation to Space Curve}.

 

\end{question}



\end{document}

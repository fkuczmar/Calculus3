\documentclass{ximera}
\title{Quadratic Approximations}

\newcommand{\pskip}{\vskip 0.1 in}

\begin{document}
\begin{abstract}
Using quadratic power series to approximate curves in the plane and in space.
\end{abstract}
\maketitle


\begin{question}  \label{Qgtyhtsry}
(a) Find quadratic approximations to the component functions of the space curve
\[
   {\bf p}(t) = \langle  3 \cos t, 3\sin t, e^t \rangle , -2\leq t \leq 2
\]
to find the linear and quadratic approximations to the curve at the point $P(3,0,1)$. Write the approximation as a linear combination of the vectors ${\bf p}^\prime(0)$ and ${\bf p}^{\prime\prime}(0)$.

(b) Explain why the two approximations lie in the same plane. Then find an equation of that plane.

(c) In the desmos activity below vary the slider $u$ on Line 13 to see the family of planes through the tangent line to the curve at $P$. Which of these planes best contains the space curve near $P$?

Desmos activity available at:

\href{https://www.desmos.com/3d/a263cd2883}{163: Quadratic Approximation to Space Curve}.

 

\end{question}



\end{document}

\documentclass{ximera}
\title{Directional Derivatives}

\newcommand{\pskip}{\vskip 0.1 in}

\begin{document}
\begin{abstract}
Introduction to directional derivatives.
\end{abstract}
\maketitle


\section{Level Sets and Directional Derivatives, One Independent Variable}

\begin{question}  \label{Qdsdfsddsf6bd}

The function
\[
   T = f(x) = 5 + 3\sin \left(   \frac{\pi}{6}x  \right)   , -9\leq x \leq 9 ,
\]
expresses the temperature (in Celsius degrees) at a point on a thin rod in terms of the displacement (in meters) from the rod's center.

The graph of the function $T=f(x)$ is shown below along with some of its level sets plotted on the $x$-axis.

(a) Describe the nature of the level sets shown below.


\begin{itemize}

\item{What is the spacing between consecutive level sets?}

\item{How many points are in most level sets? How many points are in the others?} 

\item{Some level sets are colored red, others blue. Why?}

\end{itemize}

(b) Drag the slider in Line 2 to $n=24$. Then zoom in on the origin. Use the level sets to approximate the value of $f^\prime(0)$. Include units and explain the meaning of the derivative.

(c) Compute the exact value of $f^\prime(0)$ and compare it with your estimate.

(d) At what (instantaneous) rate does the temperature of the rod change at $x=0$ in the direction of the negative $x$-axis? 
d
\pdfOnly{
Access Desmos interactives through the online version of this text at
 
\href{https://www.desmos.com/calculator/1ztmnpaymf}.
}
 
\begin{onlineOnly}
    \begin{center}
\desmos{1ztmnpaymf}{900}{600}
\end{center}
\end{onlineOnly}

Access Desmos interactives through the online version of this text at
 
\href{https://www.desmos.com/calculator/1ztmnpaymf}{163: Directional Derivative 1}.
\end{question}




\begin{exploration} \label{Edsft45t5}
Match each level set below with its function from the list below (there is one function without a bar graph):

(a) $y=e^x$

(b) $y=\ln x$

(c) $y=\arctan x$

(d) $y=x^3$

(e) $y=x^{1/3}$

(f) $y=\tan x$

\pdfOnly{
Access Desmos interactives through the online version of this text at
 
\href{https://www.desmos.com/calculator/qhto2obz1x}.
}
 
\begin{onlineOnly}
    \begin{center}
\desmos{qhto2obz1x}{900}{600}
\end{center}
\end{onlineOnly}

Access Desmos interactives through the online version of this text at
 
\href{https://www.desmos.com/calculator/qhto2obz1x}{163: Level Curves 151, Part 2}.

https://www.desmos.com/calculator/dabb7tarx3
\end{exploration}



\section{Level Sets and Directional Derivatives, Two Independent Variables}

Suppose now we're standing on a hot plate, where the temperature varies from point to point. Unlike in one dimension, where we have a  choice of only two directions to walk, we now have an entire circle of directions from which to choose. How can we compute the (instantaneous) rate of change in the temperature if we head off in an arbitrary direction from our point. As in the first example, the rate of change is with respect to distance. There is no time involved here.

It might come as a surprise that knowing the rate of change in any two independent directions (ie. directions that are not directly opposite) allows us to find the rate of change in any direction, provided that the temperature is a differentiable function of position.

Just as in one dimension, we can use the level curves of the temperature function to approximate the rate of change in any direction. But before we get to this, take a moment to think about the following question.

\begin{question}  \label{Qwerder432}
Suppose that from your current position, the temperature increases at the rate of $0.02^\circ$C/meter in the direction of the vector $\langle 1, 0 \rangle$ and that it increases at the rate of $0.04^\circ$C/meter in the direction of the vector $\langle 0,1\rangle$. What do you think is true about the rate at which the temperature changes in the direction of the vector ${\bf v} = \langle 1,1\rangle$?

\pskip

(a) The temperature increases at the rate of exactly $0.03^\circ$C/meter.

(b) The temperature increases at a rate somewhere between $0.02^\circ$C/meter and $0.04^\circ$C/meter, but not necessarily at the exact rate of $0.03^\circ$C/meter.

(c) The temperature increases at a rate less than $0.02^\circ$C/meter.

(d) The temperature increases at a rate greater than $0.04^\circ$C/meter.

(e) The temperature decreases.

(f) We cannot say, even if we assume (as do the above choices) that the temperature is a differentiable function of position.

\pskip

Try to justify your response with an intuitive argument without doing any computations.


\end{question}




\begin{question}  \label{Qds54546bd}

The function 
\[
     T= F(x,y) = x^2 -xy +y^2
\]
expresses the temperature (in Celsius degrees) at a point in the plane in terms of its coordinates, measured in meters.

Some level curves 
\[
     F(x,y)=k
\]
of the function $F$ are shown below. 

1. Use the level curves along with the sliders $\Delta k$, $\theta$, and $s$ to approximate the following instantaneous rates of change of temperature in the direction of the vector ${\bf v} = \langle \cos\theta, \sin\theta \rangle$. The slider $s$ adjusts the length of the vector ${\bf v}$ (blue). Adjust the slider $\Delta k$ to get better approximations.

\pskip

(a) At the point $A(-1,1)$ in the direction of the vector ${\bf v} = \langle 1, 0 \rangle$.

(b)  At the point $A(-1,1)$ in the direction of the vector ${\bf v} = \langle 0,1 \rangle$.

(c)  At the point $A(-1,1)$ in the direction of the vector ${\bf v} = \langle 0, -1 \rangle$.

(d)  At the point $A(-1,1)$in the direction of the vector ${\bf v} = \langle 1,1 \rangle$.

(e)  At the point $A(-1,1)$ in the direction of the vector ${\bf v} = \langle -1,-1 \rangle$.

(f)  At the point $A(-1,1)$ in the direction of the vector ${\bf v} = \langle 1,-1 \rangle$.

(g) At the origin in various directions of your choosing.

\pskip \pskip

2. In which direction from the point $A(-1,1)$ does the function $F$ appear to increase at the fastest rate? Approximate this rate of change.

\pskip \pskip

3. In which direction from the point $A(-1,1)$ does the function $F$ appear to decrease at the fastest rate? Approximate this rate of change.

\pskip \pskip

4. In which direction(s) from the point $A(-1,1)$ does the function $F$ appear to remain constant? 


\pdfOnly{
Access Geogebra interactives through the online version of this text at
 
\href{https://www.geogebra.org/classic/cfkfwfpk}.
}
 
\begin{onlineOnly}
    \begin{center}
\geogebra{cfkfwfpk}{900}{600}
\end{center}
\end{onlineOnly}

Access Geogebra interactives through the online version of this text at
 
\href{https://www.geogebra.org/classic/cfkfwfpk}{163: Directional Derivative 1}.

\end{question}


\begin{question}  \label{Ee53hhy}

Some level curves of a temperature function $T = f(x,y)$ are shown below. The output is measured in Celsius degrees and the input in meters.

(a) Find an expression for $f$.

(b) Use the sliders $\theta$ and $s$ to approximate the (instantaneous) rate at which the temperature changes from the origin in the directions of the following vectors. Include units.

\pskip

(i) ${\bf v} = \langle 4 , 3 \rangle $

(ii) ${\bf v} = \langle -3,1\rangle $

\pskip

(c) Use the algebra of difference quotients to compute the exact values of the above rates of change.

(d) Use the algebra of difference quotients to compute the rate  at which the temperature changes from the origin in the direction of the vector ${\bf v}=\langle v_x , v_y \rangle$. 

(e) Check that your expression in part (d) has the correct units.

(f) Interpret your expression for the rate of change geometrically.


\pdfOnly{
Access Geogebra interactives through the online version of this text at
 
\href{https://www.geogebra.org/classic/htee9xke}.
}
 
\begin{onlineOnly}
    \begin{center}
\geogebra{htee9xke}{900}{600}
\end{center}
\end{onlineOnly}

Access Geogebra interactives through the online version of this text at
 
\href{https://www.geogebra.org/classic/htee9xke}{163: Directional Derivative}.


\end{question}



\end{document}
\documentclass{ximera}
\title{Directional Derivatives}

\newcommand{\pskip}{\vskip 0.1 in}

\begin{document}
\begin{abstract}
Introduction to directional derivatives.
\end{abstract}
\maketitle


\section{Level Sets and Directional Derivatives, One Independent Variable}

\begin{question}  \label{Qdsdfsddsf6bd}

The function
\[
   T = f(x) = 5 + 3\sin \left(   \frac{\pi}{6}x  \right)   , -9\leq x \leq 9 ,
\]
expresses the temperature (in Celsius degrees) at a point on a thin rod in terms of the displacement (in meters) from the rod's center.

The graph of the function $T=f(x)$ is shown below along with some of its level sets plotted on the $x$-axis.

(a) Describe the nature of the level sets shown below.


\begin{itemize}

\item{What is the spacing between consecutive level sets?}

\item{How many points are in most level sets? How many points are in the others?} 

\item{Some level sets are colored red, others blue. Why?}

\end{itemize}

(b) Drag the slider in Line 2 to $n=24$. Then zoom in on the origin. Use the level sets to approximate the value of $f^\prime(0)$. Include units and explain the meaning of the derivative.

(c) Compute the exact value of $f^\prime(0)$ and compare it with your estimate.

(d) At what (instantaneous) rate does the temperature of the rod change at $x=0$ in the direction of the negative $x$-axis? 
d
\pdfOnly{
Access Desmos interactives through the online version of this text at
 
\href{https://www.desmos.com/calculator/1ztmnpaymf}.
}
 
\begin{onlineOnly}
    \begin{center}
\desmos{1ztmnpaymf}{900}{600}
\end{center}
\end{onlineOnly}

Access Desmos interactives through the online version of this text at
 
\href{https://www.desmos.com/calculator/1ztmnpaymf}{163: Directional Derivative 1}.
\end{question}




\begin{exploration} \label{Edsft45t5}
Match each level set below with its function from the list below (there is one function without a bar graph):

(a) $y=e^x$

(b) $y=\ln x$

(c) $y=\arctan x$

(d) $y=x^3$

(e) $y=x^{1/3}$

(f) $y=\tan x$

\pdfOnly{
Access Desmos interactives through the online version of this text at
 
\href{https://www.desmos.com/calculator/qhto2obz1x}.
}
 
\begin{onlineOnly}
    \begin{center}
\desmos{qhto2obz1x}{900}{600}
\end{center}
\end{onlineOnly}

Access Desmos interactives through the online version of this text at
 
\href{https://www.desmos.com/calculator/qhto2obz1x}{163: Level Curves 151, Part 2}.

https://www.desmos.com/calculator/dabb7tarx3
\end{exploration}



\section{Level Sets and Directional Derivatives, Two Independent Variables}

Suppose now we're standing on a hot plate, where the temperature varies from point to point. Unlike in one dimension, where we have a  choice of only two directions to walk, we now have an entire circle of directions from which to choose. How can we compute the (instantaneous) rate of change in the temperature if we head off in an arbitrary direction from our point. As in the first example, the rate of change is with respect to distance. There is no time involved here.

It might come as a surprise that knowing the rate of change in any two independent directions (ie. directions that are not the same nor directly opposite one another) allows us to find the rate of change in any direction, provided that the temperature is a differentiable function of position.

Just as in one dimension, we can use the level curves of the temperature function to approximate the rate of change in any direction. But before we get to this, take a moment to think about the following question.

\begin{question}  \label{Qwerder432}
Suppose that from your current position, the temperature increases at the rate of $0.02^\circ$C/meter in the direction of the vector $\langle 1, 0 \rangle$ and that it increases at the rate of $0.04^\circ$C/meter in the direction of the vector $\langle 0,1\rangle$. What do you think is true about the rate at which the temperature changes in the direction of the vector ${\bf v} = \langle 1,1\rangle$?

\begin{enumerate}

\item The temperature increases at the rate of exactly $0.03^\circ$C/meter.

\item The temperature increases at a rate somewhere between $0.02^\circ$C/meter and $0.04^\circ$C/meter, but not necessarily at the exact rate of $0.03^\circ$C/meter.

\item The temperature increases at a rate less than $0.02^\circ$C/meter.

\item The temperature increases at a rate greater than $0.04^\circ$C/meter.

\item The temperature decreases.

\item We cannot say, even if we assume (as do the above choices) that the temperature is a differentiable function of position.

\end{enumerate}

Try to justify your response with an intuitive argument without doing any computations.
\begin{freeResponse}
\end{freeResponse}

\end{question}




\begin{question}  \label{Qds54546bd}

The function 
\[
     T= F(x,y) %= x^2 -xy +y^2
\]
expresses the temperature (in Celsius degrees) at a point in the plane in terms of its coordinates, measured in meters.

Some level curves 
\[
     F(x,y)=k
\]
of the function $F$ are shown below. 

The slider $\Delta s$ controls the spacing between consecutive tick marks on the number line. The slider $\Delta k$ controls the temperature difference between consecutive level curves. The red level curve has equation $f(x,y)=k$.

\begin{enumerate}
\item Use the level curves along with the sliders $\Delta k$, $\theta$, and $s$ to approximate the following instantaneous rates of change of temperature in the direction of the vector ${\bf v} = \langle \cos\theta, \sin\theta \rangle$. 



\begin{enumerate}

\item At the point $A(2,2)$ in the direction of the vector ${\bf v} = \langle 1, 0 \rangle$.

\item  At the point $A(2,2)$ in the direction of the vector ${\bf v} = \langle 0,1 \rangle$.

\item  At the point $A(2,2)$ in the direction of the vector ${\bf v} = \langle 3,1 \rangle$ where $\theta \sim 0.3$.

\item  At the point $A(2,2)$ in the direction of the vector ${\bf v} = \langle -2,1 \rangle$ where $\theta \sim 3.6$.

\end{enumerate}

\item In which direction from the point $A(2,2)$ does the function $F$ appear to increase at the fastest rate? Approximate this rate of change.

\item In which direction from the point $A(2,2)$ does the function $F$ appear to decrease at the fastest rate? Approximate this rate of change.

\item In which direction(s) from the point $A(2,2)$ does the function $F$ appear to remain constant? 


\pdfOnly{
Access Geogebra interactives through the online version of this text at
 
\href{https://www.geogebra.org/classic/cfkfwfpk}.
}
 
\begin{onlineOnly}
    \begin{center}
\geogebra{cfkfwfpk}{900}{600}
\end{center}
\end{onlineOnly}

Access Geogebra interactives through the online version of this text at
 
\href{https://www.geogebra.org/classic/cfkfwfpk}{163: Directional Derivative 1}.

\end{enumerate}

\end{question}


\begin{question}  \label{Ee53hhy}

Some level curves of a temperature function $T = f(x,y)$ are shown below. The output is measured in Celsius degrees and the input in meters.

\pdfOnly{
Access Geogebra interactives through the online version of this text at
 
\href{https://www.geogebra.org/classic/htee9xke}.
}
 
\begin{onlineOnly}
    \begin{center}
\geogebra{htee9xke}{900}{600}
\end{center}
\end{onlineOnly}

Access Geogebra interactives through the online version of this text at
 
\href{https://www.geogebra.org/classic/htee9xke}{163: Directional Derivative}.


\begin{enumerate}
\item Find an expression for $f$.
\[
     T = f(x,y) = \answer{2x + 4y}
\]

\item Use the sliders $\theta$ and $s$ to approximate the (instantaneous) rate at which the temperature changes from the origin in the directions of the following vectors. Include units.

Note:

\begin{itemize}

\item The slider $\Delta s$ controls the distance (in meters) between consecutive tick marks on the number line through the origin. 

\item The slider $k$ controls the temperature of the red level curve. That is, the red level curve has equation $f(x,y)=k$.

\item The slider $\Delta k$ controls the spacing of the level curves. That is, $\Delta k$ is the temperature difference between consecutive level curves.  

\end{itemize}

\begin{enumerate}

\item ${\bf v} = \langle 1 ,2 \rangle$, where $\theta \sim 1.1$

\item ${\bf w} = \langle -3,1\rangle$, where $\theta \sim 2.8$.

\end{enumerate}

\item Use a  difference quotient to compute the exact values of the above rates of change. Then compare these with your estimates. Click the arrow to the lower right for a partial solution.

\begin{expandable}
Because the temperature function is linear, we can avoid limits and compute the rate of change at the origin in the direction of the vector ${\bf v} = \langle 1, 2\rangle$ as a difference quotient.

In moving from the origin to the point $(1,2)$ the average rate at which the temperature changes with respect to the distance travelled is
\[
   \frac{\Delta T}{\Delta s} = \frac{f(1,2) - f(0,0)}{|\langle 1, 2 \rangle|}\frac{^\circ C}{m} = \frac{10}{\sqrt{5}}\frac{^\circ C}{m} .
\]

Then make a similar computation for rate of change in the direction of the vector ${\bf w} = \langle -3,1\rangle$.
\end{expandable}


\item Use a difference quotient to compute the rate  at which the temperature changes from the origin in the direction of the vector ${\bf v}=\langle v_x , v_y \rangle$. 

\item Check that your expression in the previous part has the correct units.

\item Interpret your expression for the rate of change as the scalar projection of one vector onto another. Then write this scalar projection as the product of a magnitude of the gradient vector and a trigonometric function of the angle between the gradient vector and the vector ${\bf v}$.



\end{enumerate}
\end{question}


\begin{question}  \label{Qedf4t5t}
(a) Find an expression for the rate of change of the function $z=f(x,y)$ from the point $(a,b)$ in the direction of the vector ${\bf v} = \langle v_x, v_y \rangle$ by replacing the function $z=f(x,y)$ with its linear approximation at the point $(a,b)$.

(b) In what direction from $(a,b)$ does the function increase at the fastest rate? What is this maximum rate?

(c) In what direction from $(a,b)$ does the function decrease at the fastest rate? What is this maximum rate?

(d) What is the rate of change in the function from $(a,b)$ in a direction inclined at the angle $\theta$ with the direction from part (b)?

(e)  In what direction(s) from $(a,b)$ does the function neither increase nor decrease?

\end{question}

\begin{question}  \label{Qdef67kk}
Go back to Question 4 and exact answers to parts 1 (d), (e), (f), (g) and 2-4.

\end{question}






\section{Graphical Interpretation}
\begin{exploration} \label{Esdft4ew3}

The function
\[
     h = f(x,y) = \frac{3y}{2x^2+2y^2+1}-2 
\]
expresses the depth of a lake (measured in hundred hundreds of feet) in terms of position (measured in miles). The graph of the function $h= - f(x,y)$ is shown below.

\pdfOnly{
Access Geogebra interactives through the online version of this text at
 
\href{https://www.geogebra.org/classic/md2udchf}.
}
 
\begin{onlineOnly}
    \begin{center}
\geogebra{md2udchf}{900}{600}
\end{center}
\end{onlineOnly}

Access Geogebra interactives through the online version of this text at
 
\href{https://www.geogebra.org/classic/md2udchf}{163: Directional Derivatives Lake Bottom}.

\pskip

A boat on the surface is located at the point $A(1,0)$.

(a) Does the lake bottom directly below $A$ slope upward or downward in the direction toward the point $(0,-1)$? At what rate? 

First use the level curves (spaced at intervals $\Delta h = 0.2$) of the function $f$ and the graph to approximate the rate. Then find the exact rate. Be sure to include units.

(b) What angle does the lake bottom make with the horizontal at $A$ in the direction toward the point $(0,-1)$?

(c) In what direction does the lake bottom rise at the greatest rate at $A$? Find this rate and the angle the lake bottom makes at $A$ with the horizontal in this direction.

(d) In what directions is the lake bottom horizontal at $A$?

(e) Find an expression for the function $h=g(s)$ that gives the depth of the lake (in hundreds of feet) in terms of the signed distance from $A$ (measured on the surface in miles) in the direction toward the point $(0,-1)$. Use the parameterization to graph the function $h$. Then use the function to verify the rate from part (a). 

\end{exploration}

\begin{exploration}  \label{Edddd33332}
The function
\[
     h = f(x,y) = 2 - \frac{3y}{2x^2+2y^2+1}
\]
gives the height of a mountain (measured in hundred hundreds of feet) above the point with coordinates $(x,y,0)$. The coordinates are measured in miles.

A trail on the mountain lies directly above the vertical plane through the points $A(1,0)$ and $B(0,-1)$.

Parameterize the graph of the function $h=g(s)$ that expresses the altitude along the trail in terms of the signed distance along the trail, measured in miles from the point $(1,0,f(1,0))$. Take the positive direction to be from $(1,0,f(1,0))$ toward $(0,-1,f(0,-1))$. 


\end{exploration}


\end{document}
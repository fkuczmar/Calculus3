\documentclass{ximera}
\title{Directional Derivatives}

\newcommand{\pskip}{\vskip 0.1 in}

\begin{document}
\begin{abstract}
Introduction to directional derivatives.
\end{abstract}
\maketitle


\section{Directional Derivatives, One Independent Variable}

\begin{question}  \label{Qdsdfsddsf6bd}

The function
\[
   T = f(x) = 5 + 3\sin \left(   \frac{\pi}{6}x  \right)   , -9\leq x \leq 9 ,
\]
expresses the temperature (in Celsius degrees) at a point on a thin rod in terms of the displacement (in meters) from the rod's center.

The graph of the function $T=f(x)$ is shown below along with some of its level sets plotted on the $x$-axis.

(a) Describe the nature of the level sets shown below.


\begin{itemize}

\item{What is the spacing between consecutive level sets?}

\item{How many points are in most level sets? How many points are in the others?} 

\item{Some level sets are colored red, others blue. Why?}

\end{itemize}

(b) Drag the slider in Line 2 to $n=24$. Then zoom in on the origin. Use the level sets to approximate the value of $f^\prime(0)$. Include units and explain the meaning of the derivative.

(c) Compute the exact value of $f^\prime(0)$ and compare it with your estimate.

(d) At what (instantaneous) rate does the temperature of the rod change at $x=0$ in the direction of the negative $x$-axis? 
d
\pdfOnly{
Access Desmos interactives through the online version of this text at
 
\href{https://www.desmos.com/calculator/1ztmnpaymf}.
}
 
\begin{onlineOnly}
    \begin{center}
\desmos{1ztmnpaymf}{900}{600}
\end{center}
\end{onlineOnly}

Access Desmos interactives through the online version of this text at
 
\href{https://www.desmos.com/calculator/1ztmnpaymf}{163: Directional Derivative 1}.
\end{question}



\section{Directional Derivatives, Two Independent Variables}


\begin{question}  \label{Qds54546bd}

The function 
\[
     T= F(x,y) = x^2 -xy +y^2
\]
expresses the temperature (in Celsius degrees) at a point in the plane in terms of its coordinates, measured in meters.

Some level curves 
\[
     F(x,y)=k
\]
of the function $F$ are shown below. 

1. Use the level curves along with the sliders $\Delta k$, $\theta$, and $s$ to approximate the following instantaneous rates of change of temperature in the direction of the vector ${\bf v} = \langle \cos\theta, \sin\theta \rangle$. The slider $s$ adjusts the length of the vector ${\bf v}$ (blue). Adjust $\Delta k$ to get better approximations.

\pskip

(a) At the point $A(2,2)$ in the direction of the vector ${\bf v} = \langle 1, 0 \rangle$.

(b)  At the point $A(2,2)$ in the direction of the vector ${\bf v} = \langle 0,1 \rangle$.

(c)  At the point $A(2,2)$ in the direction of the vector ${\bf v} = \langle 0, -1 \rangle$.

(d)  At the point $A(2,2)$in the direction of the vector ${\bf v} = \langle 1,1 \rangle$.

(e)  At the point $A(2,2)$ in the direction of the vector ${\bf v} = \langle -1,-1 \rangle$.

(f)  At the point $A(2,2)$ in the direction of the vector ${\bf v} = \langle 1,-1 \rangle$..

(g) At the origin in various directions of your choosing.

\pskip \pskip

2. In which direction from the point $A(2,2)$ does the function $F$ appear to increase at the fastest rate? Approximate this rate of change.

\pskip \pskip

3. In which direction from the point $A(2,2)$ does the function $F$ appear to decrease at the fastest rate? Approximate this rate of change.

\pskip \pskip

4. 2. In which direction(s) from the point $A(2,2)$ does the function $F$ appear to remain constant? 


\pdfOnly{
Access Geogebra interactives through the online version of this text at
 
\href{https://www.geogebra.org/classic/cfkfwfpk}.
}
 
\begin{onlineOnly}
    \begin{center}
\geogebra{cfkfwfpk}{900}{600}
\end{center}
\end{onlineOnly}

Access Geogebra interactives through the online version of this text at
 
\href{https://www.geogebra.org/classic/cfkfwfpk}{163: Directional Derivative 1}.



\end{question}




\end{document}
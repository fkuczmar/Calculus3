\documentclass{ximera}
\title{Rotating About the Origin}

\newcommand{\pskip}{\vskip 0.1 in}

\begin{document}
\begin{abstract}
Rates of rotation. Angular momentum. 
\end{abstract}
\maketitle

\section{Introduction}

Here's the question for this chapter.

Suppose at a certain instant, you are $20$ miles from home and driving at a speed of $50$ miles/hour. 

\begin{enumerate}
\item What can you say, if anything, about the rate (with respect to time) at which you are rotating about your home at this instant? Can you find lower and upper bounds for this rate?

\item What additional information would you need to determine this rate? 
\end{enumerate}


\begin{question} \label{QRAZxdr3r3}

The function
\[
 {\bf p}(t) = \langle  3+5\cos t,4+2\ \sin t \rangle \, , \, 0\leq t \leq 2\pi ,
\]
expresses the position (in meters) of a beetle relative to the origin in terms of the number of minutes past noon.

\begin{onlineOnly}
    \begin{center}
\desmos{fne4ma8yx2}{900}{600}  % x4nun5yb
\end{center}
\end{onlineOnly}

               %\href{https://www.geogebra.org/classicx4nun5yb}{163: Distance to Origin}

\href{https://www.desmos.com/calculator/fne4ma8yx2}{163: Rotating About the Origin 1}

\begin{enumerate}
\item Play the slider $u$ in Line 2. Then sketch a graph of the function
\[
       \theta = f(t) \, , \, 0\leq t \leq 2\pi ,
\] 
that expresses the marked angle $\theta$ (measured in radians) as a function of the number of minutes past noon.

\item Use your graph in part (a) to sketch a graph of the function 
\[
  r = \frac{d\theta}{dt} = f^\prime(t) \, , \, 0\leq t \leq 2\pi .
\]

\item Explain the meaning of the function $r=f^\prime(t)$ in this context. Include units in your explanation.

\item Activate the folder \emph{Graphs} in Line 8 to see how you did.

\end{enumerate}
\end{question}



\begin{question}  \label{Q8fre3rr}
This is a continuation of the first problem, but instead of working with the specific position function there, we'll make things easier and work in general.

So we'll suppose the position function (relative to the origin, measured in meters) is
\[
      {\bf p}(t) = \langle f(t), g(t) \rangle ,
\]
with input $t$ measured in the number of seconds past noon.

Our goal is to find an expression for the function $r = h(t)$ that gives the beetle's rotation rate (in rad/sec) about the origin at time $t$ seconds past noon. This is really a related-rates problem and we'll use ideas from differential calculus to solve it.

\begin{enumerate}
\item Write an equation that relates the component functions $f(t)$, $g(t$) of the position vector and the signed angle $\theta$ from the positive $x$-axis to the position vector, measured counterclockwise.

\item Then differentiate both sides of your equation with respect to time to find an expression for the rotation rate function $r=h(t)$.

\item Check that your function is dimensionally correct.

\item Use your function and the position function from Question 1 to determine these rates of rotation at times $t=0$ and $t=3\pi/2$ minutes past noon.
\end{enumerate}

\end{question}

\begin{question}  \label{QLDferfMAZ}
This question relates the ideas in the previous question to vectors.

The idea is to relate the signed rotation rate about the origin to the vector product 
\[
 {\bf L}(t) = {\bf p}(t) \times {\bf v}(t)
\]
of the position and velocity vectors of the motion.


\begin{onlineOnly}
    \begin{center}
\desmosThreeD{lq9fvpsrh0}{900}{600}  % x4nun5yb
\end{center}
\end{onlineOnly}

               %\href{https://www.geogebra.org/classicx4nun5yb}{163: Distance to Origin}

\href{https://www.desmos.com/3d/lq9fvpsrh0}{163: Rotating About the Origin 3D}

\begin{enumerate}
\item Watch the motion above (the orange vector is ${\bf L}(t)$) and make a conjecture about the relationship between the signed rotation rate and the vector ${\bf L}(t)$. Be sure you understand why the vector ${\bf L}(t)$ points upward at sometimes and downward at others.

\item Prove  your conjecture by finding an expression for ${\bf L}(t)$ for the general motion
\[
  {\bf p}(t) = \langle f(t), g(t) \rangle
\]
\end{enumerate}
\end{question}


The takeaway is this:

The signed rotation rate about the origin for a motion ${\bf p}(t)$ in the $xy$-plane is equal to %the scalar projection
\[
   r(t) =  \frac{ ( {\bf p}(t) \times {\bf v}(t) ) \cdot {\bf k}}{\left| {\bf p}(t) \right|^2} .
\]
This is the quotient of the scalar projection of the vector 
\[
    {\bf L}(t) = {\bf p}(t) \times {\bf p}^\prime(t)
\] 
in the direction ${\bf k} = \langle 0,0,1 \rangle$ of the positive $z$-axis and the square of the distance $|{\bf p}(t)|$ from the origin.

\end{document}

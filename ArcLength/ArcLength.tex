\documentclass{ximera}
\title{ArcLength}

\newcommand{\pskip}{\vskip 0.1 in}

\begin{document}
\begin{abstract}
Parameterizing curves by arclength. 
\end{abstract}
\maketitle

Shoreline Community College is just off Exit 176 of I-5. The exit number is not random. It gives the distance to the Washingto-Oregon border in \emph{miles}, not as the crow flies but along the twists and turns of the interstate.

It helps to keep this image in mind when we parameterize a curve by arclength. What this means is to first establish a number line along the curve, much like the exit numbers on I-5 in Washington state. We then express the position (relative to the origin) of each point on the curve in terms of its exit number, or really in terms of its place on the number line.

In establishing a number line along a curve we have two choices. The first is where to put the zero marker (the Washington-Oregon border in our example) and the second is which direction to take as positive along the curve (the northward direction along I-5 in our example). One subtle point. Despite the name, arclength is negative for points on the negative half of the number line. So rightly speaking, were we to continue our arclength parameterization of I-5 into Oregon, the exits numbers in Oregon would be negative. Oregonians might object.

Here are some simple examples to illustrate this idea.

\section{Parameterizing Lines by ArcLength}

\begin{example} \label{E324tg4tg45te}
Parameterize the line $x=3$ in the $xy$-plane by arclength. Take the arclength parameter $s$ to be zero at the point $(3,5)$ and the positive direction to point ``downward" (ie. in the direction of the vector $\langle 0, -1 \rangle$). Suppose the coordinates are measured in meters.

\begin{explanation}
The point $(x,y)$ on the line $x=3$ at arclength parameter $s$ on our number line has coordinates 
\[
   (x,y) = (3 , 5 - s) \, , \, s\in\mathbb{R}.
\] 
So the function
\[
    {\bf p}(s) = \langle 3, 5 - s \rangle \, , \, s\in\mathbb{R}
\]
expresses the postion (relative to the origin) of a point on the line $x=3$ in terms of its arclength parameter. Just to check,
\[
 {\bf p}(-8) = \langle 3, 5- (-8) \rangle = \langle 3, 13 \rangle .
\]
This is correct since the point with coordinates $(3,13)$ lies on the line $x=3$ and is $8$ units in the negative direction from our zero marker $(3,5)$.

\begin{question}\label{Qdefrd54555354}
What are the units of the arclength parameter in this example?
\begin{multipleChoice}
\choice{pounds}
\choice{None. The parameter is dimensionless.}
\choice[correct]{meters}
\end{multipleChoice}
\end{question}

\end{explanation}
\end{example}


\begin{question}  \label{Egrett4tt3}
Parameterize the line in $\mathbb{R}^3$ through the point $A(4,-1,3)$ and $B(-1,1,0)$ by arclength. Take the arlength parameter to be zero at $A$ and the positive direction to point in the direction of the vector $\overrightarrow{AB}$. The coordinates are measured in meters.

\begin{onlineOnly}
    \begin{center}
\desmosThreeD{dyxjd322mo}{900}{600}
\end{center}
\end{onlineOnly}

\href{https://www.desmos.com/3d/dyxjd322mo}{163: ArcLength Parameterization Line 1}


\begin{explanation}
\begin{enumerate}
\item The key idea is to compute the unit vector 
\[
  {\bf u} = \frac{\overrightarrow{AB}}{|\overrightarrow{AB}|}
\]
in the positive direction of our number line. This unit vector is
\[
   {\bf  u} =   \answer{\frac{1}{\sqrt{38}}} \overrightarrow{AB} .
\]

So the position (relative to the origin $O$) of the point $Q$ at arclength parameter $s=4$, for example, is
\begin{align*}
  \overrightarrow{OP} &= \overrightarrow{OA} + \answer{4}{\bf u} \\
                                &= \langle 4, -1, 3\rangle + \frac{4}{\sqrt{38}} \langle \answer{-5}, \answer{2}, \answer{-3} \rangle .
\end{align*}

And more generally, the position of the point $P$ at arclength parameter $s\in \mathbb{R}$ is
\begin{align*}
   \overrightarrow{OP} &= \overrightarrow{OA} + \answer{s}{\bf u} \\ 
                                 &= \langle 4, -1, 3\rangle + \answer{\frac{s}{\sqrt{38}}} \langle \answer{-5}, \answer{2}, \answer{-3} \rangle .
\end{align*}
And our arclength parameterization of the line is
\[
    {\bf p}(t) =    \langle 4, -1, 3\rangle + \frac{s}{\sqrt{38}} \langle \answer -5, 2, -3 \rangle              \, , \, s\in\mathbb{R}.
\]

\item Here are a few questions about this example.

\begin{enumerate}

\item Check this parameterization in the desmos worksheet above by activating Line 15 and thereby plotting the points on the line at the arclength parameters $s=-5, -4, \ldots , 14, 15$. These should be spaced $\answer{1}$ meter apart. Do they look like it?

\item Compute the derivative 
\[
     {\bf T}(s) = \frac{d}{ds}\left( {\bf p}(s) \right) . 
\]
What are its units?

\item Compute the scalar
\[
   \Big| {\bf T}(s) \Big|  = \Big|  \frac{d}{ds}\left( {\bf p}(s) \right) \Big| .
\]
What are its units? 
\begin{freeResponse}
\end{freeResponse}
\end{enumerate}

\end{enumerate}

\end{explanation}
 
\end{question}


\section{Parameterizing Circles by ArcLength}
\begin{question}  \label{Qewr9erergg}
Parameterize by arclength the circle of radius $a$ centered at the origin that lies in the $xz$-plane. Take $s=0$ (the arclength parameter) at the point $A(a,0,0)$ and the positive direction to be counterclockwise as viewed from the point $(0,1,0)$. The coordinates have units of meters.

\begin{onlineOnly}
    \begin{center}
\desmosThreeD{6uvzmng195}{900}{600}
\end{center}
\end{onlineOnly}

\href{https://www.desmos.com/3d/6uvzmng195}{163: ArcLength Parameterization Circle}



\begin{explanation}
Looking from the point $(0,1,0)$ toward our circle, the counterclockwise sense of rotation swings from the vector $\overrightarrow{OA} = \langle a, 0, 0\rangle$ toward the vector $\overrightarrow{OB} = \langle 0,0,-a \rangle$. So we'll start by expressing the position of a point $P$ (relative to the origin) in terms of its polar angle $\theta = \angle AOP$, and the vectors $\overrightarrow{OA}$ and $\overrightarrow{OB}$.  

\begin{align}
   \overrightarrow{OP} &= (\answer{\cos \theta}) \overrightarrow{OA} + (\answer{\sin\theta} )\overrightarrow{OB} \\
                                 &= \langle  \answer{a\cos\theta} , \answer{0}   ,  \answer{-a\sin\theta}   \rangle .
\end{align}


Now suppose $P$ has arclength parameter $s$. The next step is to express the radian measure $\theta$ of the angle $\angle AOP$ in terms of $s$ and $a$. By the definition of what it means to measure an angle in radians,
\[
    \theta = \answer{s/a} .
\]

Finally, by substitution, we get the arclength parameterization of our circle:
\[
   {\bf p}(s) = \langle  \answer{a\cos (s/a)} , \answer{0}   ,  \answer{-a\sin (s/a)}   \rangle , -\pi a \leq s \leq \pi a.
\]

Here are a few questions about this example.

\begin{enumerate}
\item Check this parameterization in the desmos worksheet above by activating Line 17 and thereby plotting the points on the circle $1$ meter apart. Does it look correct? %Show the screenshot.

\item Compute the derivative 
\[
     {\bf T}(s) = \frac{d}{ds}\left( {\bf p}(s) \right) . 
\]
What are its units?

\item Compute the scalar
\[
   \Big| {\bf T}(s) \Big|  = \Big|  \frac{d}{ds}\left( {\bf p}(s) \right) \Big| .
\]
What are its units? 

\item Find another possible domain for the above parameterization.

\begin{freeResponse}
\end{freeResponse}

\end{enumerate}


\end{explanation}
\end{question}



\section{Sliding Down a Cylindrical Helix}
\begin{question}  \label{Qdef54356452}
Draw a line on a piece of paper that makes the \emph{pitch} angle $\phi$ with the horizontal. Then bend the paper into a cylinder of radius $a$ meters symmetric about the $z$-axis by dragging the slider $k$ to $k=1$ in the demonstration below. This action bends the line into a cylindrical helix.

\begin{onlineOnly}
    \begin{center}
\desmosThreeD{vrrpxfddv2}{900}{600}
\end{center}
\end{onlineOnly}

\href{https://www.desmos.com/3d/vrrpxfddv2}{163: Cylindrical Helix 1}

We'll assume the helix passes through the point $A(a,0,0)$ and wraps counterclocwise around the $z$-axis when viewed from above. 

\begin{enumerate}
\item  Parameterize the helix by arclength, taking $s=0$ (the arc length parameter) at $A$ with the positive direction pointing upward along the helix. Click on the arrow (below right) to reveal some hints.

\begin{expandable}
Let $P$ be a point on the helix at arclength parameter $s$. Let $Q$ be the orthogonal projection of $P$ onto the $xy$-plane.
Let $O$ be the origin as usual. And let $\theta = \angle AOQ$, measured in radians.
Let $B$ be the point (not on the helix) with coordinates $(0,a,0)$.

Make sure $k=1$ on Line 2 of the above worksheet.

\begin{enumerate}
\item Express the vector $\overrightarrow{OQ}$ in terms of the vectors $\overrightarrow{OA}$ and $\overrightarrow{OB}$. To better see this, deactivate Line 8 above to hide the cylinder.
\[
   \overrightarrow{OQ} = (\answer{\cos\theta})\overrightarrow{OA} + (\answer{\sin\theta})\overrightarrow{OB}
\]

\item Express the vector $\overrightarrow{OQ}$ in terms of $a$ and $\theta$. %To better see this, deactivate Line 8 above to hide the cylinder.
\[
    \overrightarrow{OQ} = (\cos\theta)\langle \answer{a},\answer{0},\answer{0} \rangle + (\sin\theta)\langle \answer{0},\answer{a},\answer{0} \rangle
\]


%\item Express the length $\Big| \overrightarrow{QP}\Big|$ in terms of $a$, $\theta$, and $\phi$.

\item Express the vector $\overrightarrow{QP}$ in terms of $s$ and $\phi$, and $\phi$. %\emph{Hint:} First express the length of this vector in terms of $t$ and $\phi$, and then in terms of $a$, $\theta$, and $\phi$.
\[
         \overrightarrow{QP} = \left( \answer{s(\sin\phi)} \right) \langle 0,0,1  \rangle
\]

%\item Use the previous parts to express $\overrightarrow{OP}$ in terms of $a$, $\theta$, and $\phi$.
%\[
%         \overrightarrow{OP} =    (\cos\theta)\langle \answer{a},\answer{0},\answer{0} \rangle + (\sin\theta)\langle \answer{0},%\answer{a},\answer{0} \rangle +  \left( \answer{a\theta \tan\phi} \right) \langle 0,0,1  \rangle
%\]


\item We're almost there. All that's left is to express $\theta$ in terms of $a$, $s$, and $\phi$.  As a first step, express $\theta$ in terms of the radius $a$ and the length $t$ of circular arc $AQ$. 
\[
            \theta = \answer{t/a} .
\]

\item Now Express $\theta$ in terms of $s$, $a$, and $\phi$. %\emph{Hint:} First express the arclength parameter $s$ in terms of $a$, $\theta$, and $\phi$.
\[
     \theta = \frac{t}{a} = \frac{\answer{s \cos\phi}}{a}
\]

\item Finally, express ${\bf p}(s) = \overrightarrow{OP}$ in terms of $s$, $a$, and $\phi$. Include the appropriate domain for the helix to wrap once around the cylinder as shown.
\begin{align*}
    {\bf p}(s) &= \overrightarrow{OP}  + \overrightarrow{PQ} \\
                  & = \langle  a \cos \left( \frac{s\cos\phi}{a}\right)      ,   a \sin \left( \frac{s\cos\phi}{a}\right) , s \sin\phi          \rangle
\end{align*}

\end{enumerate}
\end{expandable}

\item Input your parameterization in Line 4 of the worksheet below. Use $\phi_1$ for the pitch angle instead of $\phi$.

\begin{onlineOnly}
    \begin{center}
\desmosThreeD{ts3pqrjnlo}{900}{600}
\end{center}
\end{onlineOnly}

\href{https://www.desmos.com/3d/ts3pqrjnlo}{163: Cylindrical Helix Student Copy}




\end{enumerate}
\end{question}





\section{Exercises}

\begin{question}  \label{Q43esdfgtgre}
Suppose ${\bf p}(s)$ is an arclength parameterization of the curve shown below, where $s$  and ${\bf p}(s)$ are measured in meters. Suppose also that
\[
   \overrightarrow{AB} = {\bf p}(2) \,\,\,\, \text{and} \,\,\,\, \overrightarrow{AC} = {\bf p}(5) .
\]
1. Sketch the following vectors as accurately as possible. Explain your reasoning.

(a) ${\bf p}(4)$

(b) ${\bf p}^\prime(4)$

(c) ${\bf p}(2) + 1000\left( {\bf p}(2.001)-{\bf p}(2)  \right)$

(d) $\frac{{\bf p}(5)\times (  {\bf p}(2)\times {\bf p}(5) )}{| {\bf p}(5)\times (  {\bf p}(2)\times {\bf p}(5) ) |}$

\pskip \pskip

2. Consider the expression
\[
   {\bf p}^\prime(2)\cdot {\bf p}^\prime(5) .
\]

(a) Is this expression a vector or a scalar? Explain your reasoning.

(b) What are the units of this expression? Explain your reasoning.

(c) Approximate the value of the expression. Explain your reasoning.

(d) Interpret the meaning of the expression. 


\pskip \pskip

3. Consider the expression
\[
    \frac{d}{ds} \left( \sqrt{{\bf p}(s) \cdot {\bf p}(s)} \right)\Big|_{s=5} .
\]

(a) Is this expression a vector or a scalar? Explain your reasoning.

(b) What are the units of this expression? Explain your reasoning.

(c) Approximate the value of the expression. Explain your reasoning. {\it Hint:} Take the derivative and simplify your expression. Then interpret the derivative geometrically. 

(d) Interpret the meaning of the expression. 

(e) Explain the meaning of the difference
\[
      \sqrt{{\bf p}(5.02) \cdot {\bf p}(5.02)} - \sqrt{{\bf p}(5) \cdot {\bf p}(5)} .
\]
Then use your responses to parts (c) and (d) to approximate the difference. Include units and explain your reasoning.


\pdfOnly{
Access Geogebra interactives through the online version of this text at
 
\href{https://www.geogebra.org/classic/fje6rwph}.
}
 
\begin{onlineOnly}
    \begin{center}
\geogebra{fje6rwph}{900}{600}
\end{center}
\end{onlineOnly}
\end{question}


\begin{question}

\end{question}



\section{Unwrapping a Spool of Thread}

\begin{question}  \label{Qgdghubhgfdg}

Imagine a thread wrapped around a spool of radius $a$ cm. Grab the free end of the thread, and uwrap it from the spool, keeping the thread taut at all times. Drag the slider $\theta$ in the demonstration below to see this.

\pdfOnly{
Access Geogebra interactives through the online version of this text at
 
\href{https://www.geogebra.org/classic/ynujwbty}.
}
 
\begin{onlineOnly}
    \begin{center}
\geogebra{ynujwbty}{900}{600}
\end{center}
\end{onlineOnly}


Our aim is to express the speed of the free end of the thread ($P$) at any particular instant in terms of the radius $a$, the length of unwound thread $s$ (in cm), and the rate $v_A$ (in cm/sec) at which the thread is unwound from the spool. In the demonstration above, the unwinding rate is constant, but we will not assume this. 

We'll proceed as follow:

(a) First find an arclength parameterization of the circle of radius $a$ centered at the origin. Measure the arclength parameter $s$ counterclockwise around the circle, taking $s=0$ at the point $(a,0)$. Enter this parameterization in the line below, where the position vector ${\bf p}(s) = \overrightarrow{OA}$ runs from the origin to the point on the circle with arclength parameter $s$:
\[
    {\bf p}(s) = \langle \answer{a \cos (s/a)} , \answer{a \sin (s/a)}  \rangle \, , s\geq 0 
\]

(b) Next find a unit vector ${\bf T}(s)$ pointing in the direction of motion of point $A$ (this is the point where the unwrapped portion of the thread is tangent to the spool). 
\[
    {\bf T}(s) = \langle \answer{- \sin (s/a)} , \answer{\cos (s/a)}  \rangle \, , s\geq 0
\]

(c) Now find the components of the vector ${\bf q}(s) = \overrightarrow{AP}$, from $A$ to the free end of the thread. {\it Hint:} Think about the length of this vector and its direction.
\[
     {\bf q}(s) = \overrightarrow{AP} = \langle \answer{s \sin (s/a)} , \answer{-s \cos (s/a)}  \rangle \, , s\geq 0
\]

(d) Write the vector ${\bf r}(s) = \overrightarrow{OP}$ in terms of the vectors ${\bf p}(s)$ and ${\bf q}(s)$:
\begin{align*}
  {\bf r}(s)   &= {\bf p}(s) + {\bf q}(s)   \\
                  &= a \langle \answer{\cos (s/a)} , \answer{\sin (s/a)}  \rangle  +  s \langle \answer{\sin (s/a)} , \answer{-\cos (s/a)}
\end{align*}


(e) What is the meaning of the derivative $d{\bf r}(s)/dt$, where $t$ is time measured in seconds?
\begin{multipleChoice}  
\choice{The speed of point $A$}  
\choice{The speed of point $P$}  
\choice{The velocity of point $A$}  
\choice[correct]{The velocity of point $P$} 
\choice{The acceleration of point $P$} 
\end{multipleChoice} 


(f) Use your expression for ${\bf r}(s)$ from part (d), to find an expression for the derivative $d{\bf r}(s)/dt$, simplified as much as possible. Your expression should be in terms of $s$, $\theta$, and the unwinding rate $v_A$. Keep in mind that $v_A$ is also the speed of point $A$. Remember also that $s$ is a function of $t$.
\[
      \frac{d{\bf r}(s)}{dt} = \langle  \answer{\frac{s v_A}{a} \cos(s/a)}  , \answer{\frac{sv_A}{a} \sin(s/a)}   \rangle \, , \, s\geq 0 
\]


(f) What is the direction of the vector $d{\bf r}(s)/dt$? Choose all that apply.
\begin{selectAll}  
    \choice{$d{\bf r}(s)/dt$ is parallel to ${\bf q}(s)$}  
    \choice[correct]{$d{\bf r}(s)/dt$ is perpendicular to ${\bf q}(s)$}  
    \choice[correct]{$d{\bf r}(s)/dt$ is parallel to ${\bf p}(s)$}  
    \choice{$d{\bf r}(s)/dt$ is perpendicular to ${\bf p}(s)$}  
  \end{selectAll} 


(g) What is the speed of $P$?
\[
   v_P = \answer{\frac{sv_A}{a}} \,    \frac{\text cm}{\text{sec}}
\]


(h) Check that your result for the speed $v_P$ has the correct units. Then write $v_P$ as a multiple of the unwiding rate $v_A$ and interpret the meaning of the multiple.

(i) Optional: Parameterize the motion of $A$ around the circle so that $P$ travels at a constant speed $v_P$ cm/sec. {\it Hint:} Start by finding an arclength parameterization of the path traced by $P$. Then use this to express the arclength parameter $s$ around the circle as a function of time.

\end{question}




\begin{question}  \label{Qgtu5uhgfdg}
Repeat Question 1 for a string wrapped around the cylindrical helix
\[
   {\bf p}(\phi) = \langle a \cos \phi, a \sin \phi, b \phi  \rangle , \phi \geq 0 ,
\]
where $a$ and $b$ are constants with units of centimeters. Start by finding an arclength parameterization of the helix taking $s=0$ at the point $(a,0,0)$. Unwrap the string from this point.  

Drag the slider $s$ in the demonstration below to see the unwrapping motion. 

\pdfOnly{
Access Geogebra interactives through the online version of this text at
 
\href{https://www.geogebra.org/classic/waamu43s}.
}
 
\begin{onlineOnly}
    \begin{center}
\geogebra{waamu43s}{900}{600}
\end{center}
\end{onlineOnly}

\end{question}



\end{document}
\documentclass{ximera}
\title{ArcLength}

\newcommand{\pskip}{\vskip 0.1 in}

\begin{document}
\begin{abstract}
Parameterizing curves by arclength. 
\end{abstract}
\maketitle


\section{Exercises}

\begin{question}  \label{Q43esdfgtgre}
Suppose ${\bf p}(s)$ is an arclength parameterization of the curve shown below, where $s$  and ${\bf p}(s)$ are measured in meters. Suppose also that
\[
   \overrightarrow{AB} = {\bf p}(2) \,\,\,\, \text{and} \,\,\,\, \overrightarrow{AC} = {\bf p}(5) .
\]
1. Sketch the following vectors as accurately as possible. Explain your reasoning.

(a) ${\bf p}(4)$

(b) ${\bf p}^\prime(4)$

(c) $1000\left( {\bf p}(2.001)-{\bf p}(2)  \right)$

(d) $\frac{{\bf p}(5)\times (  {\bf p}(2)\times {\bf p}(5) )}{| {\bf p}(5)\times (  {\bf p}(2)\times {\bf p}(5) ) |}$

2. Consider the expression
\[
    \frac{d}{ds} \left( \sqrt{{\bf p}(s) \cdot {\bf p}(s)} \right)\Big|_{s=5} .
\]

(a) Is this expression a vector or a scalar? Explain your reasoning.

(b) What are the units of this expression? Explain your reasoning.

(c) Approximate the value of the expression. Explain your reasoning. {\it Hint:} Take the derivative and simplify your expression. Then interpret the derivative geometrically. 

(d) Interpret the meaning of the expression. 

(e) Explain the meaning of the difference
\[
      \sqrt{{\bf p}(5.02) \cdot {\bf p}(5.02)} - \sqrt{{\bf p}(5) \cdot {\bf p}(5)} .
\]
Then use your responses to parts (c) and (d) to approximate the difference. Include units and explain your reasoning.


\pdfOnly{
Access Geogebra interactives through the online version of this text at
 
\href{https://www.geogebra.org/classic/fje6rwph}.
}
 
\begin{onlineOnly}
    \begin{center}
\geogebra{fje6rwph}{900}{600}
\end{center}
\end{onlineOnly}
\end{question}


\begin{question}

\end{question}



\section{Unwrapping a Spool of Thread}

\begin{question}  \label{Qgdghubhgfdg}

Imagine a thread wrapped around a spool of radius $a$ cm. Grab the free end of the thread, and uwrap it from the spool, keeping the thread taut at all times. Drag the slider $\theta$ in the demonstration below to see this.

\pdfOnly{
Access Geogebra interactives through the online version of this text at
 
\href{https://www.geogebra.org/classic/ynujwbty}.
}
 
\begin{onlineOnly}
    \begin{center}
\geogebra{ynujwbty}{900}{600}
\end{center}
\end{onlineOnly}


Our aim is to express the speed of the free end of the thread ($P$) at any particular instant in terms of the radius $a$, the length of unwound thread $s$ (in cm), and the rate $v_A$ (in cm/sec) at which the thread is unwound from the spool. In the demonstration above, the unwinding rate is constant, but we will not assume this. 

We'll proceed as follow:

(a) First find an arclength parameterization of the circle of radius $a$ centered at the origin. Measure the arclength parameter $s$ counterclockwise around the circle, taking $s=0$ at the point $(a,0)$. Enter this parameterization in the line below, where the position vector ${\bf p}(s) = \overrightarrow{OA}$ runs from the origin to the point on the circle with arclength parameter $s$:
\[
    {\bf p}(s) = \langle \answer{a \cos (s/a)} , \answer{a \sin (s/a)}  \rangle \, , s\geq 0 
\]

(b) Next find a unit vector ${\bf T}(s)$ pointing in the direction of motion of point $A$ (this is the point where the unwrapped portion of the thread is tangent to the spool). 
\[
    {\bf T}(s) = \langle \answer{- \sin (s/a)} , \answer{\cos (s/a)}  \rangle \, , s\geq 0
\]

(c) Now find the components of the vector ${\bf q}(s) = \overrightarrow{AP}$, from $A$ to the free end of the thread. {\it Hint:} Think about the length of this vector and its direction.
\[
     {\bf q}(s) = \overrightarrow{AP} = \langle \answer{s \sin (s/a)} , \answer{-s \cos (s/a)}  \rangle \, , s\geq 0
\]

(d) Write the vector ${\bf r}(s) = \overrightarrow{OP}$ in terms of the vectors ${\bf p}(s)$ and ${\bf q}(s)$:
\begin{align*}
  {\bf r}(s)   &= {\bf p}(s) + {\bf q}(s)   \\
                  &= a \langle \answer{\cos (s/a)} , \answer{\sin (s/a)}  \rangle  +  s \langle \answer{\sin (s/a)} , \answer{-\cos (s/a)}
\end{align*}


(e) What is the meaning of the derivative $d{\bf r}(s)/dt$, where $t$ is time measured in seconds?
\begin{multipleChoice}  
\choice{The speed of point $A$}  
\choice{The speed of point $P$}  
\choice{The velocity of point $A$}  
\choice[correct]{The velocity of point $P$} 
\choice{The acceleration of point $P$} 
\end{multipleChoice} 


(f) Use your expression for ${\bf r}(s)$ from part (d), to find an expression for the derivative $d{\bf r}(s)/dt$, simplified as much as possible. Your expression should be in terms of $s$, $\theta$, and the unwinding rate $v_A$. Keep in mind that $v_A$ is also the speed of point $A$. Remember also that $s$ is a function of $t$.
\[
      \frac{d{\bf r}(s)}{dt} = \langle  \answer{\frac{s v_A}{a} \cos(s/a)}  , \answer{\frac{sv_A}{a} \sin(s/a)}   \rangle \, , \, s\geq 0 
\]


(f) What is the direction of the vector $d{\bf r}(s)/dt$? Choose all that apply.
\begin{selectAll}  
    \choice{$d{\bf r}(s)/dt$ is parallel to ${\bf q}(s)$}  
    \choice[correct]{$d{\bf r}(s)/dt$ is perpendicular to ${\bf q}(s)$}  
    \choice[correct]{$d{\bf r}(s)/dt$ is parallel to ${\bf p}(s)$}  
    \choice{$d{\bf r}(s)/dt$ is perpendicular to ${\bf p}(s)$}  
  \end{selectAll} 


(g) What is the speed of $P$?
\[
   v_P = \answer{\frac{sv_A}{a}} \,    \frac{\text cm}{\text{sec}}
\]


(h) Check that your result for the speed $v_P$ has the correct units. Then write $v_P$ as a multiple of the unwiding rate $v_A$ and interpret the meaning of the multiple.

(i) Optional: Parameterize the motion of $A$ around the circle so that $P$ travels at a constant speed $v_P$ cm/sec. {\it Hint:} Start by finding an arclength parameterization of the path traced by $P$. Then use this to express the arclength parameter $s$ around the circle as a function of time.

\end{question}




\begin{question}  \label{Qgtu5uhgfdg}
Repeat Question 1 for a string wrapped around the cylindrical helix
\[
   {\bf p}(\phi) = \langle a \cos \phi, a \sin \phi, b \phi  \rangle , \phi \geq 0 ,
\]
where $a$ and $b$ are constants with units of centimeters. Start by finding an arclength parameterization of the helix taking $s=0$ at the point $(a,0,0)$. Unwrap the string from this point.  

Drag the slider $s$ in the demonstration below to see the unwrapping motion. 

\pdfOnly{
Access Geogebra interactives through the online version of this text at
 
\href{https://www.geogebra.org/classic/waamu43s}.
}
 
\begin{onlineOnly}
    \begin{center}
\geogebra{waamu43s}{900}{600}
\end{center}
\end{onlineOnly}

\end{question}



\end{document}
\documentclass{ximera}
\title{ArcLength}

\newcommand{\pskip}{\vskip 0.1 in}

\begin{document}
\begin{abstract}
Parameterizing curves by arclength. 
\end{abstract}
\maketitle


\section{Unwrapping a Spool of Thread}

\begin{question}  \label{Qgdghubhgfdg}

Imagine a thread wrapped around a spool of radius $a$ cm. Grab the free end of the thread, and uwrap it from the spool, keeping the thread taut at all times. Drag the slider $u$ in the demonstration below to see this.

\pdfOnly{
Access Geogebra interactives through the online version of this text at
 
\href{https://www.geogebra.org/classic/ynujwbty}.
}
 
\begin{onlineOnly}
    \begin{center}
\geogebra{ynujwbty}{900}{600}
\end{center}
\end{onlineOnly}


Our aim is to express the speed of the free end of the thread ($P$) at any particular instant in terms of the radius $a$, the length of unwound thread $s$ (in cm), and the rate $v$ (in cm/sec) at which the thread is unwound from the spool. In the demonstration above, $v$ is constant, but we will not assume this. 

We'll proceed as follow:

(a) First find an arclength parameterization of the circle of radius $a$ centered at the origin. Measure the arclength parameter $s$ counterclockwise around the circle, taking $s=0$ at the point $(a,0)$. Enter this parameterization in the line below, where the position vector ${\bf p}(s) = \overrightarrow{OA}$ runs from the origin to the point on the circle with arclength parameter $s$:
\[
    {\bf p}(s) = \langle \answer{a \cos (s/a)} , \answer{a \sin (s/a)}  \rangle \, , s\geq 0 
\]

(b) Next find a unit vector ${\bf T}(s)$ pointing in the direction of motion of point $A$ (this is the point where the unwrapped portion of the thread is tangent to the spool). 
\[
    {\bf T}(s) = \langle \answer{- \sin (s/a)} , \answer{\cos (s/a)}  \rangle \, , s\geq 0
\]

(c) Now find the components of the vector ${\bf q}(s) = \overrightarrow{AP}$, from $A$ to the free end of the thread. {\it Hint:} Think about the length of this vector and its direction.
\[
     {\bf q}(s) = \overrightarrow{AP} = \langle \answer{s \sin (s/a)} , \answer{-s \cos (s/a)}  \rangle \, , s\geq 0
\]

(d) Write the vector ${\bf r}(s) = \overrightarrow{OP}$ in terms of the vectors ${\bf p}(s)$ and ${\bf q}(s)$:
\begin{align*}
  {\bf r}(s)   &= {\bf p}(s) \answer{+} {\bf q}(s)   \\
                  &= \langle \answer{a \cos (s/a)} , \answer{a \sin (s/a)}  \rangle  +  \langle \answer{s \sin (s/a)} , \answer{-s \cos (s/a)}
\end{align*}

\end{question}




\begin{question}  \label{Qgtu5uhgfdg}

\pdfOnly{
Access Geogebra interactives through the online version of this text at
 
\href{https://www.geogebra.org/classic/waamu43s}.
}
 
\begin{onlineOnly}
    \begin{center}
\geogebra{waamu43s}{900}{600}
\end{center}
\end{onlineOnly}

\end{question}



\end{document}
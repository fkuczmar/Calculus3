\documentclass{ximera}
\title{Escaping from the Origin}

\newcommand{\pskip}{\vskip 0.1 in}

\begin{document}
\begin{abstract}
Speed, velocity, distance and its rate of change. 
\end{abstract}
\maketitle

\section{Introduction}

Here's the question for this chapter.

Suppose at a certain instant, you are driving at a speed of $50$ miles/hour. 

\begin{enumerate}
\item What can you say about the rate of change (with respect to time) in your distance from home at this instant? Can this distance be increasing at the rate of $40$ miles/hour? Or decreasing at the rate of $60$ miles/hour? 

\item What additional information would you need to determine this rate? 
\end{enumerate}

\begin{question}  \label{Q34r066tt}
\begin{enumerate}

\item Interpret the meanings of the expressions 
\[ 
      \Big| \frac{d}{dt} \left( {\bf p}(t) \right) \Big|
\]
and
\[
   \frac{d}{dt} \left( |{\bf p}(t)| \right) .
\]
for a motion with position function (relative to the origin) ${\bf p}(t)$ measured in meters (and $t$ in seconds). Inlude units in your interpretations. Be sure to state clearly whether each is a vector or a scalar.
\begin{freeResponse}
\end{freeResponse}

\item One of the expressions above might be negative. Which one?

\item Which of the following inequalities are true? Choose all that apply.

\begin{selectAll}
\choice{ 
\[
        \Big| \frac{d}{dt} \left( {\bf p}(t) \right) \Big| \leq  \frac{d}{dt} \left( |{\bf p}(t)| \right) 
\]}

\choice[correct]{ 
\[
        \frac{d}{dt} \left( |{\bf p}(t)| \right)  \leq   \Big| \frac{d}{dt} \left( {\bf p}(t) \right) \Big| 
\]}

\choice[correct]{ 
\[
     \left|   \frac{d}{dt} \left( |{\bf p}(t)| \right) \right|  \leq   \Big| \frac{d}{dt} \left( {\bf p}(t) \right) \Big| 
\]}
\end{selectAll} 

\end{enumerate}
\end{question}



\begin{question} \label{Qdr3r3}

The function
\[
 {\bf p}(t) = \langle  3+5\cos t,4+2\ \sin t \rangle \, , \, 0\leq t \leq 2\pi ,
\]
expresses the position (in meters) of a beetle relative to the origin in terms of the number of minutes past noon.

\begin{onlineOnly}
    \begin{center}
\desmos{itv9aazspj}{900}{600}  % x4nun5yb
\end{center}
\end{onlineOnly}

               %\href{https://www.geogebra.org/classicx4nun5yb}{163: Distance to Origin}

\href{https://www.desmos.com/calculator/itv9aazspj}{163: Escape from Origin 1}

\begin{enumerate}
\item Play the slider $u$ in Line 2 to start the motion.

\item Find an expression for the function 
\[
       v = f(t) \, , \, 0\leq t \leq 2\pi
\]
that gives the beetle's speed (in meters/min) at time $t$ minutes past noon.

\item Find the beetle's speed at noon.

\item Is the distance of the beetle from the origin increasing or decreasing at noon? Use calculus to determine this rate. Then compare the rate with the beetle's speed at noon.

\item Resolve the beetle's velocity vector at noon into two components, one parallel to the position vector, the other normal to the position vector. What do you notice? Activate the folder \emph{Resolution of Velocity Vector} in Line 4 to see these components.

\item Is the distance of the beetle from the origin increasing or decreasing at time $t=\pi$ minutes past noon? Use calculus to determine this rate. Compare the rate with the beetle's speed at this time.

\item Resolve the beetle's velocity vector at time $t=\pi$ minutes past  into two components, one parallel to the position vector, the other normal to the position vector. What do you notice?

\item Use the animation to sketch by hand graphs of the functions

\begin{enumerate}
\item $s=f(t)$ that expresses the distance from the beetle to the origin at time $t$ minutes past noon, and

\item $r=g(t)$  that expresses the rate at which this distance is changing at time $t$ minutes past noon.

\item Activate the folder \emph{Graphs} on Line 8 to see these graphs. Be sure you can reconcile the graph of the function $g$ with the component of the velocity vector in the direction of the position vector. 
\end{enumerate}

\item Use the animation to approximate the times when the beetle is closest to and farthest from the origin. Then describe the relationship between the velocity and position vectors at these times. Explain why.

\end{enumerate}

\end{question}



\section{Escaping from the Origin} 
There are two ideas that I would like to emphasize repeatedly throughout the quarter:

\begin{enumerate}
\item Math is usually about working in general, not with specific numbers or functions. 

\item The whole point of working with vectors is to give a \emph{coordinate-free} description of math and the laws of physics. Avoid working with their components whenever possible.
\end{enumerate}

With these ideas in mind, let's address the problem of the last question in general.

The question is this:

Given a motion ${\bf p}(t)$ (in the plane, in space or in any number of dimensions) that expresses position relative to the origin as a function of time, we wish to find an expression for the (instantaneous) rate of change in the distance to the origin.

We'll answer this question twice:

\begin{enumerate}
\item once (in violation of the second principle above) working with the components of vectors, where we'll assume the motion 
\[
   {\bf p}(t) = \langle x , y \rangle = \langle f(t), g(t) \rangle 
\]
is in the plane,

\item and again, working with just vectors, not with their components. 

\end{enumerate}

\begin{question}  \label{QE56rertdstg}

Given a parameterization ${\bf p}(t)$ of a motion (in the plane, in space, or in any number of dimensions), find an expression for the (instantaneous) rate of change in the distance to the origin. 

Here are the two computations. On the left is the Math 151 version. On the right is the \emph{identical} computation written with vectors.

\begin{align*}
\frac{d}{dt}\Big| {\bf p}(t)  \Big| &=  \frac{d}{dt} \left(\sqrt{x^2+y^2}\right)    &&   \frac{d}{dt}\Big| {\bf p}(t)  \Big| = \frac{d}{dt} \left(  \sqrt{{\bf p} \cdot {\bf p}} \right) \\
 &= \frac{2x \frac{\answer{dx}}{\answer{dt}}+ 2y \frac{\answer{dy}}{\answer{dt}}}{2\sqrt{x^2 + y^2}}  && \hskip 0.5 in = \frac{{\bf v}\cdot {\bf p} + {\bf p}\cdot {\bf v}}{2\sqrt{{\bf p}\cdot {\bf p}}} \\
&= \frac{x \frac{dx}{dt} + y \frac{dy}{dt}}{\sqrt{x^2+y^2}}  &&  \hskip 0.5 in = \frac{{\bf v}\cdot {\bf p}}{|{\bf p}|} 
\end{align*}
\end{question}

The takeaway is this:

\emph{The rate of change in the distance to the origin is the scalar component of the velocity vector in the direction of the position (relative to the origin) vector.} This rate is is equal to the product $v\cos\theta$ of the speed and the cosine of the angle $\theta$ between the position and velocity vectors. Symbolically,
\begin{align*}
 \frac{d}{dt} \left( \left| {\bf p(t)} \right| \right) & = \frac{{\bf v}\cdot {\bf p}}{|{\bf p}|}  \\
                                                                      &= v\cos\theta.
\end{align*}



%In the context of our opening example, $\theta$ is the angle between the direction you're driving and your direction from home.

Two remarks:

\begin{enumerate}

\item We could have obtained this same result geometrically, \emph{without any computation}. Here's how.

Suppose as shown below we are given the position ${\bf p}(t)$ and velocity ${\bf v}(t)$ vectors of a beetle's motion at time $t$ when the beetle is at point $P$. 

\begin{onlineOnly}
    \begin{center}
\desmos{9kkzjv8vq9}{900}{600}  % x4nun5yb
\end{center}
\end{onlineOnly}

\href{https://www.desmos.com/calculator/9kkzjv8vq9}{163: Escape Lemma 54}

\begin{enumerate}

\item The first thing to notice is that the angle $\theta$ between the position and velocity vectors is \emph{not} obtuse (between $\pi/2$ and $\pi$) as it might appear. Rather it is acute (between $0$ and $\pi/2$). Remember, the angle $\theta$ is the angle between the \emph{directions} of the vectors. And to see this angle we should draw the vectors tail to tail.

\item Activate the folder \emph{Angle Between Vectors} on Line 1 to see the angle $\theta$.

\item To approximate the rate of change
\[
    \frac{d}{dt} \left( \left| {\bf p(t)} \right| \right)
\]
in the distance of the beetle from the origin, we'll suppose it crawls with constant velocity ${\bf p}(t)$ over a small time interval from time $t$ to time $t+\Delta t$ as it moves a distance
\[
      \Delta s = |{\bf v}(t)| \Delta t
\]
fom point $P$ to point $Q$. To see the beetle at point $Q$, activate the folder \emph{Solution} on Line 10.

\item Now we draw a circular arc centered at $O$ through $Q$ and let this arc intersect the line $OP$ at $R$. Then the change
\[
 \Delta \left( |{\bf p}(t)| \right) = |  {\bf p}(t+\Delta t) - {\bf p}(t)  |
\]
in the beetle's distance to the origin over this small time interval is just the length of $\overrightarrow{PR}$. But because $\Delta t$ is small, the circular arc from $Q$ to $R$ is approximately a line segment. So we can suppose points $P$, $Q$, and $R$ are the vertices of a right triangle $\Delta QRP$, with right angle at $R$.

Using right triangle trigonometry in $\Delta QRP$ tells us
\[
        PR \sim PQ \cos\theta 
\]
or that
\[
     \Delta \left( |{\bf p}(t)| \right) \sim (\Delta s)\cos\theta . 
\]
This equation approximates the change in the beetle's distance from the origin in terms of the angle $\theta$ and the distance it crawls during a small time interval $\Delta t$.

Dividing both sides by $\Delta t$ we get
\[
    \frac{ \Delta \left( |{\bf p}(t)| \right)}{\Delta t} \sim \frac{\Delta s}{\Delta t}\cos\theta . 
\]

Finally, taking a limit as $\Delta t\to 0$ suggests our result that
\[
   \frac{d}{dt} \left( \left| {\bf p(t)} \right| \right) = v\cos\theta.
\]

\end{enumerate}

\item This is really a related rates problem and you've probably seen one like this before.

\begin{question} \label{Q9odfhonn}
A kite drifts due east as it maintains a constant altitude. Play the slider $u$ in Line 1 of the worksheet below to watch the motion.

\pdfOnly{
Access Desmos interactives through the online version of this text at
 
\href{https://www.desmos.com/calculator/qdrzdt4erc}.
}
 
\begin{onlineOnly}
   \begin{center}
\desmos{qdrzdt4erc}{900}{600}
\end{center}
\end{onlineOnly}

\href{https://www.desmos.com/calculator/qdrzdt4erc}{151: Kite}

\begin{enumerate}
\item Which do you think is greater, the rate at which the string is being let out or the speed of the kite? No computations, just go with your intuition.

\item Use ideas of Math 151, \emph{not} vectors, to express the speed of the kite (not assumed constant) in terms of the rate at which the string is being let out (\emph{not} assumed constant) and the angle $\theta$ the string makes with the horizontal.

\item Compare the result of part (b) with what you just learned. Are they consistent?

\item What can you say if $\theta =  \pi/3$?

\item Was your intuition correct?
\end{enumerate}
\end{question}



\end{enumerate}





\section{Exercises}

\begin{exercise} \label{ExLdmfee}
At a certain instant a fly has position
\[
  {\bf p} = \langle -4, 5, 9  \rangle \text{ meters} 
\]
relative to the origin and velocity
\[
  {\bf v} = \langle 4, 2, -1  \rangle \text{ m/sec}.
\]

\begin{enumerate}
\item Find the speed of the fly at this instant.

\item Is the distance between the fly and the origin increasing or decreasing at this instant? At what rate?

\item Is the distance between the fly and the point $Q$ with coordinates $(-3, 1,5)$ increasing or decreasing at this instant? At what rate (with respect to time)?

\item At what rate, with respect to the distance the fly moves, is the fly's distance from the origin changing at this instant? Be sure to include units.
\end{enumerate}
\end{exercise}


\begin{question}  \label{QPferr34394}

The position and velocity vectors of a beetle at a moment in time are shown below.

\begin{onlineOnly}
   \begin{center}
\desmos{qw2v4hzsdj}{900}{600}
\end{center}
\end{onlineOnly}

\href{https://www.desmos.com/calculator/qw2v4hzsdj}{151: Kite}

Suppose 
\[
       |{\bf v}(t) | = 8 m/sec.
\]

Approximate

\begin{enumerate}

\item the rate of change (with respect to time) in beetle's distance from $O$ at this moment.

\item the rate of change in the beetle's distance from $O$ at this moment. This time the rate of change is with respect to the distance the beetle crawls.

\end{enumerate}


\begin{explanation}
Activate the folder \emph{Solution} in Line 1 above. From this we can see the following:

\begin{enumerate}
\item At this moment the beetle is approaching the origin at a rate equal to approximately $1/4$ of its speed. So at this moment the rate of change in its distance to the origin is about 
\[
  - \frac{1}{4}\left( 8 \text{ m/s}\right) = -2 \text{ m/s}.
\]

\item And at this moment it distance to the origin is changing at the rate of about 
\[
  - \frac{1}{4} \text{ m/m}.
\]
More simply put, the beetle is approaching the origin at a rate equal to about $1/4$ of its speed (whatever that speed might be).

\end{enumerate}

\end{explanation}

\end{question}







\begin{exercise} \label{ExLdmf347}
At a certain instant a fly has position
\[
  {\bf p} = \langle -4, 3  \rangle \text{ meters} 
\]
relative to the origin and a speed of $10$ m/sec.


\begin{enumerate}
\item Which of the following are possible? Choose all that apply.
\begin{selectAll}
  \choice[correct]{The fly is approaching the origin at a rate of $8$ m/sec.}
  \choice[correct]{The fly is moving away from the origin at a rate of $8$ m/sec.}
\choice{The fly is approaching the origin at a rate of $12$ m/sec.}
\end{selectAll}

\item Suppose at this moment the fly is neither approaching nor moving away from the origin. Find a possibility for the fly's velocity  at this time.

\item Suppose at this moment the fly is moving away from the origin at the rate of $7$ m/sec. Use vector arithmetic, \emph{not} algebra, to find all possibilities for the fly's velocity  at this time.

\end{enumerate}
\end{exercise}

\begin{exercise} \label{ExLdf3eeRE}
At a certain moment you are driving at a speed of $50$ miles/hour. What can you say about the rate of change in your distance to home at this moment. The rate of change here is \emph{not} with respect to time but with respect to the odometer reading on your car. 
\end{exercise}


\begin{exercise} \label{ExLdfr3r3}
\begin{enumerate}
\item As a mosquito flies its velocity is always perpendicular to its position relative to the point $A$. What can you say about the path of the mosquito? Prove your assertion.

\item As a mosquito flies its distance from the point $A$ remains constant. What can you say about the velocity of the mosquito? Prove your assertion.
  
\end{enumerate}
\end{exercise}



\end{document}
\documentclass{ximera}
\title{Functions of Two Independent Variables}

\newcommand{\pskip}{\vskip 0.1 in}

\begin{document}
\begin{abstract}
Introduction to functions of two or more independent variables.
\end{abstract}
\maketitle


\section{Daylight Hours}

\begin{example}   \label{E3eertr45r43324}
The function
\begin{align*}
      L &= f(\delta, \phi) \\
         & = \frac{24}{\pi}\arccos\left( -\tan \delta \tan \phi  \right) , 
\end{align*}
for
\[
             -\delta_0 \leq \delta \leq \delta_0 \text{ and } -\pi/2+\delta \leq \phi \leq \pi/2-\delta ,
\]
expresses the length of the day (in hours, from sunrise to sunset) in terms of the declination $\delta$ of the sun and the latitude $\phi$ of a location on the earth. The declination is the angle that the sun's rays make with the plane of the equator. It attains its maximum value of 
\[
   \delta_0 = 23.45^\circ
\]
on the summer solstice (in the northern hemisphere) and its minimum $-\delta_0 \sim -23.45^\circ$ on the winter solstice.

\begin{enumerate}

\item Write the domain of $f$ in set notation. Then sketch the domain.

\item Write the range of $f$ in set notation.

\item The graph of the function $f$ is shown below.

\begin{onlineOnly}
    \begin{center}
\desmosThreeD{kzyp9rm9i2}{900}{600}
\end{center}
\end{onlineOnly}

\href{https://www.desmos.com/3d/kzyp9rm9i2}{163: Length of Day}

\begin{enumerate}
\item Drag the slider $L$ and explain what the animation illustrates.

\item Sketch the level set $L=12$.  
\end{enumerate}


\item Some level curves of the function $f$ are shown below.

\begin{onlineOnly}
    \begin{center}
\desmos{ub47uf52hj}{900}{600}
\end{center}
\end{onlineOnly}

\href{https://www.desmos.com/calculator/ub47uf52hj}{163: Length of Day 1}

\begin{enumerate}
\item Interpret the meanings of the partial derivatives $\partial L/\partial \delta$ and $\partial L / \partial \phi$. Include units in your interpretation.

\item Drag point $P$ above to a position where the two partials above have \emph{different} signs. Then use the level curves above to approximate the two partials at the point $P$. Include units and interpret their meanings.

\item Use calculus to compute the two partials at $P$ and compare these values with your estimates.

\item Are there any points in the domain at which the two partials are undefined? Explain your reasoning.
\end{enumerate}


\end{enumerate}
\end{example}




\begin{example}  \label{E345432565554}
\begin{onlineOnly}
    \begin{center}
\desmosThreeD{jhuok4umw3}{900}{600}
\end{center}
\end{onlineOnly}

\href{https://www.desmos.com/3d/jhuok4umw3}{163: Length of Day Geometry}

\end{example}


\end{document}
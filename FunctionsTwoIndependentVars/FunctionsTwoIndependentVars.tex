\documentclass{ximera}
\title{Functions of Two Independent Variables}

\newcommand{\pskip}{\vskip 0.1 in}

\begin{document}
\begin{abstract}
Introduction to functions of two or more independent variables.
\end{abstract}
\maketitle


\section{Daylight Hours}


\begin{example} \label{Edft45346t34}
\begin{onlineOnly}
    \begin{center}
\desmosThreeD{jhuok4umw3}{900}{600}
\end{center}
\end{onlineOnly}

\href{https://www.desmos.com/3d/jhuok4umw3}{163: Length of Day Geometry}



The function
\begin{align*}
      L &= f(\delta, \phi) \\
         & = \frac{24}{\pi}\arccos\left( -\tan \delta \tan \phi  \right) , 
\end{align*}
for
\[
             -\delta_0 \leq \delta \leq \delta_0 \text{ and } -\pi/2+\delta \leq \phi \leq \pi/2-\delta ,
\]
expresses the length of the day (in hours, from sunrise to sunset) in terms of the declination $\delta$ of the sun and the latitude $\phi$ of a location on the earth. The declination is the angle that the sun's rays make with the plane of the equator. It attains its maximum value of 
\[
   \delta_0 = 23.45^\circ
\]
on the summer solstice (in the northern hemisphere) and its minimum $-\delta_0 \sim 23.45^\circ$ on the winter solstice.

The graph of the function $f$ is shown below.

\begin{onlineOnly}
    \begin{center}
\desmosThreeD{kzyp9rm9i2}{900}{600}
\end{center}
\end{onlineOnly}

\href{https://www.desmos.com/3d/kzyp9rm9i2}{163: Length of Day}


Some level curves of the function $f$ are shown below.

\begin{onlineOnly}
    \begin{center}
\desmos{skdmdgnui1}{900}{600}
\end{center}
\end{onlineOnly}

\href{https://www.desmos.com/calculator/skdmdgnui1}{163: Length of Day 1}



\end{example}


\end{document}
\documentclass{ximera}
\title{Lines in Space}

\newcommand{\pskip}{\vskip 0.1 in}

\begin{document}
\begin{abstract}
Lines in $\mathbb{R}^3$.
\end{abstract}
\maketitle


\begin{exploration}   \label{Exsd67g:Line}
Let ${\bf a} = \overrightarrow{OA}$ and  ${\bf b} = \overrightarrow{OB}$.

The parameterization 
\[
    {\bf r} = {\bf a} + t({\bf b} - {\bf a}) = t{\bf b} + (1-t){\bf a} , t\in \mathbb{R}
\]
is shown below. Slide $t$ and interpret the geometric meaning of the parameter $t$.

\pdfOnly{
Access Geogebra interactives through the online version of this text at
 
\href{https://www.geogebra.org/classic/xbrvnrtp}.
}
 
\begin{onlineOnly}
    \begin{center}
\geogebra{xbrvnrtp}{900}{600}
\end{center}
\end{onlineOnly}

\end{exploration}


\begin{question}  \label{Qdr5577:Lines}
A light ray parallel to the vector ${\bf v} = \langle 3,2,-4 \rangle$ and passing through the point $P(12, 17,20)$ reflects off the $xy$-plane as if the plane were a mirror.

(a) Parameterize the path of the incoming light ray. Include a domain.

(b) Parameterize the path of the reflected ray. Include a domain.

(c) Repeat the same two questions in general for a light ray parallel to the vector ${\bf v}$ passing through the point $P$ that reflects off the plane through the origin normal to the vector ${\bf n}$. Work with the vectors ${\bf v}$, ${\bf n}$, and ${\bf p} = \overrightarrow{OP}$ giving the position of $P$ relative to the origin. Avoid working with their components. 
\end{question}



Drawing a Square in Space

\href{https://www.desmos.com/3d/3a19b63ef0}{Square in Space 2}


\end{document}
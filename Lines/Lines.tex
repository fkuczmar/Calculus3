\documentclass{ximera}
\title{Lines in Space}

\newcommand{\pskip}{\vskip 0.1 in}

\begin{document}
\begin{abstract}
Lines in $\mathbb{R}^3$.
\end{abstract}
\maketitle


\begin{exploration}   \label{Exsd67g:Line}
Let ${\bf a} = \overrightarrow{OA}$ and  ${\bf b} = \overrightarrow{OB}$.

The parameterization 
\[
    {\bf r} = {\bf a} + t({\bf b} - {\bf a}) = t{\bf b} + (1-t){\bf a} , t\in \mathbb{R}
\]
is shown below. Slide $t$ and interpret the geometric meaning of the parameter $t$.

\pdfOnly{
Access Geogebra interactives through the online version of this text at
 
\href{https://www.geogebra.org/classic/xbrvnrtp}.
}
 
\begin{onlineOnly}
    \begin{center}
\geogebra{xbrvnrtp}{900}{600}
\end{center}
\end{onlineOnly}

\end{exploration}


\begin{question}  \label{Qdr5577:Lines}
A light ray parallel to the vector ${\bf v} = \langle 3,2,-4 \rangle$ and passing through the point $P(12, 17,20)$ reflects off the $xy$-plane as if the plane were a mirror.

(a) Parameterize the path of the incoming light ray. Include a domain.

(b) Parameterize the path of the reflected ray. Include a domain.

(c) Repeat parts (a) and (b) for a light ray parallel to the vector ${\bf v} = \langle 2,1,4 \rangle$ through the point $P(-1,3,-7)$ that reflects offf the plane through the origin and is normal to the vector ${\bf n} = \langle  1, -1, 2 \rangle$.

(d) More generally, suppose a light ray parallel to the vector ${\bf v}$ passing through the point $P$ reflects off the plane through the origin normal to the vector ${\bf n}$. Parameterize the reflected ray in terms of the vectors ${\bf v}$, ${\bf n}$, and ${\bf p} = \overrightarrow{OP}$ giving the position of $P$ relative to the origin. Avoid working with the components of these vectors. Include a domain for the parameterization.

Here is an outline of the solution to part (c):

To parameterize the reflected ray, we need to find the coordinates of a point $Q$ on the reflected ray and a vector parallel to the reflected ray. 

The only possible choice for $Q$ is the point where the incoming ray intersects the plane. To find the vector $\overrightarrow{OQ}$ giving the position of $Q$ relative to the origin, do the following:

\begin{itemize}

\item{First write a vector equation of the incoming ray with $t$ as the parameter. This equation will involve the vectors ${\bf v}$ and ${\bf  p}$ and is similar (but not identical) to the parameterization in Exploration 1 above.} 

\item{Next use your vector equation of the incoming ray to find the value of the parameter $t$ that gives the vector $\overrightarrow{OQ}$. Do this by solving an equation that describes the geometric relationship between the vectors $\overrightarrow{OQ}$ and ${\bf n}$. }

\item{Finally, use the value of the parameter in the previous step to find an expression for $\overrightarrow{OQ}$.}

\end{itemize}

Next we need to find a vector parallel to the reflected ray. For this, see the section {\bf Light Reflecting off a Mirror} in the {\bf Scalar Product} chapter of our class notes:

\href{https://ximera.osu.edu/calc3/Calculus3/ScalarProduct/ScalarProduct}{Class Notes}

Then put all this together to find a vector equation of the reflected ray. Include the appropriate domain.

The last step is to check your work in Desmos. Unfortunately, Desmos does not have a function for the scalar product of two vectors. But we can get around this by defining three scalars,
\[
    s_{vn} = {\bf v}\cdot {\bf n} ,
\]
\[
    s_{pn} =  {\bf p}\cdot {\bf n},
\]
and 
\[
       s_{nn} =  {\bf n}\cdot {\bf n} .
\]

These scalars are imbedded in the Desmos worksheet below and you will need to use them for the scalar products. Follow the directions in the worksheet to check your work. 


Access the Desmos worksheet at:

\href{https://www.desmos.com/3d/85f82721d5}{163: Ray Reflecting Off a Plane Student Copy}



\end{question}



Drawing a Square in Space

\href{https://www.desmos.com/3d/3a19b63ef0}{Square in Space 2}


\end{document}
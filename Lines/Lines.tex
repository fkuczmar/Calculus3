\documentclass{ximera}
\title{Lines in Space}

\newcommand{\pskip}{\vskip 0.1 in}

\begin{document}
\begin{abstract}
Lines in $\mathbb{R}^3$.
\end{abstract}
\maketitle


\begin{exploration}   \label{Exsd67g:Line}
Let ${\bf a} = \overrightarrow{OA}$ and  ${\bf b} = \overrightarrow{OB}$.

The parameterization 
\[
    {\bf r} = {\bf a} + t({\bf b} - {\bf a}) = t{\bf b} + (1-t){\bf a} , t\in \mathbb{R}
\]
is shown below. Slide $t$ and interpret the geometric meaning of the parameter $t$ in terms of the number line through $A$ and $B$ with its zero at $A$. You may drag points $A$ and $B$ in the demonstration below.

\pdfOnly{
Access Geogebra interactives through the online version of this text at
 
\href{https://www.geogebra.org/classic/xbrvnrtp}.
}
 
\begin{onlineOnly}
    \begin{center}
\geogebra{xbrvnrtp}{900}{600}
\end{center}
\end{onlineOnly}

\end{exploration}


\begin{question} \label{Qf4tlhlnm4}
\begin{enumerate}

\item Which of the following determine a unique line in $\mathbb{R}^2$? Select all that apply.
\begin{selectAll}
\choice[correct]{Two distinct points}
\choice[correct]{A point and a slope}
\choice[correct]{A point and a vector parllel to the line}
\choice[correct]{A point and a vector orthogonal to the line}
\end{selectAll}

\item Which of the following determine a unique line in $\mathbb{R}^3$? Select all that apply.
\begin{selectAll}
\choice[correct]{Two distinct points}
\choice{A point and a slope}
\choice[correct]{A point and a vector parllel to the line}
\choice{A point and a vector orthogonal to the line}
\end{selectAll}

\end{enumerate}
\end{question}

\begin{example} \label{Ex3434523093}
\begin{enumerate}
\item Find a vector equation of the line in $\mathbb{R}^3$ through the point $A(3,-2,1)$ parallel to the vector ${\bf v}  =\langle 1,4,-2\rangle$.

\item Parameterize the line in part (a).

\item Find the coordinates of the point where the line intesects the $xz$-plane.

\end{enumerate}

\begin{explanation}
\begin{enumerate}
\item A vector equation expresses the positions of all points on the line relative to the origin $O$. For this line (call it ${\cal L}$), a point $P$ lies on ${\cal L}$ if and only if the vector $\overrightarrow{AP}$ is parallel (or anti-parallel) to the vector ${\bf v}$. And this is true if and only if
\[
       \overrightarrow{AP} = t {\bf v} \text{ for some scalar } t\in \mathbb{R} .
\]
But since
\[
    \overrightarrow{AP} = \overrightarrow{OP} - \overrightarrow{OA} ,
\]
a point $P$ lies on ${\cal L}$ if and only if
\[
    \overrightarrow{OP} = \overrightarrow{OA} + t {\bf v} \text{ for some scalar } t\in \mathbb{R}.
\]
This is a vector equation of ${\cal L}$. Think of it as giving directions from the origin to a point $P$ on ${\cal L}$ - first go to the point $A$ and then move some distance in the same or the opposite direction of the vector ${\bf v}$.

What distance?
\begin{multipleChoice}
\choice{$t$}
\choice{$|{\bf v}|$}
\choice{$t|{\bf v}|$}
\choice[correct]{$|t{\bf v}|$}
\end{multipleChoice}

\item A line is a set of points, not a set of vectors, and a parameterization of the line expresses the coordinates of all points on the line in terms of some parameter. Supposing a general point $P$ on the line has coordinates $(x,y,z)$, that $A$ has coordinates $(3,-2,1)$, and ${\bf v} = \langle 1,4,-2\rangle$, we can rewrite the vector equation
\[
    \overrightarrow{OP} = \overrightarrow{OA} + t {\bf v} \text{ for some scalar } t\in \mathbb{R}.
\]
as
\[
    \langle x, y, z \rangle = \langle \answer{3}, \answer{-2}, \answer{1} \rangle + t \langle \answer{1}, \answer{4}, \answer{-2} \rangle \text{ for some scalar} t\in \mathbb{R}.
\]
Equating the components of these vectors gives a parameterization of the line ${\cal L}$ as 
\[
   (x,y,z) = (3+t , -2 + 4t, \answer{1-2t}) \, , \, t\in \mathbb{R} .
\]

\item The point $P(x,y,z)$ lies on both the line and the $xz$-plane exactly when
\[
     y = -2 + 4t = 0 
\]
or exactly when $t=1/2$. So the point where the line intersects the $xz$-plane has coordinates $(\answer{5/2}, \answer{0}, \answer{2})$.
\end{enumerate} 
\end{explanation}

\end{example} 


\begin{question}  \label{Qdr5577:Lines}
A light ray parallel to the vector ${\bf v} = \langle 3,2,-4 \rangle$ and passing through the point $P(12, 17,20)$ reflects off the $xy$-plane as if the plane were a mirror.

\begin{enumerate}

\item Parameterize the path of the incoming light ray. Include a domain.

\item Parameterize the path of the reflected ray. Include a domain.

\item Repeat parts (a) and (b) for a light ray parallel to the vector ${\bf v} = \langle 2,1,4 \rangle$ through the point $P(-1,3,-7)$ that reflects off the plane through the origin that is normal to the vector ${\bf n} = \langle  1, -1, 2 \rangle$.

\end{enumerate}

\end{question}

\begin{question}  \label{Q324g4t324r}
This question generalizes the previous one.

A light ray parallel to the vector ${\bf v}$ passing through the point $P$ reflects off the plane through the origin normal to the vector ${\bf n}$. Parameterize the reflected ray in terms of the vectors ${\bf v}$, ${\bf n}$, and ${\bf p} = \overrightarrow{OP}$ giving the position of $P$ relative to the origin. Avoid working with the components of these vectors. Include a domain for the parameterization.

Here is an outline of the approach:

\begin{enumerate}
\item To parameterize the reflected ray, we need to find the coordinates of a point $Q$ on the reflected ray and a vector parallel to the reflected ray. 

The only possible choice for $Q$ is the point where the incoming ray intersects the plane. To find the vector $\overrightarrow{OQ}$ giving the position of $Q$ relative to the origin, do the following:

\begin{enumerate}

\item{First write a vector equation of the incoming ray with $t$ as the parameter. This equation will involve the vectors ${\bf v}$ and ${\bf  p}$ and is similar (but not identical) to the parameterization in Exploration 1 above.} 

\item{Next use your vector equation of the incoming ray to find the value of the parameter $t$ that gives the vector $\overrightarrow{OQ}$. Do this by solving an equation that describes the geometric relationship between the vectors $\overrightarrow{OQ}$ and ${\bf n}$. }

\item{Finally, use the value of the parameter in the previous step to find an expression for $\overrightarrow{OQ}$. Input this expression on Line 25 of the desmos worksheet below.}

\end{enumerate}

\item Next find a vector ${\bf w}$ parallel to the reflected ray. Input this on Line 27. %For this, see the section {\bf Light Reflecting off a Mirror} in the {\bf Scalar Product} chapter of our class notes:

%\href{https://ximera.osu.edu/calc3/Calculus3/ScalarProduct/ScalarProduct}{Class Notes}


\item Finally put all this together to find a vector equation of the reflected ray. Include the appropriate domain. Input this on Line 29. Does your result look reasonable?


\begin{onlineOnly}
    \begin{center}
\desmosThreeD{znxtexpsnt}{900}{600}
\end{center}
\end{onlineOnly}

\href{https://www.desmos.com/3d/znxtexpsnt}{163: Ray Reflecting Off a Plane Student Copy 2}

%\item The last step is to check your work in Desmos. %Unfortunately, Desmos does not have a function for the scalar product of two vectors. But we can get around this by defining three scalars,
%\[
%    s_{vn} = {\bf v}\cdot {\bf n} ,
%\]
%\[
%    s_{pn} =  {\bf p}\cdot {\bf n},
%\]
%and 
%\[
%       s_{nn} =  {\bf n}\cdot {\bf n} .
%\]

%These scalars are imbedded in the Desmos worksheet below and you will need to use them for the scalar products. Follow the directions in the worksheet to check your work. 


\end{enumerate}

\end{question}



%Drawing a Square in Space

%\href{https://www.desmos.com/3d/3a19b63ef0}{Square in Space 2}


\end{document}